\documentclass{ximera}

\newcommand{\RR}{\mathbb R}
\renewcommand{\d}{\,d}
\newcommand{\dd}[2][]{\frac{d #1}{d #2}}
\renewcommand{\l}{\ell}
\newcommand{\ddx}{\frac{d}{dx}}
\newcommand{\dfn}{\textbf}
\newcommand{\eval}[1]{\bigg[ #1 \bigg]}
\renewcommand{\theenumii}{\textup{(\roman{enumii})}}
\renewcommand{\labelenumii}{\theenumii}

\usepackage{graphicx}
\usepackage{multicol}
\usepackage{tkz-euclide}
%\usepackage{unicode-math}

\usepackage{pgfplots}   % <- for graphics
\pgfplotsset{compat=newest}


\renewenvironment{freeResponse}{
\ifhandout\setbox0\vbox\bgroup\else
\begin{trivlist}\item[\hskip \labelsep\bfseries Solution:\hspace{2ex}]
\fi}
{\ifhandout\egroup\else
\end{trivlist}
\fi}

\newcommand*{\ZeroOverZero}{\ensuremath{\dfrac{0}{0}}}

\providecommand{\HCCondition}{0}
\newcommand{\WkstHop}[1][1]{\if\HCCondition 0
	\vspace*{\stretch{#1}} \fi} 
\newcommand{\WkstNew}{\if\HCCondition 0
	\newpage
	 \fi} 


\title[Problem 2]{Problem 2}

\begin{document}
\begin{abstract} \end{abstract}
\maketitle


% Extracted from whatIsALimit.tex, problem #2
\begin{problem}
Use the graphs and the given definitions of the following two functions to answer the questions below.
	
	\begin{image}
		\includegraphics[scale = 0.7, alt={Graph of f(x) = (x^2-1)/(x-1)}]{graphRationalFunction.png}
	\end{image}

	\begin{image}
	        \includegraphics[scale = 0.7, alt={Graph of g(x) = x+1}]{graphLinearFunction.png}
	\end{image}

	\begin{enumerate}
	
	\item Find the domain of $f$ and the domain of $g$.
	\begin{explanation}
		        The domain of $f$ is $(-\infty, 1) \cup (1, \infty)$ (all real numbers except $1$).
		        The domain of $g$ is $(-\infty, \infty)$ (all real numbers).
	      \end{explanation}
	
  	\item Is $f = g$? (Why or why not?)
	\begin{explanation}
	        No, these two functions are not equal.
	        Two functions are equal if and only if they have identical domains and their values agree on all points in the domain.
	
	        Since $f$ and $g$ have different domains, by part (a), they cannot be equal functions.
	\end{explanation}
	
	 \item  Looking at the graphs, find $\lim_{x \to 1} f(x)$ and $\lim_{x \to 1} g(x)$.
\begin{explanation}
        From the graphs we have that $\lim_{x \to 1} f(x) = 2$ and $\lim_{x \to 1} g(x) = 2$.
      \end{explanation}

	\end{enumerate}
\end{problem}



\end{document}
