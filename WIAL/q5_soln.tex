\documentclass{ximera}

\newcommand{\RR}{\mathbb R}
\renewcommand{\d}{\,d}
\newcommand{\dd}[2][]{\frac{d #1}{d #2}}
\renewcommand{\l}{\ell}
\newcommand{\ddx}{\frac{d}{dx}}
\newcommand{\dfn}{\textbf}
\newcommand{\eval}[1]{\bigg[ #1 \bigg]}
\renewcommand{\theenumii}{\textup{(\roman{enumii})}}
\renewcommand{\labelenumii}{\theenumii}

\usepackage{graphicx}
\usepackage{multicol}
\usepackage{tkz-euclide}
%\usepackage{unicode-math}

\usepackage{pgfplots}   % <- for graphics
\pgfplotsset{compat=newest}


\renewenvironment{freeResponse}{
\ifhandout\setbox0\vbox\bgroup\else
\begin{trivlist}\item[\hskip \labelsep\bfseries Solution:\hspace{2ex}]
\fi}
{\ifhandout\egroup\else
\end{trivlist}
\fi}

\newcommand*{\ZeroOverZero}{\ensuremath{\dfrac{0}{0}}}

\providecommand{\HCCondition}{0}
\newcommand{\WkstHop}[1][1]{\if\HCCondition 0
	\vspace*{\stretch{#1}} \fi} 
\newcommand{\WkstNew}{\if\HCCondition 0
	\newpage
	 \fi} 


\title[Problem 5]{Problem 5}

\begin{document}
\begin{abstract} \end{abstract}
\maketitle


% Extracted from whatIsALimit.tex, problem #5
\begin{problem}
True/False:  Give an explanation or counterexample.  Assume $a$ and $L$ are finite numbers.
	
	\begin{enumerate}
		\item  If $ \lim_{x \to a} f(x) = L$, then $f(a) = L$.
		\begin{explanation}
			False.  In the graph below $ \lim_{x \to 1} f(x) = 2 $, but $f(1)$ does not exist.
			
				\begin{image}
				      \includegraphics[alt={graph of function with f(x)=x+1, for x not equal to 1. it has an open circle at the point (1, 2)}]{Figure1.png}
				\end{image}
			\end{explanation}
			
		\item  If $  \lim_{x \to a^-} f(x) = L$, then $  \lim_{x \to a^+} f(x) = L $.
		\begin{explanation}
			False.  In the graph below $ \lim_{x \to 2^-} f(x) = -1$ but $ \lim_{x \to 2^+} f(x) = 1$.
			
				\begin{image}
					\includegraphics[scale=.7, alt={graph of a function that is 1 for x>2, but -1 for x < 2.}]{Figure2.png}		
				\end{image}
			\end{explanation}
			
		\item  If $ \lim_{x \to a} f(x) = L $ and $  \lim_{x \to a} g(x) = L $, then $f(a) = g(a)$.
		\begin{explanation}
				 False.  If we let 
			 
			 	\begin{image}
			       \includegraphics[alt={graph of a line and parabola, with open circle at their intersection. points are filled in above the intersection for the line and below the intersection for the parabola.}]{Figure3.png}
			 	\end{image}
				
			 we see that $ \lim_{x \to 1} f(x) = \lim_{x \to 1} g(x) = 2$, but $f(1) \ne g(1)$.
			\end{explanation}
			
		\item  $ \lim_{x \to a} \frac{f(x)}{g(x)} $ does not exist if $g(a) = 0$.
		\begin{explanation}
				False.  If $f(x) = 5x^2$ and $g(x) = x^2$, then $g(0) = 0$ but 
				$$ \lim_{x \to 0} \frac{f(x)}{g(x)} = \lim_{x \to 0} \frac{5x^2}{x^2} = \lim_{x \to 0} 5 = 5.$$
			\end{explanation}
			
			
			
		\end{enumerate}
\end{problem}



\end{document}
