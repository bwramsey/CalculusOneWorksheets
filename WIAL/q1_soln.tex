\documentclass{ximera}

\newcommand{\RR}{\mathbb R}
\renewcommand{\d}{\,d}
\newcommand{\dd}[2][]{\frac{d #1}{d #2}}
\renewcommand{\l}{\ell}
\newcommand{\ddx}{\frac{d}{dx}}
\newcommand{\dfn}{\textbf}
\newcommand{\eval}[1]{\bigg[ #1 \bigg]}
\renewcommand{\theenumii}{\textup{(\roman{enumii})}}
\renewcommand{\labelenumii}{\theenumii}

\usepackage{graphicx}
\usepackage{multicol}
\usepackage{tkz-euclide}
%\usepackage{unicode-math}

\usepackage{pgfplots}   % <- for graphics
\pgfplotsset{compat=newest}


\renewenvironment{freeResponse}{
\ifhandout\setbox0\vbox\bgroup\else
\begin{trivlist}\item[\hskip \labelsep\bfseries Solution:\hspace{2ex}]
\fi}
{\ifhandout\egroup\else
\end{trivlist}
\fi}

\newcommand*{\ZeroOverZero}{\ensuremath{\dfrac{0}{0}}}

\providecommand{\HCCondition}{0}
\newcommand{\WkstHop}[1][1]{\if\HCCondition 0
	\vspace*{\stretch{#1}} \fi} 
\newcommand{\WkstNew}{\if\HCCondition 0
	\newpage
	 \fi} 


\title[Problem 1]{Problem 1}

\begin{document}
\begin{abstract} \end{abstract}
\maketitle


% Extracted from whatIsALimit.tex, problem #1
\begin{problem} \hfil
	\begin{enumerate}

	\item  True or False: To find $\lim_{x \to 2} f(x)$, it's enough to know the values of $f(2.1)$, $f(2.01)$, $f(2.001)$, and so on.
\begin{explanation}
		 False.  These values will only help us make a guess at $\lim_{x \to 2^+} f(x)$, the right hand limit of $f$ 
		as $x$ approaches $2$.  To determine $\lim_{x \to 2} f(x)$, we also need to know $\lim_{x \to 2^-} f(x)$ which
 		we cannot determine from the above values.  For example, consider the function
	 	 \[
	          f(x) =
	        \begin{cases}
		         -1 & \mbox{if $x < 2$,}\\
		          0 & \mbox{if $x = 2$, and}\\
		          1 & \mbox{if $2 < x$.}
	        \end{cases}
	        \]

	Looking at the graph of this function below, we can see that $\lim_{x \to 2^+} f(x) = 1$ and $\lim_{x \to 2^-} f(x) = -1$.  Thus, since $\lim_{x \to 2^+} f(x) \neq \lim_{x \to 2^-} f(x)$, $\lim_{x \to 2} f(x)$ does not exist.
	
		\begin{image}
          \includegraphics[scale = 0.6, alt={graph of function that is 1 for x>2, -1 for x<2, and with f(2)=0}]{graphPiecewiseFunction.png}
		\end{image}

	\end{explanation}
	
	
	
	\item  True or False: If we know the value of $f(2)$, then can we conclude that $\lim_{x \to 2} f(x)=f(2)$?
\begin{explanation}
		False.  In the above example, we have that $f(2) = 0$, while $\lim_{x \to 2} f(x)$ does not exist.
		\end{explanation}

	\end{enumerate}


\end{problem}



\end{document}
