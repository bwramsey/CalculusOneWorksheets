\documentclass{ximera}

\newcommand{\RR}{\mathbb R}
\renewcommand{\d}{\,d}
\newcommand{\dd}[2][]{\frac{d #1}{d #2}}
\renewcommand{\l}{\ell}
\newcommand{\ddx}{\frac{d}{dx}}
\newcommand{\dfn}{\textbf}
\newcommand{\eval}[1]{\bigg[ #1 \bigg]}
\renewcommand{\theenumii}{\textup{(\roman{enumii})}}
\renewcommand{\labelenumii}{\theenumii}

\usepackage{graphicx}
\usepackage{multicol}
\usepackage{tkz-euclide}
%\usepackage{unicode-math}

\usepackage{pgfplots}   % <- for graphics
\pgfplotsset{compat=newest}


\renewenvironment{freeResponse}{
\ifhandout\setbox0\vbox\bgroup\else
\begin{trivlist}\item[\hskip \labelsep\bfseries Solution:\hspace{2ex}]
\fi}
{\ifhandout\egroup\else
\end{trivlist}
\fi}

\newcommand*{\ZeroOverZero}{\ensuremath{\dfrac{0}{0}}}

\providecommand{\HCCondition}{0}
\newcommand{\WkstHop}[1][1]{\if\HCCondition 0
	\vspace*{\stretch{#1}} \fi} 
\newcommand{\WkstNew}{\if\HCCondition 0
	\newpage
	 \fi} 

\title[Problem 5]{Problem 5}

\begin{document}
\begin{abstract} \end{abstract}
\maketitle

% Extracted from whatIsALimit.tex, problem #5
\begin{problem}
True/False:  Give an explanation or counterexample.  Assume $a$ and $L$ are finite numbers.
	
	\begin{enumerate}
		\item  If $ \lim_{x \to a} f(x) = L$, then $f(a) = L$.
		\item  If $  \lim_{x \to a^-} f(x) = L$, then $  \lim_{x \to a^+} f(x) = L $.
		\item  If $ \lim_{x \to a} f(x) = L $ and $  \lim_{x \to a} g(x) = L $, then $f(a) = g(a)$.
		\item  $ \lim_{x \to a} \frac{f(x)}{g(x)} $ does not exist if $g(a) = 0$.
		\end{enumerate}
\end{problem}

\end{document}
