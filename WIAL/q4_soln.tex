\documentclass{ximera}

\newcommand{\RR}{\mathbb R}
\renewcommand{\d}{\,d}
\newcommand{\dd}[2][]{\frac{d #1}{d #2}}
\renewcommand{\l}{\ell}
\newcommand{\ddx}{\frac{d}{dx}}
\newcommand{\dfn}{\textbf}
\newcommand{\eval}[1]{\bigg[ #1 \bigg]}
\renewcommand{\theenumii}{\textup{(\roman{enumii})}}
\renewcommand{\labelenumii}{\theenumii}

\usepackage{graphicx}
\usepackage{multicol}
\usepackage{tkz-euclide}
%\usepackage{unicode-math}

\usepackage{pgfplots}   % <- for graphics
\pgfplotsset{compat=newest}


\renewenvironment{freeResponse}{
\ifhandout\setbox0\vbox\bgroup\else
\begin{trivlist}\item[\hskip \labelsep\bfseries Solution:\hspace{2ex}]
\fi}
{\ifhandout\egroup\else
\end{trivlist}
\fi}

\newcommand*{\ZeroOverZero}{\ensuremath{\dfrac{0}{0}}}

\providecommand{\HCCondition}{0}
\newcommand{\WkstHop}[1][1]{\if\HCCondition 0
	\vspace*{\stretch{#1}} \fi} 
\newcommand{\WkstNew}{\if\HCCondition 0
	\newpage
	 \fi} 


\title[Problem 4]{Problem 4}

\begin{document}
\begin{abstract} \end{abstract}
\maketitle


% Extracted from whatIsALimit.tex, problem #4
\begin{problem}
Sketch a possible graph of a function that satisfies all of the given properties. (You \emph{do not} need to find a formula for the function.)

	\begin{align*}
		 \text{Domain:}\ [-4,3)&\cup(3,5)&  f(1) &= 3 & f(-1) &= 1& f(-4)=1\\
		 \lim_{x \to -1^-} f(x) &= -2 & \lim_{x \to -1^+} f(x) &= 1 &   \lim_{x \to 1} f(x) &= 2&  \lim_{x \to 3^+} f(x) &=-1\\
		\lim_{x \to 3^-} f(x)&=1& & &  \lim_{x \to -4^+} f(x) &= 1 &  \lim_{x \to 5^-} f(x) &= 1 &
	\end{align*}
\begin{explanation} 
	One way to go about sketching the graph is as follows.  First, on the x-axis, indicate the domain (seen here in purple)
		\begin{image}
			\includegraphics[scale=.5, alt={Graph marking the domain along x-axis}]{sketchStage1.png}
		\end{image}

	Next plot the given points (seen here in blue)

	    \begin{image}
		\includegraphics[scale=.5, alt={Graph marking points based on function values given. }]{sketchStage2.png}
  	  \end{image}
	
	Then indicate the limits using open circles and tails to indicate if the limit is the value as the function approaches from 
	the left or right or both(seen here in green)
	
	\begin{image}
		\includegraphics[scale=.5, alt={Graph with open circles indicating the limit values, with tails indicating if the limit is from the left or right.}]{sketchStage3.png}
	\end{image}

	Finally connect the graph.
	
	\begin{image}
		\includegraphics[scale=.5, alt={Graph connecting the lines from previous, to give function.}]{sketchStage4.png}
	 \end{image}
	
	\end{explanation}
\end{problem}



\end{document}
