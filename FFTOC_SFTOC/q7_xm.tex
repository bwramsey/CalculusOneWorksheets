\documentclass{ximera}

\newcommand{\RR}{\mathbb R}
\renewcommand{\d}{\,d}
\newcommand{\dd}[2][]{\frac{d #1}{d #2}}
\renewcommand{\l}{\ell}
\newcommand{\ddx}{\frac{d}{dx}}
\newcommand{\dfn}{\textbf}
\newcommand{\eval}[1]{\bigg[ #1 \bigg]}
\renewcommand{\theenumii}{\textup{(\roman{enumii})}}
\renewcommand{\labelenumii}{\theenumii}

\usepackage{graphicx}
\usepackage{multicol}
\usepackage{tkz-euclide}
%\usepackage{unicode-math}

\usepackage{pgfplots}   % <- for graphics
\pgfplotsset{compat=newest}


\renewenvironment{freeResponse}{
\ifhandout\setbox0\vbox\bgroup\else
\begin{trivlist}\item[\hskip \labelsep\bfseries Solution:\hspace{2ex}]
\fi}
{\ifhandout\egroup\else
\end{trivlist}
\fi}

\newcommand*{\ZeroOverZero}{\ensuremath{\dfrac{0}{0}}}

\providecommand{\HCCondition}{0}
\newcommand{\WkstHop}[1][1]{\if\HCCondition 0
	\vspace*{\stretch{#1}} \fi} 
\newcommand{\WkstNew}{\if\HCCondition 0
	\newpage
	 \fi} 

\title[Problem 7]{Problem 7}

\begin{document}
\begin{abstract} \end{abstract}
\maketitle

% Extracted from FTC.tex, problem #7
\begin{problem}
  The graph of $g$, a continuous function $[0, 4]$, is shown in the figure.
  Let $A(x) = \int_0^x g(t) \d t$ for $0 \le x \le 4$.
  \begin{image}
    \includegraphics[scale = 0.4]{graphContFunction.png}
  \end{image}
  \begin{enumerate}
  
  	%part a
    \item
      Circle the correct statement about $A(1)$.
      \begin{enumerate}
        \item $A(1) = 0$;
        \item $A(1) < 0$;
        \item $A(1) > 0$;
        \item none of the previous answers.
      \end{enumerate}
      
%part b
    \item
      Circle the correct statement about $A(1.5)$.
      \begin{enumerate}
        \item $A(1.5) = 0$;
        \item $A(1.5) < 0$;
        \item $A(1.5) > 0$;
        \item  none of the previous answers.
      \end{enumerate}
      
%part c
    \item
      Circle the correct statement about $A'(1.5)$.
      \begin{enumerate}
        \item $A'(1.5) = 0$;
        \item $A'(1.5) < 0$;
        \item $A'(1.5) > 0$;
        \item none of the previous answers.
      \end{enumerate}
      
%part d
    \item
      Circle the correct expression for $\int_1^4 (g(t) + 1) \d t$.
      \begin{enumerate}
        \item $A(4) + 1$;
        \item $A(4) - A(1)$;
        \item $A(4) - A(1) + 1$;
        \item $A(4) + 3$;
        \item $A(4) - A(1) + 3$;
        \item none of the previous answers.
      \end{enumerate}
      
[3]

% part e
    \item
      Circle the correct statement about $A(0)$.
      \begin{enumerate}
        \item $A(0) = 0$;
        \item $A(0) = 1$;
        \item $0 < A(0) < 1$;
        \item $A(0) > 1$;
        \item $A(0) < 0$;
        \item none of the previous answers.
      \end{enumerate}
      
% part f
    \item
      Circle the interval (or intervals) where the function $A$ is DECREASING.
      \begin{enumerate}
        \item $(0, 1)$;
        \item $(1, 2)$;
        \item $(2, 4)$;
        \item none of the previous answers.
      \end{enumerate}
      
%part g
    \item
      Circle the value (or values) where the function $A$ attains its MAXIMUM.
      \begin{enumerate}
        \item $x = 0$;
        \item $x = 1$;
        \item $x = 2$;
        \item $x = 3$;
        \item $x = 4$;
        \item none of the previous answers.
      \end{enumerate}
      
%part h
    \item
      Circle the value  (or values) where the function $A$ attains its MINIMUM.
      \begin{enumerate}
        \item $x = 0$;
        \item $x = 1$;
        \item $x = 2$;
        \item $x = 3$;
        \item $x = 4$;
        \item none of the previous answers.
      \end{enumerate}
      
%item i
    \item
      Circle the interval (or intervals) where the function $A$ is CONCAVE DOWN.
      \begin{enumerate}
        \item $(0, 1)$;
        \item $(1, 2)$;
        \item $(2, 4)$;
        \item none of the previous answers.
      \end{enumerate}
      
\item
      Circle the value (or values) of $x$ where the function $A$ has an inflection point.
      \begin{enumerate}
        \item $x = 1$;
        \item $x = 2$;
        \item $x = 3$;
        \item none of the previous answers.
      \end{enumerate}
      
\item Sketch the graph of $A$, based on items (e)-(j)

  \end{enumerate}
\end{problem}

\end{document}
