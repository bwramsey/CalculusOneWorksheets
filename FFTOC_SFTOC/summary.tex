\documentclass{ximera}

\newcommand{\RR}{\mathbb R}
\renewcommand{\d}{\,d}
\newcommand{\dd}[2][]{\frac{d #1}{d #2}}
\renewcommand{\l}{\ell}
\newcommand{\ddx}{\frac{d}{dx}}
\newcommand{\dfn}{\textbf}
\newcommand{\eval}[1]{\bigg[ #1 \bigg]}
\renewcommand{\theenumii}{\textup{(\roman{enumii})}}
\renewcommand{\labelenumii}{\theenumii}

\usepackage{graphicx}
\usepackage{multicol}
\usepackage{tkz-euclide}
%\usepackage{unicode-math}

\usepackage{pgfplots}   % <- for graphics
\pgfplotsset{compat=newest}


\renewenvironment{freeResponse}{
\ifhandout\setbox0\vbox\bgroup\else
\begin{trivlist}\item[\hskip \labelsep\bfseries Solution:\hspace{2ex}]
\fi}
{\ifhandout\egroup\else
\end{trivlist}
\fi}

\newcommand*{\ZeroOverZero}{\ensuremath{\dfrac{0}{0}}}

\providecommand{\HCCondition}{0}
\newcommand{\WkstHop}[1][1]{\if\HCCondition 0
	\vspace*{\stretch{#1}} \fi} 
\newcommand{\WkstNew}{\if\HCCondition 0
	\newpage
	 \fi} 

\title[Summary]{Summary}

\begin{document}
\begin{abstract} \end{abstract}
\maketitle



\textbf{The First Fundamental Theorem of Calculus:}

Given a continuous function $f$ on the interval $[a,b]$, we define an accumulation function, $A$ by 
\[ A(x)= \int_{a}^{x} f(t)\d t\]

 The First Fundamental Theorem of Calculus (FFTOC) states that
\[A'(x)=f(x)\]
 or, equivalently, that
\[ \frac{d}{dx} \int_{a}^{x} f(t)\d t =f(x) \]

 \textbf{Note 1: } $A(a)=0$\\[0.5em]
 \textbf{Note 2: } The function $A$ is an antiderivative of $f$.\\
  Therefore, any antiderivative, $G$ of $f$ can be written as
 {\large{$G(x)=A(x)+C $}},  or, equivalently,  \[G(x)=\int_{a}^{x} f(t)\d t +C.\]
Since $G(a)=A(a)+C=0+C=C$, it follows that   {\large{$G(x)=A(x)+G(a) $}},\\
 or, equivalently, that   \[G(x)= G(a)+\int_{a}^{x} f(t)\d t\]
 
 \textbf{Note 3: } Let $f'$, the derivative of $f$, be continuous on $[a,b]$. Since $f$ is its antiderivative, we have
 \[f(x)= f(a)+\int_{a}^{x} f'(t)\d t\]

\textbf{The Second Fundamental Theorem of Calculus:}

 FFTOC implies that, for $x=b$, $A(b)= \int_{a}^{b} f(t)\d t $. Since $A(a)=0$, we can write\\
\[\int_{a}^{b} f(t)\d t =A(b)-A(a)\]
  \textbf{Note 4: } This is the Second Fundamental Theorem of Calculus (SFTOC).\\
  \textbf{Note 5: }Equivalently, using the result in Note 3, for $x=b$, $f(b)= f(a)+\int_{a}^{b} f'(t)\d t  $. So $\int_{a}^{b} f'(t)\d t=f(b)-f(a)$, or equivalently,
  \[ \int_a^b \dfrac{d}{dx} f(x) \d x = f(b) - f(a). \]



\end{document}
