\documentclass{ximera}

\newcommand{\RR}{\mathbb R}
\renewcommand{\d}{\,d}
\newcommand{\dd}[2][]{\frac{d #1}{d #2}}
\renewcommand{\l}{\ell}
\newcommand{\ddx}{\frac{d}{dx}}
\newcommand{\dfn}{\textbf}
\newcommand{\eval}[1]{\bigg[ #1 \bigg]}
\renewcommand{\theenumii}{\textup{(\roman{enumii})}}
\renewcommand{\labelenumii}{\theenumii}

\usepackage{graphicx}
\usepackage{multicol}
\usepackage{tkz-euclide}
%\usepackage{unicode-math}

\usepackage{pgfplots}   % <- for graphics
\pgfplotsset{compat=newest}


\renewenvironment{freeResponse}{
\ifhandout\setbox0\vbox\bgroup\else
\begin{trivlist}\item[\hskip \labelsep\bfseries Solution:\hspace{2ex}]
\fi}
{\ifhandout\egroup\else
\end{trivlist}
\fi}

\newcommand*{\ZeroOverZero}{\ensuremath{\dfrac{0}{0}}}

\providecommand{\HCCondition}{0}
\newcommand{\WkstHop}[1][1]{\if\HCCondition 0
	\vspace*{\stretch{#1}} \fi} 
\newcommand{\WkstNew}{\if\HCCondition 0
	\newpage
	 \fi} 


\title[Problem 4]{Problem 4}

\begin{document}
\begin{abstract} \end{abstract}
\maketitle


% Extracted from FTC.tex, problem #4
\begin{problem}
  Compute the following integrals:
  \begin{enumerate}
    

  \item  $\int_0^1 e^{5x} \d x$
    \begin{explanation}
      $\int_0^1 e^{5x} \d x  = \eval{\frac{1}{5} e^{5x}}_0^1= \frac{1}{5} \left( e^5 - e^0 \right) = \frac{1}{5} (e^5 - 1)$
    \end{explanation}
    \item  $\int\limits_{-2}^{-1} \frac{1}{x^3} \d x$
    \begin{explanation}
      \begin{align*}
        \int\limits_{-2}^{-1} \frac{1}{x^3} \d x &= \int\limits_{-2}^{-1} x^{-3} \d x  \\
                                                 &= \eval{\frac{x^{-2}}{-2}}_{-2}^{-1}  \\
                                                 &= \eval{\frac{-1}{2x^2}}_{-2}^{-1}  \\
                                                 &= - \frac{1}{2} - \left( - \frac{1}{8} \right)  \\
                                                 &= - \frac{1}{2} + \frac{1}{8} = - \frac{3}{8}
      \end{align*}
    \end{explanation}
\item  $\int _0^4 \left( 3x - 5 + 7 \sqrt{16-x^2} \right) \d x$
    \begin{explanation}
      First notice that:
      \begin{equation}\label{eq1}
        \int _0^4 \left( 3x - 5 + 7 \sqrt{16-x^2} \right) \d x = \int_0^4 ( 3x - 5) \d x + 7 \int_0^4 \sqrt{16-x^2} \d x.
      \end{equation}
      To get the easy part out of the way first:
      \begin{equation}\label{eq2}
        \int_0^4 (3x-5) \d x = \eval{\frac{3}{2} x^2 - 5x}_0^4 = (24 - 20) - (0-0) = 4. 
      \end{equation}
      So the real issue is in computing $\int_0^4 \sqrt{16-x^2} \d x$.  
      
      We do not know how to integrate $\sqrt{16-x^2}$ directly, but recall that the solution set to the equation $x^2 + y^2 = 16$ is a circle with radius $4$ centered at the origin.  
      Solving for $y$ we get $y = \pm \sqrt{16-x^2}$.  
      So if we restrict ourselves to $y=\sqrt{16-x^2}$ and $0 \leq x \leq 4$, this gives us the upper-right quarter of the circle (draw a picture and convince yourself of this).  
      Since the total area of the circle of $\pi (4)^2 = 16\pi$, the area under this portion of the curve is $\frac{1}{4} (16\pi) = 4\pi$.  
      Thus we have computed (geometrically) that 
      \begin{equation}\label{eq3}
        \int_0^4 \sqrt{16-x^2} \d x = 4\pi. 
      \end{equation}
      So, $\int _0^4 \left( 3x - 5 + 7 \sqrt{16-x^2} \right) \d x = 4 + 7(4\pi) = 4 + 28\pi$ by (1),(2), and (3)
      \end{explanation}
	[3]

  \end{enumerate}
\end{problem}



\end{document}
