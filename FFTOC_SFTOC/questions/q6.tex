% Extracted from FTC.tex, problem #6
\begin{problem}
  Given the following graph of $y=f(x)$, let $g(x) = \int_{-1}^x f(t) \d t$.
  
      \begin{image}
      \includegraphics{Figure2.png}
    \end{image}
  
  \begin{enumerate}

  \item  Is $g$ continuous?  Why or why not?
     \begin{freeResponse}
     $g(x)$ represents the area function (or the signed area between the curve $y=f(x)$ and the $x$-axis from -1 to $x$).  By the FTC(I), $g(x)$ is continuous.
    \end{freeResponse}
    \WkstHop
    
    
    % part b
  \item  Is $g$ differentiable?  Why or why not?
     \begin{freeResponse}
    Similarly by the FTC(I) the function $g$ is differentiable, and moreover:
        $$\ddx \left( g(x) \right) = \ddx \left( \int_{-1}^x f(t) \d t \right)= f(x)$$
     \end{freeResponse}
    \WkstHop
    
    
    % part c
  \item  Where does $g$ achieve its absolute maximum and minimum values?  Where does $g$ achieve any local extreme values?  Assume the domain of $g$ is $[-1,7.6]$.
    \begin{freeResponse}
    Since $g$ is continuous on the closed interval $[-1,7.6]$, $g$ attains its absolute extreme values at either critical points or endpoints.  
        But we just saw that $g^\prime(x) = f(x)$, and so the critical points of $g$ are just the points where $f(x) = 0$.  
        Thus, the critical points of $g$ are approximately $3.1$ and $6.25$.
        
        Since $f = g^\prime$, $g$ is increasing when $f$ is positive and $g$ is decreasing when $f$ is negative.  
        So we have that $x=3.1$ is a local maximum while $x=6.25$ is a local minimum of $g$.  
        This takes care of the local extreme values.  
        
        To find the absolute extreme values of $g$, we need to determine what are the largest and smallest values among $g(-1), g(3.1), g(6.25),$ and $g(7.6)$.  
        And do not forget the $g(x)$ denotes the (signed) area of the region bounded by the curve and the $x-axis$.
        
        Clearly $g(-1) = 0$ since a line segment does not have any area.  
        At any time after $x=-1$ we have a positive (net) area.  
        Thus, our absolute minimum occurs at $x=-1$.
        To find the absolute maximum of $g$, we need to compare $g(3.1)$ and $g(7.6)$. 
        We see that we have the greatest area under the curve at $x=3.1$ because 
        by the time we get to $x=7.6$, the area we have subtracted off between $x=3.1$ and $x=6.25$ is greater than 
        what we have added back on from $x=6.25$ to $x=7.6$.  
        Therefore, our absolute maximum occurs at $x=3.1$.  
     \end{freeResponse}
    \WkstHop[4]
    
    
    % part d
  \item  Where is the graph of $g$ concave up?  Concave down?
    \begin{freeResponse}
     $g$ is concave up when when $f$ is increasing, which is on $(-1,0) \cup (4.25,7.6)$.  
        Similarly, $g$ is concave down when $f$ is decreasing, which is on $(0,4.25)$.  Remember, $g'=f$.
    \end{freeResponse}
	\WkstHop[2]

  \end{enumerate}
\end{problem}
