\documentclass{ximera}

\newcommand{\RR}{\mathbb R}
\renewcommand{\d}{\,d}
\newcommand{\dd}[2][]{\frac{d #1}{d #2}}
\renewcommand{\l}{\ell}
\newcommand{\ddx}{\frac{d}{dx}}
\newcommand{\dfn}{\textbf}
\newcommand{\eval}[1]{\bigg[ #1 \bigg]}
\renewcommand{\theenumii}{\textup{(\roman{enumii})}}
\renewcommand{\labelenumii}{\theenumii}

\usepackage{graphicx}
\usepackage{multicol}
\usepackage{tkz-euclide}
%\usepackage{unicode-math}

\usepackage{pgfplots}   % <- for graphics
\pgfplotsset{compat=newest}


\renewenvironment{freeResponse}{
\ifhandout\setbox0\vbox\bgroup\else
\begin{trivlist}\item[\hskip \labelsep\bfseries Solution:\hspace{2ex}]
\fi}
{\ifhandout\egroup\else
\end{trivlist}
\fi}

\newcommand*{\ZeroOverZero}{\ensuremath{\dfrac{0}{0}}}

\providecommand{\HCCondition}{0}
\newcommand{\WkstHop}[1][1]{\if\HCCondition 0
	\vspace*{\stretch{#1}} \fi} 
\newcommand{\WkstNew}{\if\HCCondition 0
	\newpage
	 \fi} 


\title[Problem 7]{Problem 7}

\begin{document}
\begin{abstract} \end{abstract}
\maketitle


% Extracted from FTC.tex, problem #7
\begin{problem}
  The graph of $g$, a continuous function $[0, 4]$, is shown in the figure.
  Let $A(x) = \int_0^x g(t) \d t$ for $0 \le x \le 4$.
  \begin{image}
    \includegraphics[scale = 0.4]{graphcontfunction.png}
  \end{image}
  \begin{enumerate}
  
  	%part a
    \item
      Circle the correct statement about $A(1)$.
      \begin{enumerate}
        \item $A(1) = 0$;
        \item $A(1) < 0$;
        \item $A(1) > 0$;
        \item none of the previous answers.
      \end{enumerate}
      \begin{explanation}
        Correct choice: (ii) $A(1) < 0$.  $A(1)$ is the net area of the region bounded by the curve and the $x$-axis between $x=0$ and $x=1$.  On this interval, $g(t)$ is negative and therefore, $ A(1)$ is negative.
      \end{explanation}
%part b
    \item
      Circle the correct statement about $A(1.5)$.
      \begin{enumerate}
        \item $A(1.5) = 0$;
        \item $A(1.5) < 0$;
        \item $A(1.5) > 0$;
        \item  none of the previous answers.
      \end{enumerate}
      \begin{explanation}
        Correct choice: (ii) $A(1.5) < 0$  The net area on the interval $(1,1.5)$ is postive and smaller than the negative net area on the interval $(0,1)$.  Therefore, the net area on the interval $(0,1.5)$ is negative.
      \end{explanation}
%part c
    \item
      Circle the correct statement about $A'(1.5)$.
      \begin{enumerate}
        \item $A'(1.5) = 0$;
        \item $A'(1.5) < 0$;
        \item $A'(1.5) > 0$;
        \item none of the previous answers.
      \end{enumerate}
      \begin{explanation}
        Correct choice: (iii) $A'(1.5) > 0$  because $A'(1.5)=g(1.5)$
      \end{explanation}
%part d
    \item
      Circle the correct expression for $\int_1^4 (g(t) + 1) \d t$.
      \begin{enumerate}
        \item $A(4) + 1$;
        \item $A(4) - A(1)$;
        \item $A(4) - A(1) + 1$;
        \item $A(4) + 3$;
        \item $A(4) - A(1) + 3$;
        \item none of the previous answers.
      \end{enumerate}
      \begin{explanation}
        Correct choice: (v) $A(4) - A(1) + 3$
        
        	We can split the integral: $\int_1^4 (g(t) + 1) \d t = \int_1^4 g(t) \d t +\int_1^4  1 \d t =A(4)-A(1)+3$ 
        
        We can visualize this integal by imagining shifting the graph of $g$ up one unit. See figure below.  Our area we are looking for consists of the area under the curve $g$ on the interval $(1,4)$ (the blue region) plus the rectangular region formed below (the purple region).\\
        

            
              \begin{image}
    \includegraphics[scale = 0.5]{figure3.png}
  \end{image}
            
            The area of the blue region is  $\int_1^4 g(t) \d t= \int_0^4 g(t) \d t- \int_0^1 g(t) \d t=A(4)-A(1)$.
            
            The area of the purple region is $\int_1^4  1 \d t$.  This integral is the area under the line $y=1$ on the interval $(1,4)$ which is a rectangular region of area $3$.
            
             Combining these, the total area  $\int_1^4 (g(t) + 1) \d t = A(4)-A(1)+3$
            
            
            
      \end{explanation}
[3]

% part e
    \item
      Circle the correct statement about $A(0)$.
      \begin{enumerate}
        \item $A(0) = 0$;
        \item $A(0) = 1$;
        \item $0 < A(0) < 1$;
        \item $A(0) > 1$;
        \item $A(0) < 0$;
        \item none of the previous answers.
      \end{enumerate}
      \begin{explanation}
        Correct choice: (i) $A(0) = 0$.  By the definition of $A$, $A(x)=\int_0^x g(t) \d t$ for $0 \le x \le 4$. So, $A(0)=\int_0^0 g(t) \d t=0$.  
      \end{explanation}
% part f
    \item
      Circle the interval (or intervals) where the function $A$ is DECREASING.
      \begin{enumerate}
        \item $(0, 1)$;
        \item $(1, 2)$;
        \item $(2, 4)$;
        \item none of the previous answers.
      \end{enumerate}
      \begin{explanation}
        Correct choice: (i) $(0, 1)$.  $A$ is decreasing means that the area is decreasing or we are accumulating negative area.  This happens when $A$'s derivative, $g$ is negative.
      \end{explanation}
%part g
    \item
      Circle the value (or values) where the function $A$ attains its MAXIMUM.
      \begin{enumerate}
        \item $x = 0$;
        \item $x = 1$;
        \item $x = 2$;
        \item $x = 3$;
        \item $x = 4$;
        \item none of the previous answers.
      \end{enumerate}
      \begin{explanation}
        Correct choice: (v) $x = 4$.  It appears that $A(x)<0$ on $(0,2)$ and $A(x)>0$ on $(2,4]$.  So, the maximum occurs on $[2,4]$.  Since $A'=g$ is postive on $(2,4)$, and $A$ is increasing there, the maximum is attained at the end point, $x=4$.  In other words, $A$ achieves a maximum when we have the largest potive area.  Because we start with negative area and then continue to accumulate positive area until $x=4$, the maximum occurs at the end point.
      \end{explanation}
%part h
    \item
      Circle the value  (or values) where the function $A$ attains its MINIMUM.
      \begin{enumerate}
        \item $x = 0$;
        \item $x = 1$;
        \item $x = 2$;
        \item $x = 3$;
        \item $x = 4$;
        \item none of the previous answers.
      \end{enumerate}
      \begin{explanation}
        Correct choice: (ii) $x = 1$.  $A'=g$ is negative on $(0,1)$ and positive on $(1,4)$.  So $A$ is decreasing on $(0,1)$ and increasing on $(1,4)$.  So the local and absolute maximum occurs at $x=1$.  In other words, $A$ achieves a minimum when we have the largest negative area.  We start by accumulating negative area and then switch to accumulating positive area at $x=1$.
      \end{explanation}
%item i
    \item
      Circle the interval (or intervals) where the function $A$ is CONCAVE DOWN.
      \begin{enumerate}
        \item $(0, 1)$;
        \item $(1, 2)$;
        \item $(2, 4)$;
        \item none of the previous answers.
      \end{enumerate}
      \begin{explanation}
        Correct choice: (iii) $(2, 4)$.  $A$ is concave down where $A'=g$ is decreasing.  $g$ is decreasing on $(2,4)$. 
      \end{explanation}
\item
      Circle the value (or values) of $x$ where the function $A$ has an inflection point.
      \begin{enumerate}
        \item $x = 1$;
        \item $x = 2$;
        \item $x = 3$;
        \item none of the previous answers.
      \end{enumerate}
      \begin{explanation}
        Correct choice: (ii) $x=2$.  $A'=g$ changes from increasing to decreasing.
      \end{explanation}
\item Sketch the graph of $A$, based on items (e)-(j)
  
        \begin{explanation} \hfil
                      \begin{image}
    \includegraphics[scale = 0.7]{figure4.png}
  \end{image}
      \end{explanation}
  \end{enumerate}
\end{problem}



\end{document}
