% Extracted from linearApproximation.tex, problem #1
\begin{problem}




  \begin{enumerate}
    \item
      Find the linearization, $L(x)$, of the function $f(x) = e^{2x}$ at $a = 0$.
      \WkstHop
      \begin{freeResponse}
        Recall that $L(x) = f(a) + f'(a)(x - a)$.\\
        Hence  $f'(x) = 2e^{2x} \implies f'(0) = 2e^{0} = 2$  and $ f(0) = e^{0} = 1$.\\
        Therefore $L(x) = 1 + 2\cdot x$
      \end{freeResponse}

    \item
    Using the linearization, $L(x)$, from the part (a), approximate $e$.
      \WkstHop
      \begin{freeResponse}
        \begin{align*}
          e &= e^{2\cdot(1/2)}\\
          &= f(1/2) \approx L(1/2)\\
          &\approx 1 + 2\cdot(1/2)\\
          &\approx 2
        \end{align*}
      \end{freeResponse}

    \item
    Is the estimation found in part (b) an overestimate or an underestimate? \textbf{EXPLAIN}.
      \WkstHop[2]
      \begin{freeResponse}
	$f''(x) = 4e^{2x}$ is positive on the interval $\left(0, \dfrac{1}{2}\right)$. That means the graph of $f$ is concave up between $x=0$ 
	and $x=\dfrac{1}{2}$, so this estimate is an underestimate. (The graph of the linearization $L$ lies below the graph of the function $f$.)
      \end{freeResponse}
      
  \end{enumerate}
\end{problem}
