\documentclass{ximera}

\newcommand{\RR}{\mathbb R}
\renewcommand{\d}{\,d}
\newcommand{\dd}[2][]{\frac{d #1}{d #2}}
\renewcommand{\l}{\ell}
\newcommand{\ddx}{\frac{d}{dx}}
\newcommand{\dfn}{\textbf}
\newcommand{\eval}[1]{\bigg[ #1 \bigg]}
\renewcommand{\theenumii}{\textup{(\roman{enumii})}}
\renewcommand{\labelenumii}{\theenumii}

\usepackage{graphicx}
\usepackage{multicol}
\usepackage{tkz-euclide}
%\usepackage{unicode-math}

\usepackage{pgfplots}   % <- for graphics
\pgfplotsset{compat=newest}


\renewenvironment{freeResponse}{
\ifhandout\setbox0\vbox\bgroup\else
\begin{trivlist}\item[\hskip \labelsep\bfseries Solution:\hspace{2ex}]
\fi}
{\ifhandout\egroup\else
\end{trivlist}
\fi}

\newcommand*{\ZeroOverZero}{\ensuremath{\dfrac{0}{0}}}

\providecommand{\HCCondition}{0}
\newcommand{\WkstHop}[1][1]{\if\HCCondition 0
	\vspace*{\stretch{#1}} \fi} 
\newcommand{\WkstNew}{\if\HCCondition 0
	\newpage
	 \fi} 


\title[Problem 3]{Problem 3}

\begin{document}
\begin{abstract} \end{abstract}
\maketitle


% Extracted from linearApproximation.tex, problem #3
\begin{problem}
	Estimate the value of $\sin\left(\dfrac{178\pi}{180}\right)$. Indicate whether your value is an overestimate or an underestimate. 
	\begin{explanation}
		Set $f(x) = \sin(x)$. The problem is asking us to use linear approximation to estimate the value of $f\left( \dfrac{178\pi}{180}\right)$.
		Since $\dfrac{178\pi}{180}$ is very close to $\pi$, we will use $L$ as the linearization of $\sin(x)$ at $a=\pi$.
		
		$f'(x) = \cos(x)$ so $f'(a) = \cos(\pi) = -1$.
		$f(a) = \sin(\pi) = 0$.
		The linear approximation is given by $L(x) = -(x-\pi) + 0$.		
		\begin{align*}
			\sin\left(\dfrac{178\pi}{180}\right) &\approx L\left(\dfrac{178\pi}{180}\right) \\
				&= - \left(\dfrac{178\pi}{180} - \pi \right)\\
				&= \dfrac{\pi}{90}.
		\end{align*}

		$f''(x) = -\sin(x)$. That means $f''(\pi) = 0$, BUT $f''$ is negative for all $x$ in the interval $\left( \dfrac{178 \pi}{180}, \pi\right)$. The graph of $f$ is concave down across this interval, so the estimate is an overestimate. 		
	\end{explanation}
\end{problem}



\end{document}
