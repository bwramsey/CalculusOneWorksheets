\documentclass{ximera}

\newcommand{\RR}{\mathbb R}
\renewcommand{\d}{\,d}
\newcommand{\dd}[2][]{\frac{d #1}{d #2}}
\renewcommand{\l}{\ell}
\newcommand{\ddx}{\frac{d}{dx}}
\newcommand{\dfn}{\textbf}
\newcommand{\eval}[1]{\bigg[ #1 \bigg]}
\renewcommand{\theenumii}{\textup{(\roman{enumii})}}
\renewcommand{\labelenumii}{\theenumii}

\usepackage{graphicx}
\usepackage{multicol}
\usepackage{tkz-euclide}
%\usepackage{unicode-math}

\usepackage{pgfplots}   % <- for graphics
\pgfplotsset{compat=newest}


\renewenvironment{freeResponse}{
\ifhandout\setbox0\vbox\bgroup\else
\begin{trivlist}\item[\hskip \labelsep\bfseries Solution:\hspace{2ex}]
\fi}
{\ifhandout\egroup\else
\end{trivlist}
\fi}

\newcommand*{\ZeroOverZero}{\ensuremath{\dfrac{0}{0}}}

\providecommand{\HCCondition}{0}
\newcommand{\WkstHop}[1][1]{\if\HCCondition 0
	\vspace*{\stretch{#1}} \fi} 
\newcommand{\WkstNew}{\if\HCCondition 0
	\newpage
	 \fi} 

\title[Summary]{Summary}

\begin{document}
\begin{abstract} \end{abstract}
\maketitle

\textbf{Definition} \\
If a function $f$ is \textbf{differentiable} at $x=a$, then a \textbf{linear approximation to $f$ at $a$} is a function given by\\
\begin{center}
{\large{$L_a(x)=f(a)+f'(a)(x-a)$}}
\end{center}
\textbf{Note 1: }When the value of $a$ is understood, the subscript is sometimes dropped from the notation. In this case, it is written as just $L(x)$ instead of $L_a(x)$.\\
\textbf{Note 2: }The graph of $L$ is the line tangent to the graph of $f$ at the point where $x=a$.\\
\textbf{Note 3: } If $x$ is near $a$, the value $f(x)$ can be approximated by the value of $L(x)$.\\
\textbf{Note 4: } If the graph of $f$ is concave down \emph{on the interval} between $a$ and  $x$, then the approximation $f(x) \approx L(x)$ is an overestimate. If the graph of $f$ is concave up \emph{on the interval} between $a$ and  $x$, then the approximation $f(x) \approx L(x)$ is an underestimate. (The concavity has to be consistent across the interval, not just at a point.)\\[1em]


A \textbf{differential}, $df$ of $f$ at $x$ is given by
\begin{center}
{\large{$df=f'(x)dx$}}
\end{center}
\textbf{Note 3: } If we consider a point $x$ and if $L$ is the linear approximation of $f$ at $x$, then for  any point $x+dx$ the following holds
\begin{center}
{\large{$f(x+dx)\approx L(x+dx)=f(x)+f'(x)dx=f(x)+df$}}
\end{center}
\textbf{Note 4: } The \textbf{increment} of $f$, $\Delta f$, is given by
\begin{center}
{\large{$\Delta f=f(x+dx)-f(x)\approx L(x+dx)-f(x)=f(x)+f'(x)dx-f(x)=f'(x)dx=df$}}
\end{center}
Therefore,{\large{ $\Delta f\approx df$.}}




\end{document}
