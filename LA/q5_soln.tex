\documentclass{ximera}

\newcommand{\RR}{\mathbb R}
\renewcommand{\d}{\,d}
\newcommand{\dd}[2][]{\frac{d #1}{d #2}}
\renewcommand{\l}{\ell}
\newcommand{\ddx}{\frac{d}{dx}}
\newcommand{\dfn}{\textbf}
\newcommand{\eval}[1]{\bigg[ #1 \bigg]}
\renewcommand{\theenumii}{\textup{(\roman{enumii})}}
\renewcommand{\labelenumii}{\theenumii}

\usepackage{graphicx}
\usepackage{multicol}
\usepackage{tkz-euclide}
%\usepackage{unicode-math}

\usepackage{pgfplots}   % <- for graphics
\pgfplotsset{compat=newest}


\renewenvironment{freeResponse}{
\ifhandout\setbox0\vbox\bgroup\else
\begin{trivlist}\item[\hskip \labelsep\bfseries Solution:\hspace{2ex}]
\fi}
{\ifhandout\egroup\else
\end{trivlist}
\fi}

\newcommand*{\ZeroOverZero}{\ensuremath{\dfrac{0}{0}}}

\providecommand{\HCCondition}{0}
\newcommand{\WkstHop}[1][1]{\if\HCCondition 0
	\vspace*{\stretch{#1}} \fi} 
\newcommand{\WkstNew}{\if\HCCondition 0
	\newpage
	 \fi} 


\title[Problem 5]{Problem 5}

\begin{document}
\begin{abstract} \end{abstract}
\maketitle


% Extracted from linearApproximation.tex, problem #5
\begin{problem}
Consider a square with a side $x$. Let $A$ be the area of the square.
\begin{enumerate}
\item Compute $\Delta A$, the change in area if the side increases by $\Delta x=\d x$.
 	\begin{explanation}
  $\Delta A= A(x+\d x)-A(x)= (x+\d x)^2-x^2= x^2+2x\cdot \d x +(\d x)^2-x^2=2x\cdot \d x +(\d x)^2$
    \end{explanation}
    \item Compute $\d A$, the differential of $A$ at $x$,  and compare it to  $\Delta A$.
 	\begin{explanation}
  $\d A= A'(x)\d x= 2x\cdot \d x$
  
  $\Delta A=[2x\cdot \d x] +(\d x)^2=\d A +(\d x)^2$
    \end{explanation}
     \item In the figure below the shaded part represents the change $\Delta A$. Shade the part that represents $\d A$.
       \begin{image}
      \includegraphics[scale = .3]{LAfigure1.png}
    \end{image}
 	\begin{explanation}
 Since  $\d A= A'(x)\d x= 2x\cdot \d x$, we shade the part as shown in the figure below.
\begin{image}
      \includegraphics[scale = .3]{LAfigure2.png}
    \end{image}
    The picture illustrates that $\Delta A\approx\d A$.
    \end{explanation}
\end{enumerate}
\end{problem}



\end{document}
