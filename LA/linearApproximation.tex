
%Add code to compile both versions from makefile at same time
\providecommand{\HCCondition}{0}
%Define each of the conditions
\ifcase\HCCondition
	%\condition=0 -> handout
	\documentclass[nooutcomes,noauthor,space,handout]{ximera}
	\title{ Linear Approximation (LA)}
\or	%\condition=1 -> Soln
	\documentclass[nooutcomes,noauthor]{ximera}
	\title{ Linear Approximation (LA)- Solutions}  
\fi

\newcommand{\RR}{\mathbb R}
\renewcommand{\d}{\,d}
\newcommand{\dd}[2][]{\frac{d #1}{d #2}}
\renewcommand{\l}{\ell}
\newcommand{\ddx}{\frac{d}{dx}}
\newcommand{\dfn}{\textbf}
\newcommand{\eval}[1]{\bigg[ #1 \bigg]}
\renewcommand{\theenumii}{\textup{(\roman{enumii})}}
\renewcommand{\labelenumii}{\theenumii}

\usepackage{graphicx}
\usepackage{multicol}
\usepackage{tkz-euclide}
%\usepackage{unicode-math}

\usepackage{pgfplots}   % <- for graphics
\pgfplotsset{compat=newest}


\renewenvironment{freeResponse}{
\ifhandout\setbox0\vbox\bgroup\else
\begin{trivlist}\item[\hskip \labelsep\bfseries Solution:\hspace{2ex}]
\fi}
{\ifhandout\egroup\else
\end{trivlist}
\fi}

\newcommand*{\ZeroOverZero}{\ensuremath{\dfrac{0}{0}}}

\providecommand{\HCCondition}{0}
\newcommand{\WkstHop}[1][1]{\if\HCCondition 0
	\vspace*{\stretch{#1}} \fi} 
\newcommand{\WkstNew}{\if\HCCondition 0
	\newpage
	 \fi}  



\begin{document}
\begin{abstract}
\end{abstract}
\makeTagTitle


\ifcase\HCCondition
%summary in here
\section*{SUMMARY of Linear Approximation:}

\textbf{Definition} \\
If a function $f$ is \textbf{differentiable} at $x=a$, then a \textbf{linear approximation to $f$ at $a$} is a function given by\\
\begin{center}
{\large{$L_a(x)=f(a)+f'(a)(x-a)$}}
\end{center}
\textbf{Note 1: }When the value of $a$ is understood, the subscript is sometimes dropped from the notation. In this case, it is written as just $L(x)$ instead of $L_a(x)$.\\
\textbf{Note 2: }The graph of $L$ is the line tangent to the graph of $f$ at the point where $x=a$.\\
\textbf{Note 3: } If $x$ is near $a$, the value $f(x)$ can be approximated by the value of $L(x)$.\\
\textbf{Note 4: } If the graph of $f$ is concave down \emph{on the interval} between $a$ and  $x$, then the approximation $f(x) \approx L(x)$ is an overestimate. If the graph of $f$ is concave up \emph{on the interval} between $a$ and  $x$, then the approximation $f(x) \approx L(x)$ is an underestimate. (The concavity has to be consistent across the interval, not just at a point.)\\[1em]


A \textbf{differential}, $df$ of $f$ at $x$ is given by
\begin{center}
{\large{$df=f'(x)dx$}}
\end{center}
\textbf{Note 3: } If we consider a point $x$ and if $L$ is the linear approximation of $f$ at $x$, then for  any point $x+dx$ the following holds
\begin{center}
{\large{$f(x+dx)\approx L(x+dx)=f(x)+f'(x)dx=f(x)+df$}}
\end{center}
\textbf{Note 4: } The \textbf{increment} of $f$, $\Delta f$, is given by
\begin{center}
{\large{$\Delta f=f(x+dx)-f(x)\approx L(x+dx)-f(x)=f(x)+f'(x)dx-f(x)=f'(x)dx=df$}}
\end{center}
Therefore,{\large{ $\Delta f\approx df$.}}

\newpage

\section*{Recitation Questions}
\fi

%problem 1
\begin{problem}




  \begin{enumerate}
    \item
      Find the linearization, $L(x)$, of the function $f(x) = e^{2x}$ at $a = 0$.
      \WkstHop
      \begin{freeResponse}
        Recall that $L(x) = f(a) + f'(a)(x - a)$.\\
        Hence  $f'(x) = 2e^{2x} \implies f'(0) = 2e^{0} = 2$  and $ f(0) = e^{0} = 1$.\\
        Therefore $L(x) = 1 + 2\cdot x$
      \end{freeResponse}

    \item
    Using the linearization, $L(x)$, from the part (a), approximate $e$.
      \WkstHop
      \begin{freeResponse}
        \begin{align*}
          e &= e^{2\cdot(1/2)}\\
          &= f(1/2) \approx L(1/2)\\
          &\approx 1 + 2\cdot(1/2)\\
          &\approx 2
        \end{align*}
      \end{freeResponse}

    \item
    Is the estimation found in part (b) an overestimate or an underestimate? \textbf{EXPLAIN}.
      \WkstHop[2]
      \begin{freeResponse}
	$f''(x) = 4e^{2x}$ is positive on the interval $\left(0, \dfrac{1}{2}\right)$. That means the graph of $f$ is concave up between $x=0$ 
	and $x=\dfrac{1}{2}$, so this estimate is an underestimate. (The graph of the linearization $L$ lies below the graph of the function $f$.)
      \end{freeResponse}
      
  \end{enumerate}
\end{problem}

\WkstNew
%problem 2
\begin{problem}

Complete steps (i)-(vii) below in order to estimate the following values using linear approximation:
\begin{enumerate}
	\item $\cos \left( \frac{31 \pi}{180} \right)$	
	\item $\sqrt[3]{8.13} $
\begin{enumerate}
\item Identify the function, $f(x)$.
      \WkstHop
\item  Find the nearby value where the function can be easily calculated, $x=a$.
      \WkstHop
\item Find $\Delta x=dx$.
      \WkstHop
\item  Find the linear approximation, $L(x)$.  
      \WkstHop[2]
\item  Compute the approximate value of the expression using the linear approximation.
	\WkstHop
\item  Compare the approximated value to the value given by your calculator.
      \WkstHop
\item  Compare $dy$ and $\Delta y$ using the value given by your calculator.
      \WkstHop[2]
\end{enumerate}
\end{enumerate}

\begin{freeResponse}
\begin{enumerate}
	\item $ \cos \left( \frac{31 \pi}{180} \right)$
  \begin{enumerate}
    \item  $f(x) = \cos x$
    \item $a = \frac{30 \pi}{180} = \frac{\pi}{6}$
    \item  $\Delta x = \frac{31 \pi}{180} - \frac{\pi}{6} = \frac{\pi}{180}$
    \item
    \begin{align*}
      L(x) &= f\left( \frac{\pi}{6} \right) + f^\prime \left(\frac{\pi}{6} \right) \left( x - \frac{\pi}{6} \right) \\
           &= \cos \left( \frac{\pi}{6} \right) - \sin \left(\frac{\pi}{6} \right) \left( x - \frac{\pi}{6} \right) \\
           &= \frac{\sqrt{3}}{2} - \frac{1}{2} \left( x - \frac{\pi}{6} \right) 
    \end{align*}
    \item 
    \begin{align*}
      L \left( \frac{31 \pi}{180} \right) &= \frac{\sqrt{3}}{2} - \frac{1}{2} \left( \frac{31 \pi}{180} - \frac{\pi}{6} \right) \\
                                          &=  \frac{\sqrt{3}}{2} - \frac{1}{2} \left( \frac{\pi}{180} \right) \\
                                          &= \frac{1}{2} \left( \sqrt{3} - \frac{\pi}{180} \right) \\
                                          &\approx 0.857299
    \end{align*}
    \item  $\cos \left( \frac{31 \pi}{180} \right) \approx 0.857167$
    \item
      $$ dy = L\left( \frac{31\pi}{180} \right) - L \left( \frac{\pi}{6} \right) \approx -0.008727 $$
      $$ \Delta y = \cos \left( \frac{31 \pi}{180} \right) - \cos \left( \frac{\pi}{6} \right) \approx -0.008858 $$
  \end{enumerate}
  
  \item $ \sqrt[3]{8.13}$
  \begin{enumerate}
    \item  $f(x) = \sqrt[3]{x}$.
    \item  $a=8$.
    \item  $\Delta x = 8.13 - 8 = 0.13$.
    \item 
      \begin{align*}
        L(x) &= f(8) + f^\prime (8) (x-8) \\
             &= \sqrt[3]{8} + \frac{1}{3 (\sqrt[3]{8})^2} \left( x - 8 \right) \\
            &= 2 + \frac{1}{12} (x-8) 
      \end{align*}
    \item  
      \begin{align*}
       L(8.13) &= 2 + \frac{1}{12} (8.13 - 8) \\
                &= 2 + \left( \frac{1}{12} \right) \left( \frac{13}{100} \right) \\
                &= 2 + \frac{13}{1200} = \frac{2413}{1200} \\
                &\approx 2.010833
      \end{align*}
    \item  $\sqrt[3]{8.13} \approx 2.010775$.
    \item  
      $$ dy = L(8.13) - L(8) \approx 0.010833 $$
      $$ \Delta y = \sqrt[3]{8.13} - \sqrt[3]{8} \approx 0.010775 $$
    \end{enumerate}
    \end{enumerate}
\end{freeResponse}
\end{problem}

\WkstNew

\begin{problem}
	Estimate the value of $\sin\left(\dfrac{178\pi}{180}\right)$. Indicate whether your value is an overestimate or an underestimate. 
	\WkstHop
	
	\begin{freeResponse}
		Set $f(x) = \sin(x)$. The problem is asking us to use linear approximation to estimate the value of $f\left( \dfrac{178\pi}{180}\right)$.
		Since $\dfrac{178\pi}{180}$ is very close to $\pi$, we will use $L$ as the linearization of $\sin(x)$ at $a=\pi$.
		
		$f'(x) = \cos(x)$ so $f'(a) = \cos(\pi) = -1$.
		$f(a) = \sin(\pi) = 0$.
		The linear approximation is given by $L(x) = -(x-\pi) + 0$.		
		\begin{align*}
			\sin\left(\dfrac{178\pi}{180}\right) &\approx L\left(\dfrac{178\pi}{180}\right) \\
				&= - \left(\dfrac{178\pi}{180} - \pi \right)\\
				&= \dfrac{\pi}{90}.
		\end{align*}

		$f''(x) = -\sin(x)$. That means $f''(\pi) = 0$, BUT $f''$ is negative for all $x$ in the interval $\left( \dfrac{178 \pi}{180}, \pi\right)$. The graph of $f$ is concave down across this interval, so the estimate is an overestimate. 		
	\end{freeResponse}
\end{problem}

\WkstNew
	
%problem 3
\begin{problem}

  Consider the graph of $f' (x)$ given below.
  Suppose you know that $f(3) = 7$.
  Can you approximate $f(2.98)$ and $f(3.02)$?
  Explain your answer. Are these overestimates or underestimates?
 % \begin{center}
 %   \begin{image}
 %     \includegraphics{Figure1.png}
 %   \end{image}
 % \end{center}
	  \begin{center}
		\begin{tikzpicture}
			\begin{axis}[
				xmin=-1.3, xmax=4.3, ymin=-1.3,ymax=4.3,    
				axis lines =middle, 
				every axis y label/.style={at=(current axis.above origin),anchor=south},
				every axis x label/.style={at=(current axis.right of origin),anchor=west},
				xtick={-1,...,4}, ytick={-1,...,4},
				grid=major, width=3.5in, height = 3.5in,
				grid style={dashed, gray!40}
				]
				\addplot[color=blue, ultra thick, smooth, domain=-1.3:4.3]{(-1/3)*x^2+3} node[pos=0.4, color=blue, above right]{$f'$};				

			\end{axis}
		\end{tikzpicture}
	\end{center}
	\WkstHop
  \begin{freeResponse}
        $f(3)=7$ and $f^\prime (3)=0$ 
    $$ L(x) =7+0(x-3) = 7$$
    This is a constant function, and so our approximations are $f(2.98)\approx 7$ and $f(3.02)\approx 7$.
    These are overestimates for the graph of $f$ is concave DOWN on the intervals $(2.98, 3)$ and $(3, 3.02)$.
      \end{freeResponse}
\end{problem}


\WkstNew
\begin{problem}
Consider a square with a side $x$. Let $A$ be the area of the square.
\begin{enumerate}
\item Compute $\Delta A$, the change in area if the side increases by $\Delta x=\d x$.
 	\WkstHop
 \begin{freeResponse}
  $\Delta A= A(x+\d x)-A(x)= (x+\d x)^2-x^2= x^2+2x\cdot \d x +(\d x)^2-x^2=2x\cdot \d x +(\d x)^2$
    \end{freeResponse}
    \item Compute $\d A$, the differential of $A$ at $x$,  and compare it to  $\Delta A$.
 	\WkstHop
 \begin{freeResponse}
  $\d A= A'(x)\d x= 2x\cdot \d x$
  
  $\Delta A=[2x\cdot \d x] +(\d x)^2=\d A +(\d x)^2$
    \end{freeResponse}
     \item In the figure below the shaded part represents the change $\Delta A$. Shade the part that represents $\d A$.
       \begin{image}
      \includegraphics[scale = .3]{LAfigure1.png}
    \end{image}
 	\WkstHop
 \begin{freeResponse}
 Since  $\d A= A'(x)\d x= 2x\cdot \d x$, we shade the part as shown in the figure below.
\begin{image}
      \includegraphics[scale = .3]{LAfigure2.png}
    \end{image}
    The picture illustrates that $\Delta A\approx\d A$.
    \end{freeResponse}
\end{enumerate}
\end{problem}



\WkstNew


%problem 4
\begin{problem}

  Estimate the amount of paint needed to apply a coat of paint $.05$ cm thick to a hemispherical dome with diameter $50$m. Is this value an underestimate or an overestimate?
	\WkstHop
 
  \begin{freeResponse}
	    The radius of the dome is $\frac{50}{2}m=25m$.
	    The paint increases this by 0.0005m (0.05cm to meters).
	    The volume of a ``hemispherical dome'' (or half of a sphere) is $V = \frac{1}{2} \left(\frac{4}{3} \pi r^3 \right) = \frac{2}{3} \pi r^3$.
	    Then
	    $$ \d V = 2 \pi r^2 \d r  $$
	    The amount of paint needed is approximately the change in volume ($\d V$).
	    We also have that $r=25m$ and $\d r = (5 \times 10^{-4}) m$.
	    Thus, the amount of paint needed to paint the dome is approximately
	    $$ \eval{\d V}_{r=25,\d r = 0.0005} = 2 \pi (25)^2 (5 \times 10^{-4}) = 5 \pi (0.125) = 0.625 \pi \approx 1.9635 m^3$$
    
	   The function $V(r) = \frac{2}{3} \pi r^3$ has derivatives $\dfrac{dV}{dr} = 2 \pi r^2$, and $\dfrac{d^2 V}{dr^2} = 4 \pi r$.
	   This second derivative is positive for $r$ in the interval $(25, 25.0005)$, so the graph of $V$ is concave up in that entire interval. 
	   This means the estimate is an underestimate.
 
  \end{freeResponse}
\end{problem}



\WkstNew


%problem 5
\begin{problem}
The graph of a function $f$ is given below.
    % \begin{image}
    %  \includegraphics{figure3.png}
    %\end{image}
    	  \begin{center}
		\begin{tikzpicture}
			\begin{axis}[
				xmin=-0.3, xmax=4.3, ymin=-0.3,ymax=4.3,    
				axis lines =middle, 
				every axis y label/.style={at=(current axis.above origin),anchor=south},
				every axis x label/.style={at=(current axis.right of origin),anchor=west},
				xtick={0,...,4}, ytick={0,...,4},
				grid=major, width=3in, height = 3in,
				grid style={dashed, gray!40}
				]
				\addplot[color=blue, ultra thick, smooth, domain=0:4]{(1/4)*x^2+1} node[pos=0.4, color=blue, above left]{\large{$y=f(x)$}};				

			\end{axis}
		\end{tikzpicture}
	\end{center}
%part a
\begin{enumerate}
\item Given that $f'(2)=1$, find the linear approximation $L$ to the function $f$ at $a=2$.
	\WkstHop[2]

\begin{freeResponse}
$L(x)=f(2)+f'(2)(x-2)=2+(x-2)=x$
\end{freeResponse}

%part b
\item Sketch the graph of $L$ in the figure above.
\begin{freeResponse}
    \begin{image}
      \includegraphics[scale = .7]{figure4.png}
    \end{image}
\end{freeResponse}

%part c
\item Use the linear approximation $L$ to estimate the value of $f(3)$.  Is this an underestimate or overestimate?  \textbf{EXPLAIN}.
	\WkstHop[2]
\begin{freeResponse}
$f(3)\approx L(3)=3$.  It is an underestimate because $f$ is concave up on the interval $(2, 3)$.
\end{freeResponse}

%part d
\item When $x$ changes from $a=2$ to $a+\Delta x=3$, the change in the {\bf function} $y=f(x)$, $\Delta y$, is given by $\Delta y=f(a+ \Delta x)-f(a)$.  Draw and label $\Delta y$ and $\Delta x$ in the figure above.


\begin{freeResponse}
   \begin{image}
      \includegraphics[scale = .6]{figure5.png}
    \end{image}
\end{freeResponse}

%part e
\item When $x$ changes from $a=2$ to $a+ \Delta x=3$, the change in the {\bf linear approximaton}, $dy$, is given by $dy=L(a+\Delta x)-L(a)=f'(a)\Delta x$.  Draw and label $L(x)$, $dx$ and $dy$ (differential) in the figure above.
%   \begin{image}
%      \includegraphics[scale = .7]{figure6.png}
%    \end{image}

\begin{freeResponse}
   \begin{image}
      \includegraphics[scale = .7]{figure7.png}
    \end{image}
\end{freeResponse}

\end{enumerate}
\end{problem}


\WkstNew


%problem 6
\begin{problem}

  The figure shows the graph of a function $f$.
  Let $L_a(x)$ be the linear approximation of $f$ at $a$.
  \begin{image}
    \includegraphics[scale = 1]{LAGraph.png}
  \end{image}
  Circle ALL the correct statements below.
  \begin{enumerate}
    \item
      $L_a(b) < f(b)$
    \item
      $L_a(b) > f(b)$
    \item
      $L_a(a) < f(a)$
    \item
      $L_a(a) > f(a)$
    \item
      No statement (a)~--~(d) is correct.
  \end{enumerate}
	\WkstHop
  \begin{freeResponse}
    From the graph of $f$ and the the graph of $L_a(x)$ we see that the correct statement is (a):
    \begin{image}
      \includegraphics[scale = .5]{figure2.png}
    \end{image}
  \end{freeResponse}
\end{problem}

	\WkstNew

%problem 7
\begin{problem}

  By using linear approximation, determine which of the following is the best estimate of $e^{0.002}$.
  \begin{enumerate}
    \item1.00100050016679834166
    \item 1.00200200133400026675
    \item 1.00300450450337702601
    \item 1.02020134002675581016
  \end{enumerate}
	\WkstHop
  \begin{freeResponse}
    Let $f(x) = e^x$ and $a=0$.  Then since $f'(x) = e^x$, we have that
    \begin{align*}
      L(x) &= f(a) + f'(a)(x-a) \\
           &=  f(0) + f'(0)(x-0) \\
           &= e^0 + e^0x \\
           &= 1 + x
    \end{align*}
    Then since $L(0.002) = 1 + 0.002 = 1.002$, the answer is (b).
  \end{freeResponse}	
\end{problem}

	\WkstNew

\begin{problem}
	Find a formula for the differential of the following functions.   
	\begin{enumerate}
	    \item $\displaystyle y = 3x^6e^x$.
	\WkstHop
		  \begin{freeResponse}
			$\displaystyle \dfrac{dy}{dx} = 18x^5e^x + 3x^6 e^x$ so 
			\begin{align*}
				dy &= \dfrac{dy}{dx} dx \\
				&= \left(18x^5e^x + 3x^6 e^x\right) dx.
			\end{align*}
		  \end{freeResponse}
	    \item $\displaystyle z = \ln(1+t^2)$.
	\WkstHop
		  \begin{freeResponse}
			$\displaystyle dz = \dfrac{2t}{1+t^2} dt$
		  \end{freeResponse}
	    \item $\displaystyle \theta = \tan^{-1}(r^3)$.
	\WkstHop
		  \begin{freeResponse}
			$\displaystyle d\theta = \dfrac{3r^2}{1+r^6} dr$
		  \end{freeResponse}
	  \end{enumerate}

\end{problem}

	\WkstNew

\begin{problem}
	In your own words, explain why $L_a(x)$ is a good approximation of the function $f$ for x values $x \approx a$.
	\WkstHop
	
	\begin{freeResponse}
		The graph of $L_a(x)$ is the line tangent to the graph of $f$ at $(a, f(a) )$. 
		For a differentiable function, the tangent line is very close to the graph of the
		function for x-values near that point. 
	\end{freeResponse}
\end{problem}

\end{document} 
