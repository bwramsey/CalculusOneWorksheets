\documentclass{ximera}

\newcommand{\RR}{\mathbb R}
\renewcommand{\d}{\,d}
\newcommand{\dd}[2][]{\frac{d #1}{d #2}}
\renewcommand{\l}{\ell}
\newcommand{\ddx}{\frac{d}{dx}}
\newcommand{\dfn}{\textbf}
\newcommand{\eval}[1]{\bigg[ #1 \bigg]}
\renewcommand{\theenumii}{\textup{(\roman{enumii})}}
\renewcommand{\labelenumii}{\theenumii}

\usepackage{graphicx}
\usepackage{multicol}
\usepackage{tkz-euclide}
%\usepackage{unicode-math}

\usepackage{pgfplots}   % <- for graphics
\pgfplotsset{compat=newest}


\renewenvironment{freeResponse}{
\ifhandout\setbox0\vbox\bgroup\else
\begin{trivlist}\item[\hskip \labelsep\bfseries Solution:\hspace{2ex}]
\fi}
{\ifhandout\egroup\else
\end{trivlist}
\fi}

\newcommand*{\ZeroOverZero}{\ensuremath{\dfrac{0}{0}}}

\providecommand{\HCCondition}{0}
\newcommand{\WkstHop}[1][1]{\if\HCCondition 0
	\vspace*{\stretch{#1}} \fi} 
\newcommand{\WkstNew}{\if\HCCondition 0
	\newpage
	 \fi} 


\title[Problem 4]{Problem 4}

\begin{document}
\begin{abstract} \end{abstract}
\maketitle


% Extracted from linearApproximation.tex, problem #4
\begin{problem}

  Consider the graph of $f' (x)$ given below.
  Suppose you know that $f(3) = 7$.
  Can you approximate $f(2.98)$ and $f(3.02)$?
  Explain your answer. Are these overestimates or underestimates?
 % \begin{center}
 %   \begin{image}
 %     \includegraphics{Figure1.png}
 %   \end{image}
 % \end{center}
	  \begin{center}
		\begin{tikzpicture}
			\begin{axis}[
				xmin=-1.3, xmax=4.3, ymin=-1.3,ymax=4.3,    
				axis lines =middle, 
				every axis y label/.style={at=(current axis.above origin),anchor=south},
				every axis x label/.style={at=(current axis.right of origin),anchor=west},
				xtick={-1,...,4}, ytick={-1,...,4},
				grid=major, width=3.5in, height = 3.5in,
				grid style={dashed, gray!40}
				]
				\addplot[color=blue, ultra thick, smooth, domain=-1.3:4.3]{(-1/3)*x^2+3} node[pos=0.4, color=blue, above right]{$f'$};				

			\end{axis}
		\end{tikzpicture}
	\end{center}
	\begin{explanation}
        $f(3)=7$ and $f^\prime (3)=0$ 
    $$ L(x) =7+0(x-3) = 7$$
    This is a constant function, and so our approximations are $f(2.98)\approx 7$ and $f(3.02)\approx 7$.
    These are overestimates for the graph of $f$ is concave DOWN on the intervals $(2.98, 3)$ and $(3, 3.02)$.
      \end{explanation}
\end{problem}



\end{document}
