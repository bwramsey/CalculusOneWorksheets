\documentclass{ximera}

\newcommand{\RR}{\mathbb R}
\renewcommand{\d}{\,d}
\newcommand{\dd}[2][]{\frac{d #1}{d #2}}
\renewcommand{\l}{\ell}
\newcommand{\ddx}{\frac{d}{dx}}
\newcommand{\dfn}{\textbf}
\newcommand{\eval}[1]{\bigg[ #1 \bigg]}
\renewcommand{\theenumii}{\textup{(\roman{enumii})}}
\renewcommand{\labelenumii}{\theenumii}

\usepackage{graphicx}
\usepackage{multicol}
\usepackage{tkz-euclide}
%\usepackage{unicode-math}

\usepackage{pgfplots}   % <- for graphics
\pgfplotsset{compat=newest}


\renewenvironment{freeResponse}{
\ifhandout\setbox0\vbox\bgroup\else
\begin{trivlist}\item[\hskip \labelsep\bfseries Solution:\hspace{2ex}]
\fi}
{\ifhandout\egroup\else
\end{trivlist}
\fi}

\newcommand*{\ZeroOverZero}{\ensuremath{\dfrac{0}{0}}}

\providecommand{\HCCondition}{0}
\newcommand{\WkstHop}[1][1]{\if\HCCondition 0
	\vspace*{\stretch{#1}} \fi} 
\newcommand{\WkstNew}{\if\HCCondition 0
	\newpage
	 \fi} 

\title[Problem 7]{Problem 7}

\begin{document}
\begin{abstract} \end{abstract}
\maketitle

% Extracted from linearApproximation.tex, problem #7
\begin{problem}
The graph of a function $f$ is given below.
    % \begin{image}
    %  \includegraphics{figure3.png}
    %\end{image}
    	  \begin{center}
		\begin{tikzpicture}
			\begin{axis}[
				xmin=-0.3, xmax=4.3, ymin=-0.3,ymax=4.3,    
				axis lines =middle, 
				every axis y label/.style={at=(current axis.above origin),anchor=south},
				every axis x label/.style={at=(current axis.right of origin),anchor=west},
				xtick={0,...,4}, ytick={0,...,4},
				grid=major, width=3in, height = 3in,
				grid style={dashed, gray!40}
				]
				\addplot[color=blue, ultra thick, smooth, domain=0:4]{(1/4)*x^2+1} node[pos=0.4, color=blue, above left]{\large{$y=f(x)$}};				

			\end{axis}
		\end{tikzpicture}
	\end{center}
%part a
\begin{enumerate}
\item Given that $f'(2)=1$, find the linear approximation $L$ to the function $f$ at $a=2$.
	[2]

%part b
\item Sketch the graph of $L$ in the figure above.

%part c
\item Use the linear approximation $L$ to estimate the value of $f(3)$.  Is this an underestimate or overestimate?  \textbf{EXPLAIN}.
	[2]

%part d
\item When $x$ changes from $a=2$ to $a+\Delta x=3$, the change in the {\bf function} $y=f(x)$, $\Delta y$, is given by $\Delta y=f(a+ \Delta x)-f(a)$.  Draw and label $\Delta y$ and $\Delta x$ in the figure above.

%part e
\item When $x$ changes from $a=2$ to $a+ \Delta x=3$, the change in the {\bf linear approximaton}, $dy$, is given by $dy=L(a+\Delta x)-L(a)=f'(a)\Delta x$.  Draw and label $L(x)$, $dx$ and $dy$ (differential) in the figure above.
%   \begin{image}
%      \includegraphics[scale = .7]{figure6.png}
%    \end{image}

\end{enumerate}
\end{problem}

\end{document}
