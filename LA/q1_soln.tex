\documentclass{ximera}

\newcommand{\RR}{\mathbb R}
\renewcommand{\d}{\,d}
\newcommand{\dd}[2][]{\frac{d #1}{d #2}}
\renewcommand{\l}{\ell}
\newcommand{\ddx}{\frac{d}{dx}}
\newcommand{\dfn}{\textbf}
\newcommand{\eval}[1]{\bigg[ #1 \bigg]}
\renewcommand{\theenumii}{\textup{(\roman{enumii})}}
\renewcommand{\labelenumii}{\theenumii}

\usepackage{graphicx}
\usepackage{multicol}
\usepackage{tkz-euclide}
%\usepackage{unicode-math}

\usepackage{pgfplots}   % <- for graphics
\pgfplotsset{compat=newest}


\renewenvironment{freeResponse}{
\ifhandout\setbox0\vbox\bgroup\else
\begin{trivlist}\item[\hskip \labelsep\bfseries Solution:\hspace{2ex}]
\fi}
{\ifhandout\egroup\else
\end{trivlist}
\fi}

\newcommand*{\ZeroOverZero}{\ensuremath{\dfrac{0}{0}}}

\providecommand{\HCCondition}{0}
\newcommand{\WkstHop}[1][1]{\if\HCCondition 0
	\vspace*{\stretch{#1}} \fi} 
\newcommand{\WkstNew}{\if\HCCondition 0
	\newpage
	 \fi} 


\title[Problem 1]{Problem 1}

\begin{document}
\begin{abstract} \end{abstract}
\maketitle


% Extracted from linearApproximation.tex, problem #1
\begin{problem}




  \begin{enumerate}
    \item
      Find the linearization, $L(x)$, of the function $f(x) = e^{2x}$ at $a = 0$.
      \begin{explanation}
        Recall that $L(x) = f(a) + f'(a)(x - a)$.\\
        Hence  $f'(x) = 2e^{2x} \implies f'(0) = 2e^{0} = 2$  and $ f(0) = e^{0} = 1$.\\
        Therefore $L(x) = 1 + 2\cdot x$
      \end{explanation}

    \item
    Using the linearization, $L(x)$, from the part (a), approximate $e$.
      \begin{explanation}
        \begin{align*}
          e &= e^{2\cdot(1/2)}\\
          &= f(1/2) \approx L(1/2)\\
          &\approx 1 + 2\cdot(1/2)\\
          &\approx 2
        \end{align*}
      \end{explanation}

    \item
    Is the estimation found in part (b) an overestimate or an underestimate? \textbf{EXPLAIN}.
      [2]
      \begin{explanation}
	$f''(x) = 4e^{2x}$ is positive on the interval $\left(0, \dfrac{1}{2}\right)$. That means the graph of $f$ is concave up between $x=0$ 
	and $x=\dfrac{1}{2}$, so this estimate is an underestimate. (The graph of the linearization $L$ lies below the graph of the function $f$.)
      \end{explanation}
      
  \end{enumerate}
\end{problem}



\end{document}
