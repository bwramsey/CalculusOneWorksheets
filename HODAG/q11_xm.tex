\documentclass{ximera}

\newcommand{\RR}{\mathbb R}
\renewcommand{\d}{\,d}
\newcommand{\dd}[2][]{\frac{d #1}{d #2}}
\renewcommand{\l}{\ell}
\newcommand{\ddx}{\frac{d}{dx}}
\newcommand{\dfn}{\textbf}
\newcommand{\eval}[1]{\bigg[ #1 \bigg]}
\renewcommand{\theenumii}{\textup{(\roman{enumii})}}
\renewcommand{\labelenumii}{\theenumii}

\usepackage{graphicx}
\usepackage{multicol}
\usepackage{tkz-euclide}
%\usepackage{unicode-math}

\usepackage{pgfplots}   % <- for graphics
\pgfplotsset{compat=newest}


\renewenvironment{freeResponse}{
\ifhandout\setbox0\vbox\bgroup\else
\begin{trivlist}\item[\hskip \labelsep\bfseries Solution:\hspace{2ex}]
\fi}
{\ifhandout\egroup\else
\end{trivlist}
\fi}

\newcommand*{\ZeroOverZero}{\ensuremath{\dfrac{0}{0}}}

\providecommand{\HCCondition}{0}
\newcommand{\WkstHop}[1][1]{\if\HCCondition 0
	\vspace*{\stretch{#1}} \fi} 
\newcommand{\WkstNew}{\if\HCCondition 0
	\newpage
	 \fi} 

\title[Problem 11]{Problem 11}

\begin{document}
\begin{abstract} \end{abstract}
\maketitle

% Extracted from higherOrderDerivativesAndGraphs.tex, problem #11
\begin{problem}
The total number of people, $N$ who have contracted a common cold by a time $t$ days after its outbreak on an island is given by $N=N(t)= \frac{200000}{1+100e^{-0.1t}},t\geq0$.

	\begin{enumerate}
		%a
		\item Evaluate and interpret the limit $\lim_{t \to \infty} N(t)$
%b
		\item How long will it take for the number of people who have contracted the cold to reach 40,000?
\item The graph of the function $N$ on the interval $[0,100]$ is given below.  Sketch (as best you can) the graph of its derivative, $N'(t)$.
			\begin{image}
			\includegraphics[scale=.4]{figure8.png}
			\end{image}
\item Calculate $N'(t)$.  What does $N'(t)$ represent?
\item Evaluate and interpret the limit $\lim_{t \to \infty} N'(t)$
\item Find the average growth rate of the number of people who have contracted the disease during the time interval $[5,6]$ (or during the sixth day after the outbreak).
\item Find the instantaneous growth rate of the number of people who have contracted the disease for $t=5$ (round to a whole number).
\end{enumerate}
\end{problem}

\end{document}
