% Extracted from higherOrderDerivativesAndGraphs.tex, problem #8
\begin{problem}
Suppose that a stone is thrown vertically upward from a cliff on Mars with an initial velocity of $24$ ft/s from a height of $192$ ft.  The height $s$ of the stone above the ground after $t$ seconds is given by $s(t) = -6t^2 + 24t + 192$.

	\begin{enumerate}
	
	%part a
	\item  Determine the velocity and acceleration of the stone after $t$ seconds.
\WkstHop

			\begin{freeResponse}
			The velocity $v(t)$ is:  $v(t) = s'(t) = -12t + 24$.
			
			The acceleration $a(t)$ is:  $a(t) = v'(t) = s''(t) = -12$.
			\end{freeResponse}
			
			
			
	%part b
	\item  What is the greatest height of the stone and when does it occur?  What are the velocity and acceleration at that time?
\WkstHop

			\begin{freeResponse}
			Since the function $s(t)$ is differentiable everywhere (it is a polynomial), the maximum height must occur at a time when the velocity is $0$.  So we solve:
			\begin{align*}
			v(t) = -12t+24 &= 0 \\
			12t &= 24 \\
			t &= 2
			\end{align*}
			
			It is easy to check that $v(t) > 0$ for $0 \leq t < 2$ and $v(t) < 0$ for $2 < t$, and so the greatest height of the stone really does occur at time $t=2$.  The greatest height is $s(2) = -24 + 48 + 192 = 216$ ft.  For the second question we have already seen that $v(2) = 0$ ft/sec, and we have a constant acceleration of $-12$ ft/sec$^2$.  So $a(2) = -12$ ft/sec$^2$.
			\end{freeResponse}
			
			
			
	%part c
	\item  When does the stone hit the ground?  What are the velocity and acceleration at that time?
\WkstHop

			\begin{freeResponse}
			The stone hits the ground when $s(t) = 0$.  So we solve:
			\begin{align*}
			s(t) = -6t^2 + 24t + 192 &= 0 \\
			-6(t^2 - 4t - 32) &= 0 \\
			-6(t-8)(t+4) &= 0 
			\end{align*}
			
			Since we are only considering $t \geq 0$, we must have $t=8$.  At this instant, $v(8) = -96 + 24 = -72$ ft/sec and $a(8) = -12$ ft/sec$^2$.  
			
			\end{freeResponse}
			
			
			
	\end{enumerate}
			
			
			
		
\end{problem}
