% Extracted from higherOrderDerivativesAndGraphs.tex, problem #11
\begin{problem}
The total number of people, $N$ who have contracted a common cold by a time $t$ days after its outbreak on an island is given by $N=N(t)= \frac{200000}{1+100e^{-0.1t}},t\geq0$.

	\begin{enumerate}
		%a
		\item Evaluate and interpret the limit $\lim_{t \to \infty} N(t)$
\WkstHop
			\begin{freeResponse}	
			$\lim_{t \to \infty} N(t)=\lim_{t \to \infty} \frac{200000}{1+100e^{-0.1t}}=200000$.  In the long run, 200000 people will get the cold.
			\end{freeResponse}
		%b
		\item How long will it take for the number of people who have contracted the cold to reach 40,000?
\WkstHop
			\begin{freeResponse}
				\begin{align*}
				N(t)&=40000\\
				\frac{200000}{1+100e^{-0.1t}}&=40000\\
				200000&=40000(1+100e^{-0.1t})\\
				1+100e^{-0.1t}&=\frac{200000}{40000}\\
				100e^{-0.1t}&=5-1\\
				e^{-0.1t}&=\frac{4}{100}\\
				-0.1t&=\ln(.04)\\
				t&=\frac{\ln(0.04)}{-0.1}=32.1888
				\end{align*}
			It will take 33 days for the number of people who have contracted the cold to reach 40,000.
			\end{freeResponse}

		\item The graph of the function $N$ on the interval $[0,100]$ is given below.  Sketch (as best you can) the graph of its derivative, $N'(t)$.
			\begin{image}
			\includegraphics[scale=.4]{Figure8.png}
			\end{image}
\WkstHop
			\begin{freeResponse}\hfil
				\begin{image}
			\includegraphics[scale=.5]{Figure9.png}
			\end{image}		
			\end{freeResponse}
		\item Calculate $N'(t)$.  What does $N'(t)$ represent?
\WkstHop

			\begin{freeResponse}	
				$N'(t)=\frac{-200000(10)e^{-0.1t}}{(1+100e^{-0.1t})^2}=\frac{2000000e^{-0.1t}}{(1+100e^{-0.1t})^2}$.\\
				$N'(t)$ represents the instantaneous growth rate of the number of people who have contracted the cold at the time $t$.
			\end{freeResponse}
\WkstNew

		\item Evaluate and interpret the limit $\lim_{t \to \infty} N'(t)$
\WkstHop
			\begin{freeResponse}	
				$\lim_{t \to \infty} N'(t)=\lim_{t \to \infty}\frac{2000000e^{-0.1t}}{(1+100e^{-0.1t})^2}=\frac{2000000(0)}{(1+100(0))^2}=\frac{0}{1}=0$.\\
				In the long run, the number of people who have contracted the cold will stabilize.  There will be no growth in the long run.
			\end{freeResponse}

		\item Find the average growth rate of the number of people who have contracted the disease during the time interval $[5,6]$ (or during the sixth day after the outbreak).
\WkstHop
			\begin{freeResponse}	
				$AVRG=\frac{N(6)-N(5)}{6-5}=N(6)-N(5)=\frac{200000}{1+100e^{-0.1(6)}}-\frac{200000}{1+100e^{-0.1(5)}} \approx 335.1$
			\end{freeResponse}

		\item Find the instantaneous growth rate of the number of people who have contracted the disease for $t=5$ (round to a whole number).
\WkstHop
			\begin{freeResponse}	
				$N'(5)=\frac{2000000e^{-0.1(5)}}{(1+100e^{-0.1(5)})^2}=319$ people per day
			\end{freeResponse}

	\end{enumerate}
\end{problem}
