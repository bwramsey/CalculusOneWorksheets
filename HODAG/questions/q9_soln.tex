\documentclass{ximera}

\newcommand{\RR}{\mathbb R}
\renewcommand{\d}{\,d}
\newcommand{\dd}[2][]{\frac{d #1}{d #2}}
\renewcommand{\l}{\ell}
\newcommand{\ddx}{\frac{d}{dx}}
\newcommand{\dfn}{\textbf}
\newcommand{\eval}[1]{\bigg[ #1 \bigg]}
\renewcommand{\theenumii}{\textup{(\roman{enumii})}}
\renewcommand{\labelenumii}{\theenumii}

\usepackage{graphicx}
\usepackage{multicol}
\usepackage{tkz-euclide}
%\usepackage{unicode-math}

\usepackage{pgfplots}   % <- for graphics
\pgfplotsset{compat=newest}


\renewenvironment{freeResponse}{
\ifhandout\setbox0\vbox\bgroup\else
\begin{trivlist}\item[\hskip \labelsep\bfseries Solution:\hspace{2ex}]
\fi}
{\ifhandout\egroup\else
\end{trivlist}
\fi}

\newcommand*{\ZeroOverZero}{\ensuremath{\dfrac{0}{0}}}

\providecommand{\HCCondition}{0}
\newcommand{\WkstHop}[1][1]{\if\HCCondition 0
	\vspace*{\stretch{#1}} \fi} 
\newcommand{\WkstNew}{\if\HCCondition 0
	\newpage
	 \fi} 


\title[Problem 9]{Problem 9}

\begin{document}
\begin{abstract} \end{abstract}
\maketitle


% Extracted from higherOrderDerivativesAndGraphs.tex, problem #9
\begin{problem}
	Suppose the function $f$ has derivative given by $\displaystyle f'(x) = \dfrac{2x}{4+x^2}$.

	\begin{enumerate}
	
	%part a
	\item  Find a formula for $f''(x)$.
\begin{explanation}
				The second derivative is given by $\displaystyle f''(x) = \dfrac{2(4-x^2)}{(4+x^2)^2}$.
			\end{explanation}
			
			
			
	%part b
	\item  Construct a sign-chart for $f'$ and $f''$.
[2]
		\begin{explanation}
			For $f'$ we obtain:

			\begin{tikzpicture} 
				\tkzTabInit[lgt=2,espcl=1] 
					{$x$         /1, 
					$2x$   /1, 
					$4+x^2$ /1, 
					$f'(x)$      /1}% 
					{   , $0$ ,  }% 
				\tkzTabLine{ , - , t , + ,}
				\tkzTabLine{ , + , t , + ,}
				\tkzTabLine{ , - , t , + ,}
			\end{tikzpicture} 
			
			For $f''$, using the factorization $4-x^2 = (2+x)(2-x)$ , we obtain:			

			\begin{tikzpicture} 
				\tkzTabInit[lgt=2,espcl=1] 
					{$x$         /1, 
					$2$   /1, 
					$2+x$  /1,
					$2-x$  /1,
					$(4+x^2)^2$   /1, 
					$f''(x)$    /1}% 
					{  , $-2$ ,$2$,  }% 
				\tkzTabLine{  , + , t , + , t , + ,}
				\tkzTabLine{  , - , t , + , t , + ,}
				\tkzTabLine{  , + , t , + , t , - ,}
				\tkzTabLine{  , + , t , + , t , + ,}
				\tkzTabLine{  , - , t , + , t , - ,}
			\end{tikzpicture} 
		\end{explanation}
			
			
			
	%part c
	\item  On what interval(s) is $f$ increasing? On what interval(s) is $f$ concave down?
\begin{explanation}
				From the sign-chart for $f'$, we see that $f'$ is positive on $(0,\infty)$ 
				and negative on $(-\infty, 0)$.  We know that $f$ is increasing on an interval if $f'$ is positive on that interval. 
				That means: $f$ is increasing on the interval $(0,\infty)$.

				From the sign-chart for $f''$, we see that $f'$ is positive on $(-2,2)$ 
				and negative on $(-\infty, -2)$ and $(2, \infty)$.  We know that $f$ is concave down on an interval if $f''$ is negative on that interval. 
				That means: $f$ is concave down on the intervals $(-\infty, -2)$ and $(2,\infty)$.
			
			\end{explanation}
			
			
			
	\end{enumerate}
			
			
			
		
\end{problem}



\end{document}
