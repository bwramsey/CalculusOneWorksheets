\documentclass{ximera}

\newcommand{\RR}{\mathbb R}
\renewcommand{\d}{\,d}
\newcommand{\dd}[2][]{\frac{d #1}{d #2}}
\renewcommand{\l}{\ell}
\newcommand{\ddx}{\frac{d}{dx}}
\newcommand{\dfn}{\textbf}
\newcommand{\eval}[1]{\bigg[ #1 \bigg]}
\renewcommand{\theenumii}{\textup{(\roman{enumii})}}
\renewcommand{\labelenumii}{\theenumii}

\usepackage{graphicx}
\usepackage{multicol}
\usepackage{tkz-euclide}
%\usepackage{unicode-math}

\usepackage{pgfplots}   % <- for graphics
\pgfplotsset{compat=newest}


\renewenvironment{freeResponse}{
\ifhandout\setbox0\vbox\bgroup\else
\begin{trivlist}\item[\hskip \labelsep\bfseries Solution:\hspace{2ex}]
\fi}
{\ifhandout\egroup\else
\end{trivlist}
\fi}

\newcommand*{\ZeroOverZero}{\ensuremath{\dfrac{0}{0}}}

\providecommand{\HCCondition}{0}
\newcommand{\WkstHop}[1][1]{\if\HCCondition 0
	\vspace*{\stretch{#1}} \fi} 
\newcommand{\WkstNew}{\if\HCCondition 0
	\newpage
	 \fi} 


\title[Problem 2]{Problem 2}

\begin{document}
\begin{abstract} \end{abstract}
\maketitle


% Extracted from higherOrderDerivativesAndGraphs.tex, problem #2
\begin{problem}
Suppose we know that function $f$ is positive and increasing, and the function $g$ is negative and decreasing. Determine whether the following functions are increasing or decreasing.
	\begin{enumerate}
		\item The product function: $f(x)g(x)$.
\begin{explanation}
				The derivative of $f(x)g(x)$ is $f'(x) g(x) + f(x) g'(x)$. Since $f$ is increasing and $g$ is decreasing, $f'(x)$ is positive and $g'(x)$ is 
				negative in the interval we are considering. That means $f'(x)g(x)$ is the product of a positive and a negative number, so it is negative. 
				Similary, $f(x)g'(x)$ is the product of a positive number and a negative number, so it is also negative. The result is that 
				$\dfrac{d}{dx}\left( f(x)g(x)\right)$ is negative, so $f(x)g(x)$ is decreasing in the interval.
			\end{explanation}
		\item The composite function: $f(g(x))$.
\begin{explanation}
				The derivative of $f(g(x))$ is $f'( g(x) ) g'(x)$. Since $f$ is increasing and $g$ is decreasing, $f'(x)$ is positive and $g'(x)$ is 
				negative in the interval we are considering. That means $f'(g(x)) g'(x)$ is the product of a positive and a negative number, so it is 		
				negative. The result is that $\dfrac{d}{dx}\left( f(g(x))\right)$ is negative, so $f(x)g(x)$ is decreasing in the interval.
			\end{explanation}
	\end{enumerate}
\end{problem}



\end{document}
