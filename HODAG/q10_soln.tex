\documentclass{ximera}

\newcommand{\RR}{\mathbb R}
\renewcommand{\d}{\,d}
\newcommand{\dd}[2][]{\frac{d #1}{d #2}}
\renewcommand{\l}{\ell}
\newcommand{\ddx}{\frac{d}{dx}}
\newcommand{\dfn}{\textbf}
\newcommand{\eval}[1]{\bigg[ #1 \bigg]}
\renewcommand{\theenumii}{\textup{(\roman{enumii})}}
\renewcommand{\labelenumii}{\theenumii}

\usepackage{graphicx}
\usepackage{multicol}
\usepackage{tkz-euclide}
%\usepackage{unicode-math}

\usepackage{pgfplots}   % <- for graphics
\pgfplotsset{compat=newest}


\renewenvironment{freeResponse}{
\ifhandout\setbox0\vbox\bgroup\else
\begin{trivlist}\item[\hskip \labelsep\bfseries Solution:\hspace{2ex}]
\fi}
{\ifhandout\egroup\else
\end{trivlist}
\fi}

\newcommand*{\ZeroOverZero}{\ensuremath{\dfrac{0}{0}}}

\providecommand{\HCCondition}{0}
\newcommand{\WkstHop}[1][1]{\if\HCCondition 0
	\vspace*{\stretch{#1}} \fi} 
\newcommand{\WkstNew}{\if\HCCondition 0
	\newpage
	 \fi} 


\title[Problem 10]{Problem 10}

\begin{document}
\begin{abstract} \end{abstract}
\maketitle


% Extracted from higherOrderDerivativesAndGraphs.tex, problem #10
\begin{problem}	
	\begin{enumerate}
	
	%part a
	\item  True or False:  If the acceleration of an object is constant, then its velocity is constant.
\begin{explanation}
		This is false in general, and is true if and only if the acceleration is 0.  Let $a$ denote your non-zero constant acceleration.  If $a > 0$ then your velocity is increasing, and if $a < 0$ then your velocity is decreasing.  See the accompanying picture for when $a>0$ (note that the units for $v(t)$ and $a(t)$ are \dfn{not} the same!)
		
			\begin{image}
			\includegraphics[trim= 170 420 250 180]{figure1.pdf}
			\end{image}
		\end{explanation}	
		
		
	
	%part b	
	\item  True or False:  A moving object can have negative acceleration and increasing speed.
\begin{explanation}
		True.  If your velocity is negative (which means that you are moving in the negative direction) then a negative acceleration increases the magnitude (or absolute value) of the velocity.  But the magnitude of the velocity is the speed.
		\end{explanation}	
		
		
		
	\end{enumerate}
	
	
\end{problem}



\end{document}
