%Add code to compile both versions from makefile at same time
\providecommand{\HCCondition}{0}
%Define each of the conditions
\ifcase\HCCondition
	%\condition=0 -> handout
	\documentclass[nooutcomes,noauthor,space,handout]{ximera}
	\title{Higher order derivatives and graphs (HODAG)} 
\or	%\condition=1 -> Soln
	\documentclass[nooutcomes,noauthor]{ximera}
	\title{Higher order derivatives and graphs (HODAG) - Solutions}  
\fi


\newcommand{\RR}{\mathbb R}
\renewcommand{\d}{\,d}
\newcommand{\dd}[2][]{\frac{d #1}{d #2}}
\renewcommand{\l}{\ell}
\newcommand{\ddx}{\frac{d}{dx}}
\newcommand{\dfn}{\textbf}
\newcommand{\eval}[1]{\bigg[ #1 \bigg]}
\renewcommand{\theenumii}{\textup{(\roman{enumii})}}
\renewcommand{\labelenumii}{\theenumii}

\usepackage{graphicx}
\usepackage{multicol}
\usepackage{tkz-euclide}
%\usepackage{unicode-math}

\usepackage{pgfplots}   % <- for graphics
\pgfplotsset{compat=newest}


\renewenvironment{freeResponse}{
\ifhandout\setbox0\vbox\bgroup\else
\begin{trivlist}\item[\hskip \labelsep\bfseries Solution:\hspace{2ex}]
\fi}
{\ifhandout\egroup\else
\end{trivlist}
\fi}

\newcommand*{\ZeroOverZero}{\ensuremath{\dfrac{0}{0}}}

\providecommand{\HCCondition}{0}
\newcommand{\WkstHop}[1][1]{\if\HCCondition 0
	\vspace*{\stretch{#1}} \fi} 
\newcommand{\WkstNew}{\if\HCCondition 0
	\newpage
	 \fi}  %% we can turn off input when making a master document
\usepackage{fullpage}
\usepackage{tkz-tab}
\usetikzlibrary{decorations.pathreplacing}

\begin{document}
\begin{abstract}		\end{abstract}
\maketitle
\ifcase\HCCondition
\section*{Recitation Questions}
\fi
\begin{problem}
Fill in the following blanks with the correct choice of the words from this list:
$$ \text{Increasing, decreasing, positive, negative, concave up, concave down} $$

	\begin{enumerate}
	
	%part a
	\item  If you know  that  $f''(x) > 0$, for all $x$ in some open interval $I$, then you know $f'(x)$ is \underline{\hspace{3cm}} and $f(x)$ is \underline{\hspace{3cm}}, for all $x$ in $I$.
\WkstHop
		\begin{freeResponse}
		increasing, concave up
		\end{freeResponse}
		
			
				
	%part b
	\item  If you know $g'(x) < 0$ and decreasing, for all $x$ in some open interval $I$, then you know $g(x)$ is \underline{\hspace{3cm}} and \underline{\hspace{3cm}}, for all $x$ in $I$.
\WkstHop
		\begin{freeResponse}
		decreasing, concave down
		\end{freeResponse}
		
			
				
	%part c
	\item  If you know $h(x)$ is positive, increasing, and concave down, for all $x$ in some open interval $I$, then you know $h'(x)$ is \underline{\hspace{3cm}} and \underline{\hspace{3cm}} and that $h''(x)$ is \underline{\hspace{3cm}}, for all $x$ in $I$.
\WkstHop
		\begin{freeResponse}
		positive, decreasing, negative
		\end{freeResponse}
		
			
				
	\end{enumerate}	
		
\end{problem}
\WkstNew



\begin{problem}
Suppose we know that function $f$ is positive and increasing, and the function $g$ is negative and decreasing. Determine whether the following functions are increasing or decreasing.
	\begin{enumerate}
		\item The product function: $f(x)g(x)$.
\WkstHop
			\begin{freeResponse}
				The derivative of $f(x)g(x)$ is $f'(x) g(x) + f(x) g'(x)$. Since $f$ is increasing and $g$ is decreasing, $f'(x)$ is positive and $g'(x)$ is 
				negative in the interval we are considering. That means $f'(x)g(x)$ is the product of a positive and a negative number, so it is negative. 
				Similary, $f(x)g'(x)$ is the product of a positive number and a negative number, so it is also negative. The result is that 
				$\dfrac{d}{dx}\left( f(x)g(x)\right)$ is negative, so $f(x)g(x)$ is decreasing in the interval.
			\end{freeResponse}
		\item The composite function: $f(g(x))$.
\WkstHop
			\begin{freeResponse}
				The derivative of $f(g(x))$ is $f'( g(x) ) g'(x)$. Since $f$ is increasing and $g$ is decreasing, $f'(x)$ is positive and $g'(x)$ is 
				negative in the interval we are considering. That means $f'(g(x)) g'(x)$ is the product of a positive and a negative number, so it is 		
				negative. The result is that $\dfrac{d}{dx}\left( f(g(x))\right)$ is negative, so $f(x)g(x)$ is decreasing in the interval.
			\end{freeResponse}
	\end{enumerate}
\end{problem}


\WkstNew


\begin{problem}
Let $f(x) = 2x^3-4x^2+1151$ and $g(x) = x^3\sin(x)$.  Compute $f^{(4)}(x)$ and $g^{(4)}(x)$. 
\WkstHop

\begin{freeResponse}
Let's find the first four derivatives of $f$.

	\begin{align*}
	f^{(0)}(x)&=2x^3-4x^2+1151\\
	f^{(1)}(x)&=6x^2-8x\\
	f^{(2)}(x)&=12x-8\\
	f^{(3)}(x)&=12\\
	f^{(4)}(x)&=0
	\end{align*}
	Let's find the first four derivatives of $g$.
\begin{align*}
	g^{(0)}(x)&=x^3\sin{x}\\
	g^{(1)}(x)&=3x^2\sin{x}+x^3\cos{x}\\
	g^{(2)}(x)&=6x\sin{x}+3x^2\cos{x}+3x^2\cos{x}-x^3\sin{x}\\
	&=(6x-x^3)\sin{x}+(6x^2)\cos{x}\\
	g^{(3)}(x)&=(6-3x^2)\sin{x}+(6x-x^3)\cos{x}+(12x)\cos{x}-(6x^2)\sin{x}\\
	&=(6-3x^2-6x^2)\sin{x}+(6x-x^3+12x)\cos{x}\\
	&=(6-9x^2)\sin{x}+(18x-x^3)\cos{x}\\
	g^{(4)}(x)&=(-18x)\sin{x}+(6-9x^2)\cos{x}+(18-3x^2)\cos{x}-(18x-x^3)\sin{x}\\
	&=(-18x-18x+x^3)\sin{x}+(6-9x^2+18-3x^2)\cos{x}\\
	&=(-36x+x^3)\sin{x}+(24-12x^2)\cos{x}
	\end{align*}
\end{freeResponse}
\end{problem}




\begin{problem}
Let $f(x) = \sin(x)$ and $g(x) = \cos(x)$.  Can you compute $f^{(48)}(x)$, $g^{(42)}(x)$, and $g^{(39)}(x)$ ?  
\WkstHop

\begin{freeResponse}
Let's find the first several derivatives of $f(x)=sin(x)$.

	\begin{align*}
	f^{(0)}(x)&=\sin(x)\\
	f^{(1)}(x)&=\cos(x)\\
	f^{(2)}(x)&=-\sin(x)\\
	f^{(3)}(x)&=-\cos(x)\\
	f^{(4)}(x)&=\sin(x)
	\end{align*}

  Recall that, for $n$ a nonnegative integer,
  $$f^{(4n)}(x) = f(x) = \sin(x) \quad \text{and} \quad g^{(4n)}(x) = g(x) = \cos(x).$$
  Thus: \\ 
  $f^{(48)}(x) =f(x)= \sin(x)$ \\
 $g^{(42)}(x) =g^{(40+2)}(x)=\bigl(g^{(40)}\bigr)^{(2)}(x)= g^{(2)}(x) = g''(x) = - \cos(x)$  \\
$g^{(39)}(x) =g^{(36+3)}(x)=\bigl(g^{(36)}\bigr)^{(3)}(x)= g^{(3)}(x) = g'''(x) = \sin(x)$  



\end{freeResponse}

\end{problem}


\WkstNew


\begin{problem}

  \begin{enumerate}
  %part a
    \item
      The (entire) graph of a function $f$ is shown in the figure below.
      \begin{image}
        \includegraphics[scale = 0.3]{figure3.png}
      \end{image}

      	%part i
      
	%part ii
         

     
          Find the interval (or intervals) on which the \emph{derivative of $f$} is increasing.
\WkstHop
          \begin{freeResponse}
            $f'$ is increasing  on (0,1): $f$ is concave up on this interval
          \end{freeResponse}



    \item
      The figure below shows the graphs of $f$, $f'$, and $f''$.
      Which curve is which?
      \begin{image}
        \includegraphics[scale = 0.3]{figure2.png}
      \end{image}
\WkstHop
      \begin{freeResponse}
        Graph A is $f$, Graph B is $f'$ and Graph C is $f''$.
      \end{freeResponse}


  \end{enumerate}
\end{problem}

\WkstNew


\begin{problem}
Consider the parabola $f(x)=ax^2+bx+c$ where $a,b,c$ are constants.  For what values of $a,b,c$ is $f$ concave up?  For what values of $a,b,c$ is $f$ concave down?				
\WkstHop
\begin{freeResponse}
\begin{align*}
f'(x)&=2ax+b\\
f''(x)&=2a
\end{align*}
$\implies$ the sign of $a$ will determine is $f$ is concave up or down.  When $a<0$, $f$ is concave down.  When $a>0$, $f$ is concave up.  Note: If $a=0$, $f$ has no concavity because it is a linear function.


				
\end{freeResponse}
				
\WkstNew

\end{problem}	
%problem 1
\begin{problem}
John is walking along a straight path.  His position at the time $t>0$ is given by $s=f(t)$.  He starts at $t=0$ from his house ($f(0)=0$) and the graph of $f$ is given below.


	\begin{image}
	\includegraphics[scale=.4]{Figure13.png}
	\end{image}

	\begin{image}
	\includegraphics{Figure2a.png}
	\end{image}
	
	\begin{enumerate}
	
	%part a
	\item  Describe the motion of John as precisely as you can.
\WkstHop
		\begin{freeResponse}
		Let us assume that $f(t)$ gives the position of John from his house at time $t$.  If $f(t) < 0$, then John is west of his house;  if $f(t) > 0$, then John is east of his house; if $f(t)=0$, then John is in front of his house.  
		
		For the first hour, that is for $0 \leq t \leq 1$ John walks east away from his house.
		
		\begin{image}
		\includegraphics[trim= 140 570 290 200]{Figure3.pdf}
		\end{image}
	
		For the second hour, that is, for $1 \leq t \leq 2$, John turns around and starts walking back west.  However, he does not walk all the way back to his house.
		
		\begin{image}
		\includegraphics[trim= 140 570 290 200]{Figure4.pdf}
		\end{image}
		
		For the third hour, that is, for $2\le t\le 3$, John realizes he needs to pick up an apple from the store, which is way east of his house, for Elaine.  So he turns around again and walks east to the store.
		
		\begin{image}
		\includegraphics[trim= 140 570 290 200]{Figure5.pdf}
		\end{image}
		
		For the fourth hour, that is, for $t \ge 3$, John needs to drop the apple off at Elaine's house, which is west of his house.  He turns west and walks past his house until he makes it to Elaine's house.
		
		\begin{image}
		\includegraphics[trim= 140 540 290 200]{Figure6.pdf}
		\end{image}

		\end{freeResponse}
		
		
		
	%part b
	\item  When is John's velocity zero?  What is happening to John at those times? 
\WkstHop

		\begin{freeResponse}
		John's velocity is zero at times $t=0,t=1,t=2$, and $t=3$ (the times where the slopes of the tangent lines are zero). At these times, John changes direction since the sign of the slope of the tangent line to the graph of $f$ changes then.
		\end{freeResponse}
		
		
		
	%part c
	\item  When is John moving in the positive direction?  When is John the furthest in the positive direction and the furthest in the negative direction?
\WkstHop

		\begin{freeResponse}
		John is moving in the positive direction (in this story, east) during times in the region $(0,1) \cup (2,3)$.  John is furthest in the positive direction at time $t=3$ and furthest in the negative direction at (approximately) time $t=3.5$.
		\end{freeResponse}
		
		
		
	%part d
	\item  When is John's velocity increasing?  And when is it decreasing?  When is John going at maximum velocity?
\WkstHop
		\begin{freeResponse}
		John's velocity is increasing on the time intervals $(0,0.3),(1.5,2.5)$.  John's velocity is decreasing on the time intervals $(0.3, 1.5),(2.5,3.5)$.  John is moving the fastest at time $t = 3.5$, but that maximizes his speed, not his velocity, because he is moving in the negative direction at that time.  It appears that he maximizes his velocity at about $t=2.5$.  
		\end{freeResponse}
		
		
		
	\end{enumerate}
			
			
	
\end{problem}

\WkstNew



%problem 2
\begin{problem}
Suppose that a stone is thrown vertically upward from a cliff on Mars with an initial velocity of $24$ ft/s from a height of $192$ ft.  The height $s$ of the stone above the ground after $t$ seconds is given by $s(t) = -6t^2 + 24t + 192$.

	\begin{enumerate}
	
	%part a
	\item  Determine the velocity and acceleration of the stone after $t$ seconds.
\WkstHop

			\begin{freeResponse}
			The velocity $v(t)$ is:  $v(t) = s'(t) = -12t + 24$.
			
			The acceleration $a(t)$ is:  $a(t) = v'(t) = s''(t) = -12$.
			\end{freeResponse}
			
			
			
	%part b
	\item  What is the greatest height of the stone and when does it occur?  What are the velocity and acceleration at that time?
\WkstHop

			\begin{freeResponse}
			Since the function $s(t)$ is differentiable everywhere (it is a polynomial), the maximum height must occur at a time when the velocity is $0$.  So we solve:
			\begin{align*}
			v(t) = -12t+24 &= 0 \\
			12t &= 24 \\
			t &= 2
			\end{align*}
			
			It is easy to check that $v(t) > 0$ for $0 \leq t < 2$ and $v(t) < 0$ for $2 < t$, and so the greatest height of the stone really does occur at time $t=2$.  The greatest height is $s(2) = -24 + 48 + 192 = 216$ ft.  For the second question we have already seen that $v(2) = 0$ ft/sec, and we have a constant acceleration of $-12$ ft/sec$^2$.  So $a(2) = -12$ ft/sec$^2$.
			\end{freeResponse}
			
			
			
	%part c
	\item  When does the stone hit the ground?  What are the velocity and acceleration at that time?
\WkstHop

			\begin{freeResponse}
			The stone hits the ground when $s(t) = 0$.  So we solve:
			\begin{align*}
			s(t) = -6t^2 + 24t + 192 &= 0 \\
			-6(t^2 - 4t - 32) &= 0 \\
			-6(t-8)(t+4) &= 0 
			\end{align*}
			
			Since we are only considering $t \geq 0$, we must have $t=8$.  At this instant, $v(8) = -96 + 24 = -72$ ft/sec and $a(8) = -12$ ft/sec$^2$.  
			
			\end{freeResponse}
			
			
			
	\end{enumerate}
			
			
			
		
\end{problem}
\WkstNew

%problem 2
\begin{problem}
	Suppose the function $f$ has derivative given by $\displaystyle f'(x) = \dfrac{2x}{4+x^2}$.

	\begin{enumerate}
	
	%part a
	\item  Find a formula for $f''(x)$.
\WkstHop
			\begin{freeResponse}
				The second derivative is given by $\displaystyle f''(x) = \dfrac{2(4-x^2)}{(4+x^2)^2}$.
			\end{freeResponse}
			
			
			
	%part b
	\item  Construct a sign-chart for $f'$ and $f''$.
\WkstHop[2]
		\begin{freeResponse}
			For $f'$ we obtain:

			\begin{tikzpicture} 
				\tkzTabInit[lgt=2,espcl=1] 
					{$x$         /1, 
					$2x$   /1, 
					$4+x^2$ /1, 
					$f'(x)$      /1}% 
					{   , $0$ ,  }% 
				\tkzTabLine{ , - , t , + ,}
				\tkzTabLine{ , + , t , + ,}
				\tkzTabLine{ , - , t , + ,}
			\end{tikzpicture} 
			
			For $f''$, using the factorization $4-x^2 = (2+x)(2-x)$ , we obtain:			

			\begin{tikzpicture} 
				\tkzTabInit[lgt=2,espcl=1] 
					{$x$         /1, 
					$2$   /1, 
					$2+x$  /1,
					$2-x$  /1,
					$(4+x^2)^2$   /1, 
					$f''(x)$    /1}% 
					{  , $-2$ ,$2$,  }% 
				\tkzTabLine{  , + , t , + , t , + ,}
				\tkzTabLine{  , - , t , + , t , + ,}
				\tkzTabLine{  , + , t , + , t , - ,}
				\tkzTabLine{  , + , t , + , t , + ,}
				\tkzTabLine{  , - , t , + , t , - ,}
			\end{tikzpicture} 
		\end{freeResponse}
			
			
			
	%part c
	\item  On what interval(s) is $f$ increasing? On what interval(s) is $f$ concave down?
\WkstHop

			\begin{freeResponse}
				From the sign-chart for $f'$, we see that $f'$ is positive on $(0,\infty)$ 
				and negative on $(-\infty, 0)$.  We know that $f$ is increasing on an interval if $f'$ is positive on that interval. 
				That means: $f$ is increasing on the interval $(0,\infty)$.

				From the sign-chart for $f''$, we see that $f'$ is positive on $(-2,2)$ 
				and negative on $(-\infty, -2)$ and $(2, \infty)$.  We know that $f$ is concave down on an interval if $f''$ is negative on that interval. 
				That means: $f$ is concave down on the intervals $(-\infty, -2)$ and $(2,\infty)$.
			
			\end{freeResponse}
			
			
			
	\end{enumerate}
			
			
			
		
\end{problem}

\WkstNew

%problem3	
\begin{problem}	
	\begin{enumerate}
	
	%part a
	\item  True or False:  If the acceleration of an object is constant, then its velocity is constant.
\WkstHop

		\begin{freeResponse}
		This is false in general, and is true if and only if the acceleration is 0.  Let $a$ denote your non-zero constant acceleration.  If $a > 0$ then your velocity is increasing, and if $a < 0$ then your velocity is decreasing.  See the accompanying picture for when $a>0$ (note that the units for $v(t)$ and $a(t)$ are \dfn{not} the same!)
		
			\begin{image}
			\includegraphics[trim= 170 420 250 180]{Figure1.pdf}
			\end{image}
		\end{freeResponse}	
		
		
	
	%part b	
	\item  True or False:  A moving object can have negative acceleration and increasing speed.
\WkstHop

		\begin{freeResponse}
		True.  If your velocity is negative (which means that you are moving in the negative direction) then a negative acceleration increases the magnitude (or absolute value) of the velocity.  But the magnitude of the velocity is the speed.
		\end{freeResponse}	
		
		
		
	\end{enumerate}
	
	
\end{problem}	

\WkstNew

\begin{problem}
The total number of people, $N$ who have contracted a common cold by a time $t$ days after its outbreak on an island is given by $N=N(t)= \frac{200000}{1+100e^{-0.1t}},t\geq0$.

	\begin{enumerate}
		%a
		\item Evaluate and interpret the limit $\lim_{t \to \infty} N(t)$
\WkstHop
			\begin{freeResponse}	
			$\lim_{t \to \infty} N(t)=\lim_{t \to \infty} \frac{200000}{1+100e^{-0.1t}}=200000$.  In the long run, 200000 people will get the cold.
			\end{freeResponse}
		%b
		\item How long will it take for the number of people who have contracted the cold to reach 40,000?
\WkstHop
			\begin{freeResponse}
				\begin{align*}
				N(t)&=40000\\
				\frac{200000}{1+100e^{-0.1t}}&=40000\\
				200000&=40000(1+100e^{-0.1t})\\
				1+100e^{-0.1t}&=\frac{200000}{40000}\\
				100e^{-0.1t}&=5-1\\
				e^{-0.1t}&=\frac{4}{100}\\
				-0.1t&=\ln(.04)\\
				t&=\frac{\ln(0.04)}{-0.1}=32.1888
				\end{align*}
			It will take 33 days for the number of people who have contracted the cold to reach 40,000.
			\end{freeResponse}

		\item The graph of the function $N$ on the interval $[0,100]$ is given below.  Sketch (as best you can) the graph of its derivative, $N'(t)$.
			\begin{image}
			\includegraphics[scale=.4]{Figure8.png}
			\end{image}
\WkstHop
			\begin{freeResponse}\hfil
				\begin{image}
			\includegraphics[scale=.5]{Figure9.png}
			\end{image}		
			\end{freeResponse}
		\item Calculate $N'(t)$.  What does $N'(t)$ represent?
\WkstHop

			\begin{freeResponse}	
				$N'(t)=\frac{-200000(10)e^{-0.1t}}{(1+100e^{-0.1t})^2}=\frac{2000000e^{-0.1t}}{(1+100e^{-0.1t})^2}$.\\
				$N'(t)$ represents the instantaneous growth rate of the number of people who have contracted the cold at the time $t$.
			\end{freeResponse}
\WkstNew

		\item Evaluate and interpret the limit $\lim_{t \to \infty} N'(t)$
\WkstHop
			\begin{freeResponse}	
				$\lim_{t \to \infty} N'(t)=\lim_{t \to \infty}\frac{2000000e^{-0.1t}}{(1+100e^{-0.1t})^2}=\frac{2000000(0)}{(1+100(0))^2}=\frac{0}{1}=0$.\\
				In the long run, the number of people who have contracted the cold will stabilize.  There will be no growth in the long run.
			\end{freeResponse}

		\item Find the average growth rate of the number of people who have contracted the disease during the time interval $[5,6]$ (or during the sixth day after the outbreak).
\WkstHop
			\begin{freeResponse}	
				$AVRG=\frac{N(6)-N(5)}{6-5}=N(6)-N(5)=\frac{200000}{1+100e^{-0.1(6)}}-\frac{200000}{1+100e^{-0.1(5)}} \approx 335.1$
			\end{freeResponse}

		\item Find the instantaneous growth rate of the number of people who have contracted the disease for $t=5$ (round to a whole number).
\WkstHop
			\begin{freeResponse}	
				$N'(5)=\frac{2000000e^{-0.1(5)}}{(1+100e^{-0.1(5)})^2}=319$ people per day
			\end{freeResponse}

	\end{enumerate}
\end{problem}	
\WkstNew	
		
%problem6			


%problem7
\begin{problem}
The graph of $h(t)$ represents the height in feet of water in a pool at time $t$ minutes.
			\begin{image}
			\includegraphics[scale=.5]{Figure7.png}
			\end{image}
	

	\begin{enumerate}
		\item During what time intervals is the height of water in the pool increasing? Decreasing?
\WkstHop

			\begin{freeResponse}
			The height of the water in the pool is increasing when the function $h$ is increasing or has positive slope.  This is $(0,3),(7,10)$.  The height of the water is decreasing when the function $h$ is decreasing or has negative slope.  This is $(3,7),(10,12)$.
			\end{freeResponse}
		\item  Assume the rate of change of the water in the pool is $0$ when $t=0$.  For what other values of $t$ is the rate of change of the water in the pool $0$?
\WkstHop

			\begin{freeResponse}
			The rate of the change of water in the pool is $0$ when the slope of $h$ is zero.  This is at $t=3,7,10$.
			\end{freeResponse}
		\item If $h(t)$ represents the height of the feet in water at time $t$ minutes.  What function represents the rate of change of the height over time?
\WkstHop
			\begin{freeResponse}
				$h'(t)$, the derivative of $h(t)$.  $h'(t)$ tells us about the slope, or rate of change, of height with respect to time.
			\end{freeResponse}
		\item Sketch a graph of the rate of change of the height over time, $h'(t)$.
\WkstHop
			\begin{freeResponse}
			We'll learn how to sketch graphs of derivatives more precisely in 4.3 but for now we can use information about the slope to help us graph $h'(t)$.\\
			First, we'll plot the points where the slope of the graph of $h(t)$ is zero.  This is at $t=0,3,7,10$.
			\begin{image}
			\includegraphics[scale=.4]{Figure10.png}
			\end{image}
			Next, we'll use information about whether the slope is positive or negative to add tails to the zeros to indicate if the derivative should be postive or negative.  Remember, the derivative tells us about the slope. The slope of $h$ is postive on $(0,3),(7,10)$, so $h'(t)$ is positive on $(0,3),(7,10)$.  The slope of $h$ is negative on $(3,7),(10,12)$, so $h'(t)$ is negative on $(3,7),(10,12)$.
			\begin{image}
			\includegraphics[scale=.4]{Figure11.png}
			\end{image}
			Finally, we'll connect our tails.  If you've already taken calculus, you may recall that we know when our derivitive $h'$ has its highest and lowest points based on where $h$ changes concavity.  If you haven't had calculus, don't worry, we'll get there.  For now, just guess where you think these might be.
			\begin{image}
			\includegraphics[scale=.4]{Figure12.png}
			\end{image}
			\end{freeResponse}
	\end{enumerate}

\end{problem}






\end{document} 






\begin{problem}
An oil tank is to be drained for cleaning. There are $V(t)$ gallons of oil left in the tank $t$ minutes after the draining began, where $V(t)=45(60-t)^2$.

	\begin{enumerate} 
		%part a
		\item Find the average rate at which the oil drians during the first 15 minutes.
			\begin{freeResponse}
				\begin{align*}
					AR&= \frac{V(15)-V(0)}{15-0}\\
					&=\frac{45(60-15)^2-45(60-0)^2}{15-0}\\
					&=\frac{45(45)^2-45(60)^2}{15}\\
					&=\frac{45(2025-3600)}{15}\\
					&=-4725\ \text{gallons per minute}
				\end{align*}


			\end{freeResponse}
			
		%part b
		\item Find the average rate at which the oil drains during the time interval $[10,15]$.
			\begin{freeResponse}
					\begin{align*}
					AR&= \frac{V(15)-V(10)}{15-10}\\
					&=\frac{45(60-15)^2-45(60-10)^2}{15-10}\\
					&=\frac{45(45)^2-45(50)^2}{5}\\
					&=\frac{45(2025-2500)}{5}\\
					&=-4275\ \text{gallons per minute}
				\end{align*}	


			\end{freeResponse}
		%part c
		\item Find the rate at which the oil is flowing out of the tank 15 minutes after the draining began.
			\begin{freeResponse}
			\begin{align*}
			V'(t)&= \eval{\dd{t} 45(60-t)^2)}_{t=15}\\
			&=\eval{45 \dd{t} (60-t)^2}_{t=15}\\
			&=45\eval{\dd{t}(3600-120t+t^2)}_{t=15}\\
			&=\eval{45(-120+2t)}_{t=15}\\
			&=45(-120+2(15))\\
			&=45(-120+30)\\
			&=45(-90)\\
			&=-4050\ \text{gallons per minute}
			\end{align*}
			\end{freeResponse}
		%part d
		\item Find the average rate, $AR \Delta t$, at which the oil drains during the time interval:
			\begin{enumerate}
				\item $[15+ \Delta t,15],\ \text{if}\ -1< \Delta t<0$
			\begin{freeResponse}
					\begin{align*}
					AR&= \frac{V(15)-V(15+\Delta t)}{15-(15+\Delta t}\\
					&=\frac{45(60-15)^2-45(60-(15+\Delta t))^2}{-\Delta t}\\
					&=\frac{-45(45)^2+45(45-\Delta t)^2}{\Delta t}\\
					&=\frac{45(-2025+2025-90\Delta t+(\Delta t)^2)}{\Delta t}\\
					&=\frac{45(-90\Delta t+(\Delta t)^2)}{\Delta t}\\
					&=45(-90+\Delta t)\\
					&=(-4050+45\Delta t)\ \text{gallons per minute}			
					\end{align*}
			\end{freeResponse}			

				\item $[15,15+ \Delta t],\ \text{if}\ 0< \Delta t<1$
			\begin{freeResponse}
					\begin{align*}
					AR&= \frac{V(15+\Delta t)-V(15)}{15+\Delta t-15}\\
					&=\frac{45(60-(15+\Delta t))^2-45(60-15)^2}{\Delta t}\\
					&=\frac{45(45-\Delta t)^2-45(45)^2}{\Delta t}\\
					&=\frac{45(2025-90\Delta t+(\Delta t)^2)-2025}{\Delta t}\\
					&=\frac{45(-90\Delta t+(\Delta t)^2)}{\Delta t}\\
					&=45(-90+\Delta t)\\
					&=(-4050+45\Delta t)\ \text{gallons per minute}			
					\end{align*}
			\end{freeResponse}
			\end{enumerate}

		\item Use the result in part (d) to find the limit $\lim_{\Delta t \to 0} AR(\Delta t)$.  What does this limit represent?
			\begin{freeResponse}
			$\lim_{\Delta t \to 0} AR(\Delta t)=\lim_{\Delta t \to 0} (-4050+45\Delta t) =-4050$ gallons per minute.  This is the instantaneous rate at which the oil is flowing out of the tank 15 minutes after the draining began.
			\end{freeResponse}
	\end{enumerate}
	


\end{problem}












