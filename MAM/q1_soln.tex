\documentclass{ximera}

\newcommand{\RR}{\mathbb R}
\renewcommand{\d}{\,d}
\newcommand{\dd}[2][]{\frac{d #1}{d #2}}
\renewcommand{\l}{\ell}
\newcommand{\ddx}{\frac{d}{dx}}
\newcommand{\dfn}{\textbf}
\newcommand{\eval}[1]{\bigg[ #1 \bigg]}
\renewcommand{\theenumii}{\textup{(\roman{enumii})}}
\renewcommand{\labelenumii}{\theenumii}

\usepackage{graphicx}
\usepackage{multicol}
\usepackage{tkz-euclide}
%\usepackage{unicode-math}

\usepackage{pgfplots}   % <- for graphics
\pgfplotsset{compat=newest}


\renewenvironment{freeResponse}{
\ifhandout\setbox0\vbox\bgroup\else
\begin{trivlist}\item[\hskip \labelsep\bfseries Solution:\hspace{2ex}]
\fi}
{\ifhandout\egroup\else
\end{trivlist}
\fi}

\newcommand*{\ZeroOverZero}{\ensuremath{\dfrac{0}{0}}}

\providecommand{\HCCondition}{0}
\newcommand{\WkstHop}[1][1]{\if\HCCondition 0
	\vspace*{\stretch{#1}} \fi} 
\newcommand{\WkstNew}{\if\HCCondition 0
	\newpage
	 \fi} 


\title[Problem 1]{Problem 1}

\begin{document}
\begin{abstract} \end{abstract}
\maketitle


% Extracted from maximumsAndMinimums.tex, problem #1
\begin{problem}
Determine whether the following statements are true or false and give either an explanation or a counterexample.

	\begin{enumerate}
	
	%part a
	\item  The function $f(x) = \sqrt{x}$ has a local maximum on the interval $[0,1]$.
		\begin{explanation}
		False.  Since $\sqrt{x}$ is increasing over the entire region $[0,1]$, the only candidate for a local maximum would be $x=1$.  But by definition, endpoints are never local extrema.  So $f(x) = \sqrt{x}$ has no local maximum on the interval $[0,1]$.
		\end{explanation}	
%part c
	\item  If $f'(2)=0$, then $x=2$ is either a local maximum or local minimum of $f$.
					\begin{explanation}
				False.  Consider $f(x)=(x-2)^3$.  This has derivative $f'(x) = 3(x-2)^2$, so that $f'(2)=0$.   We can see from the graph of $f$ below, that $x=2$ is not a local extremum for $f$.
				\begin{center}
					\begin{tikzpicture}
						\begin{axis}[
							xmin=-1.5, xmax=5.5, ymin=-6.5,ymax=12.5,    
							axis lines =middle, 
							every axis y label/.style={at=(current axis.above origin),anchor=south},
							every axis x label/.style={at=(current axis.right of origin),anchor=west},
							xtick={-1,...,5}, ytick={-6, -4,...,12},
							grid=major, width=3in,
							grid style={dashed, gray!40} 
							]
							\addplot[color=blue, very thick, smooth, domain=-1.5:5.5]{(x-2)^3};
						\end{axis}
					\end{tikzpicture}
				\end{center}
			\end{explanation}
			\end{enumerate}
\end{problem}



\end{document}
