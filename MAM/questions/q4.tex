% Extracted from maximumsAndMinimums.tex, problem #4
\begin{problem}
Locate the critical points and use the second derivative test to determine whether they correspond to local maxima or local minima. \textbf{EXPLAIN}.

$f(x)=(x+c)^4$ where $c$ is a positive constant 
\WkstHop

\begin{freeResponse}

$$f'(x)=4(x+c)^3, f''(x)=12(x+c)^2$$

To find the critical points: $f'(x)=0=4(x+c)^3$.  This occurs when $x=-c$
 
Using the second derivative test: $f''(-c)=12(-c+c)^2=0$.  The second derivative test was inconclusive so we need to use the first derivative test.

Making a sign chart we see that if we plug something a little less than $-c$ into $f'$, we'll get a negative number.  If we plug something a little greater than $-c$ into $f'$, we'll get a positive number. 

      \begin{image}
        \includegraphics[scale = 0.5]{figure4a.png}
             \end{image}
The derivative of $f$, $f'$, does not change sign on the intervals $(-\infty,-c)$ and on $(-c,\infty)$, therefore, we have the above chart.  From the chart, we can conclude that $f$ has a local minimum at $x=-c$, and that $f$ also has a global minimum there.  We could also have used our understanding of the family of functions $g(x)=x^4$ to reason $x=-c$ would be a minimum.

At $x=-c$, $f$ has a local minimum by the first derivative test since $x=-c$ is a critical point of $f$ and the sign of $f'$ changes from negative to positive around $x=-c$.

\end{freeResponse}
\end{problem}
