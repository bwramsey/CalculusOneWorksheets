% Extracted from maximumsAndMinimums.tex, problem #7
\begin{problem}
Let $f(x) = \frac{1}{1 + x^2}$.  Find the following for $f$:

	\begin{enumerate}
	%part a
	\item  $f'$ and $f''$
\WkstHop
	
		\begin{freeResponse}
			\begin{align*}
			f'(x) &= \frac{(1+x^2)(0) - 1(2x)}{(1+x^2)^2} \\
			&= \frac{-2x}{(1+x^2)^2}
			\end{align*}
			
			\begin{align*}
			f''(x) &= \frac{(1+x^2)^2(-2) - (-2x)(2)(1+x^2)(2x)}{(1+x^2)^4} \\
			&= \frac{-2(1+x^2) + 8x^2}{(1+x^2)^3} \\
			&= \frac{6x^2 - 2}{(1+x^2)^3}
			\end{align*}
		\end{freeResponse}
	%part b	
	\item  Critical points
\WkstHop
	
		\begin{freeResponse}
		Since $1+x^2 > 0$ for all $x$, $f$ is differentiable over all real numbers.  Thus all critical points of $f$ occur when $f'(x) = 0$.  But a fraction equals 0 if and only if its numerator equals 0.  So
		$$ f'(x) = 0 \qquad \Longrightarrow \qquad -2x = 0 \qquad \Longrightarrow \qquad x=0 $$
		Hence, the only critical point is $x=0$.  
		\end{freeResponse}

\WkstNew		
	%part d	
	\item  Local extrema (and check your answers with both the first and second derivative tests)
\WkstHop
	
		\begin{freeResponse}
		At $x=0$, $f'$ changes sign from positive to negative.  Thus $f$ goes from increasing to decreasing, and therefore by the first derivative test $x=0$ is a local maximum of $f$.  
		
		For the second derivative test, we have that
		$$ f''(0) = \frac{6(0)^2 - 2}{(1+0^2)^3} = \frac{-2}{1} = -2 < 0 $$
		and thus we again conclude that $x=0$ is a local maximum of $f$.
		\end{freeResponse}

	%part f	
	\item  Inflection points. \textbf{EXPLAIN}.
\WkstHop
	
		\begin{freeResponse}
		By the results in part 
		\begin{align*}
 		f''(x) &= \frac{6x^2 - 2}{(1+x^2)^3}\\
		 &= \frac{6(x^2 - \frac{1}{3})}{(1+x^2)^3}\\
		 &= \frac{6(x - \frac{1}{\sqrt{3}})(x + \frac{1}{\sqrt{3}})}{(1+x^2)^3}\\
		\end{align*}
		  which implies that the only candidates for inflection points are points where $x=\pm\frac{1}{\sqrt{3}}$.
		On the other hand, since $-1<- \frac{1}{\sqrt{3}}<0$, $f''(-1)=\frac{1}{2}$, and $f''(0)=-2$, it follows that $f''$ changes the sign at  $x=-\frac{1}{\sqrt{3}}$. Similarly, since $0<x=\frac{1}{\sqrt{3}}<1$, $f''(0)=-2$, $f''(1)=\frac{1}{2}$, it follows that $f''$ changes the sign at  $x=\frac{1}{\sqrt{3}}$, too.
		Since,
		\begin{align*}
 		 & f\left( \frac{1}{\sqrt{3}} \right)=\frac{3}{4} \\ 
 		& f\left( -\frac{1}{\sqrt{3}} \right)=\frac{3}{4} \\ 
		\end{align*}  
		The function $f$ has inflection points at $\left( -\frac{1}{\sqrt{3}},\frac{3}{4} \right),\left( \frac{1}{\sqrt{3}},\frac{3}{4} \right)$ because $f$ is continuous at those points and $f$ changes concavity around those points.
		\end{freeResponse}
		
	\end{enumerate}
		
\end{problem}
