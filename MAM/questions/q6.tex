% Extracted from maximumsAndMinimums.tex, problem #6
\begin{problem}
Find the critical points of $f$ on the given interval. Determine whether the function $f$ has a local maximum, local minimum or no local extremum at each critical point. \textbf{EXPLAIN}.
		\begin{enumerate}
		
			%part a
			\item  $f(x) = x \sqrt{2-x^2}$ on $( -\sqrt{2}, \sqrt{2} )$.
			\WkstHop
			
				\begin{freeResponse}
								\begin{align*}
				f'(x) &= \sqrt{2-x^2} + x \left( \frac{1}{2 \sqrt{2-x^2}} (-2x) \right) \\
				&= \sqrt{2-x^2} - \frac{x^2}{\sqrt{2-x^2}} \\
				&= \frac{2-x^2-x^2}{\sqrt{2-x^2}} \\
				&= \frac{2(1-x^2)}{\sqrt{2-x^2}}
				\end{align*}
				
				Critical points of $f$ occur where $f'(x) = 0$ or where $f'(x)$ does not exist.  Solving $f'(x) = 0$ yields that $2(1-x^2) = 0$, or $x = \pm 1$.  $f'(x)$ does not exist when $2-x^2 \leq 0$, but the points $x=\pm \sqrt{2}$ are not in the domain.\\  

				Let us use the Second Derivative Test to determine whether the function $f$ has a local maximum or local minimum at the points where $x=\pm1$.
				\begin{align*}
				f''(x) &=\frac{d}{dx}\frac{2(1-x^2)}{\sqrt{2-x^2}}
				= \frac{2x(x^2-3)}{(2-x^2)^{\frac{3}{2}}}
				\end{align*}
				Since $f''(-1)=4$, the point $(-1,-1)$ is a local minimum by the Second Derivative Test.
				Since $f''(1)=-4$, the point $(1,1)$ is a local maximum by the Second Derivative Test.
								\end{freeResponse}
				
				
				
			%part b
			\item  $f(x) = x^3 e^{-x}$ on $(-1,5)$.
			\WkstHop

				\begin{freeResponse}
			
				\begin{align*}
				f'(x) &= 3x^2 e^{-x} + x^3(-e^{-x}) \\
				&= x^2 e^{-x} (3-x)
				\end{align*}
				
				Notice that $f'(x)$ always exists, and so all of the critical points of $f$ occur when $f'(x)=0$.  Solving this equation:
				$$ x^2 e^{-x} (3-x) = 0 $$
				$$ x^2 (3-x) = 0 $$
				$$ x = 0 \qquad \text{or} \qquad x=3 $$
				It is easier to use the First derivative test in order to determine whether the function $f$ has a local maximum, local minimum or no local extremum at the point where $x=0$. Since $-1<0<1$, $f'(-1)=4e>0$, and $f'(1)=2e^{-1}>0$, it follows that the sign of $f'$ does not change at $x=0$. The function $f$ has \textbf{no local extremum} at $x=0$.\\

				Similarly, we will use the First derivative test in order to determine whether the function $f$ has a local maximum, local minimum or no local extremum at the points where $x=3$. Since $1<3<4$, $f'(1)=2e^{-1}>0$, and  $f'(4)=-16e^{-4}<0$ it follows that the sign of $f'$ changes from positive to negative at $x=3$. 
				The function $f$ has \textbf{a local maximum} at $x=3$ by the First Derivative Test.
 
						
				\end{freeResponse}
				
			\WkstNew	
				
			%part c
			\item  $f(x) = x \ln \left( \frac{x}{5} \right)$ on $(0, 5)$.
			\WkstHop

				\begin{freeResponse}
								\begin{align*}
				f'(x) &= \ln \left( \frac{x}{5} \right) + x \cdot \frac{5}{x} \cdot \frac{1}{5} \\
				&= \ln \left( \frac{x}{5} \right) + 1
				\end{align*}
				
				Notice that $f'(x)$ exists for all values in $(0,5)$, and so all of the critical points of $f$ occur when $f'(x)=0$.  Solving this equation:
				$$ \ln \left( \frac{x}{5} \right) + 1 = 0 $$
				$$ \ln \left( \frac{x}{5} \right) = -1 $$
				$$ \frac{x}{5} = e^{-1} $$
				$$ x = 5e^{-1} = \frac{5}{e} $$
						\\
				Let's use the Second Derivative Test to determine whether the function $f$ has a local maximum, local minimum or no local extremum at the points where $x= \frac{5}{e} $.
				\begin{align*}
				f''(x) &=\frac{d}{dx}( \ln \left( \frac{x}{5} \right) + 1)\\
				&=\frac{d}{dx}( \ln \left( x \right) -\ln \left( 5 \right)+ 1)
				= \frac{1}{x}
				\end{align*}
				Since, $f''\Bigl( \frac{5}{e}\Bigr)>0$, the point $\Bigl( \frac{5}{e},- \frac{5}{e}\Bigr)$ is a local minimum by the Second Derivative Test.
						
		
				\end{freeResponse}
				
				
				
			\end{enumerate}

		
		
		

\end{problem}
