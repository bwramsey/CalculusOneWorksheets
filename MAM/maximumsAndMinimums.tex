%Add code to compile both versions from makefile at same time
\providecommand{\HCCondition}{0}
%Define each of the conditions
\ifcase\HCCondition
	%\condition=0 -> handout
	\documentclass[nooutcomes,noauthor,space,handout]{ximera}
	\title{Maximums and minimums (MAM)} 
\or	%\condition=1 -> Soln
	\documentclass[nooutcomes,noauthor]{ximera}
	\title{Maximums and minimums (MAM) - Solutions} 
\fi



\usepackage{fullpage}

\newcommand{\RR}{\mathbb R}
\renewcommand{\d}{\,d}
\newcommand{\dd}[2][]{\frac{d #1}{d #2}}
\renewcommand{\l}{\ell}
\newcommand{\ddx}{\frac{d}{dx}}
\newcommand{\dfn}{\textbf}
\newcommand{\eval}[1]{\bigg[ #1 \bigg]}
\renewcommand{\theenumii}{\textup{(\roman{enumii})}}
\renewcommand{\labelenumii}{\theenumii}

\usepackage{graphicx}
\usepackage{multicol}
\usepackage{tkz-euclide}
%\usepackage{unicode-math}

\usepackage{pgfplots}   % <- for graphics
\pgfplotsset{compat=newest}


\renewenvironment{freeResponse}{
\ifhandout\setbox0\vbox\bgroup\else
\begin{trivlist}\item[\hskip \labelsep\bfseries Solution:\hspace{2ex}]
\fi}
{\ifhandout\egroup\else
\end{trivlist}
\fi}

\newcommand*{\ZeroOverZero}{\ensuremath{\dfrac{0}{0}}}

\providecommand{\HCCondition}{0}
\newcommand{\WkstHop}[1][1]{\if\HCCondition 0
	\vspace*{\stretch{#1}} \fi} 
\newcommand{\WkstNew}{\if\HCCondition 0
	\newpage
	 \fi}  %% we can turn off input when making a master document


\begin{document}
\begin{abstract}		\end{abstract}
\maketitle
\ifcase\HCCondition
\section*{Recitation Questions}
\fi
%problem1
\begin{problem}
Determine whether the following statements are true or false and give either an explanation or a counterexample.

	\begin{enumerate}
	
	%part a
	\item  The function $f(x) = \sqrt{x}$ has a local maximum on the interval $[0,1]$.
		\begin{freeResponse}
		False.  Since $\sqrt{x}$ is increasing over the entire region $[0,1]$, the only candidate for a local maximum would be $x=1$.  But by definition, endpoints are never local extrema.  So $f(x) = \sqrt{x}$ has no local maximum on the interval $[0,1]$.
		\end{freeResponse}	
\WkstHop
				
	%part c
	\item  If $f'(2)=0$, then $x=2$ is either a local maximum or local minimum of $f$.
					\begin{freeResponse}
				False.  Consider $f(x)=(x-2)^3$.  This has derivative $f'(x) = 3(x-2)^2$, so that $f'(2)=0$.   We can see from the graph of $f$ below, that $x=2$ is not a local extremum for $f$.
				\begin{center}
					\begin{tikzpicture}
						\begin{axis}[
							xmin=-1.5, xmax=5.5, ymin=-6.5,ymax=12.5,    
							axis lines =middle, 
							every axis y label/.style={at=(current axis.above origin),anchor=south},
							every axis x label/.style={at=(current axis.right of origin),anchor=west},
							xtick={-1,...,5}, ytick={-6, -4,...,12},
							grid=major, width=3in,
							grid style={dashed, gray!40} 
							]
							\addplot[color=blue, very thick, smooth, domain=-1.5:5.5]{(x-2)^3};
						\end{axis}
					\end{tikzpicture}
				\end{center}
			\end{freeResponse}
			\end{enumerate}
\WkstHop		
\end{problem}	

\WkstNew

%problem 2
\begin{problem}
	The entire graph of a function $g$ is given below.
				\begin{center}
					\begin{tikzpicture}
						\begin{axis}[
							xmin=-6.5, xmax=6.5, ymin=-4.5,ymax=4.5,    
							axis lines =middle, 
							every axis y label/.style={at=(current axis.above origin),anchor=south},
							every axis x label/.style={at=(current axis.right of origin),anchor=west},
							xtick={-6,...,6}, ytick={-4,...,4},
							grid=major, width=3in,
							grid style={dashed, gray!40} 
							]
							\addplot[color=blue, very thick, smooth, domain=-6:-3]{-(2/3)*(x+3)^2+4};
							\draw[fill=blue] (axis cs:-6,-2) circle [color=blue,radius=3pt] node [below right] {};						
						
							\addplot[color=blue, very thick, smooth, domain=-3:0]{(2/3)*x^2-2};						
							\addplot[color=blue, very thick, smooth, domain=0:2]{-(1/2)*x^2-2};												
							\addplot[color=blue, very thick, smooth, domain=2:4]{-(8^(0.5))*(4-x)^(0.5)};																		
							\addplot[color=blue, very thick, smooth, domain=4:6]{-(3/4)*(x-4)^2+2} node[anchor=south,pos=0]{$\Large{y=g(x)}$};
							
							\draw[fill=blue] (axis cs:6,-1) circle [color=blue,radius=3pt] node [below right] {};						
							\draw[fill=blue] (axis cs:4,2) circle [color=blue,radius=3pt] node [below right] {}; 					
													
						\draw[fill=blue] (axis cs:4,0) circle [color=blue,radius=3pt] node [below right] {};
						\draw[fill=white] (axis cs:4,0) circle [color=white,radius=2pt] node [below right] {};
						
						
						
						\end{axis}
					\end{tikzpicture}
				\end{center}
	Based on the graph of $g$, answer the questions below.
	\begin{enumerate}
		\item List the $x$-coordinates of all critical points of $g$.
\WkstHop
			\begin{freeResponse}
				$x = -3$, $x=0$, $x=2$, and $x=4$.
			\end{freeResponse}
		\item List the $x$-coordinates of all critical points of $g$ where $g'(x)=0$.
\WkstHop
			\begin{freeResponse}
				$x = 0$
			\end{freeResponse}
		\item List the $x$-coordinates of all critical points of $g$ where $g'(x)$ is \textbf{undefined}.
\WkstHop
			\begin{freeResponse}
				$x = -3$, $x=2$, and $x=4$.
			\end{freeResponse}
		\item List the $x$-coordinates of all local maximums of $g$.
\WkstHop
			\begin{freeResponse}
				$x = -3$ and $x=4$.
			\end{freeResponse}
		\item List the $x$-coordinates of all local minimums of $g$.
\WkstHop
			\begin{freeResponse}
				$x=2$
			\end{freeResponse}		
		\item List all intervals where $g$ is both decreasing AND concave down.
\WkstHop
			\begin{freeResponse}
				$(0,2)$ and $(4,6)$.
			\end{freeResponse}			
		\item List all intervals where $g$ is both decreasing AND concave up.
\WkstHop
			\begin{freeResponse}
				$(-3,0)$ 
			\end{freeResponse}
		\item List all intervals where $g$ is both increasing AND concave down.
\WkstHop
			\begin{freeResponse}
				$(-6,-3)$
			\end{freeResponse}		
		\item List all intervals where $g$ is both increasing AND concave up.
\WkstHop
			\begin{freeResponse}
				$(2,4)$
			\end{freeResponse}		
		\item List the $x$-coordinates of all inflection points of $g$.
\WkstHop
			\begin{freeResponse}
				$x=-3$, $x=0$, and $x=2$.
			\end{freeResponse}		
	\end{enumerate}
\end{problem}
	
\WkstNew



%problem 2
\begin{problem}
For each point in the interval (a,b)  and identified on the graph below, determine if the function $f$ has a critical point, a local max or min at that point.\\


	\begin{center}
	\begin{image}
	\includegraphics[trim= 20 430 250 200]{Figure3.pdf}
	\end{image}
	\end{center}	
\WkstHop
	
		\begin{freeResponse}


(p) The function $f$ has a critical point and a local maximum at $x=p$.  

(q) The function $f$ has a critical point and a local minimum at $x=q$.

(r) The function $f$ has a critical point and a local maximum at $x=r$

(s) The derivative of $f$ does not exist at $x=s$ because the function $f$ has a corner at $x=s$.  The function has a critical point and a local minimum at $x=s$.

(t) The function $f$ has a critical point but not a local maximum or minimum at $x=t$.


		\end{freeResponse}	
		
\end{problem}

\WkstNew

%problem8
\begin{problem}
Locate the critical points and use the second derivative test to determine whether they correspond to local maxima or local minima. \textbf{EXPLAIN}.

$f(x)=(x+c)^4$ where $c$ is a positive constant 
\WkstHop

\begin{freeResponse}

$$f'(x)=4(x+c)^3, f''(x)=12(x+c)^2$$

To find the critical points: $f'(x)=0=4(x+c)^3$.  This occurs when $x=-c$
 
Using the second derivative test: $f''(-c)=12(-c+c)^2=0$.  The second derivative test was inconclusive so we need to use the first derivative test.

Making a sign chart we see that if we plug something a little less than $-c$ into $f'$, we'll get a negative number.  If we plug something a little greater than $-c$ into $f'$, we'll get a positive number. 

      \begin{image}
        \includegraphics[scale = 0.5]{figure4a.png}
             \end{image}
The derivative of $f$, $f'$, does not change sign on the intervals $(-\infty,-c)$ and on $(-c,\infty)$, therefore, we have the above chart.  From the chart, we can conclude that $f$ has a local minimum at $x=-c$, and that $f$ also has a global minimum there.  We could also have used our understanding of the family of functions $g(x)=x^4$ to reason $x=-c$ would be a minimum.

At $x=-c$, $f$ has a local minimum by the first derivative test since $x=-c$ is a critical point of $f$ and the sign of $f'$ changes from negative to positive around $x=-c$.

\end{freeResponse}
\end{problem}
\WkstNew



\begin{problem}
Let $f$ be a  function that is continuous on its domain $(-6,4)$.
 The graph of  $f'$, the derivative of $f$, is given below.
	%\begin{center}
	%\begin{image}
	%\includegraphics[scale=0.7]{Figure14a.png}
	%\end{image}
	%\end{center}	
				\begin{center}
					\begin{tikzpicture}
						\begin{axis}[
							xmin=-6.5, xmax=4.5, ymin=-2.5,ymax=5.5,    
							axis lines =middle, 
							every axis y label/.style={at=(current axis.above origin),anchor=south},
							every axis x label/.style={at=(current axis.right of origin),anchor=west},
							xtick={-6,...,4}, ytick={-2,...,5},
							grid=major, width=4in,
							grid style={dashed, gray!40} 
							]
							\addplot[color=blue, very thick, smooth, domain=-6:0]{x+4}  node[anchor=south east,pos=0.75]{$\LARGE{y=f'(x)}$};
							\draw[fill=blue] (axis cs:-6,-2) circle [color=blue,radius=3pt] node [below right] {};
							\draw[fill=white] (axis cs:-6,-2) circle [color=white,radius=2pt] node [below right] {};
							\draw[fill=blue] (axis cs:0,4) circle [color=blue,radius=3pt] node [below right] {};
							\draw[fill=white] (axis cs:0,4) circle [color=white,radius=2pt] node [below right] {};
						
							\addplot[color=blue, very thick, smooth, domain=0:4]{(x-2)^2-1};
							
							\draw[fill=blue] (axis cs:0,3) circle [color=blue,radius=3pt] node [below right] {};
							\draw[fill=white] (axis cs:0,3) circle [color=white,radius=2pt] node [below right] {};

							\draw[fill=blue] (axis cs:4,3) circle [color=blue,radius=3pt] node [below right] {};
							\draw[fill=white] (axis cs:4,3) circle [color=white,radius=2pt] node [below right] {};
						
						
						
						\end{axis}
					\end{tikzpicture}
				\end{center}

	Based on the graph of $f'$, answer  the question below.\\
	\begin{enumerate}
		\item List x-coordinates of all critical points of $f$.
		\WkstHop
			\begin{freeResponse}
			Notice, $f'(-4)=f'(1)=f'(3)=0$, and $f'(0)$ is not defined. Therefore, the function $f$ has critical points at
			$x=-4$, $x=0$,  $x=1$ and  $x=3$.  
			\end{freeResponse}
		
		\item List x-coordinates of all critical points of $f$ where $f'(x)=0$.\\ 
		\WkstHop
			\begin{freeResponse}
			$x=-4$, $x=1$ and  $x=3$.  
			\end{freeResponse}
		\item List x-coordinates of all critical points of $f$ where $f'(x)$ is \textbf{undefined}.
		\WkstHop
			\begin{freeResponse}
			$x=0$  
			\end{freeResponse}
		\item List x-coordinates of all local minimums of $f$. 
		\WkstHop
			\begin{freeResponse}
			$x=-4$ and $x=3$. 
			At these points the sign of $f'(x)$ changes from negative to positive. 
			\end{freeResponse}
		\item List x-coordinates of all local maximums of $f$. 
		\WkstHop
			\begin{freeResponse}
			 $x=1$
			 At this point the sign of $f'(x)$ changes from positive to  negative. 
			\end{freeResponse}
		\item List all intervals where $f$ is decreasing and concave down.
		\WkstHop
			\begin{freeResponse}
			$(1,2)$  
			Notice, $f'(x)$ is negative and decreasing there. 
			\end{freeResponse}
		\item List all intervals where $f$ is decreasing and concave up.
		\WkstHop
			\begin{freeResponse}
			$(-6,-4)$ and $(2,3)$.  
			Notice, $f'(x)$ is negative and increasing there. 
			\end{freeResponse}
		\item List all intervals where $f$ is increasing and concave down.
		\WkstHop
			\begin{freeResponse}
			$(0,1)$ 
			Notice, $f'(x)$ is positive and decreasing there. 
			\end{freeResponse}
		\item List all intervals where $f$ is increasing and concave up.
		\WkstHop
			\begin{freeResponse}
			$(-4,0)$ and $(3,4)$.
			Notice, $f'(x)$ is positive and increasing there. 
			\end{freeResponse}
		\item List x-coordinates of all inflection points of $f$. 
		\WkstHop
			\begin{freeResponse}
			$x=0$, and $x=2$  $f'(x)$ is increasing on $(-6,0)$ and decreasing on $(0,2)$, then increasing on $(2,4)$.
			\end{freeResponse}
	\end{enumerate}

\end{problem}
	
\WkstNew


%problem 3
\begin{problem}
Find the critical points of $f$ on the given interval. Determine whether the function $f$ has a local maximum, local minimum or no local extremum at each critical point. \textbf{EXPLAIN}.
		\begin{enumerate}
		
			%part a
			\item  $f(x) = x \sqrt{2-x^2}$ on $( -\sqrt{2}, \sqrt{2} )$.
			\WkstHop
			
				\begin{freeResponse}
								\begin{align*}
				f'(x) &= \sqrt{2-x^2} + x \left( \frac{1}{2 \sqrt{2-x^2}} (-2x) \right) \\
				&= \sqrt{2-x^2} - \frac{x^2}{\sqrt{2-x^2}} \\
				&= \frac{2-x^2-x^2}{\sqrt{2-x^2}} \\
				&= \frac{2(1-x^2)}{\sqrt{2-x^2}}
				\end{align*}
				
				Critical points of $f$ occur where $f'(x) = 0$ or where $f'(x)$ does not exist.  Solving $f'(x) = 0$ yields that $2(1-x^2) = 0$, or $x = \pm 1$.  $f'(x)$ does not exist when $2-x^2 \leq 0$, but the points $x=\pm \sqrt{2}$ are not in the domain.\\  

				Let us use the Second Derivative Test to determine whether the function $f$ has a local maximum or local minimum at the points where $x=\pm1$.
				\begin{align*}
				f''(x) &=\frac{d}{dx}\frac{2(1-x^2)}{\sqrt{2-x^2}}
				= \frac{2x(x^2-3)}{(2-x^2)^{\frac{3}{2}}}
				\end{align*}
				Since $f''(-1)=4$, the point $(-1,-1)$ is a local minimum by the Second Derivative Test.
				Since $f''(1)=-4$, the point $(1,1)$ is a local maximum by the Second Derivative Test.
								\end{freeResponse}
				
				
				
			%part b
			\item  $f(x) = x^3 e^{-x}$ on $(-1,5)$.
			\WkstHop

				\begin{freeResponse}
			
				\begin{align*}
				f'(x) &= 3x^2 e^{-x} + x^3(-e^{-x}) \\
				&= x^2 e^{-x} (3-x)
				\end{align*}
				
				Notice that $f'(x)$ always exists, and so all of the critical points of $f$ occur when $f'(x)=0$.  Solving this equation:
				$$ x^2 e^{-x} (3-x) = 0 $$
				$$ x^2 (3-x) = 0 $$
				$$ x = 0 \qquad \text{or} \qquad x=3 $$
				It is easier to use the First derivative test in order to determine whether the function $f$ has a local maximum, local minimum or no local extremum at the point where $x=0$. Since $-1<0<1$, $f'(-1)=4e>0$, and $f'(1)=2e^{-1}>0$, it follows that the sign of $f'$ does not change at $x=0$. The function $f$ has \textbf{no local extremum} at $x=0$.\\

				Similarly, we will use the First derivative test in order to determine whether the function $f$ has a local maximum, local minimum or no local extremum at the points where $x=3$. Since $1<3<4$, $f'(1)=2e^{-1}>0$, and  $f'(4)=-16e^{-4}<0$ it follows that the sign of $f'$ changes from positive to negative at $x=3$. 
				The function $f$ has \textbf{a local maximum} at $x=3$ by the First Derivative Test.
 
						
				\end{freeResponse}
				
			\WkstNew	
				
			%part c
			\item  $f(x) = x \ln \left( \frac{x}{5} \right)$ on $(0, 5)$.
			\WkstHop

				\begin{freeResponse}
								\begin{align*}
				f'(x) &= \ln \left( \frac{x}{5} \right) + x \cdot \frac{5}{x} \cdot \frac{1}{5} \\
				&= \ln \left( \frac{x}{5} \right) + 1
				\end{align*}
				
				Notice that $f'(x)$ exists for all values in $(0,5)$, and so all of the critical points of $f$ occur when $f'(x)=0$.  Solving this equation:
				$$ \ln \left( \frac{x}{5} \right) + 1 = 0 $$
				$$ \ln \left( \frac{x}{5} \right) = -1 $$
				$$ \frac{x}{5} = e^{-1} $$
				$$ x = 5e^{-1} = \frac{5}{e} $$
						\\
				Let's use the Second Derivative Test to determine whether the function $f$ has a local maximum, local minimum or no local extremum at the points where $x= \frac{5}{e} $.
				\begin{align*}
				f''(x) &=\frac{d}{dx}( \ln \left( \frac{x}{5} \right) + 1)\\
				&=\frac{d}{dx}( \ln \left( x \right) -\ln \left( 5 \right)+ 1)
				= \frac{1}{x}
				\end{align*}
				Since, $f''\Bigl( \frac{5}{e}\Bigr)>0$, the point $\Bigl( \frac{5}{e},- \frac{5}{e}\Bigr)$ is a local minimum by the Second Derivative Test.
						
		
				\end{freeResponse}
				
				
				
			\end{enumerate}

		
		
		

\end{problem}
	
\WkstNew	

\begin{problem}
Let $f(x) = \frac{1}{1 + x^2}$.  Find the following for $f$:

	\begin{enumerate}
	%part a
	\item  $f'$ and $f''$
\WkstHop
	
		\begin{freeResponse}
			\begin{align*}
			f'(x) &= \frac{(1+x^2)(0) - 1(2x)}{(1+x^2)^2} \\
			&= \frac{-2x}{(1+x^2)^2}
			\end{align*}
			
			\begin{align*}
			f''(x) &= \frac{(1+x^2)^2(-2) - (-2x)(2)(1+x^2)(2x)}{(1+x^2)^4} \\
			&= \frac{-2(1+x^2) + 8x^2}{(1+x^2)^3} \\
			&= \frac{6x^2 - 2}{(1+x^2)^3}
			\end{align*}
		\end{freeResponse}
	%part b	
	\item  Critical points
\WkstHop
	
		\begin{freeResponse}
		Since $1+x^2 > 0$ for all $x$, $f$ is differentiable over all real numbers.  Thus all critical points of $f$ occur when $f'(x) = 0$.  But a fraction equals 0 if and only if its numerator equals 0.  So
		$$ f'(x) = 0 \qquad \Longrightarrow \qquad -2x = 0 \qquad \Longrightarrow \qquad x=0 $$
		Hence, the only critical point is $x=0$.  
		\end{freeResponse}

\WkstNew		
	%part d	
	\item  Local extrema (and check your answers with both the first and second derivative tests)
\WkstHop
	
		\begin{freeResponse}
		At $x=0$, $f'$ changes sign from positive to negative.  Thus $f$ goes from increasing to decreasing, and therefore by the first derivative test $x=0$ is a local maximum of $f$.  
		
		For the second derivative test, we have that
		$$ f''(0) = \frac{6(0)^2 - 2}{(1+0^2)^3} = \frac{-2}{1} = -2 < 0 $$
		and thus we again conclude that $x=0$ is a local maximum of $f$.
		\end{freeResponse}

	%part f	
	\item  Inflection points. \textbf{EXPLAIN}.
\WkstHop
	
		\begin{freeResponse}
		By the results in part 
		\begin{align*}
 		f''(x) &= \frac{6x^2 - 2}{(1+x^2)^3}\\
		 &= \frac{6(x^2 - \frac{1}{3})}{(1+x^2)^3}\\
		 &= \frac{6(x - \frac{1}{\sqrt{3}})(x + \frac{1}{\sqrt{3}})}{(1+x^2)^3}\\
		\end{align*}
		  which implies that the only candidates for inflection points are points where $x=\pm\frac{1}{\sqrt{3}}$.
		On the other hand, since $-1<- \frac{1}{\sqrt{3}}<0$, $f''(-1)=\frac{1}{2}$, and $f''(0)=-2$, it follows that $f''$ changes the sign at  $x=-\frac{1}{\sqrt{3}}$. Similarly, since $0<x=\frac{1}{\sqrt{3}}<1$, $f''(0)=-2$, $f''(1)=\frac{1}{2}$, it follows that $f''$ changes the sign at  $x=\frac{1}{\sqrt{3}}$, too.
		Since,
		\begin{align*}
 		 & f\left( \frac{1}{\sqrt{3}} \right)=\frac{3}{4} \\ 
 		& f\left( -\frac{1}{\sqrt{3}} \right)=\frac{3}{4} \\ 
		\end{align*}  
		The function $f$ has inflection points at $\left( -\frac{1}{\sqrt{3}},\frac{3}{4} \right),\left( \frac{1}{\sqrt{3}},\frac{3}{4} \right)$ because $f$ is continuous at those points and $f$ changes concavity around those points.
		\end{freeResponse}
		
	\end{enumerate}
		
\end{problem}

\WkstNew




\begin{problem}
Sketch a possible graph of a function $f$ that has the following properties:\\
	\begin{enumerate}
		\item $f$ is defined on the interval $(0,6)$.
		\item $f$ has no local maximums.
		\item $f$ has exactly two local minimums.
	\end{enumerate}
\WkstHop
	
		\begin{freeResponse}
		This could be the graph of $f$. Notice two local minimums: at $x=2$ and at $x=4$.
		%\begin{image}
		%\includegraphics{Figure 21.pdf}
		%\end{image}
			\begin{center}
					\begin{tikzpicture}
						\begin{axis}[
							xmin=-0.3, xmax=6.3, ymin=-0.3,ymax=5.3,    
							axis lines =middle, 
							every axis y label/.style={at=(current axis.above origin),anchor=south},
							every axis x label/.style={at=(current axis.right of origin),anchor=west},
							xtick={0,...,6}, ytick={0,...,5},
							grid=major, width=4in,
							grid style={dashed, gray!40} 
							]
							\addplot[color=blue, very thick, smooth, domain=0:4]{(x-2)^2};
							\addplot[color=blue, very thick, smooth, domain=4:6]{(x-6)^2};

							\draw[fill=blue] (axis cs:0,4) circle [color=blue,radius=3pt] node [below right] {};
							\draw[fill=white] (axis cs:0,4) circle [color=white,radius=2pt] node [below right] {};
							\draw[fill=blue] (axis cs:4,4) circle [color=blue,radius=3pt] node [below right] {};
							\draw[fill=white] (axis cs:4,4) circle [color=white,radius=2pt] node [below right] {};
						
							
							\draw[fill=blue] (axis cs:6,0) circle [color=blue,radius=3pt] node [below right] {};
							\draw[fill=white] (axis cs:6,0) circle [color=white,radius=2pt] node [below right] {};
							\draw[fill=blue] (axis cs:4,2) circle [color=blue,radius=3pt] node [below right] {};
						
						
						
						\end{axis}
					\end{tikzpicture}
				\end{center}

		\end{freeResponse}	
		
\end{problem}


\WkstNew


\begin{problem}
	Consider the parabola $f(x)=ax^2+bx+c$ where $a,b,c$ are constants.  For what values of $a,b,c$ is $f$ concave up?  For what values of $a,b,c$ is $f$ concave down?				
	\WkstHop
	
	\begin{freeResponse}
		\begin{align*}
			f'(x)&=2ax+b\\
			f''(x)&=2a
		\end{align*}
		This means the sign of $a$ will determine whether $f$ is concave up or down.  When $a<0$, $f$ is concave down, which makes sense 
		because then the graph of $f$ is a downward opening parabola.  When $a>0$, $f$ is concave up, which makes sense because then the 
		graph of $f$ is an upward openng parabola.  If $a=0$, $f$ has no concavity because it is a linear function.
						
	\end{freeResponse}
					
\end{problem}	




\end{document} 


















