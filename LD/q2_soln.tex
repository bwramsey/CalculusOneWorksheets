\documentclass{ximera}

\newcommand{\RR}{\mathbb R}
\renewcommand{\d}{\,d}
\newcommand{\dd}[2][]{\frac{d #1}{d #2}}
\renewcommand{\l}{\ell}
\newcommand{\ddx}{\frac{d}{dx}}
\newcommand{\dfn}{\textbf}
\newcommand{\eval}[1]{\bigg[ #1 \bigg]}
\renewcommand{\theenumii}{\textup{(\roman{enumii})}}
\renewcommand{\labelenumii}{\theenumii}

\usepackage{graphicx}
\usepackage{multicol}
\usepackage{tkz-euclide}
%\usepackage{unicode-math}

\usepackage{pgfplots}   % <- for graphics
\pgfplotsset{compat=newest}


\renewenvironment{freeResponse}{
\ifhandout\setbox0\vbox\bgroup\else
\begin{trivlist}\item[\hskip \labelsep\bfseries Solution:\hspace{2ex}]
\fi}
{\ifhandout\egroup\else
\end{trivlist}
\fi}

\newcommand*{\ZeroOverZero}{\ensuremath{\dfrac{0}{0}}}

\providecommand{\HCCondition}{0}
\newcommand{\WkstHop}[1][1]{\if\HCCondition 0
	\vspace*{\stretch{#1}} \fi} 
\newcommand{\WkstNew}{\if\HCCondition 0
	\newpage
	 \fi} 


\title[Problem 2]{Problem 2}

\begin{document}
\begin{abstract} \end{abstract}
\maketitle


% Extracted from logarithmicDifferentiation.tex, problem #2
\begin{problem}

  Find all real numbers $x$ which satisfy each of the following equations.
  \begin{enumerate}
    \item
      $\log_x(25) = 2$.
\begin{explanation}
        Recall that, by definition of the inverse to an exponential function,
        \[
        \log_b(x) = y \iff x = b^y.
        \]
        Using this relationship we have $\log_x(25) = 2 \iff x^2 = 25$.
        Therefore
        \begin{align*}
          x^2 = 25 &\implies x = \pm 5,\\
                   &\implies x = 5 \hspace{1em} \mbox{(base of log is always $>0$)}.
        \end{align*}
        Therefore $x = 5$ is the only solution to $\log_x(25) = 2$.
      \end{explanation}


    \item
      $7^x = 15$
\begin{explanation}
        Similar to the previous problem,
        \[
        \log_b(x) = y \iff x = b^y .
        \]
        Using this relationship we have $7^x = 15  \iff x = \log_{7}(15)$.
        Therefore $x = \log_7(15)$ is the only solution to $7^x = 15$.
      \end{explanation}
      
    \item
      $\ln(x) + 1 = 0$.
\begin{explanation}
        Similar to the previous two problems we'll use the relationship 
        \[
          \ln(x) = y \iff x = e^y.
        \]
        Before applying this relationship we perform a bit of algebra first:
        \[
          \ln(x) + 1 = 0 \implies \ln(x) = -1.
        \]
        Now we have $\ln(x) = -1 \iff x = e^{-1}$.
        Therefore $x = e^{-1} (= 1/e)$ is the only solution to $\ln(x) + 1 = 0$.
      \end{explanation}
  \end{enumerate}
\end{problem}



\end{document}
