%Add code to compile both versions from makefile at same time
\providecommand{\HCCondition}{0}
%Define each of the conditions
\ifcase\HCCondition
	%\condition=0 -> handout
	\documentclass[nooutcomes,noauthor,space,handout]{ximera}
	\title{Logarithmic differentiation (LD)}  
\or	%\condition=1 -> Soln
	\documentclass[nooutcomes,noauthor]{ximera}
	\title{Logarithmic differentiation (LD) - Solutions}
\fi


\usepackage{fullpage}
\newcommand{\RR}{\mathbb R}
\renewcommand{\d}{\,d}
\newcommand{\dd}[2][]{\frac{d #1}{d #2}}
\renewcommand{\l}{\ell}
\newcommand{\ddx}{\frac{d}{dx}}
\newcommand{\dfn}{\textbf}
\newcommand{\eval}[1]{\bigg[ #1 \bigg]}
\renewcommand{\theenumii}{\textup{(\roman{enumii})}}
\renewcommand{\labelenumii}{\theenumii}

\usepackage{graphicx}
\usepackage{multicol}
\usepackage{tkz-euclide}
%\usepackage{unicode-math}

\usepackage{pgfplots}   % <- for graphics
\pgfplotsset{compat=newest}


\renewenvironment{freeResponse}{
\ifhandout\setbox0\vbox\bgroup\else
\begin{trivlist}\item[\hskip \labelsep\bfseries Solution:\hspace{2ex}]
\fi}
{\ifhandout\egroup\else
\end{trivlist}
\fi}

\newcommand*{\ZeroOverZero}{\ensuremath{\dfrac{0}{0}}}

\providecommand{\HCCondition}{0}
\newcommand{\WkstHop}[1][1]{\if\HCCondition 0
	\vspace*{\stretch{#1}} \fi} 
\newcommand{\WkstNew}{\if\HCCondition 0
	\newpage
	 \fi}  %% we can turn off input when making a master document

  

\begin{document}
\begin{abstract}		\end{abstract}
\maketitle
\ifcase\HCCondition
%summary in here
\section*{Basic properties of the natural logarithm:}
\begin{itemize}
	\item $\displaystyle \ln\left( yz \right) = \ln(y) + \ln(z)$
	\item $\displaystyle \ln\left( \dfrac{y}{z} \right) = \ln(y) - \ln(z)$
	\item $\displaystyle \ln\left( y^z \right) = z\ln(y)$
   	 \item For any function $g$: $\displaystyle \ln( e^{g(x)} ) = g(x)$
	\item $\displaystyle \ddx \left( \ln(f(x) \right) = \dfrac{f'(x)}{f(x)}$.
\end{itemize}

\newpage

\section*{Recitation Questions}
\fi




\begin{problem}	
	\begin{enumerate}
		\item Write as an exponential with base $5$.   $\displaystyle 7^{3x}$.
\WkstHop
		\begin{freeResponse}
				$7 = 5^{\log_5(7)}$, so \[ 7^{3x} = \left( 5^{\log_5(7)} \right)^{3x} = 5^{3x \log_5(7)} \]
			\end{freeResponse}	
		\item Write in terms of the natural logarithm.  $\displaystyle \log_3\left( 4\right)$.
\WkstHop
			\begin{freeResponse}
				The change of base formula says $\displaystyle \log_a(u) = \frac{\log_b(u)}{\log_b(a)}$, so \[ \log_3(4) = \frac{\ln(4)}{\ln(3)} \]
			\end{freeResponse}
		\item Expand the following:  $\displaystyle \log_{1/2} \left( \frac{6x^5(2+\tan(x))^x}{\sqrt[5]{e^{4x}+1}} \right)$.
\WkstHop
			\begin{freeResponse}
				\[ \log_{1/2}\left( \frac{6x^5(2+\tan(x))^x}{\sqrt[5]{e^{4x}+1}} \right) = \log_{1/2}(6) + 5\log_{1/2}(x) + x\log_{1/2}(2+\tan(x)) - \frac{1}{5}\log_{1/2}(e^{4x}+1)\]
			\end{freeResponse}
	\end{enumerate}
\end{problem}	

\WkstNew

%problem 1
\begin{problem}

  Find all real numbers $x$ which satisfy each of the following equations.
  \begin{enumerate}
    \item
      $\log_x(25) = 2$.
\WkstHop

      \begin{freeResponse}
        Recall that, by definition of the inverse to an exponential function,
        \[
        \log_b(x) = y \iff x = b^y.
        \]
        Using this relationship we have $\log_x(25) = 2 \iff x^2 = 25$.
        Therefore
        \begin{align*}
          x^2 = 25 &\implies x = \pm 5,\\
                   &\implies x = 5 \hspace{1em} \mbox{(base of log is always $>0$)}.
        \end{align*}
        Therefore $x = 5$ is the only solution to $\log_x(25) = 2$.
      \end{freeResponse}


    \item
      $7^x = 15$
\WkstHop
      \begin{freeResponse}
        Similar to the previous problem,
        \[
        \log_b(x) = y \iff x = b^y .
        \]
        Using this relationship we have $7^x = 15  \iff x = \log_{7}(15)$.
        Therefore $x = \log_7(15)$ is the only solution to $7^x = 15$.
      \end{freeResponse}
      
    \item
      $\ln(x) + 1 = 0$.
\WkstHop
      \begin{freeResponse}
        Similar to the previous two problems we'll use the relationship 
        \[
          \ln(x) = y \iff x = e^y.
        \]
        Before applying this relationship we perform a bit of algebra first:
        \[
          \ln(x) + 1 = 0 \implies \ln(x) = -1.
        \]
        Now we have $\ln(x) = -1 \iff x = e^{-1}$.
        Therefore $x = e^{-1} (= 1/e)$ is the only solution to $\ln(x) + 1 = 0$.
      \end{freeResponse}
  \end{enumerate}
\end{problem}

\WkstNew

%problem7
\begin{problem}
  True or False:
  \begin{enumerate}
    \item[(1)]
      If $f(x) = (x-2)^x$, then $f'(x) = x (x-2)^{x-1}$.
\WkstHop
      \begin{freeResponse}
        False.
        Any time that you have a function of $x$ raised to a function of $x$, in order to compute the derivative you need to use logarithmic differentiation (or something equivalent).

        Correct derivative of $f$:
        \begin{align*}
          f(x) &= (x-2)^x \implies f(x) = e^{x \ln(x-2)} \\
          &\implies f'(x) = e^{x \ln(x-2)} \cdot \left(1\cdot\ln(x-2) + x \cdot \frac{1}{x-2}\right) \\
          &\implies f'(x) = (x-2)^x \cdot \left(\ln(x-2) + \frac{x}{x-2}\right)
        \end{align*}
        \end{freeResponse}	
		
      \item[(2)]
        If $f(x) = (3x)^x$, then $f'(x) = (3x)^x \ln (3x)$.
\WkstHop
	\begin{freeResponse}
          False.  Same as part (1).

          Correct derivative of $f$:
          \begin{align*}
            f(x) &= (3x)^x \implies f(x) = e^{x \cdot \ln(3x)} \\
            &\implies f'(x)  = e^{x \cdot \ln(3x)} \left(1 \cdot \ln(3x) + x \cdot \frac{3}{3x} \right) \\
            &\implies f'(x)  = (3x)^x \left(\ln(3x) + 1 \right)
          \end{align*}
	\end{freeResponse}	
	\end{enumerate}
\end{problem}

\WkstNew

%problem8
\begin{problem}
Find the derivatives of the following functions:
	\begin{enumerate}
	
	%part a
	\item  $f(x) = x^{e^x} + 7x$
\WkstHop
		\begin{freeResponse}
		$f'(x) = \ddx \left(x^{e^x} \right) + \ddx(7x) = \ddx \left(x^{e^x} \right) + 7$.  So the real problem is to find $\ddx \left(x^{e^x} \right)$.\\
		 
		 We will use logarithmic differentiation. \\
		 
		First, we take natural logarithm of both sides of the equation $g(x)=x^{e^x}$.
		 
		 $\ln{(g(x))}=e^{x}\ln(x)$.
		 
		 Now, we differentiate both sides.
		 
		  $\dfrac{g'(x)}{g(x)}=e^x\ln(x)+\dfrac{e^{x}}{x}$.
		  
		  Next, we solve for $g'(x)$.\\
		  
		    $g'(x)=g(x)\Bigl(e^x\ln(x)+\dfrac{e^{x}}{x}\Bigr)$.
		    
		    Now, we substitute.
		    
		     $g'(x)=x^{e^x}\Bigl(e^x\ln(x)+\dfrac{e^{x}}{x}\Bigr)$.

		  
		ALTERNATIVE APPROACH
		
		\begin{align*}
		\ddx \left( x^{e^x} \right) &= \ddx \left( e^{\ln(x)^{e^x}} \right) \\
		&= \ddx \left( e^{e^x \ln(x)} \right) \\
		&= e^{e^x \ln(x)} \left( e^x \ln(x) + \frac{e^x}{x} \right) \\
		&= x^{e^x} \left( e^x \ln(x) + \frac{e^x}{x} \right).
		\end{align*}
		
		Thus, $f'(x) = x^{e^x} \left( e^x \ln(x) + \frac{e^x}{x} \right) + 7$.  
		
		\end{freeResponse}

	\item  $g(x) = (\ln(x) + 9)^{\sec(x^4)}$\\
\WkstHop
	\begin{freeResponse}
	We will use logarithmic differentiation.\\
	
	First, we take natural logarithm .\\
	 
	 $\ln{(g(x))} = \sec(x^4)\ln{(\ln(x) + 9)}$\\
	 
	Then, we differentiate.\\
	
	 $\dfrac{g'(x)}{g(x)} = \sec(x^4)\tan(x^4)4x^3\ln{(\ln(x) + 9)}+\dfrac{\sec(x^4)}{x(\ln(x) + 9)}$\\
	 
	We  multiply by $g(x)$.\\
	
		 $g'(x) = g(x)\Bigl(4x^3\sec(x^4)\tan(x^4)\ln{(\ln(x) + 9)}+\dfrac{\sec(x^4)}{x(\ln(x) + 9)}\Bigr)$\\
		 
		 We substitute.
		 \\
		 
		  $g'(x) =(\ln(x) + 9)^{\sec(x^4)}\Bigl(4x^3\sec(x^4)\tan(x^4)\ln{(\ln(x) + 9)}+\dfrac{\sec(x^4)}{x(\ln(x) + 9)}\Bigr)$\\
		 
		 ALTERNATIVE APPROACH\\
		 
			\begin{align*}
			g'(x) &= \ddx \left( (\ln(x) + 9)^{\sec(x^4)} \right) \\
			&= \ddx \left( e^{\sec(x^4) \ln ( \ln(x) + 9) } \right) \\
			&= e^{\sec(x^4) \ln ( \ln(x) + 9)} \left( 4x^3 \sec(x^4) \tan(x^4) \ln(\ln(x) + 9) + \sec(x^4) \frac{\frac{1}{x}}{\ln(x) + 9} \right) \\
			&= (\ln(x) + 9)^{\sec(x^4)} \left( 4x^3 \sec(x^4) \tan(x^4) \ln(\ln(x) + 9) + \frac{\sec(x^4)}{x(\ln(x) + 9)} \right) .
			\end{align*}
		\end{freeResponse}
\WkstNew

\item  $f(x) = \frac{(x+1)^5(\sin(x)+5)^4}{(x^2 + 5)\sqrt{x-3}}$
\WkstHop
          \begin{freeResponse}
           \begin{align*}
            \ln{(f(x))} &= \ln{(x+1)^5}+\ln{(\sin(x)+5)^4}-\ln{(x^2 + 5)}-\ln{\sqrt{x-3}}\\
            &= 5\ln{(x+1)}+4\ln{(\sin(x)+5)}-\ln{(x^2 + 5)}-\dfrac{1}{2}\ln{(x-3)}\\
             \end{align*}
             Now we differentiate.\\
             
              $\dfrac{f'(x)}{f(x)} =\dfrac{5}{x+1}+\dfrac{4\cos(x)}{\sin(x)+5}-\dfrac{2x}{x^2+5}-\dfrac{1}{2(x-3)}$
              
              Then we solve for $f'(x)$.
              
              
                $f'(x) =f(x)\Bigl(\dfrac{5}{x+1}+\dfrac{4\cos(x)}{\sin(x)+5}-\dfrac{2x}{x^2+5}-\dfrac{1}{2(x-3)}\Bigr)$
                Then we substitute.\\
                
                   $f'(x) = \frac{(x+1)^5(\sin(x)+5)^4}{(x^2 + 5)\sqrt{x-3}}\Bigl(\dfrac{5}{x+1}+\dfrac{4\cos(x)}{\sin(x)+5}-\dfrac{2x}{x^2+5}-\dfrac{1}{2(x-3)}\Bigr)$
                
          \end{freeResponse}
	%part c
	\item  $h(x) = \frac{(x^2 - 7)^5}{\cos^7(x^2 - 5)}$
\WkstHop
          \begin{freeResponse}
            Since $h(x)$ changes sign on its domain, we write :
            \begin{align*}
              \ln |h(x)| &= \ln \left( \frac{|x^2 - 7|^5}{\ |cos^7(x^2 - 5)|}\right) \\
                       &= 5 \cdot \ln (|x^2 - 7|) - 7 \cdot \ln(|\cos(x^2-5)|)
            \end{align*}
  Differentiate both sides with respect to $x$, (explanation given on the next page):\\[1em]
             $ \dfrac{h'(x)}{h(x)} = 5 \dfrac{2x}{x^2 - 7}  - 7 \dfrac{-\sin(x^2-5)2x}{\cos(x^2 - 5)} $\\[1em]
 $ h'(x) =h(x)\bigl( 5 \dfrac{2x}{x^2 - 7}  - 7 \dfrac{-\sin(x^2-5)2x}{\cos(x^2 - 5)} \bigr)=$\\[1em]
 $  =\frac{(x^2 - 7)^5}{\cos^7(x^2 - 5)}\bigl(  \dfrac{10x}{x^2 - 7}  + \dfrac{14x\sin(x^2-5)}{\cos(x^2 - 5)} \bigr)=$\\[1em]
 $  =\frac{(x^2 - 7)^4}{\cos^7(x^2 - 5)}\bigl(  10x  + 14x(x^2-7) \tan(x^2-5) \bigr)$\\[1em]

EXPLANATION: The function $h(x)$ assumes negative values on some intervals. On those intervals\\[1em]

 $\bigl(\ln(|h(x)|)\bigr)'=\bigl(\ln(-h(x))\bigr)'=\dfrac{-h'(x)}{-h(x)}=\dfrac{h'(x)}{h(x)}$, \\[2em]
 Also,  e.g., if   \hspace{0.2in} $x^2-7<0$\\[1em]

$\bigl(\ln(|x^2-7|)\bigr)'=\bigl(\ln(-(x^2-7))\bigr)'=\dfrac{-2x}{-(x^2-7)}=\dfrac{2x}{x^2-7}$ .
\end{freeResponse}
	\end{enumerate}
\end{problem}
\end{document} 






Derivative of $h$:\\[2em]
         
              \ddx \ln h(x) &\implies \frac{h'(x)}{h(x)} = 5 \cdot \frac{1}{x^2 - 7} \cdot 2x - 7 \cdot\frac{1}{\cos(x^2 - 5)} \cdot - \sin(x^2-5) \cdot 2x\\
                            &= \frac{10x}{x^2-7} + \frac{14x \cdot \sin(x^2 - 5)}{\cos(x^2 - 5)}\\
              &= \frac{10x}{x^2-7} + 14x \tan(x^2-5) \\
              &\implies h'(x) = h(x)\cdot \left( \frac{10x}{x^2-7} + 14x \tan(x^2-5) \right) \\
              &\implies h'(x) = \frac{(x^2 - 7)^4}{\cos^7(x^2 - 5)} \cdot \left( 10x + 14x(x^2-7) \tan(x^2-5) \right)\\

    
	\end{freeResponse}
	\end{enumerate}
\end{problem}






\end{document} 


















