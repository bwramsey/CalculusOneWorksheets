%Add code to compile both versions from makefile at same time
\providecommand{\HCCondition}{0}
%Define each of the conditions
\ifcase\HCCondition
	%\condition=0 -> handout
	\documentclass[nooutcomes,noauthor,space,handout]{ximera}
	\title{Chain Rule (CR)}  
\or	%\condition=1 -> Soln
	\documentclass[nooutcomes,noauthor]{ximera}
	\title{Chain Rule (CR) - Solutions}
\fi

\usepackage{fullpage}

\newcommand{\RR}{\mathbb R}
\renewcommand{\d}{\,d}
\newcommand{\dd}[2][]{\frac{d #1}{d #2}}
\renewcommand{\l}{\ell}
\newcommand{\ddx}{\frac{d}{dx}}
\newcommand{\dfn}{\textbf}
\newcommand{\eval}[1]{\bigg[ #1 \bigg]}
\renewcommand{\theenumii}{\textup{(\roman{enumii})}}
\renewcommand{\labelenumii}{\theenumii}

\usepackage{graphicx}
\usepackage{multicol}
\usepackage{tkz-euclide}
%\usepackage{unicode-math}

\usepackage{pgfplots}   % <- for graphics
\pgfplotsset{compat=newest}


\renewenvironment{freeResponse}{
\ifhandout\setbox0\vbox\bgroup\else
\begin{trivlist}\item[\hskip \labelsep\bfseries Solution:\hspace{2ex}]
\fi}
{\ifhandout\egroup\else
\end{trivlist}
\fi}

\newcommand*{\ZeroOverZero}{\ensuremath{\dfrac{0}{0}}}

\providecommand{\HCCondition}{0}
\newcommand{\WkstHop}[1][1]{\if\HCCondition 0
	\vspace*{\stretch{#1}} \fi} 
\newcommand{\WkstNew}{\if\HCCondition 0
	\newpage
	 \fi}  %% we can turn off input when making a master document



\begin{document}
\begin{abstract}		\end{abstract}
\maketitle

\ifcase\HCCondition
%summary in here

\begin{itemize}
	\item \textbf{The Chain Rule}: If $f(x)$ and $g(x)$ are differentiable functions, then $\displaystyle \ddx f\left(g(x)\right) = f'\left( g(x)\right) g'(x)$.
		
	\item \textbf{The Derivatives of Trigonometric Functions}:
		\begin{align*}
			&\ddx\sin(x) = \cos(x)  \quad & \ddx \cos(x) = -\sin(x) \\
			&\ddx\tan(x) = \sec^2(x)  &\ddx \cot(x) = -\csc^2(x) \\
			&\ddx\sec(x) = \sec(x)\tan(x)  &\ddx \csc(x) = -\csc(x)\cot(x)
		\end{align*}
\end{itemize}
\WkstHop
\section*{Recitation Questions}
\fi



%problem1
\begin{problem}
For the following problems, the derivative is given.  Determine which function was the original function.
	\begin{enumerate}
	
		\item  The derivative is $f'(x) = \cos (x) e^{\sin (x)}$.  Which is the original function?
		
			\begin{enumerate}
			\item  $f(x) = (\sin(x))(e^x)$
			\item  $f(x) = \sin (e^x)$
			\item  $f(x) = e^{\sin(x)}$
			\item  $f(x) = e^{x \sin(x)}$

				\begin{freeResponse}
				Since
				$$ \ddx \left( e^{\sin(x)} \right) = e^{\sin(x)} \cdot \cos(x) = \cos(x) e^{\sin(x)} $$
				the correct answer is (iii).  
				\end{freeResponse}
				
			\end{enumerate}
			
			\WkstHop
			
		\item  The derivative is $g'(x) = 4 \left( \tan (x^4 - 5x) \right)^3 \sec^2(x^4-5x)(4x^3-5)$.  Which is the original function?
		
			\begin{enumerate}
			
			\item  $g(x) = \left( \tan(x) - 5x \right)^4$
			\item  $g(x) = \tan^4(x) - 5x^4$
			\item  $g(x) = \tan(x^4 - 5x)$
			\item  $g(x) = \tan^4(x^4-5x)$
			
				\begin{freeResponse}
				Since 
				$$\ddx \left( \tan^4(x^4-5x) \right) = 4 \tan^3(x^4-5x) \sec^2(x^4-5x) (4x^3-5)$$ 
				the correct answer is (iv).  Notice that $\tan^3(x^4-5x) = \left( \tan(x^4-5x) \right)^3$ are (slightly) different notations for the exact same expression.
				\end{freeResponse}
				
			\end{enumerate}
\WkstHop
	\end{enumerate}
\end{problem}	
	

\WkstNew


%problem 2
\begin{problem}
 A table of values for $f(x)$ and $f'(x)$ is shown below:

% replaced this png with a MathMode array
%\begin{image}[0.25\linewidth]
%	\includegraphics{figure1.png}
%\end{image}

$$\begin{array}{c|c|c}
	\hline
	& & \\
	x & f(x) & f'(x) \\
	& & \\
	\hline
	& & \\
	1 & 3 & 4\\
	2 & 2 & 3\\
	3 & 4 & 5\\
	4 & 6 & 3\\
	\hline
\end{array} $$

	\begin{itemize}
		\item	Evaluate the limit $\displaystyle \lim_{x \to 2} \dfrac{f(x^2)-6}{x-2}$. \textbf{EXPLAIN}.
\WkstHop
			\begin{freeResponse}
				If we set $g(x) = f(x^2)$, this limit is $\displaystyle \lim_{x\to 2} \dfrac{g(x)-g(2)}{x-2}$. In other words, we're being asked to
				evaluate $\displaystyle\ddx f(x^2)$ at $x = 2$. By the Chain Rule, $\displaystyle\ddx f(x^2) = 2x f'(x^2)$. When evaluated at $x=2$
				this is $2(2) f'(4) = 4 (3) = 12$. The value of this limit is $12$.
				
				Our full explanation is then:
				$\displaystyle \lim_{x\to 2} \dfrac{f(x^2)}{x-2} = \eval{\ddx f(x^2) }_{x=2}$ by the limit definition of the derivative. Our calculation above shows that this equals $12$.
			  \end{freeResponse}	

	  	\item Evaluate $\displaystyle\ddx f(f(x))$ at $x = 3$.
		  \begin{enumerate}
			    \item 6
			    \item 25
			    \item 5
			    \item 15
			    \item DNE
			    \item None of the previous answers.
			  \end{enumerate}
\WkstHop
			  \begin{freeResponse}
			    The answer is (d):
			    \begin{align*}
			      \eval{\ddx f(f(x))}_{x = 3} &= f'(f(3)) \cdot f'(3) \\
			      &= f'(4)\cdot 5 \\
			      &= 3 \cdot 5 = 15
			    \end{align*}
			  \end{freeResponse}
	\end{itemize}

\end{problem}

\WkstNew

%problem 3
\begin{problem}
  Given the following graphs of $f$ and $g$ (both piecewise
  linear functions), define new functions $u(x) = f(g(x))$ and
  $v(x) = f(x)g(x)$.  Find:

% image replaced with tikzpic below
%  \begin{image}[0.4\linewidth]
%    \includegraphics[trim= 170 420 250 230]{Figure7.pdf}
%  \end{image}

	  \begin{center}
		\begin{tikzpicture}
			\begin{axis}[
				xmin=-2.3, xmax=7.3, ymin=-1.5,ymax=5.5,    
				axis lines =middle, 
				every axis y label/.style={at=(current axis.above origin),anchor=south},
				every axis x label/.style={at=(current axis.right of origin),anchor=west},
				xtick={-2,...,7}, ytick={-1,...,5},
				grid=major, width=4in, height = 3in,
				grid style={dashed, gray!40}
				]
				\path[draw, color=blue, very thick] (axis cs:-3,1) --  (axis cs:0,0) -- (axis cs:2,4) node[pos=0.75, below right]{\large{$f$}}  -- (axis cs: 8,2);
				\path[draw, color=red, very thick] (axis cs:-3,5) --  (axis cs:2,0) -- (axis cs:8,4) node[pos=0.5, below right]{\large{$g$}};	
			\end{axis}
		\end{tikzpicture}
	\end{center}


  \begin{enumerate}	
  \item $u'(1)$
\WkstHop
	\begin{freeResponse}
      $u'(x) = \ddx(f(g(x))) = f'(g(x)) \cdot g'(x)$.  So,
      \begin{align*}
        u'(1) &= f'(g(1)) \cdot g'(1) \\
              &= f'(1) \cdot (-1) \\
              &= (2)(-1) = -2 
      \end{align*}  
    \end{freeResponse}
		
		
		
	
  \item $v'(1)$
\WkstHop
	\begin{freeResponse}
      $v'(x) = \ddx(f(x)g(x)) = f'(x)g(x) + f(x)g'(x)$.  So,
      \begin{align*}
        v'(1) &= f'(1) g(1) + f(1) g'(1) \\
              &= (2)(1) + (2)(-1) \\
              &= 0
      \end{align*}
    \end{freeResponse}

  \item $\displaystyle \lim_{x\to 1} \dfrac{\sqrt{g(x)}-1}{x-1}$
\WkstHop
	\begin{freeResponse}
	This is asking for $\eval{\ddx( \sqrt{g(x)} )}_{x=1}$. 
	By the Chain Rule, $\ddx( \sqrt{g(x)} ) = \dfrac{g'(x)}{2\sqrt{g(x)}}$ so 
	\begin{align*}
		\eval{\ddx( \sqrt{g(x)} )}_{x=1} &= \dfrac{g'(1)}{2\sqrt{g(1)}}\\
			&= \dfrac{-1}{2\sqrt{1}}\\
			&= -\dfrac{1}{2}.
	\end{align*}
    \end{freeResponse}
  \end{enumerate}		
			
	
		
\end{problem}
	
	
\WkstNew
		
			

%problem 4		
\begin{problem}
Suppose the line tangent to the graph of $f$ at $x=1$ is $y=6x-7$.  Find an equation of the line tangent to the following curves at $x=1$:
	\begin{enumerate}
	
	\item  $y=g(x) = 5(f(x))^4$  
\WkstHop
		\begin{freeResponse}
		First, in the equation of the tangent line to $f(x)$ at $x=1$, when $x=1$ we have that $y= 6(1) - 7 = -1$.  Thus $f(1) = -1$ and therefore $g(1) = 5(-1)^4 = 5$.  Hence, a point on our line is $(1,5)$.  Also, since the slope of the given tangent line is $m=6$, we know that $f'(1) = 6$.
		
		Now, by the chain rule we have that $g'(x) = 5 \cdot 4 (f(x))^3 \cdot f'(x)$.  So $g'(1) = 20(f(1))^3 \cdot f'(1) = 20(-1)^3 (6) = -120.$  Thus, the equation of the line tangent to the graph of $y = g(x)$ at $x=1$ is
		$$ y - 5 = -120(x-1) $$
		$$ y = -120x + 125 $$
		\end{freeResponse}

	\item  $y=h(x) = x^2 (f(x^3))$
\WkstHop
		\begin{freeResponse}
		$h(1) = 1^2 \cdot f(1^3) = f(1) = -1$.  By the product and chain rules:
		$$ h'(x) = 2x(f(x^3)) + x^2(f'(x^3) \cdot 3x^2) = 2x(f(x^3)) + 3x^4(f'(x^3)) $$
		Thus, $h'(1) = 2f(1) + 3f'(1) = 2(-1) + 3(6) = -2 + 18 = 16$.  So, the equation of the line tangent to the graph of $h$ at $x=1$ is
		$$ y- (-1) = 16(x-1) $$
		$$ y = 16x - 17 $$
		\end{freeResponse}
	\end{enumerate}		
		
		
	
\end{problem}

\WkstNew



%problem 5		
\begin{problem}
Differentiate each function (with respect to $x$)
	\begin{enumerate}
	
	%part a
	\item  $\cos \left( \sqrt{x+7} \right) $
		\begin{freeResponse}
		\begin{align*}
		\ddx \left( \cos \left( \sqrt{x+7} \right) \right) &= -\sin \left( \sqrt{x+7} \right) \cdot \frac{1}{2} (x+7)^{\frac{-1}{2}} (1) \\
		&= \frac{-\sin \left( \sqrt{x+7} \right) }{2 \sqrt{x+7}}
		\end{align*}
		\end{freeResponse}
\WkstHop

	%part b
	\item  $\sqrt{ \cos(x) + 7}$
		\begin{freeResponse}
		\begin{align*}
		\ddx \left( \sqrt{ \cos(x) + 7} \right) &= \frac{1}{2} \left( \cos(x) + 7 \right)^{\frac{-1}{2}} \left( -\sin(x) \right) \\
		&= \frac{- \sin(x) }{2 \sqrt{\cos(x) + 7 }}
		\end{align*}
		\end{freeResponse}
\WkstHop

	%part c
	\item $\sqrt{\cos(x)} + 7$
		\begin{freeResponse}
		\begin{align*}
		\ddx \left( \sqrt{\cos(x)} + 7 \right) &=  \frac{1}{2} \left( \cos(x) \right)^{\frac{-1}{2}} \left( -\sin(x) \right) + 0  \\
		&= \frac{- \sin(x)}{2 \sqrt{\cos(x)}}
		\end{align*}
		\end{freeResponse}
\WkstHop

	%part d
	\item  $\cos \left( \sqrt{x} + 7 \right)$
		\begin{freeResponse}
		\begin{align*}
		\ddx \left( \cos \left( \sqrt{x} + 7 \right) \right) &= - \sin \left( \sqrt{x} + 7 \right) \cdot \frac{1}{2} x^{\frac{-1}{2}} \\
		&=  \frac{- \sin \left( \sqrt{x} + 7 \right) }{2 \sqrt{x}}
		\end{align*}
		\end{freeResponse}
\WkstHop

		\item  $\cos(x) \cdot\left( \sqrt{x} + 7 \right) $
		\begin{freeResponse}
		\begin{align*}
		\ddx\left(\cos(x)\cdot \left( \sqrt{x} + 7 \right) \right) &= - \sin(x)\cdot \left( \sqrt{x} + 7 \right)+\cos(x)\cdot\frac{1}{2} x^{\frac{-1}{2}} \\
		&= - \sin(x)\cdot \left( \sqrt{x} + 7 \right)+\frac{\cos(x)}{2\sqrt{x}}
		\end{align*}
		\end{freeResponse}
\WkstHop
	\end{enumerate}
		
		
		
		
\end{problem}
\WkstNew

\begin{problem}
Find the derivative of the following functions:

	\begin{enumerate}
	\item  $f(x) = \sin(x) \cos(x)$
			\begin{freeResponse}
			$$f'(x) = (\cos(x))(\cos(x)) + (\sin(x))(-\sin(x)) = \cos^2(x) - \sin^2(x).$$
			\end{freeResponse}			
\WkstHop

	\item  $f(x) = \frac{e^x \tan(x)}{\sec(x) + 2}$
			\begin{freeResponse}
			\begin{align*}
			f'(x) &= \frac{(\sec(x)+2)(e^x \tan(x) + e^x \sec^2(x)) - e^x \tan(x) (\sec(x) \tan(x))}{(\sec(x) + 2)^2}  \\
			&= \frac{e^x[(\sec(x) + 2)(\tan(x) + \sec^2(x)) - \sec(x) \tan^2(x)]}{(\sec(x) + 2)^2}.
			\end{align*}
			\end{freeResponse}
\WkstHop			
			\item  $f(x) = e^{x \tan(x)}$
			\begin{freeResponse}
			\[	
			f'(x) = e^{x \tan(x)}\left(\tan{(x)}+x\sec^{2}(x)\right) 
		         \]
			\end{freeResponse}			
\WkstHop
\WkstNew
		\item  $f(x) = \sin(x) \cos(x) e^{3x}$
			\begin{freeResponse}
			\begin{align*}
			f'(x) &= \ddx[\sin(x) \cos(x)] e^{3x} + \left(\sin(x) \cos(x)\right) \ddx(e^{3x})  \\
			&= \left(\cos^2(x) - \sin^2(x)\right)e^{3x} + 3e^{3x} \sin(x) \cos(x)  \\
			&= e^{3x}\left(\cos^2(x) + 3\sin(x) \cos(x) - \sin^2(x)\right).
			\end{align*}
			\end{freeResponse}
\WkstHop

	\item  $f(x) = \frac{x+5}{7x^6 + \cot(x)}$
			\begin{freeResponse}
			\begin{align*}
			f'(x) &= \frac{(7x^6 + \cot(x))(1) - (x+5)(42x^5 - \csc^2(x))}{(7x^6 + \cot(x))^2}  \\
			&= \frac{7x^6 + \cot(x) - (x+5)(42x^5 - \csc^2(x))}{(7x^6 + \cot(x))^2}.
			\end{align*}
			\end{freeResponse}	
\WkstHop

	\item  $f(x) = \sin{\left(2x\right)}\sec^{3}{\left(x^2+4x\right)}$
			\begin{freeResponse}
			\begin{align*}
			f'(x) &=2\cos{\left(2x\right)}\sec^{3}{\left(x^2+4x\right)}+\sin{\left(2x\right)}3\sec^{2}{\left(x^2+4x\right)} \sec{\left(x^2+4x\right)}\tan{\left(x^2+4x\right)}\left(2x+4\right)  \\
			&=2\cos{\left(2x\right)}\sec^{3}{\left(x^2+4x\right)}+3\sin{\left(2x\right)}\sec^{3}{\left(x^2+4x\right)} \tan{\left(x^2+4x\right)}\left(2x+4\right)  .
			\end{align*}
			\end{freeResponse}
\WkstHop
	\end{enumerate}
		
\end{problem}	
\WkstNew

\begin{problem}
Find values for $a$, $b$, and $c$ so that the following function is differentiable everywhere.

\[f(x) =   \begin{cases}
	a \sin(x) + b \cos(x)		 	&	\qquad \text{if } x < 0					\\
	ax^2 + bx + c   				&	\qquad \text{if } x \geq 0	 \end{cases}   \]

	\WkstHop

	\begin{freeResponse}
		The function $f$ is differentiable for $x<0$ since $f$ is a combination of trigonometric functions.  $f$ is also differentiable $x>0$ since $f$ is a polynomial on that interval.  We need to focus on $x=0$.  Since $f$ is differentiable everywhere, $f$ must also be differentiable at $x=0$ and therefore continuous at $x=0$.  Therefore: \\\\
We need that $\lim_{x \to 0^-} f(x) = \lim_{x \to 0^+} f(x)=\lim_{x \to 0} f(x)=f(0)$.  Observe that
		
		\begin{itemize}
		
		\item $\lim_{x \to 0^-} f(x) 
		= \lim_{x \to 0^-} (a\sin(x) + b\cos(x))
		= b(1) = b$.
		
		\item  $ \lim_{x \to 0^+} f(x)
		= \lim_{x \to 0^+} (ax^2 + bx + c)
		= c$.
		
		\end{itemize}
		
		Thus, we must have that $b = c$\\\\
		Next, if $f'(0)$ exists, then:
		\begin{align*}
		f'(0)&=\lim_{x\to 0} \frac{f(x)-f(0)}{x-0}\\\\
		& \text{since}\ f(0)=a \cdot 0^2+b \cdot 0 +c =c\\\\
		& \lim_{x\to 0} \frac{f(x)-f(0)}{x-0}=\lim_{x\to 0} \frac{f(x)-c}{x}
		\end{align*}
		Since this limit exists, the left and right limits must be equal.
		
		\begin{align*}
		\lim_{x\to 0^+} \frac{f(x)-c}{x}&=\lim_{x\to 0^+} \frac{ax^2+bx+c-c}{x}\\
		&=\lim_{x\to 0^+} \frac{ax^2+bx}{x}\\
		&=\lim_{x\to 0^+} (ax+b)\\
		&=b
		\end{align*}
		\begin{align*}
		\lim_{x\to 0^-} \frac{f(x)-c}{x}&=\lim_{x\to 0^-} \frac{a \sin(x)+b \cos(x)-c}{x}\\
		&=\lim_{x\to 0^-} \frac{a \sin(x)}{x}+\lim_{x\to 0^-} \frac{b\cos(x)-c}{x}\\ \\
		& \text{we already found}\ b=c \ \text{so we have}\\
		&\lim_{x\to 0^-} \frac{a \sin(x)}{x}+\lim_{x\to 0^-} \frac{c\cdot \cos(x)-c}{x}\\
		&=a\cdot \lim_{x\to 0^-} \frac{ \sin(x)}{x}+c \cdot \lim_{x\to 0^-} \frac{\cos(x)-1}{x}\\ \\
		& \text{since}\ \lim_{x\to 0} \frac{\sin(x)}{x}=1\ \text{and}\  \lim_{x\to 0} \frac{\cos(x)-1}{x}=0 \ \text{we have}\\
		&a\cdot \lim_{x\to 0^-} \frac{\sin(x)}{x}+c \cdot \lim_{x\to 0^-} \frac{\cos(x)-1}{x}=a
		\end{align*}

		This means in order for the derivative to exist, $a=b=c$.
		\end{freeResponse}
\end{problem}	




\end{document} 


















