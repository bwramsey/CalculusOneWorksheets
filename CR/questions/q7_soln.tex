\documentclass{ximera}

\newcommand{\RR}{\mathbb R}
\renewcommand{\d}{\,d}
\newcommand{\dd}[2][]{\frac{d #1}{d #2}}
\renewcommand{\l}{\ell}
\newcommand{\ddx}{\frac{d}{dx}}
\newcommand{\dfn}{\textbf}
\newcommand{\eval}[1]{\bigg[ #1 \bigg]}
\renewcommand{\theenumii}{\textup{(\roman{enumii})}}
\renewcommand{\labelenumii}{\theenumii}

\usepackage{graphicx}
\usepackage{multicol}
\usepackage{tkz-euclide}
%\usepackage{unicode-math}

\usepackage{pgfplots}   % <- for graphics
\pgfplotsset{compat=newest}


\renewenvironment{freeResponse}{
\ifhandout\setbox0\vbox\bgroup\else
\begin{trivlist}\item[\hskip \labelsep\bfseries Solution:\hspace{2ex}]
\fi}
{\ifhandout\egroup\else
\end{trivlist}
\fi}

\newcommand*{\ZeroOverZero}{\ensuremath{\dfrac{0}{0}}}

\providecommand{\HCCondition}{0}
\newcommand{\WkstHop}[1][1]{\if\HCCondition 0
	\vspace*{\stretch{#1}} \fi} 
\newcommand{\WkstNew}{\if\HCCondition 0
	\newpage
	 \fi} 


\title[Problem 7]{Problem 7}

\begin{document}
\begin{abstract} \end{abstract}
\maketitle


% Extracted from chainRule.tex, problem #7
\begin{problem}
Find values for $a$, $b$, and $c$ so that the following function is differentiable everywhere.

\[f(x) =   \begin{cases}
	a \sin(x) + b \cos(x)		 	&	\qquad \text{if } x < 0					\\
	ax^2 + bx + c   				&	\qquad \text{if } x \geq 0	 \end{cases}   \]

	\begin{explanation}
		The function $f$ is differentiable for $x<0$ since $f$ is a combination of trigonometric functions.  $f$ is also differentiable $x>0$ since $f$ is a polynomial on that interval.  We need to focus on $x=0$.  Since $f$ is differentiable everywhere, $f$ must also be differentiable at $x=0$ and therefore continuous at $x=0$.  Therefore: \\\\
We need that $\lim_{x \to 0^-} f(x) = \lim_{x \to 0^+} f(x)=\lim_{x \to 0} f(x)=f(0)$.  Observe that
		
		\begin{itemize}
		
		\item $\lim_{x \to 0^-} f(x) 
		= \lim_{x \to 0^-} (a\sin(x) + b\cos(x))
		= b(1) = b$.
		
		\item  $ \lim_{x \to 0^+} f(x)
		= \lim_{x \to 0^+} (ax^2 + bx + c)
		= c$.
		
		\end{itemize}
		
		Thus, we must have that $b = c$\\\\
		Next, if $f'(0)$ exists, then:
		\begin{align*}
		f'(0)&=\lim_{x\to 0} \frac{f(x)-f(0)}{x-0}\\\\
		& \text{since}\ f(0)=a \cdot 0^2+b \cdot 0 +c =c\\\\
		& \lim_{x\to 0} \frac{f(x)-f(0)}{x-0}=\lim_{x\to 0} \frac{f(x)-c}{x}
		\end{align*}
		Since this limit exists, the left and right limits must be equal.
		
		\begin{align*}
		\lim_{x\to 0^+} \frac{f(x)-c}{x}&=\lim_{x\to 0^+} \frac{ax^2+bx+c-c}{x}\\
		&=\lim_{x\to 0^+} \frac{ax^2+bx}{x}\\
		&=\lim_{x\to 0^+} (ax+b)\\
		&=b
		\end{align*}
		\begin{align*}
		\lim_{x\to 0^-} \frac{f(x)-c}{x}&=\lim_{x\to 0^-} \frac{a \sin(x)+b \cos(x)-c}{x}\\
		&=\lim_{x\to 0^-} \frac{a \sin(x)}{x}+\lim_{x\to 0^-} \frac{b\cos(x)-c}{x}\\ \\
		& \text{we already found}\ b=c \ \text{so we have}\\
		&\lim_{x\to 0^-} \frac{a \sin(x)}{x}+\lim_{x\to 0^-} \frac{c\cdot \cos(x)-c}{x}\\
		&=a\cdot \lim_{x\to 0^-} \frac{ \sin(x)}{x}+c \cdot \lim_{x\to 0^-} \frac{\cos(x)-1}{x}\\ \\
		& \text{since}\ \lim_{x\to 0} \frac{\sin(x)}{x}=1\ \text{and}\  \lim_{x\to 0} \frac{\cos(x)-1}{x}=0 \ \text{we have}\\
		&a\cdot \lim_{x\to 0^-} \frac{\sin(x)}{x}+c \cdot \lim_{x\to 0^-} \frac{\cos(x)-1}{x}=a
		\end{align*}

		This means in order for the derivative to exist, $a=b=c$.
		\end{explanation}
\end{problem}



\end{document}
