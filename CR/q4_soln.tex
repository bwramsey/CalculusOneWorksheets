\documentclass{ximera}

\newcommand{\RR}{\mathbb R}
\renewcommand{\d}{\,d}
\newcommand{\dd}[2][]{\frac{d #1}{d #2}}
\renewcommand{\l}{\ell}
\newcommand{\ddx}{\frac{d}{dx}}
\newcommand{\dfn}{\textbf}
\newcommand{\eval}[1]{\bigg[ #1 \bigg]}
\renewcommand{\theenumii}{\textup{(\roman{enumii})}}
\renewcommand{\labelenumii}{\theenumii}

\usepackage{graphicx}
\usepackage{multicol}
\usepackage{tkz-euclide}
%\usepackage{unicode-math}

\usepackage{pgfplots}   % <- for graphics
\pgfplotsset{compat=newest}


\renewenvironment{freeResponse}{
\ifhandout\setbox0\vbox\bgroup\else
\begin{trivlist}\item[\hskip \labelsep\bfseries Solution:\hspace{2ex}]
\fi}
{\ifhandout\egroup\else
\end{trivlist}
\fi}

\newcommand*{\ZeroOverZero}{\ensuremath{\dfrac{0}{0}}}

\providecommand{\HCCondition}{0}
\newcommand{\WkstHop}[1][1]{\if\HCCondition 0
	\vspace*{\stretch{#1}} \fi} 
\newcommand{\WkstNew}{\if\HCCondition 0
	\newpage
	 \fi} 


\title[Problem 4]{Problem 4}

\begin{document}
\begin{abstract} \end{abstract}
\maketitle


% Extracted from chainRule.tex, problem #4
\begin{problem}
Suppose the line tangent to the graph of $f$ at $x=1$ is $y=6x-7$.  Find an equation of the line tangent to the following curves at $x=1$:
	\begin{enumerate}
	
	\item  $y=g(x) = 5(f(x))^4$  
\begin{explanation}
		First, in the equation of the tangent line to $f(x)$ at $x=1$, when $x=1$ we have that $y= 6(1) - 7 = -1$.  Thus $f(1) = -1$ and therefore $g(1) = 5(-1)^4 = 5$.  Hence, a point on our line is $(1,5)$.  Also, since the slope of the given tangent line is $m=6$, we know that $f'(1) = 6$.
		
		Now, by the chain rule we have that $g'(x) = 5 \cdot 4 (f(x))^3 \cdot f'(x)$.  So $g'(1) = 20(f(1))^3 \cdot f'(1) = 20(-1)^3 (6) = -120.$  Thus, the equation of the line tangent to the graph of $y = g(x)$ at $x=1$ is
		$$ y - 5 = -120(x-1) $$
		$$ y = -120x + 125 $$
		\end{explanation}

	\item  $y=h(x) = x^2 (f(x^3))$
\begin{explanation}
		$h(1) = 1^2 \cdot f(1^3) = f(1) = -1$.  By the product and chain rules:
		$$ h'(x) = 2x(f(x^3)) + x^2(f'(x^3) \cdot 3x^2) = 2x(f(x^3)) + 3x^4(f'(x^3)) $$
		Thus, $h'(1) = 2f(1) + 3f'(1) = 2(-1) + 3(6) = -2 + 18 = 16$.  So, the equation of the line tangent to the graph of $h$ at $x=1$ is
		$$ y- (-1) = 16(x-1) $$
		$$ y = 16x - 17 $$
		\end{explanation}
	\end{enumerate}		
		
		
	
\end{problem}



\end{document}
