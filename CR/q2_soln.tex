\documentclass{ximera}

\newcommand{\RR}{\mathbb R}
\renewcommand{\d}{\,d}
\newcommand{\dd}[2][]{\frac{d #1}{d #2}}
\renewcommand{\l}{\ell}
\newcommand{\ddx}{\frac{d}{dx}}
\newcommand{\dfn}{\textbf}
\newcommand{\eval}[1]{\bigg[ #1 \bigg]}
\renewcommand{\theenumii}{\textup{(\roman{enumii})}}
\renewcommand{\labelenumii}{\theenumii}

\usepackage{graphicx}
\usepackage{multicol}
\usepackage{tkz-euclide}
%\usepackage{unicode-math}

\usepackage{pgfplots}   % <- for graphics
\pgfplotsset{compat=newest}


\renewenvironment{freeResponse}{
\ifhandout\setbox0\vbox\bgroup\else
\begin{trivlist}\item[\hskip \labelsep\bfseries Solution:\hspace{2ex}]
\fi}
{\ifhandout\egroup\else
\end{trivlist}
\fi}

\newcommand*{\ZeroOverZero}{\ensuremath{\dfrac{0}{0}}}

\providecommand{\HCCondition}{0}
\newcommand{\WkstHop}[1][1]{\if\HCCondition 0
	\vspace*{\stretch{#1}} \fi} 
\newcommand{\WkstNew}{\if\HCCondition 0
	\newpage
	 \fi} 


\title[Problem 2]{Problem 2}

\begin{document}
\begin{abstract} \end{abstract}
\maketitle


% Extracted from chainRule.tex, problem #2
\begin{problem}
 A table of values for $f(x)$ and $f'(x)$ is shown below:

% replaced this png with a MathMode array
%\begin{image}[0.25\linewidth]
%	\includegraphics{figure1.png}
%\end{image}

$$\begin{array}{c|c|c}
	\hline
	& & \\
	x & f(x) & f'(x) \\
	& & \\
	\hline
	& & \\
	1 & 3 & 4\\
	2 & 2 & 3\\
	3 & 4 & 5\\
	4 & 6 & 3\\
	\hline
\end{array} $$

	\begin{itemize}
		\item	Evaluate the limit $\displaystyle \lim_{x \to 2} \dfrac{f(x^2)-6}{x-2}$. \textbf{EXPLAIN}.
\begin{explanation}
				If we set $g(x) = f(x^2)$, this limit is $\displaystyle \lim_{x\to 2} \dfrac{g(x)-g(2)}{x-2}$. In other words, we're being asked to
				evaluate $\displaystyle\ddx f(x^2)$ at $x = 2$. By the Chain Rule, $\displaystyle\ddx f(x^2) = 2x f'(x^2)$. When evaluated at $x=2$
				this is $2(2) f'(4) = 4 (3) = 12$. The value of this limit is $12$.
				
				Our full explanation is then:
				$\displaystyle \lim_{x\to 2} \dfrac{f(x^2)}{x-2} = \eval{\ddx f(x^2) }_{x=2}$ by the limit definition of the derivative. Our calculation above shows that this equals $12$.
			  \end{explanation}	

	  	\item Evaluate $\displaystyle\ddx f(f(x))$ at $x = 3$.
		  \begin{enumerate}
			    \item 6
			    \item 25
			    \item 5
			    \item 15
			    \item DNE
			    \item None of the previous answers.
			  \end{enumerate}
\begin{explanation}
			    The answer is (d):
			    \begin{align*}
			      \eval{\ddx f(f(x))}_{x = 3} &= f'(f(3)) \cdot f'(3) \\
			      &= f'(4)\cdot 5 \\
			      &= 3 \cdot 5 = 15
			    \end{align*}
			  \end{explanation}
	\end{itemize}

\end{problem}



\end{document}
