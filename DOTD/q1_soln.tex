\documentclass{ximera}

\newcommand{\RR}{\mathbb R}
\renewcommand{\d}{\,d}
\newcommand{\dd}[2][]{\frac{d #1}{d #2}}
\renewcommand{\l}{\ell}
\newcommand{\ddx}{\frac{d}{dx}}
\newcommand{\dfn}{\textbf}
\newcommand{\eval}[1]{\bigg[ #1 \bigg]}
\renewcommand{\theenumii}{\textup{(\roman{enumii})}}
\renewcommand{\labelenumii}{\theenumii}

\usepackage{graphicx}
\usepackage{multicol}
\usepackage{tkz-euclide}
%\usepackage{unicode-math}

\usepackage{pgfplots}   % <- for graphics
\pgfplotsset{compat=newest}


\renewenvironment{freeResponse}{
\ifhandout\setbox0\vbox\bgroup\else
\begin{trivlist}\item[\hskip \labelsep\bfseries Solution:\hspace{2ex}]
\fi}
{\ifhandout\egroup\else
\end{trivlist}
\fi}

\newcommand*{\ZeroOverZero}{\ensuremath{\dfrac{0}{0}}}

\providecommand{\HCCondition}{0}
\newcommand{\WkstHop}[1][1]{\if\HCCondition 0
	\vspace*{\stretch{#1}} \fi} 
\newcommand{\WkstNew}{\if\HCCondition 0
	\newpage
	 \fi} 


\title[Problem 1]{Problem 1}

\begin{document}
\begin{abstract} \end{abstract}
\maketitle


% Extracted from definitionOfTheDerivative.tex, problem #1
\begin{problem}
 Consider the following two figures depicting the same graph $y=f(x)$ and the same two lines, and the same two points $P$ and $Q$:
  \[
    \begin{array}{lr}
      \includegraphics[trim= 150 350 250 180]{Figure2.pdf} &		   \includegraphics[trim= 140 350 250 180]{Figure3.pdf}
    \end{array}
  \]
  \begin{enumerate}
    \item
      In Figure 1 on the left, what are the coordinates of $P$ and $Q$?  In Figure 2 on the right, what are the expressions for the coordinates of $P$ and $Q$?

	\begin{explanation}
	Figure 1 on the left: $P(a,f(a))$ and $Q(x,f(x))$ \\
	Figure 2 on the right: $P(a,f(a))$ and $Q((a+h),f(a+h))$

	\end{explanation}

	\item Express the slope of the secant line through $P$ and $Q$ in terms of the above coordinates for each Figure 1 and Figure 2.

      \begin{explanation}
	Figure 1: $\frac{f(x)-f(a)}{x-a}$ \quad
	Figure 2: $\frac{f(a+h)-f(a)}{h}$
        \end{explanation}
	
	\item
	Express the slope of the tangent line at the point $P$ in terms of the above coordinates for each Figure 1 and Figure 2.

	\begin{explanation}
        	Figure 1:  $ \lim_{x \to a} \frac{f(x)-f(a)}{x-a}$\quad
	Figure 2: $\lim_{h \to 0} \frac{f(a+h)-f(a)}{h}$
      \end{explanation}

    \item 
      What is the difference?
      \begin{explanation}
	Just the variable we use, $x$ or $h$ where $x=a+h \iff h=x-a$
      \end{explanation}


    \item 
      For each of the two graphs, which lines are the secant lines?
      \begin{explanation}
        The green line is the secant line in each of the two graphs.        
      \end{explanation}

    \item 
      For each of the two graphs, which lines are the tangent lines?
      \begin{explanation}
        The red line is the tangent line in each of the two graphs.        
      \end{explanation}

  \end{enumerate}
\end{problem}



\end{document}
