\documentclass{ximera}

\newcommand{\RR}{\mathbb R}
\renewcommand{\d}{\,d}
\newcommand{\dd}[2][]{\frac{d #1}{d #2}}
\renewcommand{\l}{\ell}
\newcommand{\ddx}{\frac{d}{dx}}
\newcommand{\dfn}{\textbf}
\newcommand{\eval}[1]{\bigg[ #1 \bigg]}
\renewcommand{\theenumii}{\textup{(\roman{enumii})}}
\renewcommand{\labelenumii}{\theenumii}

\usepackage{graphicx}
\usepackage{multicol}
\usepackage{tkz-euclide}
%\usepackage{unicode-math}

\usepackage{pgfplots}   % <- for graphics
\pgfplotsset{compat=newest}


\renewenvironment{freeResponse}{
\ifhandout\setbox0\vbox\bgroup\else
\begin{trivlist}\item[\hskip \labelsep\bfseries Solution:\hspace{2ex}]
\fi}
{\ifhandout\egroup\else
\end{trivlist}
\fi}

\newcommand*{\ZeroOverZero}{\ensuremath{\dfrac{0}{0}}}

\providecommand{\HCCondition}{0}
\newcommand{\WkstHop}[1][1]{\if\HCCondition 0
	\vspace*{\stretch{#1}} \fi} 
\newcommand{\WkstNew}{\if\HCCondition 0
	\newpage
	 \fi} 


\title[Problem 4]{Problem 4}

\begin{document}
\begin{abstract} \end{abstract}
\maketitle


% Extracted from reviewOfFamousFunctions.tex, problem #4
\begin{problem}

  Find all real numbers which satisfy each of the equations.  In the previous problem, we used $\theta$ to denote the radian measure of the angle.  However, we can use any variable to represent the angle measure.  For example, in part a, $x$ is the variable representing the radian measure of the angle.
  \begin{enumerate}
    \item
      $\cos(x) = 1$
\begin{explanation}
        This is asking for the collection of all angles such that cosine of that angle equals 1.

        The unit circle shows that one such angle is $0$ (since $\cos(0) = 1$).
        There is a slight trick here: since cosine has period $2\pi$ we actually have $\cos(0 + 2\pi n) = 1$ for every integer $n$.
        In summary, $x = 2\pi n$, where $n$ is any integer, gives all the solutions to this equation.
      \end{explanation}

    \item
      $\sin(3 \theta) = \sqrt{3}/2$ for $0 \leq \theta \leq 2\pi$
\begin{explanation}
        Finding all numbers $\theta$ with $0 \leq \theta \leq 2\pi$ that satisfy $\sin(3 \theta) = \sqrt{3}/2$ is a bit tricky.
        We first perform a useful trick from algebra~---~variable substitution.

        Let $x = 3\theta$.
        So, we are trying to find all numbers $x$ such that $\sin(x) = \sqrt{3}/2$ for $0 \leq x/3 \leq 2\pi$.
        Then $ x= \frac{\pi}{3} + 2 \pi n$ or $ x = \frac{2 \pi }{3} + 2 \pi n $ for $n$ any integer as long as $0 \leq \theta \leq 2\pi$
        Since $x = 3 \theta$, we can solve for $\theta$ to obtain $\theta = \pi/9 + (2 \pi n)/3$ or $\theta = (2\pi)/9 + (2\pi n)/3$, where $n$ is again any integer as long as $0 \leq \theta \leq 2\pi$.
        We are only looking for solutions of $\theta$ in $[0, 2\pi ]$, and so our solutions are
        \[
        \theta = \frac{\pi}{9}, \frac{2\pi}{9}, \frac{7\pi}{9}, \frac{8\pi}{9}, \frac{13\pi}{9}, \frac{14\pi}{9}. 
        \]
      \end{explanation}
  \end{enumerate}
\end{problem}



\end{document}
