


%Add code to compile both versions from makefile at same time
\providecommand{\HCCondition}{0}

%Define each of the conditions
\ifcase\HCCondition
	%\condition=0 -> handout
	\documentclass[nooutcomes,noauthor,space,handout]{ximera}
	\title{Review of famous functions (ROFF)}
\or	%\condition=1 -> Soln
	\documentclass[nooutcomes,noauthor]{ximera}
	\title{Review of famous functions (ROFF) - Solutions} 
\fi

\newcommand{\RR}{\mathbb R}
\renewcommand{\d}{\,d}
\newcommand{\dd}[2][]{\frac{d #1}{d #2}}
\renewcommand{\l}{\ell}
\newcommand{\ddx}{\frac{d}{dx}}
\newcommand{\dfn}{\textbf}
\newcommand{\eval}[1]{\bigg[ #1 \bigg]}
\renewcommand{\theenumii}{\textup{(\roman{enumii})}}
\renewcommand{\labelenumii}{\theenumii}

\usepackage{graphicx}
\usepackage{multicol}
\usepackage{tkz-euclide}
%\usepackage{unicode-math}

\usepackage{pgfplots}   % <- for graphics
\pgfplotsset{compat=newest}


\renewenvironment{freeResponse}{
\ifhandout\setbox0\vbox\bgroup\else
\begin{trivlist}\item[\hskip \labelsep\bfseries Solution:\hspace{2ex}]
\fi}
{\ifhandout\egroup\else
\end{trivlist}
\fi}

\newcommand*{\ZeroOverZero}{\ensuremath{\dfrac{0}{0}}}

\providecommand{\HCCondition}{0}
\newcommand{\WkstHop}[1][1]{\if\HCCondition 0
	\vspace*{\stretch{#1}} \fi} 
\newcommand{\WkstNew}{\if\HCCondition 0
	\newpage
	 \fi}  %% we can turn off input when making a master document
\usepackage{fullpage}
\usepackage{tkz-tab}
\usetikzlibrary{decorations.pathreplacing}

\begin{document}
\begin{abstract}		\end{abstract}
\makeTagTitle
\ifcase\HCCondition
{\large{\textbf{SUMMARY: Polynomial, rational, exponential, logarithmic,}}}\\
 {\large{\textbf{trigonometric and inverse trigonometric functions }}}\\[0.8em]
\begin{itemize}
	\item Know the graphs and properties of ``famous'' functions
	 \item Understand the definition of a polynomial function
        \item Understand the definition of a rational function
        \item Know and use the properties of exponential and logarithmic functions
	\item Understand the relationship between exponential and logarithmic functions
           \item Evaluate expressions and solve equations involving\\
          exponential and logarithmic functions
	\item Understand the properties of trigonometric functions
	\item Evaluate expressions and solve equations involving\\
          trigonometric functions and inverse trigonometric functions
	\item Use sign-charts to solve inequalities involving famous functions.
\end{itemize}
\newpage
\section*{Recitation Questions}
\fi
%problem 1
\begin{problem}
The graph of $g(x)=e^x$ is given below.

%	\begin{image}		
%	\includegraphics[scale=.6]{Figure1.pdf}
%	\end{image}	
	  \begin{center}
		\begin{tikzpicture}
			\begin{axis}[
				xmin=-6.3, xmax=6.3, ymin=-6.3,ymax=6.3,    
				axis lines =middle, 
				every axis y label/.style={at=(current axis.above origin),anchor=south},
				every axis x label/.style={at=(current axis.right of origin),anchor=west},
				xtick={-6,-4,...,6}, ytick={-6,-4,...,6},
				grid=major, width=3.5in, height = 3.5in,
				grid style={dashed, gray!40}
				]
				\addplot[color=red, ultra thick, smooth, domain=-6.3:3]{exp(x)};				

			\end{axis}
		\end{tikzpicture}
	\end{center}
			

\begin{enumerate}	
	\item  Find the domain and range of $g$.
\WkstHop
		\begin{freeResponse}
			Domain: $(-\infty,\infty)$, Range: $(0,\infty)$
		\end{freeResponse}


	
	\item  Find the values of $g(1), g(0), g(-1)$ and plot the points $(1,g(1)), (0,g(0)),$ and $(-1,g(-1))$  on the graph of $g(x)=e^x$.
\WkstHop
		\begin{freeResponse}	
			$g(1)=e^1=e$ , $ g(0)=e^0=1$, and $g(-1)=e^{-1}=\frac{1}{e}$.
			 These are standard values of the famous function $g(x)=e^x$ that you should know.
		\begin{image}		
		\includegraphics[scale=.6]{Figure5.pdf}
		\end{image}


		\end{freeResponse}

	\item Graph $h(x)=\ln(x)$ on the same axis as $g(x)=e^x$.
\WkstHop
		\begin{freeResponse}
		Recall:  $\ln(x)$ is the inverse of $e^x$.  To find the graph of $\ln(x)$ we reflect the graph of $e^x$ over the line $y=x$.  Since the points $\left(-1,\frac{1}{e}\right),(0,1),(1,e)$ are on $y=e^x=g(x)$, the points $\left(\frac{1}{e}, -1\right),(1,0),(e,1)$ are on the graph of $y=\ln(x)=g^{-1}(x)=h(x)$ 

		\begin{image}		
		\includegraphics[scale=.7]{Figure4.pdf}
		\end{image}
		  \begin{center}
		\begin{tikzpicture}
			\begin{axis}[
				xmin=-6.3, xmax=6.3, ymin=-6.3,ymax=6.3,    
				axis lines =middle, 
				every axis y label/.style={at=(current axis.above origin),anchor=south},
				every axis x label/.style={at=(current axis.right of origin),anchor=west},
				xtick={-6,-4,...,6}, ytick={-6,-4,...,6},
				grid=major, width=3.5in, height = 3.5in,
				grid style={dashed, gray!40}
				]
				\addplot[color=red, ultra thick, smooth, domain=-6.3:3]{exp(x)};				
				\addplot[color=blue, ultra thick, smooth, domain=0.1:6.3]{ln(x)};				
			\end{axis}
		\end{tikzpicture}
	\end{center}	
		\end{freeResponse}

	\item Find the domain and range of $h$.
\WkstHop
		\begin{freeResponse}
		The domain of $h(x)=\ln(x)$ is $(0,\infty)$.  The range is $(-\infty,\infty)$.
		\end{freeResponse}

	\item Find the values of $h(1), h(0), h(-1), h(e), h\left(\frac{1}{e}\right)$, or say $x$ not in the domain.
\WkstHop
			\begin{freeResponse}
			 $h(1)=\ln(1)=0$, $h(0)$ is not defined because $0$ is not in the domain of $\ln(x)$.
			$ h(-1)$ is not defined because $-1$ is not in the domain of $\ln(x)$.
			$h(e)=\ln(e)=1$, and $h\left(\frac{1}{e}\right)=\ln\left(\frac{1}{e}\right)=\ln\left(e^{-1}\right)= -1\ln(e)=-1$.
			 These are standard values of the famous function $h(x)=\ln(x)$ that you should know.
		\end{freeResponse}
	\end{enumerate}
	
 		
		
	
\end{problem}
\WkstNew

\begin{problem}
	Solve the following inequalities:
	\begin{enumerate}


		\item $\displaystyle  \dfrac{x^3 \left(2x-5\right)^2}{x+1} > 0$.
\WkstHop
			\begin{freeResponse}
				Call $f(x) = \dfrac{x^3 \left(2x-5\right)^2}{x+1}$.
				Since the right-hand side of the inequality is zero, we are really asking where $f(x)$ will be positive.  The only way
				a nice function like $f$ can switch from positive to negative, or vice-versa, is if it is either zero
				or undefined at the point where it changes.
		
				Notice where the value of $f$ would be either zero or undefined.  Those occur at $x = -1, 0, \frac{5}{2}$. (These are the places
				where the corresponding factors of $f$ are themselves zero or undefined.)  
				We'll use those points to split the number line into four regions. Inside each of those regions, our factors are either
				always positive or always negative.
				
				\begin{tikzpicture} 
					\tkzTabInit[lgt=2,espcl=1] 
						{$x$         /1, 
						$x^3$   /1, 
						$\left(2x-5\right)^2$  /1,
						$x+1$       /1,
						$f(x)$       /1}% 
						{  , $-1$ , $0$ ,$\dfrac{5}{2}$,  }% 
					\tkzTabLine{ , - , t , - , d , + , d , + ,}
					\tkzTabLine{ , + , t , + , d , + , d , + ,}
					\tkzTabLine{ , - , t , + , d , + , d , +, }
					\tkzTabLine{ , + , t , - , d , + , d , +, }
				\end{tikzpicture} 
				
				For $x < -1$, we find $x^3 < 0$, $(2x-5)^2 > 0$, and $x+1 < 0$.  Together, that means the value of $f(x)$ is  
				$\displaystyle \dfrac{ \textrm{negative} \cdot \textrm{positive}}{\textrm{negative}}$.  It is, therefore, positive in that region, and makes up
				part of our solution.  We cannot include the endpoint $x=-1$ in the solution, as the fraction is undefined there.
				
				In the same way, we see that in the intervals $\left( 0, \frac{5}{2} \right)$ and $\left( \frac{5}{2}, 0\right)$, $f(x)$ is also positive.
				Notice that the inequality is NOT satisfied at $x=\frac{5}{2}$, since $f(\frac{5}{2})$ equals zero.  
				
				The solution is $\left( -\infty, -1 \right) \cup \left( 0, \frac{5}{2} \right) \cup \left( \frac{5}{2}, \infty \right)$.
			\end{freeResponse}

		\item $\displaystyle  \dfrac{2\cdot 5^x - 10^x}{x^2-5} \leq 0$.
\WkstHop
			\begin{freeResponse}
				Call $g(x) = \dfrac{2\cdot 5^x - 10^x}{x^2-5}$.
				Start by factoring the numerator and denominator:
				\begin{align*} 
					\dfrac{2\cdot 5^x - 10^x}{x^2-5} &= \dfrac{2\cdot 5^x - 2^x\cdot 5^x}{x^2-5}\\
						&= \dfrac{5^x(2 - 2^x)}{x^2-5}\\						
						&= \dfrac{5^x(2 - 2^x)}{(x+\sqrt{5})(x-\sqrt{5})}
				\end{align*}
				Notice that $5^x$ is always positive, $2-2^x=0$ only when $x=1$, $x+\sqrt{5}=0$ only at $x=-\sqrt{5}$, 
				and $x-\sqrt{5}=0$ only at $x=\sqrt{5}$. Our sign chart looks like this:
				
				\begin{tikzpicture} 
					\tkzTabInit[lgt=2,espcl=1] 
						{$x$         /1, 
						$5^x$   /1, 
						$2-2^x$  /1,
						$x+\sqrt{5}$  /1,
						$x-\sqrt{5}$       /1,
						$g(x)$ /1 }% 
						{  , $-\sqrt{5}$ , $1$ ,$\sqrt{5}$,  }% 
					\tkzTabLine{ , + , t , + , d, + , t , + ,}
					\tkzTabLine{ , + , t , + , d , - , t , - ,}
					\tkzTabLine{ , - , t , + , d , + , t , +, }
					\tkzTabLine{ , - , t , - , d , - , t , +, }
					\tkzTabLine{ , + , t , - , d , + , t , -, }
				\end{tikzpicture} 
				
				From the bottom row of the sign-chart, we see the solution to the inequality is:
				$\left( -\sqrt{5}, 1\right] \cup \left( \sqrt{5}, \infty\right)$.
			\end{freeResponse}
			
	\end{enumerate}


\end{problem}
\WkstNew

%problem 3
\begin{problem} \hfil
	\begin{enumerate}
	\item Suppose we're given the right triangle below.  Express $\sin(\theta)$ and $\cos(\theta)$ in terms of the sides of the triangle.

%	\begin{image}
%	\includegraphics[scale=.3]{figure11l.png}
%	\end{image}

\begin{image}[1.5in]
  \begin{tikzpicture}
    \coordinate (C) at (0,2);
    \coordinate (D) at (2,2);
    \coordinate (E) at (2,5);
% \tkzMarkRightAngles(C,D,E)
   \tkzMarkAngles(D,C,E)
    \draw[very thick] (D)--(E)--(C)--cycle;
    \draw (1.8, 2) -- (1.8, 2.2) -- (2, 2.2);
    \node at (1,2-.4) {$A$};
    \node at (2.4,3.5) {$B$};
    \node at (1-.5,3.7) {$C=1$};
    \node at (0.5,2.3) {$\theta$};
  \end{tikzpicture}
\end{image}

\WkstHop
	\begin{freeResponse}
	$\sin(\theta)=\frac{B}{C}=B$ and $\cos(\theta)=\frac{A}{C}=A$ 
	\end{freeResponse}

	\item Suppose we are given the triangle below.  
	%	\begin{image}
	%	\includegraphics[scale=.3]{figure22l.png}
	%	\end{image}
\begin{image}[1.5in]
  \begin{tikzpicture}
    \coordinate (C) at (0,2);
    \coordinate (D) at (4,2);
    \coordinate (E) at (4,6);
    \tkzMarkRightAngles(C,D,E)
   \tkzMarkAngles(D,C,E)
    \draw[very thick] (D)--(E)--(C)--cycle;
    \node at (2,2-.4) {$A$};
    \node at (4.4,4) {$B$};
    \node at (1+.4,4.2) {$C=1$};
    \node at (1,2.3) {$\theta = 45^\circ$};
  \end{tikzpicture}
\end{image}

		\begin{enumerate}
	\item Find the length of the sides A and B.
\WkstHop
	\begin{freeResponse}
	This is an isosceles triangle. $A=B$  Using the Pythagorean Theorem:
	\begin{align*}
	A^2+B^2&=1^2\\
	2A^2&=1 \\ 	
	A^2&=\frac{1}{2}\\
	A&=\sqrt{\frac{1}{2}}=\frac{\sqrt{2}}{2}\\
	B&=\sqrt{\frac{1}{2}}=\frac{\sqrt{2}}{2}
	\end{align*}
	\end{freeResponse}
	\item Express $\sin\left(\frac{\pi}{4}\right)$ and $\cos\left(\frac{\pi}{4}\right)$ in terms of the sides of the triangle.
\WkstHop
	\begin{freeResponse}
	$\sin\left(\frac{\pi}{4}\right)=\frac{B}{C}=\frac{\sqrt{2}}{2}$ and $\cos\left(\frac{\pi}{4}\right)=\frac{A}{C}=\frac{\sqrt{2}}{2}$

	\end{freeResponse}
	\end{enumerate}
\WkstNew
	\item Suppose we are given the triangle below.  
%		\begin{image}
%		\includegraphics[scale=.5]{figure33l.png}
%		\end{image}
\begin{image}[1.5in]
  \begin{tikzpicture}
    \coordinate (C) at (0,2);
    \coordinate (D) at (3,2);
    \coordinate (E) at (3,8);
    \tkzMarkRightAngles(C,D,E)
%   \tkzMarkAngle(D,C,E)
    \draw[very thick] (D)--(E)--(C)--cycle;
    \node at (1.5,2-.4) {$A$};
    \node at (3.4,4) {$B$};
    \node at (1-.5,4.2) {$C=1$};
    \node at (0.9,2.3) {\small{$\theta = \frac{\pi}{3}$}};
  \end{tikzpicture}
\end{image}

		\begin{enumerate}
	\item Find the length of the sides A and B.
\WkstHop
	\begin{freeResponse}
	You might remember this as a 30/60/90 triangle.  To find the lengths of the sides of the triangle, we create a triangle as in the figure below.  All the angles in this triangle are of 60 degrees, therefore, this is an equilateral triangle 
		\begin{image}
		\includegraphics[scale=.5]{figure44l.png}
		\end{image}
	Now that we have an equilateral triangle, we have $C=C=2A$.  Thus, $1=2A \implies A=\frac{1}{2}$ \\
	To find B, we use the Pythagorean Theorem.
	\begin{align*}
	\left(\frac{1}{2}\right)^2+B^2&=1^2\\
	\frac{1}{4}+B^2&=1 \\ 	
	B^2&=\frac{3}{4}\\
	B&=\sqrt{\frac{3}{4}}=\frac{\sqrt{3}}{2}
	\end{align*}
	\end{freeResponse}

	\item Express $\sin\left(\frac{\pi}{3}\right)$ and $\cos\left(\frac{\pi}{3}\right)$ in terms of the sides of the triangle.
\WkstHop
	\begin{freeResponse}
	$\sin\left(\frac{\pi}{3}\right)=\frac{B}{C}=\frac{\sqrt{3}}{2}$ and $\cos\left(\frac{\pi}{3}\right)=\frac{A}{C}=\frac{1}{2}$

	\end{freeResponse}
	\end{enumerate}
\WkstNew
\item Suppose we are given the triangle below.  
%		\begin{image}
%		\includegraphics[scale=.5]{figure55l.png}
%		\end{image}
\begin{image}[1.75in]
  \begin{tikzpicture}
    \coordinate (C) at (0,2);
    \coordinate (D) at (6,2);
    \coordinate (E) at (6,5);
    \tkzMarkRightAngles(C,D,E)
%   \tkzMarkAngle(D,C,E)
    \draw[very thick] (D)--(E)--(C)--cycle;
    \node at (3,1.6) {$B$};
    \node at (6.3,3.5) {$A$};
    \node at (2.6,3.8) {$C=1$};
    \node at (1.6,2.4) {\small{$\theta = \frac{\pi}{6}$}};
  \end{tikzpicture}
\end{image}

		\begin{enumerate}
	\item Find the length of the sides A and B.
\WkstHop
	\begin{freeResponse}
		We've actually already found the lengths of the sides for this type of triangle in part c.  
		$B=\sqrt{\frac{3}{4}}=\frac{\sqrt{3}}{2}$ and $A=\frac{1}{2}$ 
	\end{freeResponse}

	\item Write $\sin\left(\frac{\pi}{6}\right)$ and $\cos\left(\frac{\pi}{6}\right)$ in terms of the sides of the triangle.
\WkstHop
	\begin{freeResponse}
		$\sin\left(\frac{\pi}{6}\right)=\frac{A}{C}=\frac{1}{2}$ and 
		$\cos\left(\frac{\pi}{6}\right)=\frac{B}{C}=\frac{\sqrt{3}}{2}$

	\end{freeResponse}
	\end{enumerate}
\WkstNew
\item 
For any point P(x,y) on the unit circle, we can express its coordinates in terms of $\sin(\theta)$ and $\cos(\theta)$.
Here $\theta$ is the radian measure of the angle in standard position whose terminal side is the line through the origin and the point $P(x,y)$.
\begin{image}
		\includegraphics[scale=.8]{figure1313l.png}
		\end{image}
\WkstHop
		\begin{freeResponse}
			$(x,y)=(\cos(\theta),\sin(\theta))$
		\end{freeResponse}

	\item  Use all of the above information to label the given points on the unit circle.  That is, for each point on the unit circle, provide the angle measure in radians and degrees, and give the (x,y) coordinate for the point.
		\begin{image}
		\includegraphics{figure66l.png}
		\end{image}
\WkstHop

		\begin{freeResponse} \hfil
		\begin{image}
		\includegraphics[scale=.7]{figure77l.png}
		\end{image}
		\end{freeResponse}
	\end{enumerate}

\end{problem}

\WkstNew
%problem4
\begin{problem}

  Find all real numbers which satisfy each of the equations.  In the previous problem, we used $\theta$ to denote the radian measure of the angle.  However, we can use any variable to represent the angle measure.  For example, in part a, $x$ is the variable representing the radian measure of the angle.
  \begin{enumerate}
    \item
      $\cos(x) = 1$
\WkstHop
      \begin{freeResponse}
        This is asking for the collection of all angles such that cosine of that angle equals 1.

        The unit circle shows that one such angle is $0$ (since $\cos(0) = 1$).
        There is a slight trick here: since cosine has period $2\pi$ we actually have $\cos(0 + 2\pi n) = 1$ for every integer $n$.
        In summary, $x = 2\pi n$, where $n$ is any integer, gives all the solutions to this equation.
      \end{freeResponse}

    \item
      $\sin(3 \theta) = \sqrt{3}/2$ for $0 \leq \theta \leq 2\pi$
\WkstHop
      \begin{freeResponse}
        Finding all numbers $\theta$ with $0 \leq \theta \leq 2\pi$ that satisfy $\sin(3 \theta) = \sqrt{3}/2$ is a bit tricky.
        We first perform a useful trick from algebra~---~variable substitution.

        Let $x = 3\theta$.
        So, we are trying to find all numbers $x$ such that $\sin(x) = \sqrt{3}/2$ for $0 \leq x/3 \leq 2\pi$.
        Then $ x= \frac{\pi}{3} + 2 \pi n$ or $ x = \frac{2 \pi }{3} + 2 \pi n $ for $n$ any integer as long as $0 \leq \theta \leq 2\pi$
        Since $x = 3 \theta$, we can solve for $\theta$ to obtain $\theta = \pi/9 + (2 \pi n)/3$ or $\theta = (2\pi)/9 + (2\pi n)/3$, where $n$ is again any integer as long as $0 \leq \theta \leq 2\pi$.
        We are only looking for solutions of $\theta$ in $[0, 2\pi ]$, and so our solutions are
        \[
        \theta = \frac{\pi}{9}, \frac{2\pi}{9}, \frac{7\pi}{9}, \frac{8\pi}{9}, \frac{13\pi}{9}, \frac{14\pi}{9}. 
        \]
      \end{freeResponse}
  \end{enumerate}
\end{problem}

\WkstNew
\begin{problem} \hfil

Graph $f(\theta)=\sin(\theta)$ and $g(\theta)=\cos(\theta)$ from $[-\frac{\pi}{8},2\pi+\frac{\pi}{8}]$
\WkstHop
\WkstHop
		\begin{freeResponse} \hfil
		\begin{image}
		\includegraphics{figure88l.png}
		\end{image}
		\begin{image}
		\includegraphics{figure99l.png}
		\end{image}
		\end{freeResponse}
\end{problem}
%problem 4
\begin{problem}
 Without using a calculator, determine if the statement
  \[
    \cos^{-1}\bigl(\cos(7\pi/6)\bigr) = 7\pi/6
  \]
  is true or false.
\WkstHop
  \begin{freeResponse}
    This statement is \textbf{false}: the correct statement is $\cos^{-1}\bigl(\cos(7\pi/6)\bigr) = 5\pi/6$. (Why?)

    \textbf{Spoiler Alert}: the cosine function is \emph{not} invertible since its graph fails the horizontal line test.
    \begin{image}
      \includegraphics[scale = 0.8]{figure12l.png}
    \end{image}
    To produce the inverse cosine we must first restrict the domain of cosine, to the interval $[0, \pi]$, to produce an invertible function:
    \begin{image}
      \includegraphics[scale = 0.8]{figure23l.png}
    \end{image}
    So, the range of $\cos^{-1}$ is $[0, \pi]$.
    Since $7\pi/6$ is not in this range, $7\pi/6$ is \emph{never} a possible output of $\cos^{-1}$.
  \end{freeResponse}
\end{problem}
\WkstNew

%problem 5
\begin{problem}
A boat sails directly toward a 100-meter skyscraper that stands on the edge of a harbor.  The angular size $\theta$ of the building is the angle formed by lines from the top and bottom of the building to the observer on the boat (see figure below).
    \begin{image}
      \includegraphics[scale = 0.6]{figure67l.png}
    \end{image}
\begin{enumerate}
 \item Express the angle $\theta$ as the function of $x$, the distance of the boat from the building.
\WkstHop
	\begin{freeResponse}
		$\tan\theta=\frac{100}{x} \implies \theta=\tan^{-1}\left( \frac{100}{x} \right)$
	\end{freeResponse}
	\item Find the angular size, $\theta$, when the boat is $x=100\sqrt{3}$m from the building.
\WkstHop
\WkstHop
	\begin{freeResponse}
		$\theta=\tan^{-1}\left( \frac{100}{100\sqrt{3}} \right)=\tan^{-1}\left( \frac{1}{\sqrt{3}} \right)=\frac{\pi}{6}$
	\end{freeResponse}
\end{enumerate}
\end{problem}

%problem 6

\begin{problem}

  \textbf{True or False:}
  $\sin^{-1}(0) = \pi$.
\WkstHop
  \begin{freeResponse}
    This statement is \textbf{false}: the correct statement is $\sin^{-1}(0) = 0$. (Why?)

    \textbf{Spoiler Alert}: the sine function is \emph{not} invertible since its graph fails the horizontal line test.
    \begin{image}
      \includegraphics[scale = 0.4]{figure34l.png}
    \end{image}
    To produce the inverse sine we first restrict the domain of sine, to the interval $[-\pi/2, \pi/2]$, to produce an invertible function:
    \begin{image}
      \includegraphics[scale = 0.4]{figure45l.png}
    \end{image}
    So, the range of $\sin^{-1}$ is $[-\pi/2, \pi/2]$.
    Since $\pi$ is not in this range, $\pi$, is \emph{never} a possible output of $\sin^{-1}$.
  \end{freeResponse}
\end{problem}
\WkstNew

%problem7
  \begin{problem}
  Simplify each of the following expressions.
  \begin{enumerate}
    \item
      $\cos^{-1} \bigl( \sin(\pi/2) \bigr)$
\WkstHop
      \begin{freeResponse}
        By the unit circle, $\sin(\pi/2) = 1$, and so we are looking for $\cos^{-1}(1)$.
        The range of $\cos^{-1}$ is $[0, \pi]$, and so, by properties of inverse functions, $\cos^{-1}(1) = 0$.
      \end{freeResponse}

    \item
      $\tan \bigl( \sin^{-1}(x/4) \bigr)$
\WkstHop
      \begin{freeResponse}
        Let $\theta = \sin^{-1}(x/4)$, then $\sin(\theta) = x/4$.
        We can then draw the corresponding right triangle:
        \begin{image}
          \includegraphics[scale = 0.4]{figure56l.png}
        \end{image}
        Calling the adjacent side $y$, by the Pythagorean Theorem we obtain

          $$4^2 = x^2 + y^2 \implies y = \sqrt{16-x^2}$$
Remark: Since $\theta$ is in the range of $\sin^{-1}$, it follows that $-\pi /2 \le \theta \le \pi /2$. 
On the other hand, the expression $ \tan \left( \sin^{-1} \left(x/4 \right) \right)$ is defined only for $-4<x<4$ (Note: $\sin^{-1} \left(4/4 \right)=\frac{\pi}{2}$ and  $\tan$ is not defined at $\frac{\pi}{2}$; similarly for $x=-4$.)

 Therefore, $-\pi /2 < \theta < \pi /2$, which implies that  $\cos(\theta)=y/4> 0$.
  Therefore, $y> 0$.\\
 
        Then
        \begin{align*}
          \tan \left( \sin^{-1} \left(x/4 \right) \right) &= \tan( \theta), \\
                                                                   &= \frac{x}{y}, \\
                                                                 &= \frac{x}{\sqrt{16-x^2}}.
        \end{align*}
Note:  $\tan\left(\theta\right)$  has the same sign as $x$, since $y>0$. 
      \end{freeResponse}
  \end{enumerate}
\end{problem}  

%problem8


%problem9
\end{document} 









