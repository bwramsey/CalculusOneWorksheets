\documentclass{ximera}

\newcommand{\RR}{\mathbb R}
\renewcommand{\d}{\,d}
\newcommand{\dd}[2][]{\frac{d #1}{d #2}}
\renewcommand{\l}{\ell}
\newcommand{\ddx}{\frac{d}{dx}}
\newcommand{\dfn}{\textbf}
\newcommand{\eval}[1]{\bigg[ #1 \bigg]}
\renewcommand{\theenumii}{\textup{(\roman{enumii})}}
\renewcommand{\labelenumii}{\theenumii}

\usepackage{graphicx}
\usepackage{multicol}
\usepackage{tkz-euclide}
%\usepackage{unicode-math}

\usepackage{pgfplots}   % <- for graphics
\pgfplotsset{compat=newest}


\renewenvironment{freeResponse}{
\ifhandout\setbox0\vbox\bgroup\else
\begin{trivlist}\item[\hskip \labelsep\bfseries Solution:\hspace{2ex}]
\fi}
{\ifhandout\egroup\else
\end{trivlist}
\fi}

\newcommand*{\ZeroOverZero}{\ensuremath{\dfrac{0}{0}}}

\providecommand{\HCCondition}{0}
\newcommand{\WkstHop}[1][1]{\if\HCCondition 0
	\vspace*{\stretch{#1}} \fi} 
\newcommand{\WkstNew}{\if\HCCondition 0
	\newpage
	 \fi} 

\title[Problem 1]{Problem 1}

\begin{document}
\begin{abstract} \end{abstract}
\maketitle

% Extracted from reviewOfFamousFunctions.tex, problem #1
\begin{problem}
The graph of $g(x)=e^x$ is given below.

%	\begin{image}		
%	\includegraphics[scale=.6]{Figure1.pdf}
%	\end{image}	
	  \begin{center}
		\begin{tikzpicture}
			\begin{axis}[
				xmin=-6.3, xmax=6.3, ymin=-6.3,ymax=6.3,    
				axis lines =middle, 
				every axis y label/.style={at=(current axis.above origin),anchor=south},
				every axis x label/.style={at=(current axis.right of origin),anchor=west},
				xtick={-6,-4,...,6}, ytick={-6,-4,...,6},
				grid=major, width=3.5in, height = 3.5in,
				grid style={dashed, gray!40}
				]
				\addplot[color=red, ultra thick, smooth, domain=-6.3:3]{exp(x)};				

			\end{axis}
		\end{tikzpicture}
	\end{center}

\begin{enumerate}	
	\item  Find the domain and range of $g$.
\item  Find the values of $g(1), g(0), g(-1)$ and plot the points $(1,g(1)), (0,g(0)),$ and $(-1,g(-1))$  on the graph of $g(x)=e^x$.
\item Graph $h(x)=\ln(x)$ on the same axis as $g(x)=e^x$.
\item Find the domain and range of $h$.
\item Find the values of $h(1), h(0), h(-1), h(e), h\left(\frac{1}{e}\right)$, or say $x$ not in the domain.
\end{enumerate}

\end{problem}

\end{document}
