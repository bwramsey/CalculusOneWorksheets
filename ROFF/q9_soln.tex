\documentclass{ximera}

\newcommand{\RR}{\mathbb R}
\renewcommand{\d}{\,d}
\newcommand{\dd}[2][]{\frac{d #1}{d #2}}
\renewcommand{\l}{\ell}
\newcommand{\ddx}{\frac{d}{dx}}
\newcommand{\dfn}{\textbf}
\newcommand{\eval}[1]{\bigg[ #1 \bigg]}
\renewcommand{\theenumii}{\textup{(\roman{enumii})}}
\renewcommand{\labelenumii}{\theenumii}

\usepackage{graphicx}
\usepackage{multicol}
\usepackage{tkz-euclide}
%\usepackage{unicode-math}

\usepackage{pgfplots}   % <- for graphics
\pgfplotsset{compat=newest}


\renewenvironment{freeResponse}{
\ifhandout\setbox0\vbox\bgroup\else
\begin{trivlist}\item[\hskip \labelsep\bfseries Solution:\hspace{2ex}]
\fi}
{\ifhandout\egroup\else
\end{trivlist}
\fi}

\newcommand*{\ZeroOverZero}{\ensuremath{\dfrac{0}{0}}}

\providecommand{\HCCondition}{0}
\newcommand{\WkstHop}[1][1]{\if\HCCondition 0
	\vspace*{\stretch{#1}} \fi} 
\newcommand{\WkstNew}{\if\HCCondition 0
	\newpage
	 \fi} 


\title[Problem 9]{Problem 9}

\begin{document}
\begin{abstract} \end{abstract}
\maketitle


% Extracted from reviewOfFamousFunctions.tex, problem #9
\begin{problem}
  Simplify each of the following expressions.
  \begin{enumerate}
    \item
      $\cos^{-1} \bigl( \sin(\pi/2) \bigr)$
\begin{explanation}
        By the unit circle, $\sin(\pi/2) = 1$, and so we are looking for $\cos^{-1}(1)$.
        The range of $\cos^{-1}$ is $[0, \pi]$, and so, by properties of inverse functions, $\cos^{-1}(1) = 0$.
      \end{explanation}

    \item
      $\tan \bigl( \sin^{-1}(x/4) \bigr)$
\begin{explanation}
        Let $\theta = \sin^{-1}(x/4)$, then $\sin(\theta) = x/4$.
        We can then draw the corresponding right triangle:
        \begin{image}
          \includegraphics[scale = 0.4]{figure56l.png}
        \end{image}
        Calling the adjacent side $y$, by the Pythagorean Theorem we obtain

          $$4^2 = x^2 + y^2 \implies y = \sqrt{16-x^2}$$
Remark: Since $\theta$ is in the range of $\sin^{-1}$, it follows that $-\pi /2 \le \theta \le \pi /2$. 
On the other hand, the expression $ \tan \left( \sin^{-1} \left(x/4 \right) \right)$ is defined only for $-4<x<4$ (Note: $\sin^{-1} \left(4/4 \right)=\frac{\pi}{2}$ and  $\tan$ is not defined at $\frac{\pi}{2}$; similarly for $x=-4$.)

 Therefore, $-\pi /2 < \theta < \pi /2$, which implies that  $\cos(\theta)=y/4> 0$.
  Therefore, $y> 0$.\\
 
        Then
        \begin{align*}
          \tan \left( \sin^{-1} \left(x/4 \right) \right) &= \tan( \theta), \\
                                                                   &= \frac{x}{y}, \\
                                                                 &= \frac{x}{\sqrt{16-x^2}}.
        \end{align*}
Note:  $\tan\left(\theta\right)$  has the same sign as $x$, since $y>0$. 
      \end{explanation}
  \end{enumerate}
\end{problem}



\end{document}
