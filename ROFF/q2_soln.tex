\documentclass{ximera}

\newcommand{\RR}{\mathbb R}
\renewcommand{\d}{\,d}
\newcommand{\dd}[2][]{\frac{d #1}{d #2}}
\renewcommand{\l}{\ell}
\newcommand{\ddx}{\frac{d}{dx}}
\newcommand{\dfn}{\textbf}
\newcommand{\eval}[1]{\bigg[ #1 \bigg]}
\renewcommand{\theenumii}{\textup{(\roman{enumii})}}
\renewcommand{\labelenumii}{\theenumii}

\usepackage{graphicx}
\usepackage{multicol}
\usepackage{tkz-euclide}
%\usepackage{unicode-math}

\usepackage{pgfplots}   % <- for graphics
\pgfplotsset{compat=newest}


\renewenvironment{freeResponse}{
\ifhandout\setbox0\vbox\bgroup\else
\begin{trivlist}\item[\hskip \labelsep\bfseries Solution:\hspace{2ex}]
\fi}
{\ifhandout\egroup\else
\end{trivlist}
\fi}

\newcommand*{\ZeroOverZero}{\ensuremath{\dfrac{0}{0}}}

\providecommand{\HCCondition}{0}
\newcommand{\WkstHop}[1][1]{\if\HCCondition 0
	\vspace*{\stretch{#1}} \fi} 
\newcommand{\WkstNew}{\if\HCCondition 0
	\newpage
	 \fi} 


\title[Problem 2]{Problem 2}

\begin{document}
\begin{abstract} \end{abstract}
\maketitle


% Extracted from reviewOfFamousFunctions.tex, problem #2
\begin{problem}
	Solve the following inequalities:
	\begin{enumerate}


		\item $\displaystyle  \dfrac{x^3 \left(2x-5\right)^2}{x+1} > 0$.
\begin{explanation}
				Call $f(x) = \dfrac{x^3 \left(2x-5\right)^2}{x+1}$.
				Since the right-hand side of the inequality is zero, we are really asking where $f(x)$ will be positive.  The only way
				a nice function like $f$ can switch from positive to negative, or vice-versa, is if it is either zero
				or undefined at the point where it changes.
		
				Notice where the value of $f$ would be either zero or undefined.  Those occur at $x = -1, 0, \frac{5}{2}$. (These are the places
				where the corresponding factors of $f$ are themselves zero or undefined.)  
				We'll use those points to split the number line into four regions. Inside each of those regions, our factors are either
				always positive or always negative.
				
				\begin{tikzpicture} 
					\tkzTabInit[lgt=2,espcl=1] 
						{$x$         /1, 
						$x^3$   /1, 
						$\left(2x-5\right)^2$  /1,
						$x+1$       /1,
						$f(x)$       /1}% 
						{  , $-1$ , $0$ ,$\dfrac{5}{2}$,  }% 
					\tkzTabLine{ , - , t , - , d , + , d , + ,}
					\tkzTabLine{ , + , t , + , d , + , d , + ,}
					\tkzTabLine{ , - , t , + , d , + , d , +, }
					\tkzTabLine{ , + , t , - , d , + , d , +, }
				\end{tikzpicture} 
				
				For $x < -1$, we find $x^3 < 0$, $(2x-5)^2 > 0$, and $x+1 < 0$.  Together, that means the value of $f(x)$ is  
				$\displaystyle \dfrac{ \textrm{negative} \cdot \textrm{positive}}{\textrm{negative}}$.  It is, therefore, positive in that region, and makes up
				part of our solution.  We cannot include the endpoint $x=-1$ in the solution, as the fraction is undefined there.
				
				In the same way, we see that in the intervals $\left( 0, \frac{5}{2} \right)$ and $\left( \frac{5}{2}, 0\right)$, $f(x)$ is also positive.
				Notice that the inequality is NOT satisfied at $x=\frac{5}{2}$, since $f(\frac{5}{2})$ equals zero.  
				
				The solution is $\left( -\infty, -1 \right) \cup \left( 0, \frac{5}{2} \right) \cup \left( \frac{5}{2}, \infty \right)$.
			\end{explanation}

		\item $\displaystyle  \dfrac{2\cdot 5^x - 10^x}{x^2-5} \leq 0$.
\begin{explanation}
				Call $g(x) = \dfrac{2\cdot 5^x - 10^x}{x^2-5}$.
				Start by factoring the numerator and denominator:
				\begin{align*} 
					\dfrac{2\cdot 5^x - 10^x}{x^2-5} &= \dfrac{2\cdot 5^x - 2^x\cdot 5^x}{x^2-5}\\
						&= \dfrac{5^x(2 - 2^x)}{x^2-5}\\						
						&= \dfrac{5^x(2 - 2^x)}{(x+\sqrt{5})(x-\sqrt{5})}
				\end{align*}
				Notice that $5^x$ is always positive, $2-2^x=0$ only when $x=1$, $x+\sqrt{5}=0$ only at $x=-\sqrt{5}$, 
				and $x-\sqrt{5}=0$ only at $x=\sqrt{5}$. Our sign chart looks like this:
				
				\begin{tikzpicture} 
					\tkzTabInit[lgt=2,espcl=1] 
						{$x$         /1, 
						$5^x$   /1, 
						$2-2^x$  /1,
						$x+\sqrt{5}$  /1,
						$x-\sqrt{5}$       /1,
						$g(x)$ /1 }% 
						{  , $-\sqrt{5}$ , $1$ ,$\sqrt{5}$,  }% 
					\tkzTabLine{ , + , t , + , d, + , t , + ,}
					\tkzTabLine{ , + , t , + , d , - , t , - ,}
					\tkzTabLine{ , - , t , + , d , + , t , +, }
					\tkzTabLine{ , - , t , - , d , - , t , +, }
					\tkzTabLine{ , + , t , - , d , + , t , -, }
				\end{tikzpicture} 
				
				From the bottom row of the sign-chart, we see the solution to the inequality is:
				$\left( -\sqrt{5}, 1\right] \cup \left( \sqrt{5}, \infty\right)$.
			\end{explanation}
			
	\end{enumerate}


\end{problem}



\end{document}
