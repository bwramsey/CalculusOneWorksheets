%Add code to compile both versions from makefile at same time
\providecommand{\HCCondition}{0}
%Define each of the conditions
\ifcase\HCCondition
	%\condition=0 -> handout
	\documentclass[nooutcomes,noauthor,space,handout]{ximera}
	\title{Antiderivatives (A)}
\or	%\condition=1 -> Soln
	\documentclass[nooutcomes,noauthor]{ximera}
	\title{Antiderivatives (A) - Solutions}  
\fi

\usepackage{fullpage, multicol}
\newcommand{\RR}{\mathbb R}
\renewcommand{\d}{\,d}
\newcommand{\dd}[2][]{\frac{d #1}{d #2}}
\renewcommand{\l}{\ell}
\newcommand{\ddx}{\frac{d}{dx}}
\newcommand{\dfn}{\textbf}
\newcommand{\eval}[1]{\bigg[ #1 \bigg]}
\renewcommand{\theenumii}{\textup{(\roman{enumii})}}
\renewcommand{\labelenumii}{\theenumii}

\usepackage{graphicx}
\usepackage{multicol}
\usepackage{tkz-euclide}
%\usepackage{unicode-math}

\usepackage{pgfplots}   % <- for graphics
\pgfplotsset{compat=newest}


\renewenvironment{freeResponse}{
\ifhandout\setbox0\vbox\bgroup\else
\begin{trivlist}\item[\hskip \labelsep\bfseries Solution:\hspace{2ex}]
\fi}
{\ifhandout\egroup\else
\end{trivlist}
\fi}

\newcommand*{\ZeroOverZero}{\ensuremath{\dfrac{0}{0}}}

\providecommand{\HCCondition}{0}
\newcommand{\WkstHop}[1][1]{\if\HCCondition 0
	\vspace*{\stretch{#1}} \fi} 
\newcommand{\WkstNew}{\if\HCCondition 0
	\newpage
	 \fi}  %% we can turn off input when making a master document

  

\begin{document}
\begin{abstract}		\end{abstract}
\maketitle

\ifcase\HCCondition
%summary in here
\section*{SUMMARY of Antiderivatives:}

A function $F$ is called an \dfn{antiderivative} of $f$ on an
interval  $I$ if  $F'(x) = f(x)$, for all $x$ in  $I$.\\[0.8em]
            
If $F$ is an antiderivative of $f$ on an interval $I$, then the function $f$ has a whole \textbf{family of antiderivatives}.\\[0.8em] If a function $G$ is an antiderivative of $f$ on $I$, then $G(x)=F(x)+C$, for all $x$ in $I$, for some constant $C$.\\[0.8em]
The family of of \emph{all} antiderivatives of $f$ is denoted by $\int f(x) \d x$ and called \textbf{ indefinite integral of $f$}. Therefore, if $F$ is an antiderivative of $f$, \hspace{0.3in} $\int f(x) \d x= F(x)+C.$\\\\[0.8em]
\vspace*{\stretch{1}}	

\textbf{Basic Indefinite Integrals}

\begin{multicols}{2}
\begin{itemize}
	\item $\displaystyle \int k \d x= k x+C$\\
	\item $\displaystyle \int \frac{1}{x} \d x= \ln|x|+C$\\
	\item $\displaystyle \int x^n \d x= \frac{x^{n+1}}{n+1}+C\qquad(n\ne-1)$\\
	\item $\displaystyle \int e^x \d x= e^x + C$\\
	\item $\displaystyle \int a^x \d x= \frac{a^x}{\ln(a)}+C$\\
	\item $\displaystyle \int \cos(x) \d x = \sin(x) + C$\\
	\item $\displaystyle \int \sin(x) \d x = -\cos(x) + C$  \\
	\item $\displaystyle \int \sec^2(x) \d x = \tan(x) + C$\\
	\item $\displaystyle \int \csc^2(x) \d x = -\cot(x) + C$\\
	\item $\displaystyle \int \sec(x)\tan(x) \d x = \sec(x) + C$\\
	\item $\displaystyle \int \csc(x)\cot(x) \d x = -\csc(x) + C$\\
	\item $\displaystyle \int \frac{1}{x^2+1}\d x = \arctan{(x)} + C$\\
	\item $\displaystyle \int \frac{1}{\sqrt{1-x^2}}\d x= \arcsin{(x)}+C$\\
\end{itemize}
\end{multicols}
\vspace*{\stretch{1}}	

\newpage 
\section*{Recitation Questions}
\fi



%problem 1
\begin{problem}
Find the most general antiderivative of the function
$$ g(t) = e^{-2t} - 5 + 6\sqrt{t}-\frac{7}{t} + \frac{5}{1 + t^2} $$
		\begin{freeResponse}
		The family of all antiderivatives of g is given by 
		
		$$\int \left( e^{-2t} - 5 + 6\sqrt{t}-\frac{7}{t} + \frac{5}{1 + t^2} \right) \d t.$$\\
		
		We can apply the Sum Rule and Constant Multiple Rule.\\
	

		$$\int \left( e^{-2t} - 5 + 6\sqrt{t}-\frac{7}{t} + \frac{5}{1 + t^2}\right)\d t=\int e^{-2t}\d t - 5\int \d t + 6\int \sqrt{t}\d t-7\int \frac{1}{t}\d t + 5\int \frac{1}{1 + t^2}\d t$$\\
			
		And finally,
		$$\int \left( e^{-2t} - 5 + 6\sqrt{t}-\frac{7}{t} + \frac{5}{1 + t^2} \right) \d t= -\frac{1}{2} e^{-2t} - 5t + 4t^{\frac{3}{2}} - 7 \ln |t| + 5 \arctan{t} + C. $$\\
		\end{freeResponse}

\vspace*{\stretch{1}}	

	\end{problem}

\WkstNew

\begin{problem}
Determine the following indefinite integrals.
\begin{enumerate}
	\item $$\int \left(\sec^2{(x)}+5\right)\d x$$\\
\WkstHop
		\begin{freeResponse}
		$$\int \left(\sec^2{(x)}+5\right)\d x=\tan{(x)}+5x +C$$\\
		\end{freeResponse}
	\item $$\int \left(\sec^2{(x)}-\sec{(x)}\tan{(x)}\right)\d x$$\\
\WkstHop
		\begin{freeResponse}
		We apply the Sum Rule.\\
		 $$\int \left(\sec^2{(x)}-\sec{(x)}\tan{(x)}\right)\d x=\int \sec^2{(x)}\d x-\int\sec{(x)}\tan{(x)}\d x=\tan{(x)}-\sec{(x)}+C.$$\\
		\end{freeResponse}
	\item $$\int \frac{x+3x^5}{x^3}\d x$$\\
\WkstHop
		\begin{freeResponse}
		First, we express the function in a more convenient way.\\
		 $$\int \frac{x+3x^5}{x^3}\d x=\int \left(\frac{x}{x^3}+\frac{3x^5}{x^3}\right)\d x=\int \left(x^{-2}+3x^2\right)\d x.$$\\
		Now we apply the Sum Rule and the Constant Multiple Rule.\\
		 $$\int \frac{x+3x^5}{x^3}\d x=\int \left(x^{-2}+3x^2\right)\d x=\int x^{-2}\d x+3\int x^{2} \d x.$$\\
		  Now we determine each integral.\\
		 $$\int \frac{x+3x^5}{x^3}\d x=\int x^{-2}\d x+3\int x^{2} \d x=-\frac{1}{x}+x^3+C.$$\\
		\end{freeResponse}
	\item $$\int \frac{1+2x}{1+x^2}\d x$$\\ (HINT: Think about the derivative of $\displaystyle \ln\left( f(x) \right)$.
\WkstHop
		\begin{freeResponse}
		First, we express the function in a more convenient way.\\
		 $$\int \frac{1+2x}{1+x^2}\d x=\int \left(\frac{1}{1+x^2}+\frac{2x}{1+x^2}\right)\d x.$$\\
		Now we apply the Sum Rule.\\
		 $$\int \frac{1+2x}{1+x^2}\d x=\int\frac{1}{1+x^2}\d x+\int\frac{2x}{1+x^2}\d x.$$\\
		 Now we determine each integral.\\
		  $$\int \frac{1+2x}{1+x^2}\d x=\int\frac{1}{1+x^2}\d x+\int\frac{2x}{1+x^2}\d x=\arctan{(x)}+\ln{(1+x^2)}+C.$$\\
		\end{freeResponse}
	\item $$\int \frac{2+x^2}{1+x^2}\d x$$ 
\WkstHop
		\begin{freeResponse}
		First, we express the function in a more convenient way.\\
		 $$\int \frac{2+x^2}{1+x^2}\d x=\int \frac{1+1+x^2}{1+x^2}\d x=\int \left(\frac{1}{1+x^2}+\frac{1+x^2}{1+x^2}\right)\d x=\int \left(\frac{1}{1+x^2}+1\right)\d x.$$
		Now we apply the Sum Rule and the Constant Multiple Rule.\\
		$$\int \frac{2+x^2}{1+x^2}\d x=\int \frac{1}{1+x^2}\d x+\int\d x=\arctan{(x)}+x+C.$$
		\end{freeResponse}
	\end{enumerate}
	\end{problem}

\WkstNew


%problem 2
\begin{problem}
Assume that $f^\prime (t) = 4t^3 + 2t$ and $f(3) = 5$.  Find $f(t)$.
\WkstHop
		\begin{freeResponse}
		This is an Initial Value Problem. First we find the general solution of the differential equation.
		
		$$f(t) = t^4 + t^2 + C .$$
		Then, we find the unique solution of the IVP.
		$$ 5 = f(3) = 3^4 + 3^2 + C = 81 + 9 + C = 90 + C$$
		$$\Longrightarrow \quad  C = -85 $$
		and so
		$$ f(t) = t^4 + t^2 - 85. $$
		\end{freeResponse}
		
		
		

\end{problem}
	
	
\WkstNew	


%problem 3
\begin{problem}
 The graph of a function $f$ is the parabola given in the figure below.
  \begin{image}
    \includegraphics[scale=.8]{"Figure 1".png}
  \end{image}
  
  \begin{enumerate}

    \item
      Find a formula for $f(x)$.
      \begin{freeResponse}
        $f(x) = x^2$
      \end{freeResponse}
\WkstHop
    \item
      Suppose  $F_1$ is an anti-derivatives of $f$ satisfying $F_1(0)=2$. Sketch and label the graph of $F_1$. \\
Sketch the graphs of three more antiderivatives of $f$.\\  
      \begin{freeResponse} \hfil\
        \begin{image}
          \includegraphics[scale=.3]{"Figure 2".png}
        \end{image}
      \end{freeResponse}

\WkstHop

    \item
      Using the expression you found in part (a) and your sketches  in part (b), find the algebraic representation of  $F_1$.
      \begin{freeResponse}
       The general antiderivative is $F(x) = \frac{1}{3}x^3+C$.  Therefore,\\
 $F_1(x) = \frac{1}{3}x^3+C$, for some constant C.\\
 Since $F_1(0)=2$, it follows that $2 = \frac{1}{3}(0)^3+C=C$. Therefore, $F_1(x) = \frac{1}{3}x^3+2$.
      \end{freeResponse}
\WkstHop

    \item
      Suppose we're given $h(x) = (1/3)x^3 +17 $, what is $h'(x)$?
      \begin{freeResponse}
        $h'(x) = x^2$.
      \end{freeResponse}
\WkstHop

    \item
      What is the relationship between $f$, $F_1$, $h$, and $h'$? 
      \begin{freeResponse}
        We have $F_1' = f = h'$ and $h = F_1+C$, where $C$ is some constant. Determine this constant!
      \end{freeResponse}
  \end{enumerate}
\WkstHop
\end{problem}

\WkstNew

\begin{problem}
Given the acceleration function, the initial velocity and initial position of an object moving along a line, find the \textbf{position function}.\\[1em]
$a(t)=4\cos{t}$, $v(0)=2$, $s(0)=6$.
\begin{freeResponse}
$v(t)=4\sin{t}+C$.\\
Letting $t=0$, we get\\
$v(0)=4\sin{0}+C$ and\\
$2=C$.\\
Therefore, $v(t)=4\sin{t}+2$.\\
So,  $s(t)=-4\cos{t}+2t+C$.\\
Letting $t=0$, we get\\

 $s(0)=-4\cos{0}+2(0)+C$.\\
So,  \\
$6=-4+C$,\\
and $C=10$.\\
Therefore,\\
$s(t)=-4\cos{t}+2t+10$.

\end{freeResponse}
\WkstHop

\end{problem}
\WkstNew

\begin{problem}
	
Consider an object moving along a line with velocity $v(t)=\pi \sin(\pi t)$ on $[0,2]$ and initial position $s(0)=0$.  Time is measured in seconds and velocity in m/s.

\begin{enumerate}
	
	\item Determine the position function, $s(t)$, on $[0,2]$.
	
	\begin{freeResponse}	
		General antiderivative of $v(t)$: $s(t)=-\cos(\pi t)+C$\\
		Apply initial position: $s(t)=-\cos(\pi t)+C \implies 0=s(0)=-\cos(0)+C \implies C=1$\\
		Position function: $s(t)=-\cos(\pi t)+1$
	\end{freeResponse}
\WkstHop
	
	\item Mark the position of the object at the time $t=1$ on the line below.
	  \begin{image}
    \includegraphics[scale=.4]{figure3.png}
  \end{image}
		\begin{freeResponse}
		$s(1)=-\cos(\pi)+1=-(-1)+1=2$
		  \begin{image}
    \includegraphics[scale=.4]{figure4.png}
  \end{image}
	\end{freeResponse}
\WkstHop
	
	\item Determine the average velocity, $v_{av}$, of the object during the interval $[0,2]$.
	
		\begin{freeResponse}	
	\begin{align*}
	v_{av}&=\frac{s(2)-s(0)}{2-0}\\
	&=\frac{-\cos(2\pi)+1-(-\cos(0)+1)}{2}\\
	&=\frac{-\cos(2\pi)+\cos(0)}{2}\\
	&=\frac{-1+1}{2}=0	
	\end{align*}
	\end{freeResponse}
\WkstHop	
	
	\item Determine when the motion is in the positive direction.
	
	
	\begin{freeResponse}	
	Motion is in the positive direction when $v(t)>0$.
	
	\begin{align*}
	v(t)>0 & \iff \pi \sin(\pi t)>0\\
	& \iff \sin(\pi t)>0\\
	& \implies 0<\pi t <\pi\\
	& \iff 0<t<1	
	\end{align*}
	\end{freeResponse}

\WkstHop		

	\item At what time (or times) is the object farthest from the origin?

	\begin{freeResponse}	
	Object is farthest from the origin when $\mid{s(t)-0}\mid=\mid{1-\cos(\pi t)}\mid$ is maximized.  Since $s(t) \ge t$ for $t$ in $[0,2]$ we need to maximize $s(t)$.
	Finding critical points on the open interval $(0,2)$.
	\begin{align*}
	v(t)=0 &\iff \pi\sin(\pi t)=0\\
	&\implies \pi t=\pi\\
	&\iff t=1
	\end{align*}
	Finding the absolute maximum, we have to check the endpoints and the critical points:
	\begin{align*}
	s(0)&=0\\
	s(1)&=2\\
	s(2)&=0
	\end{align*}
	Hence the object is farthest from the origin at $t=1$.
	
	\end{freeResponse}
\WkstHop


\end{enumerate}
\end{problem}


\WkstNew

\begin{problem}
Consider an object moving along a straight line. The graphs of acceleration function (in $m/s^2$), the  velocity function and the position function of the object are given in the figure below.
 \begin{image}
    \includegraphics[scale=.05]{antiderivativeImage001.png}
  \end{image}
  
  Find the initial velocity, $v(0)$, and initial position, $s(0)$, of the object. (HINT: Start by determining which graph corresponds to each of $s(t)$, $v(t)$, and $a(t)$.)
\WkstHop
   \begin{freeResponse}
  It is clear that the function $B$ is the velocity. Therefore, $v(0)=3$.
    It is clear that the function $A$ is the position function. Therefore, $s(0)=0$.
  \end{freeResponse}
\end{problem}
\end{document} 


















