\documentclass{ximera}

\newcommand{\RR}{\mathbb R}
\renewcommand{\d}{\,d}
\newcommand{\dd}[2][]{\frac{d #1}{d #2}}
\renewcommand{\l}{\ell}
\newcommand{\ddx}{\frac{d}{dx}}
\newcommand{\dfn}{\textbf}
\newcommand{\eval}[1]{\bigg[ #1 \bigg]}
\renewcommand{\theenumii}{\textup{(\roman{enumii})}}
\renewcommand{\labelenumii}{\theenumii}

\usepackage{graphicx}
\usepackage{multicol}
\usepackage{tkz-euclide}
%\usepackage{unicode-math}

\usepackage{pgfplots}   % <- for graphics
\pgfplotsset{compat=newest}


\renewenvironment{freeResponse}{
\ifhandout\setbox0\vbox\bgroup\else
\begin{trivlist}\item[\hskip \labelsep\bfseries Solution:\hspace{2ex}]
\fi}
{\ifhandout\egroup\else
\end{trivlist}
\fi}

\newcommand*{\ZeroOverZero}{\ensuremath{\dfrac{0}{0}}}

\providecommand{\HCCondition}{0}
\newcommand{\WkstHop}[1][1]{\if\HCCondition 0
	\vspace*{\stretch{#1}} \fi} 
\newcommand{\WkstNew}{\if\HCCondition 0
	\newpage
	 \fi} 


\title[Problem 2]{Problem 2}

\begin{document}
\begin{abstract} \end{abstract}
\maketitle


% Extracted from antiderivatives.tex, problem #2
\begin{problem}
Determine the following indefinite integrals.
\begin{enumerate}
	\item $$\int \left(\sec^2{(x)}+5\right)\d x$$\\
\begin{explanation}
		$$\int \left(\sec^2{(x)}+5\right)\d x=\tan{(x)}+5x +C$$\\
		\end{explanation}
	\item $$\int \left(\sec^2{(x)}-\sec{(x)}\tan{(x)}\right)\d x$$\\
\begin{explanation}
		We apply the Sum Rule.\\
		 $$\int \left(\sec^2{(x)}-\sec{(x)}\tan{(x)}\right)\d x=\int \sec^2{(x)}\d x-\int\sec{(x)}\tan{(x)}\d x=\tan{(x)}-\sec{(x)}+C.$$\\
		\end{explanation}
	\item $$\int \frac{x+3x^5}{x^3}\d x$$\\
\begin{explanation}
		First, we express the function in a more convenient way.\\
		 $$\int \frac{x+3x^5}{x^3}\d x=\int \left(\frac{x}{x^3}+\frac{3x^5}{x^3}\right)\d x=\int \left(x^{-2}+3x^2\right)\d x.$$\\
		Now we apply the Sum Rule and the Constant Multiple Rule.\\
		 $$\int \frac{x+3x^5}{x^3}\d x=\int \left(x^{-2}+3x^2\right)\d x=\int x^{-2}\d x+3\int x^{2} \d x.$$\\
		  Now we determine each integral.\\
		 $$\int \frac{x+3x^5}{x^3}\d x=\int x^{-2}\d x+3\int x^{2} \d x=-\frac{1}{x}+x^3+C.$$\\
		\end{explanation}
	\item $$\int \frac{1+2x}{1+x^2}\d x$$\\ (HINT: Think about the derivative of $\displaystyle \ln\left( f(x) \right)$.
\begin{explanation}
		First, we express the function in a more convenient way.\\
		 $$\int \frac{1+2x}{1+x^2}\d x=\int \left(\frac{1}{1+x^2}+\frac{2x}{1+x^2}\right)\d x.$$\\
		Now we apply the Sum Rule.\\
		 $$\int \frac{1+2x}{1+x^2}\d x=\int\frac{1}{1+x^2}\d x+\int\frac{2x}{1+x^2}\d x.$$\\
		 Now we determine each integral.\\
		  $$\int \frac{1+2x}{1+x^2}\d x=\int\frac{1}{1+x^2}\d x+\int\frac{2x}{1+x^2}\d x=\arctan{(x)}+\ln{(1+x^2)}+C.$$\\
		\end{explanation}
	\item $$\int \frac{2+x^2}{1+x^2}\d x$$ 
\begin{explanation}
		First, we express the function in a more convenient way.\\
		 $$\int \frac{2+x^2}{1+x^2}\d x=\int \frac{1+1+x^2}{1+x^2}\d x=\int \left(\frac{1}{1+x^2}+\frac{1+x^2}{1+x^2}\right)\d x=\int \left(\frac{1}{1+x^2}+1\right)\d x.$$
		Now we apply the Sum Rule and the Constant Multiple Rule.\\
		$$\int \frac{2+x^2}{1+x^2}\d x=\int \frac{1}{1+x^2}\d x+\int\d x=\arctan{(x)}+x+C.$$
		\end{explanation}
	\end{enumerate}
	\end{problem}



\end{document}
