% Extracted from antiderivatives.tex, problem #4
\begin{problem}
 The graph of a function $f$ is the parabola given in the figure below.
  \begin{image}
    \includegraphics[scale=.8]{../xmPictures/Figure1.png}
  \end{image}
  
  \begin{enumerate}

    \item
      Find a formula for $f(x)$.
      \begin{freeResponse}
        $f(x) = x^2$
      \end{freeResponse}
\WkstHop
    \item
      Suppose  $F_1$ is an anti-derivatives of $f$ satisfying $F_1(0)=2$. Sketch and label the graph of $F_1$. \\
Sketch the graphs of three more antiderivatives of $f$.\\  
      \begin{freeResponse} \hfil\
        \begin{image}
          \includegraphics[scale=.3]{../xmPictures/Figure2.png}
        \end{image}
      \end{freeResponse}

\WkstHop

    \item
      Using the expression you found in part (a) and your sketches  in part (b), find the algebraic representation of  $F_1$.
      \begin{freeResponse}
       The general antiderivative is $F(x) = \frac{1}{3}x^3+C$.  Therefore,\\
 $F_1(x) = \frac{1}{3}x^3+C$, for some constant C.\\
 Since $F_1(0)=2$, it follows that $2 = \frac{1}{3}(0)^3+C=C$. Therefore, $F_1(x) = \frac{1}{3}x^3+2$.
      \end{freeResponse}
\WkstHop

    \item
      Suppose we're given $h(x) = (1/3)x^3 +17 $, what is $h'(x)$?
      \begin{freeResponse}
        $h'(x) = x^2$.
      \end{freeResponse}
\WkstHop

    \item
      What is the relationship between $f$, $F_1$, $h$, and $h'$? 
      \begin{freeResponse}
        We have $F_1' = f = h'$ and $h = F_1+C$, where $C$ is some constant. Determine this constant!
      \end{freeResponse}
  \end{enumerate}
\WkstHop
\end{problem}
