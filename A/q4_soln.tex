\documentclass{ximera}

\newcommand{\RR}{\mathbb R}
\renewcommand{\d}{\,d}
\newcommand{\dd}[2][]{\frac{d #1}{d #2}}
\renewcommand{\l}{\ell}
\newcommand{\ddx}{\frac{d}{dx}}
\newcommand{\dfn}{\textbf}
\newcommand{\eval}[1]{\bigg[ #1 \bigg]}
\renewcommand{\theenumii}{\textup{(\roman{enumii})}}
\renewcommand{\labelenumii}{\theenumii}

\usepackage{graphicx}
\usepackage{multicol}
\usepackage{tkz-euclide}
%\usepackage{unicode-math}

\usepackage{pgfplots}   % <- for graphics
\pgfplotsset{compat=newest}


\renewenvironment{freeResponse}{
\ifhandout\setbox0\vbox\bgroup\else
\begin{trivlist}\item[\hskip \labelsep\bfseries Solution:\hspace{2ex}]
\fi}
{\ifhandout\egroup\else
\end{trivlist}
\fi}

\newcommand*{\ZeroOverZero}{\ensuremath{\dfrac{0}{0}}}

\providecommand{\HCCondition}{0}
\newcommand{\WkstHop}[1][1]{\if\HCCondition 0
	\vspace*{\stretch{#1}} \fi} 
\newcommand{\WkstNew}{\if\HCCondition 0
	\newpage
	 \fi} 


\title[Problem 4]{Problem 4}

\begin{document}
\begin{abstract} \end{abstract}
\maketitle


% Extracted from antiderivatives.tex, problem #4
\begin{problem}
 The graph of a function $f$ is the parabola given in the figure below.
  \begin{image}
    \includegraphics[scale=.8]{Figure1.png}
  \end{image}
  
  \begin{enumerate}

    \item
      Find a formula for $f(x)$.
      \begin{explanation}
        $f(x) = x^2$
      \end{explanation}
\item
      Suppose  $F_1$ is an anti-derivatives of $f$ satisfying $F_1(0)=2$. Sketch and label the graph of $F_1$. \\
Sketch the graphs of three more antiderivatives of $f$.\\  
      \begin{explanation} \hfil\
        \begin{image}
          \includegraphics[scale=.3]{Figure2.png}
        \end{image}
      \end{explanation}

\item
      Using the expression you found in part (a) and your sketches  in part (b), find the algebraic representation of  $F_1$.
      \begin{explanation}
       The general antiderivative is $F(x) = \frac{1}{3}x^3+C$.  Therefore,\\
 $F_1(x) = \frac{1}{3}x^3+C$, for some constant C.\\
 Since $F_1(0)=2$, it follows that $2 = \frac{1}{3}(0)^3+C=C$. Therefore, $F_1(x) = \frac{1}{3}x^3+2$.
      \end{explanation}
\item
      Suppose we're given $h(x) = (1/3)x^3 +17 $, what is $h'(x)$?
      \begin{explanation}
        $h'(x) = x^2$.
      \end{explanation}
\item
      What is the relationship between $f$, $F_1$, $h$, and $h'$? 
      \begin{explanation}
        We have $F_1' = f = h'$ and $h = F_1+C$, where $C$ is some constant. Determine this constant!
      \end{explanation}
  \end{enumerate}
\end{problem}



\end{document}
