\documentclass{ximera}

\newcommand{\RR}{\mathbb R}
\renewcommand{\d}{\,d}
\newcommand{\dd}[2][]{\frac{d #1}{d #2}}
\renewcommand{\l}{\ell}
\newcommand{\ddx}{\frac{d}{dx}}
\newcommand{\dfn}{\textbf}
\newcommand{\eval}[1]{\bigg[ #1 \bigg]}
\renewcommand{\theenumii}{\textup{(\roman{enumii})}}
\renewcommand{\labelenumii}{\theenumii}

\usepackage{graphicx}
\usepackage{multicol}
\usepackage{tkz-euclide}
%\usepackage{unicode-math}

\usepackage{pgfplots}   % <- for graphics
\pgfplotsset{compat=newest}


\renewenvironment{freeResponse}{
\ifhandout\setbox0\vbox\bgroup\else
\begin{trivlist}\item[\hskip \labelsep\bfseries Solution:\hspace{2ex}]
\fi}
{\ifhandout\egroup\else
\end{trivlist}
\fi}

\newcommand*{\ZeroOverZero}{\ensuremath{\dfrac{0}{0}}}

\providecommand{\HCCondition}{0}
\newcommand{\WkstHop}[1][1]{\if\HCCondition 0
	\vspace*{\stretch{#1}} \fi} 
\newcommand{\WkstNew}{\if\HCCondition 0
	\newpage
	 \fi} 


\title[Problem 6]{Problem 6}

\begin{document}
\begin{abstract} \end{abstract}
\maketitle


% Extracted from antiderivatives.tex, problem #6
\begin{problem}
	
Consider an object moving along a line with velocity $v(t)=\pi \sin(\pi t)$ on $[0,2]$ and initial position $s(0)=0$.  Time is measured in seconds and velocity in m/s.

\begin{enumerate}
	
	\item Determine the position function, $s(t)$, on $[0,2]$.
	
	\begin{explanation}	
		General antiderivative of $v(t)$: $s(t)=-\cos(\pi t)+C$\\
		Apply initial position: $s(t)=-\cos(\pi t)+C \implies 0=s(0)=-\cos(0)+C \implies C=1$\\
		Position function: $s(t)=-\cos(\pi t)+1$
	\end{explanation}
\item Mark the position of the object at the time $t=1$ on the line below.
	  \begin{image}
    \includegraphics[scale=.4]{figure3.png}
  \end{image}
		\begin{explanation}
		$s(1)=-\cos(\pi)+1=-(-1)+1=2$
		  \begin{image}
    \includegraphics[scale=.4]{figure4.png}
  \end{image}
	\end{explanation}
\item Determine the average velocity, $v_{av}$, of the object during the interval $[0,2]$.
	
		\begin{explanation}	
	\begin{align*}
	v_{av}&=\frac{s(2)-s(0)}{2-0}\\
	&=\frac{-\cos(2\pi)+1-(-\cos(0)+1)}{2}\\
	&=\frac{-\cos(2\pi)+\cos(0)}{2}\\
	&=\frac{-1+1}{2}=0	
	\end{align*}
	\end{explanation}
\item Determine when the motion is in the positive direction.
	
	
	\begin{explanation}	
	Motion is in the positive direction when $v(t)>0$.
	
	\begin{align*}
	v(t)>0 & \iff \pi \sin(\pi t)>0\\
	& \iff \sin(\pi t)>0\\
	& \implies 0<\pi t <\pi\\
	& \iff 0<t<1	
	\end{align*}
	\end{explanation}

\item At what time (or times) is the object farthest from the origin?

	\begin{explanation}	
	Object is farthest from the origin when $\mid{s(t)-0}\mid=\mid{1-\cos(\pi t)}\mid$ is maximized.  Since $s(t) \ge t$ for $t$ in $[0,2]$ we need to maximize $s(t)$.
	Finding critical points on the open interval $(0,2)$.
	\begin{align*}
	v(t)=0 &\iff \pi\sin(\pi t)=0\\
	&\implies \pi t=\pi\\
	&\iff t=1
	\end{align*}
	Finding the absolute maximum, we have to check the endpoints and the critical points:
	\begin{align*}
	s(0)&=0\\
	s(1)&=2\\
	s(2)&=0
	\end{align*}
	Hence the object is farthest from the origin at $t=1$.
	
	\end{explanation}
\end{enumerate}
\end{problem}



\end{document}
