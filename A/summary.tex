\documentclass{ximera}

\newcommand{\RR}{\mathbb R}
\renewcommand{\d}{\,d}
\newcommand{\dd}[2][]{\frac{d #1}{d #2}}
\renewcommand{\l}{\ell}
\newcommand{\ddx}{\frac{d}{dx}}
\newcommand{\dfn}{\textbf}
\newcommand{\eval}[1]{\bigg[ #1 \bigg]}
\renewcommand{\theenumii}{\textup{(\roman{enumii})}}
\renewcommand{\labelenumii}{\theenumii}

\usepackage{graphicx}
\usepackage{multicol}
\usepackage{tkz-euclide}
%\usepackage{unicode-math}

\usepackage{pgfplots}   % <- for graphics
\pgfplotsset{compat=newest}


\renewenvironment{freeResponse}{
\ifhandout\setbox0\vbox\bgroup\else
\begin{trivlist}\item[\hskip \labelsep\bfseries Solution:\hspace{2ex}]
\fi}
{\ifhandout\egroup\else
\end{trivlist}
\fi}

\newcommand*{\ZeroOverZero}{\ensuremath{\dfrac{0}{0}}}

\providecommand{\HCCondition}{0}
\newcommand{\WkstHop}[1][1]{\if\HCCondition 0
	\vspace*{\stretch{#1}} \fi} 
\newcommand{\WkstNew}{\if\HCCondition 0
	\newpage
	 \fi} 

\title[Summary]{Summary}

\begin{document}
\begin{abstract} \end{abstract}
\maketitle



A function $F$ is called an \dfn{antiderivative} of $f$ on an
interval  $I$ if  $F'(x) = f(x)$, for all $x$ in  $I$.\\[0.8em]
            
If $F$ is an antiderivative of $f$ on an interval $I$, then the function $f$ has a whole \textbf{family of antiderivatives}.\\[0.8em] If a function $G$ is an antiderivative of $f$ on $I$, then $G(x)=F(x)+C$, for all $x$ in $I$, for some constant $C$.\\[0.8em]
The family of of \emph{all} antiderivatives of $f$ is denoted by $\int f(x) \d x$ and called \textbf{ indefinite integral of $f$}. Therefore, if $F$ is an antiderivative of $f$, \hspace{0.3in} $\int f(x) \d x= F(x)+C.$\\\\[0.8em]
\vspace*{\stretch{1}}	

\textbf{Basic Indefinite Integrals}

\begin{multicols}{2}
\begin{itemize}
	\item $\displaystyle \int k \d x= k x+C$\\
	\item $\displaystyle \int \frac{1}{x} \d x= \ln|x|+C$\\
	\item $\displaystyle \int x^n \d x= \frac{x^{n+1}}{n+1}+C\qquad(n\ne-1)$\\
	\item $\displaystyle \int e^x \d x= e^x + C$\\
	\item $\displaystyle \int a^x \d x= \frac{a^x}{\ln(a)}+C$\\
	\item $\displaystyle \int \cos(x) \d x = \sin(x) + C$\\
	\item $\displaystyle \int \sin(x) \d x = -\cos(x) + C$  \\
	\item $\displaystyle \int \sec^2(x) \d x = \tan(x) + C$\\
	\item $\displaystyle \int \csc^2(x) \d x = -\cot(x) + C$\\
	\item $\displaystyle \int \sec(x)\tan(x) \d x = \sec(x) + C$\\
	\item $\displaystyle \int \csc(x)\cot(x) \d x = -\csc(x) + C$\\
	\item $\displaystyle \int \frac{1}{x^2+1}\d x = \arctan{(x)} + C$\\
	\item $\displaystyle \int \frac{1}{\sqrt{1-x^2}}\d x= \arcsin{(x)}+C$\\
\end{itemize}
\end{multicols}



\end{document}
