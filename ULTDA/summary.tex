\documentclass{ximera}

\newcommand{\RR}{\mathbb R}
\renewcommand{\d}{\,d}
\newcommand{\dd}[2][]{\frac{d #1}{d #2}}
\renewcommand{\l}{\ell}
\newcommand{\ddx}{\frac{d}{dx}}
\newcommand{\dfn}{\textbf}
\newcommand{\eval}[1]{\bigg[ #1 \bigg]}
\renewcommand{\theenumii}{\textup{(\roman{enumii})}}
\renewcommand{\labelenumii}{\theenumii}

\usepackage{graphicx}
\usepackage{multicol}
\usepackage{tkz-euclide}
%\usepackage{unicode-math}

\usepackage{pgfplots}   % <- for graphics
\pgfplotsset{compat=newest}


\renewenvironment{freeResponse}{
\ifhandout\setbox0\vbox\bgroup\else
\begin{trivlist}\item[\hskip \labelsep\bfseries Solution:\hspace{2ex}]
\fi}
{\ifhandout\egroup\else
\end{trivlist}
\fi}

\newcommand*{\ZeroOverZero}{\ensuremath{\dfrac{0}{0}}}

\providecommand{\HCCondition}{0}
\newcommand{\WkstHop}[1][1]{\if\HCCondition 0
	\vspace*{\stretch{#1}} \fi} 
\newcommand{\WkstNew}{\if\HCCondition 0
	\newpage
	 \fi} 

\title[Summary]{Summary}

\begin{document}
\begin{abstract} \end{abstract}
\maketitle


\subsection*{Infinite Limits}
\begin{description}
	\item $\displaystyle \lim_{x\to a} f(x) = \infty$ means the values of $f(x)$ grow arbitrarily large as $x$ approaches $a$.
	\item $\displaystyle \lim_{x\to a} f(x) = -\infty$ means the values of $|f(x)|$ grow arbitrarily large as $x$ approaches $a$ with $f(x)$ negative.	
\end{description}
\subsection*{ Limits at Infinity}
\begin{description}	
	\item $\displaystyle \lim_{x\to \infty} f(x) =L$ means the values of $f(x)$ becomes arbitrarily close to $L$ by making $x$ sufficiently large.
	\item $\displaystyle \lim_{x\to -\infty} f(x) =L$ means the values of $f(x)$ becomes arbitrarily close to $L$ by making $|x|$ sufficiently large with $x$ negative.
\end{description}
 \subsection*{Vertical and Horizontal Asymptotes:} 
 
  \begin{description}
    \item A function $f$ has a  \textbf{vertical asymptote} at $x=a$ if at least one of the following conditions hold: 
    	\begin{itemize}
    		\item $\displaystyle \lim_{x\to a} f(x) = \pm \infty$
    		\item $\displaystyle \lim_{x\to a^+} f(x) = \pm \infty$
    		\item $\displaystyle \lim_{x\to a^-} f(x) = \pm \infty$
    	\end{itemize}
    \item A function $f$ has a  \textbf{horizontal asymptote} at $y=L$ if at least one of the following conditions hold: 
    	\begin{itemize}
    		\item $\displaystyle \lim_{x\to \infty} f(x) = L$
    		\item $\displaystyle \lim_{x\to -\infty} f(x) = L$
    	\end{itemize}
	\item Both vertical asymptotes and horizontal asymptotes are written as the equation of a line. Vertical asymptotes are given as
		the equation of a vertical line, and horizontal asymptotes are given as the equation of a horizontal line.
     \end{description}



\end{document}
