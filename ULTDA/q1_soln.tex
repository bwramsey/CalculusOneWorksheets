\documentclass{ximera}

\newcommand{\RR}{\mathbb R}
\renewcommand{\d}{\,d}
\newcommand{\dd}[2][]{\frac{d #1}{d #2}}
\renewcommand{\l}{\ell}
\newcommand{\ddx}{\frac{d}{dx}}
\newcommand{\dfn}{\textbf}
\newcommand{\eval}[1]{\bigg[ #1 \bigg]}
\renewcommand{\theenumii}{\textup{(\roman{enumii})}}
\renewcommand{\labelenumii}{\theenumii}

\usepackage{graphicx}
\usepackage{multicol}
\usepackage{tkz-euclide}
%\usepackage{unicode-math}

\usepackage{pgfplots}   % <- for graphics
\pgfplotsset{compat=newest}


\renewenvironment{freeResponse}{
\ifhandout\setbox0\vbox\bgroup\else
\begin{trivlist}\item[\hskip \labelsep\bfseries Solution:\hspace{2ex}]
\fi}
{\ifhandout\egroup\else
\end{trivlist}
\fi}

\newcommand*{\ZeroOverZero}{\ensuremath{\dfrac{0}{0}}}

\providecommand{\HCCondition}{0}
\newcommand{\WkstHop}[1][1]{\if\HCCondition 0
	\vspace*{\stretch{#1}} \fi} 
\newcommand{\WkstNew}{\if\HCCondition 0
	\newpage
	 \fi} 


\title[Problem 1]{Problem 1}

\begin{document}
\begin{abstract} \end{abstract}
\maketitle


% Extracted from usingLimitsToDetectAsymptotes.tex, problem #1
\begin{problem}
  \outcome{Match graphs of functions with their equations based on vertical asymptotes.}
  Without using a graphing utility, match each graph of functions in A-F with the algebraic representation of functions in a-f:

  \begin{center}
    \includegraphics[trim= 250 360 300 185]{Figure1.pdf}
  \end{center}

  \begin{enumerate}
    \item
      The function $f$ defined by $\displaystyle f(x) = \frac{x}{x^2 + 1}$.
\begin{explanation}
        Since there is no real number $x$ such that $x^2 + 1 = 0$, we have no candidates for vertical asymptotes.
        Hence the graph of $f$ should not contain any vertical asymptotes.
        The only listed graph with no vertical asymptote is graph D.
      \end{explanation}

    \item 
      The function $g$ defined by $\displaystyle g(x) = \frac{x}{x^2 -1}$.
\begin{explanation}
        \emph{Candidates} for vertical asymptotes:
        \begin{align*}
          x^2 -1 = 0 &\implies \mbox{$x = 1$ or $x = -1$.}
        \end{align*}

        Test of candidate $x = 1$:
        \begin{align*}
          \lim_{x \to 1^+} \frac{x}{x^2 -1} &= \lim_{x \to 1^+}\frac{x}{(x-1)(x+1)} = \infty \\
               \end{align*}
	\text{Note: the  limit is of the form} $\frac{ \#}{0}$, the numerator  positive, the denominator also positive. \\
       Therefore, vertical asymptote at $x = 1$.
   
        
        Test of candidate $x = -1$:
        \begin{align*}
          \lim_{x \to -1^-} \frac{x}{x^2 -1} &= \lim_{x \to -1^-} \frac{x}{(x-1)(x+1)}= -\infty \\
                \end{align*}
	Note: the limit is of the form  $\frac{ \#}{0}$, the numerator  negative, the denominator positive. \\
        Therefore, vertical asymptote at $x = -1$.
  

        Since $g(0) = 0$, the only listed graph that can match is graph C.
      \end{explanation}

\item 
      The function $h$ defined by $\displaystyle h(x) = \frac{1}{x^2 -1}$.
\begin{explanation}
        \emph{Candidates} for vertical asymptotes: $x = 1$ and $x = -1$.

        Test of candidate $x = 1$:
        \begin{align*}
          \lim_{x \to 1^+} \frac{1}{x^2 -1} &= \lim_{x \to 1^+} \frac{1}{(x-1)(x+1)} = \infty\\
             \end{align*}
	Note:  the limit is of the form $ \frac{ \#}{0}$, the numerator  positive, the denominator positive. \\
          Therefore, vertical asymptote at $x = 1$.
     

        Test of candidate $x = -1$:
        \begin{align*}
          \lim_{x \to -1^-} \frac{1}{x^2 -1} &= \lim_{x \to -1^-} \frac{1}{(x-1)(x+1)} = \infty\\
            \end{align*}
Note:  the limit is of the form $ \frac{ \#}{0}$, the numerator  positive, the denominator positive. \\
Therefore, vertical asymptote at $x = -1$.
      
        Since $h(0) = -1$, the only listed graph that can match is graph F.
      \end{explanation}

    \item 
      The function $a$ defined by $\displaystyle a(x) = \frac{x}{(x-1)^2}$.
\begin{explanation}
        Candidate for the vertical asymptote: $x = 1$.

        Test of candidate $x = 1$:

        \begin{align*}
          \lim_{x \to 1^+}\frac{x}{(x-1)^2} &= \infty \\
           \end{align*}
	Note: the limit is of the form $ \frac{ \#}{0}$, the numerator positive, the denominator positive.\\
     Therefore, vertical asymptote at $x = 1$.
       
        Since $a(0) = 0$ the graph is graph B.
      \end{explanation}

    \item 
      The function $s$ defined by $\displaystyle s(x) = \frac{1}{(x-1)^2}$.
\begin{explanation}
        Candidate for the vertical asymptote: $x = 1$.

        Test of candidate $x = 1$:

        \begin{align*}
          \lim_{x \to 1^+} \frac{1}{(x-1)^2} &= \infty \\
              \end{align*}
	Note: the limit is of the form $ \frac{\#}{0}$, the numerator positive, the denominator positive.  \\
       Therefore, vertical asymptote at $x = 1$.
    
        Since $s(0) = 1$ the graph is graph A.        
      \end{explanation}


    \item
      The function $r$ defined by $\displaystyle r(x) = \frac{x}{x+1}$.
\begin{explanation}
        Candidate for the vertical asymptote: $x = -1$.

        Test of candidate $x = -1$:

        \begin{align*}
          \lim_{x \to -1^+} \frac{x}{x+1} &= -\infty \\
              \end{align*}
	Note:  the limit is of the form $ \frac{\#}{0} $, the numerator  negative, the denominator positive.\\
       Therefore, vertical asymptote at $x = -1$.
    
        Since $r(0) = 0$ the graph is graph E.
      \end{explanation}

  \end{enumerate}

\end{problem}



\end{document}
