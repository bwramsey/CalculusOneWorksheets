\documentclass{ximera}

\newcommand{\RR}{\mathbb R}
\renewcommand{\d}{\,d}
\newcommand{\dd}[2][]{\frac{d #1}{d #2}}
\renewcommand{\l}{\ell}
\newcommand{\ddx}{\frac{d}{dx}}
\newcommand{\dfn}{\textbf}
\newcommand{\eval}[1]{\bigg[ #1 \bigg]}
\renewcommand{\theenumii}{\textup{(\roman{enumii})}}
\renewcommand{\labelenumii}{\theenumii}

\usepackage{graphicx}
\usepackage{multicol}
\usepackage{tkz-euclide}
%\usepackage{unicode-math}

\usepackage{pgfplots}   % <- for graphics
\pgfplotsset{compat=newest}


\renewenvironment{freeResponse}{
\ifhandout\setbox0\vbox\bgroup\else
\begin{trivlist}\item[\hskip \labelsep\bfseries Solution:\hspace{2ex}]
\fi}
{\ifhandout\egroup\else
\end{trivlist}
\fi}

\newcommand*{\ZeroOverZero}{\ensuremath{\dfrac{0}{0}}}

\providecommand{\HCCondition}{0}
\newcommand{\WkstHop}[1][1]{\if\HCCondition 0
	\vspace*{\stretch{#1}} \fi} 
\newcommand{\WkstNew}{\if\HCCondition 0
	\newpage
	 \fi} 


\title[Problem 3]{Problem 3}

\begin{document}
\begin{abstract} \end{abstract}
\maketitle


% Extracted from usingLimitsToDetectAsymptotes.tex, problem #3
\begin{problem}
Let $f$ be a function given by $f(x)=\ln(1+x)$.

	\begin{enumerate}
	\item Find the domain of $f$.  Write your answer in interval notation.
\begin{explanation}
	$1+x>0 \implies x>-1 \implies \text{domain:}(-1,+\infty)$
	\end{explanation}
	
	\item Find the vertical asymptotes of $f$ and \textbf{EXPLAIN} and justify your answer.
\begin{explanation}
	  The function $f(x) = \ln(1+x)$ has a vertical asymptote at $x=-1$ since $\displaystyle \lim_{x \to -1^+} \ln{(1+x)}=-\infty$. (There is no Theorem/Test to reference here, since it is checking the definition of a vertical asymptote.)\\
	Alternatively, $g(x)=\ln(x)$ has a vertical asymptote at $x=0$.  $f$ is $g$ shifted one unit left so $f$ will have a vertical asymptote at $x=-1$. 
	
	\end{explanation}

	\item Sketch a graph of $f$

\begin{explanation} 
	The graph of $f(x)=\ln(1+x)$ is show below.
  \begin{center}
    \includegraphics[scale=.5]{Figure5.pdf}
  \end{center}
	\end{explanation}
\end{enumerate}

\end{problem}



\end{document}
