% Extracted from usingLimitsToDetectAsymptotes.tex, problem #4
\begin{problem}
Let $f(x)=\frac{\ln(x)}{x - 2}$. \\
\begin{enumerate}
    \item Evaluate the limit.\\[1em]
      $$ \lim_{x \to 2^-} \frac{\ln(x)}{x - 2}$$
\WkstHop
      \begin{freeResponse}
     This limit  is of the form  $\frac{\#}{0}$, the numerator  positive, and the denominator is  negative.
Therefore, $ \lim_{x \to 2^-} \frac{\ln{x}}{x - 2} = -\infty$.
     
      \end{freeResponse}
\item Find the vertical asymptotes of $f$. \textbf{EXPLAIN} and justify your answer.\\
\WkstHop
\WkstHop

  \begin{freeResponse}
  	$x=2$ is a vertical asymptote of $f$ since $\lim_{x \to 2^-} \frac{\ln{x}}{x - 2} = -\infty$ calculated in part (a). \\
     On the other hand, we know that \\
      $\displaystyle \lim_{x \to 0^+} \ln(x)=-\infty$.\\
      So, it follows that\\
       $\displaystyle \lim_{x \to 0^+} \frac{\ln(x)}{x - 2}=\infty$,\\
       since the limit is of the form $\frac{\infty}{\#}$, the numerator negative, and  the denominator negative.
       That means $x=0$ is a vertical asymptote of $f$ since $\lim_{x \to 0^+} \frac{\ln{x}}{x - 2} = \infty$
      \end{freeResponse}
	\end{enumerate}
\end{problem}
