\documentclass{ximera}

\newcommand{\RR}{\mathbb R}
\renewcommand{\d}{\,d}
\newcommand{\dd}[2][]{\frac{d #1}{d #2}}
\renewcommand{\l}{\ell}
\newcommand{\ddx}{\frac{d}{dx}}
\newcommand{\dfn}{\textbf}
\newcommand{\eval}[1]{\bigg[ #1 \bigg]}
\renewcommand{\theenumii}{\textup{(\roman{enumii})}}
\renewcommand{\labelenumii}{\theenumii}

\usepackage{graphicx}
\usepackage{multicol}
\usepackage{tkz-euclide}
%\usepackage{unicode-math}

\usepackage{pgfplots}   % <- for graphics
\pgfplotsset{compat=newest}


\renewenvironment{freeResponse}{
\ifhandout\setbox0\vbox\bgroup\else
\begin{trivlist}\item[\hskip \labelsep\bfseries Solution:\hspace{2ex}]
\fi}
{\ifhandout\egroup\else
\end{trivlist}
\fi}

\newcommand*{\ZeroOverZero}{\ensuremath{\dfrac{0}{0}}}

\providecommand{\HCCondition}{0}
\newcommand{\WkstHop}[1][1]{\if\HCCondition 0
	\vspace*{\stretch{#1}} \fi} 
\newcommand{\WkstNew}{\if\HCCondition 0
	\newpage
	 \fi} 


\title[Problem 9]{Problem 9}

\begin{document}
\begin{abstract} \end{abstract}
\maketitle


% Extracted from usingLimitsToDetectAsymptotes.tex, problem #9
\begin{problem}
A  piecewise defined function $f$ is given by 
 
	$f(x) =   \left\{ \begin{array}{cl}
	\frac{2 x-3}{x-2}		 	&	\qquad \text{if }\hspace{0.1in}  x <  2					\\ \\
	\frac{x^2-5x +6}{x^2-4}	&	\qquad \text{if } \hspace{0.1in}   x>2 	\\ \\
		
						\end{array} \right.  $\\[1em]

  \begin{enumerate}

     \item
      Find all vertical asymptotes.   \textbf{EXPLAIN} and justify your answer by using appropriate limits.\\[1em]
\begin{explanation}
   
   Since $ \lim_{x \to 2^-} f(x)= \lim_{x \to 2^-}\frac{2x-3}{x-2}$, the form of the limit is $\frac{\#}{0}$.\\[1em]
     $ \lim_{x \to 2^-} f(x)=  \lim_{x \to 2^-}\frac{2x-3}{x-2}= -\infty$,\\[1em]
      since the numerator  positive, the denominator negative and goes to 0.\\
      Therefore, the line $x=2$ is a vertical asymptote of $f$ since $\lim_{x\to 2^-} f(x) = -\infty$.\\

      Note: The line $x=-2$ is not a vertical asymptote, since for values of $x$ near $-2$, $f(x)=\frac{2 x-3}{x-2}$. So, $f$ is continuous at $-2$.
    \end{explanation}
 \item
      Find all horizontal asymptotes. \textbf{EXPLAIN} and justify your answer by using appropriate limits.\\[1em]
\begin{explanation}
    Check End Behavior as $x \to \infty$:\\[1em]
        $\lim_{x \to \infty}\frac{x^2-5x +6}{x^2-4}$ is of the form: $\frac{\infty}{\infty}$
        
        \begin{align*}
                   \lim_{x \to \infty} \frac{x^2-5x +6}{x^2-4}
          &= \lim_{x \to \infty} \frac{x^2-5x +6}{x^2-4} \cdot \frac{\frac{1}{x^{2}}}{\frac{1}{x^{2}}} \\
          &= \lim_{x \to \infty}  \frac{1-\frac{5}{x}+\frac{6}{x^2}}{1-\frac{4}{x^{2}}}\\
          &= 1 \\
        \end{align*}
        Therefore, the line $y=1$ is a horizontal asymptote of $f$ since $\lim_{x \to \infty} f(x) = 1$.\\[2em]

         Check End Behavior as $x \to -\infty$:\\[1em]
        $\lim_{x \to -\infty}\frac{2 x-3}{x-2}$ is of the form: $\frac{\infty}{\infty}$
        
        \begin{align*}
                   \lim_{x \to -\infty}\frac{2 x-3}{x-2}	
          &= \lim_{x \to -\infty}\frac{2 x-3}{x-2}	 \cdot \frac{\frac{1}{x}}{\frac{1}{x}} \\
          &= \lim_{x \to -\infty}  \frac{2-\frac{3}{x}}{1-\frac{2}{x}}\\
          &= 2 \\
        \end{align*}
        Therefore, the line $y=2$ is a horizontal asymptote of $f$ since $\lim_{x \to -\infty} f(x) = 2$.\\[2em]  
          \end{explanation}
      \end{enumerate}
\end{problem}



\end{document}
