

%Add code to compile both versions from makefile at same time
\providecommand{\HCCondition}{0}
%Define each of the conditions
\ifcase\HCCondition
	%\condition=0 -> handout
	\documentclass[nooutcomes,noauthor,space,handout]{ximera}
	\title{Using limits to detect asymptotes (ULTDA)}  
\or	%\condition=1 -> Soln
	\documentclass[nooutcomes,noauthor]{ximera}
	\title{Using limits to detect asymptotes (ULTDA) - Solutions} 
\fi

\newcommand{\RR}{\mathbb R}
\renewcommand{\d}{\,d}
\newcommand{\dd}[2][]{\frac{d #1}{d #2}}
\renewcommand{\l}{\ell}
\newcommand{\ddx}{\frac{d}{dx}}
\newcommand{\dfn}{\textbf}
\newcommand{\eval}[1]{\bigg[ #1 \bigg]}
\renewcommand{\theenumii}{\textup{(\roman{enumii})}}
\renewcommand{\labelenumii}{\theenumii}

\usepackage{graphicx}
\usepackage{multicol}
\usepackage{tkz-euclide}
%\usepackage{unicode-math}

\usepackage{pgfplots}   % <- for graphics
\pgfplotsset{compat=newest}


\renewenvironment{freeResponse}{
\ifhandout\setbox0\vbox\bgroup\else
\begin{trivlist}\item[\hskip \labelsep\bfseries Solution:\hspace{2ex}]
\fi}
{\ifhandout\egroup\else
\end{trivlist}
\fi}

\newcommand*{\ZeroOverZero}{\ensuremath{\dfrac{0}{0}}}

\providecommand{\HCCondition}{0}
\newcommand{\WkstHop}[1][1]{\if\HCCondition 0
	\vspace*{\stretch{#1}} \fi} 
\newcommand{\WkstNew}{\if\HCCondition 0
	\newpage
	 \fi}  %% we can turn off input when making a master document
\usepackage{fullpage}



\begin{document}
\begin{abstract}		\end{abstract}
\makeTagTitle
\ifcase\HCCondition
\subsection*{Infinite Limits}
\begin{description}
	\item $\displaystyle \lim_{x\to a} f(x) = \infty$ means the values of $f(x)$ grow arbitrarily large as $x$ approaches $a$.
	\item $\displaystyle \lim_{x\to a} f(x) = -\infty$ means the values of $|f(x)|$ grow arbitrarily large as $x$ approaches $a$ with $f(x)$ negative.	
\end{description}
\subsection*{ Limits at Infinity}
\begin{description}	
	\item $\displaystyle \lim_{x\to \infty} f(x) =L$ means the values of $f(x)$ becomes arbitrarily close to $L$ by making $x$ sufficiently large.
	\item $\displaystyle \lim_{x\to -\infty} f(x) =L$ means the values of $f(x)$ becomes arbitrarily close to $L$ by making $|x|$ sufficiently large with $x$ negative.
\end{description}
 \subsection*{Vertical and Horizontal Asymptotes:} 
 
  \begin{description}
    \item A function $f$ has a  \textbf{vertical asymptote} at $x=a$ if at least one of the following conditions hold: 
    	\begin{itemize}
    		\item $\displaystyle \lim_{x\to a} f(x) = \pm \infty$
    		\item $\displaystyle \lim_{x\to a^+} f(x) = \pm \infty$
    		\item $\displaystyle \lim_{x\to a^-} f(x) = \pm \infty$
    	\end{itemize}
    \item A function $f$ has a  \textbf{horizontal asymptote} at $y=L$ if at least one of the following conditions hold: 
    	\begin{itemize}
    		\item $\displaystyle \lim_{x\to \infty} f(x) = L$
    		\item $\displaystyle \lim_{x\to -\infty} f(x) = L$
    	\end{itemize}
	\item Both vertical asymptotes and horizontal asymptotes are written as the equation of a line. Vertical asymptotes are given as
		the equation of a vertical line, and horizontal asymptotes are given as the equation of a horizontal line.
     \end{description}


	\newpage
	\subsection*{Recitation Questions}
\fi
%Problem1
\begin{problem}
  \outcome{Match graphs of functions with their equations based on vertical asymptotes.}
  Without using a graphing utility, match each graph of functions in A-F with the algebraic representation of functions in a-f:

  \begin{center}
    \includegraphics[trim= 250 360 300 185]{Figure1.pdf}
  \end{center}

  \begin{enumerate}
    \item
      The function $f$ defined by $\displaystyle f(x) = \frac{x}{x^2 + 1}$.
\WkstHop
      \begin{freeResponse}
        Since there is no real number $x$ such that $x^2 + 1 = 0$, we have no candidates for vertical asymptotes.
        Hence the graph of $f$ should not contain any vertical asymptotes.
        The only listed graph with no vertical asymptote is graph D.
      \end{freeResponse}

    \item 
      The function $g$ defined by $\displaystyle g(x) = \frac{x}{x^2 -1}$.
\WkstHop
      \begin{freeResponse}
        \emph{Candidates} for vertical asymptotes:
        \begin{align*}
          x^2 -1 = 0 &\implies \mbox{$x = 1$ or $x = -1$.}
        \end{align*}

        Test of candidate $x = 1$:
        \begin{align*}
          \lim_{x \to 1^+} \frac{x}{x^2 -1} &= \lim_{x \to 1^+}\frac{x}{(x-1)(x+1)} = \infty \\
               \end{align*}
	\text{Note: the  limit is of the form} $\frac{ \#}{0}$, the numerator  positive, the denominator also positive. \\
       Therefore, vertical asymptote at $x = 1$.
   
        
        Test of candidate $x = -1$:
        \begin{align*}
          \lim_{x \to -1^-} \frac{x}{x^2 -1} &= \lim_{x \to -1^-} \frac{x}{(x-1)(x+1)}= -\infty \\
                \end{align*}
	Note: the limit is of the form  $\frac{ \#}{0}$, the numerator  negative, the denominator positive. \\
        Therefore, vertical asymptote at $x = -1$.
  

        Since $g(0) = 0$, the only listed graph that can match is graph C.
      \end{freeResponse}

\WkstNew
    \item 
      The function $h$ defined by $\displaystyle h(x) = \frac{1}{x^2 -1}$.
\WkstHop
      \begin{freeResponse}
        \emph{Candidates} for vertical asymptotes: $x = 1$ and $x = -1$.

        Test of candidate $x = 1$:
        \begin{align*}
          \lim_{x \to 1^+} \frac{1}{x^2 -1} &= \lim_{x \to 1^+} \frac{1}{(x-1)(x+1)} = \infty\\
             \end{align*}
	Note:  the limit is of the form $ \frac{ \#}{0}$, the numerator  positive, the denominator positive. \\
          Therefore, vertical asymptote at $x = 1$.
     

        Test of candidate $x = -1$:
        \begin{align*}
          \lim_{x \to -1^-} \frac{1}{x^2 -1} &= \lim_{x \to -1^-} \frac{1}{(x-1)(x+1)} = \infty\\
            \end{align*}
Note:  the limit is of the form $ \frac{ \#}{0}$, the numerator  positive, the denominator positive. \\
Therefore, vertical asymptote at $x = -1$.
      
        Since $h(0) = -1$, the only listed graph that can match is graph F.
      \end{freeResponse}

    \item 
      The function $a$ defined by $\displaystyle a(x) = \frac{x}{(x-1)^2}$.
\WkstHop
      \begin{freeResponse}
        Candidate for the vertical asymptote: $x = 1$.

        Test of candidate $x = 1$:

        \begin{align*}
          \lim_{x \to 1^+}\frac{x}{(x-1)^2} &= \infty \\
           \end{align*}
	Note: the limit is of the form $ \frac{ \#}{0}$, the numerator positive, the denominator positive.\\
     Therefore, vertical asymptote at $x = 1$.
       
        Since $a(0) = 0$ the graph is graph B.
      \end{freeResponse}

    \item 
      The function $s$ defined by $\displaystyle s(x) = \frac{1}{(x-1)^2}$.
\WkstHop
      \begin{freeResponse}
        Candidate for the vertical asymptote: $x = 1$.

        Test of candidate $x = 1$:

        \begin{align*}
          \lim_{x \to 1^+} \frac{1}{(x-1)^2} &= \infty \\
              \end{align*}
	Note: the limit is of the form $ \frac{\#}{0}$, the numerator positive, the denominator positive.  \\
       Therefore, vertical asymptote at $x = 1$.
    
        Since $s(0) = 1$ the graph is graph A.        
      \end{freeResponse}


    \item
      The function $r$ defined by $\displaystyle r(x) = \frac{x}{x+1}$.
\WkstHop
      \begin{freeResponse}
        Candidate for the vertical asymptote: $x = -1$.

        Test of candidate $x = -1$:

        \begin{align*}
          \lim_{x \to -1^+} \frac{x}{x+1} &= -\infty \\
              \end{align*}
	Note:  the limit is of the form $ \frac{\#}{0} $, the numerator  negative, the denominator positive.\\
       Therefore, vertical asymptote at $x = -1$.
    
        Since $r(0) = 0$ the graph is graph E.
      \end{freeResponse}

  \end{enumerate}

\end{problem}
\WkstNew

%Problem2

\begin{problem}
	Sketch a possible graph of a function $g$ that satisfies the following conditions:
	\begin{align*}
	&\textrm{Domain:} [-5,-2) \cup (-2,3) \cup (3,5)
	& \lim_{x \to -2} g(x) = 3\\
	& g(1)=1
	& \lim_{x \to 1^+} g(x) = -\infty\\
	& \lim_{x \to 3} g(x) = \infty
	& \lim_{x \to 5^-} g(x) = \infty\\
	& \lim_{x \to 1^-} g(x) = 1
	& g(-5)=-1.8
	\end{align*}
\WkstHop
	\begin{freeResponse}
	First we will draw our domain on the x-axis using green.  We will use a bracket when a point is in the domain and curved parenthesis when it is not.
  \begin{center}
    \includegraphics[scale=.5]{Figure2.pdf}
  \end{center}
	Next, we draw and open circle at $(-2,3)$ because we know $\lim_{x \to -2} g(x) = 3$.  We also draw a closed circle at (1,1) and (-5,-1.8) because we know $g(1)=1$ and $g(-5)=-1.8$.  Finally we draw in our vertical asymptotes in purple as indicated by the infinite limits at $x=1, x=3, x=5$.  We draw the tails of the asmyptotes in blue to indicate whether the function $g$ approaches $ -\infty$ or $\infty$.
  \begin{center}
    \includegraphics[scale=.5]{Figure3.pdf}
  \end{center}
	Finally, we connect our graph together.
  \begin{center}
    \includegraphics[scale=.5]{Figure4.pdf}
  \end{center}
	\end{freeResponse}
\end{problem}
\WkstNew

%Problem3
\begin{problem}
Let $f$ be a function given by $f(x)=\ln(1+x)$.

	\begin{enumerate}
	\item Find the domain of $f$.  Write your answer in interval notation.
\WkstHop
	\begin{freeResponse}
	$1+x>0 \implies x>-1 \implies \text{domain:}(-1,+\infty)$
	\end{freeResponse}
	
	\item Find the vertical asymptotes of $f$ and \textbf{EXPLAIN} and justify your answer.
\WkstHop
	\begin{freeResponse}
	  The function $f(x) = \ln(1+x)$ has a vertical asymptote at $x=-1$ since $\displaystyle \lim_{x \to -1^+} \ln{(1+x)}=-\infty$. (There is no Theorem/Test to reference here, since it is checking the definition of a vertical asymptote.)\\
	Alternatively, $g(x)=\ln(x)$ has a vertical asymptote at $x=0$.  $f$ is $g$ shifted one unit left so $f$ will have a vertical asymptote at $x=-1$. 
	
	\end{freeResponse}

	\item Sketch a graph of $f$

\WkstHop
\WkstHop
	\begin{freeResponse} 
	The graph of $f(x)=\ln(1+x)$ is show below.
  \begin{center}
    \includegraphics[scale=.5]{Figure5.pdf}
  \end{center}
	\end{freeResponse}
\end{enumerate}

\end{problem}
\WkstNew

%Problem4
\begin{problem}
Let $f(x)=\frac{\ln(x)}{x - 2}$. \\
\begin{enumerate}
    \item Evaluate the limit.\\[1em]
      $$ \lim_{x \to 2^-} \frac{\ln(x)}{x - 2}$$
\WkstHop
      \begin{freeResponse}
     This limit  is of the form  $\frac{\#}{0}$, the numerator  positive, and the denominator is  negative.
Therefore, $ \lim_{x \to 2^-} \frac{\ln{x}}{x - 2} = -\infty$.
     
      \end{freeResponse}
\item Find the vertical asymptotes of $f$. \textbf{EXPLAIN} and justify your answer.\\
\WkstHop
\WkstHop

  \begin{freeResponse}
  	$x=2$ is a vertical asymptote of $f$ since $\lim_{x \to 2^-} \frac{\ln{x}}{x - 2} = -\infty$ calculated in part (a). \\
     On the other hand, we know that \\
      $\displaystyle \lim_{x \to 0^+} \ln(x)=-\infty$.\\
      So, it follows that\\
       $\displaystyle \lim_{x \to 0^+} \frac{\ln(x)}{x - 2}=\infty$,\\
       since the limit is of the form $\frac{\infty}{\#}$, the numerator negative, and  the denominator negative.
       That means $x=0$ is a vertical asymptote of $f$ since $\lim_{x \to 0^+} \frac{\ln{x}}{x - 2} = \infty$
      \end{freeResponse}
	\end{enumerate}
\end{problem}
\WkstNew


%Problem5

\begin{problem}
	Select the meaning of $\lim_{x \to \infty} f(x)=6$.  Support your explanation graphically.

	\begin{enumerate}
	\item As $x$ becomes arbitrarily negatively large, $f(x)$ approaches $6$.
	\item As $x$ becomes arbitrarily positively large, $f(x)$ approaches $6$.
	\item As $x$ approaches 6, $f(x)$ becomes arbitrarily negatively large.
	\item As $x$ approaches 6, $f(x)$ becomes arbitrarily positively large.
	\end{enumerate}
\WkstHop
	\begin{freeResponse}
	As $x$ becomes arbitrarily positively large, $f(x)$ approaches $6$.  Here are two examples of what this function might look like:
	\begin{image}
	\includegraphics[scale=.5]{figure8.png}
	\end{image}
	\begin{image}
	\includegraphics[scale=.5]{figure9.png}
	\end{image}
	\end{freeResponse}

\end{problem}


\begin{problem}
Evaluate the following limits.
	\begin{enumerate}
		\item $\lim_{x \to \infty} \frac{\sqrt[3]{x^9+5}}{3x^3+ \sqrt{4x^6+1}}$\\[1in]
\WkstHop
		\begin{freeResponse}
		This limit is of the form: $\frac{\infty}{\infty}$.\\[1em]
		This will be a very detailed solution. Make sure you can justify every step below.
		\begin{align*}
		\lim_{x \to \infty} \frac{\sqrt[3]{x^9+5}}{3x^3+ \sqrt{4x^6+1}}
		&=\lim_{x \to \infty} \frac{\sqrt[3]{x^9+5}}{3x^3+ \sqrt{4x^6+1}}\cdot \frac{\frac{1}{x^3}}{\frac{1}{x^3}}\\
		&=\lim_{x \to \infty} \frac{\frac{\sqrt[3]{x^9+5}}{x^3}}{\frac{3x^3+ \sqrt{4x^6+1}}{x^3}}\\
		&=\lim_{x \to \infty}  \frac{\sqrt[3]{\frac{x^9+5}{x^9}}}{\frac{3x^3}{x^3}+ \sqrt{\frac{4x^6+1}{x^6}}}\\
		&=\lim_{x \to \infty}  \frac{\sqrt[3]{1+\frac{5}{x^9}}}{3+ \sqrt{4+\frac{1}{x^6}}}\\
		&=  \frac{\lim_{x \to \infty}\sqrt[3]{1+\frac{5}{x^9}}}{\lim_{x \to \infty}\Bigl(3+ \sqrt{4+\frac{1}{x^6}}\Bigr)}\\
		&=  \frac{\sqrt[3]{\lim_{x \to \infty}\Bigl(1+\frac{5}{x^9}\Bigr)}}{3+ \lim_{x \to \infty}\sqrt{4+\frac{1}{x^6}}}\\
		&=  \frac{\sqrt[3]{1+\lim_{x \to \infty}\frac{5}{x^9}}}{3+\sqrt{ \lim_{x \to \infty}\Bigl(4+\frac{1}{x^6}\Bigr)}}\\
		&=  \frac{\sqrt[3]{1+5\lim_{x \to \infty}\Bigl(\frac{1}{x}\Bigr)^{9}}}{3+\sqrt{4+ \lim_{x \to \infty}\frac{1}{x^6}}}\\
		&=  \frac{\sqrt[3]{1+5\Bigl(\lim_{x \to \infty}\frac{1}{x}\Bigr)^{9}}}{3+\sqrt{4+ \Bigl(\lim_{x \to \infty}\frac{1}{x}\Bigr)^{6}}}\\
		&=  \frac{\sqrt[3]{1+5\Bigl(0\Bigr)^{9}}}{3+\sqrt{4+ \Bigl(0\Bigr)^{6}}}\\
		&=\frac{1}{3+ \sqrt{4}}\\
		&=\frac{1}{5}
		\end{align*}
		\end{freeResponse}

		\item $\lim_{x \to \infty} \frac{\sin(9x)}{5x}$
\WkstHop
		\begin{freeResponse}
		 Since $-1 \le  \sin(9x) \le 1$ for all $x$ , and we can divide by $5x>0$, $\frac{-1}{5x} \le  \frac{\sin(9x)}{5x} \le \frac{1}{5x}$.\\
		  We have $\lim_{x \to \infty} \left(-\frac{1}{5x} \right) =-\frac{1}{5}\lim_{x \to \infty} \left(\frac{1}{x} \right) = 0$ and $\lim_{x \to \infty} \frac{1}{5x} = 0$.  The Squeeze theorem implies that
        \[
          \lim_{x \to \infty} \frac{\sin(9x)}{5x} = 0
        \]
		\end{freeResponse}
	\end{enumerate}
\end{problem}

\WkstNew
%problem 7
\begin{problem}
The function $f$ is  defined by $\displaystyle f(x) = \frac{6e^x+1}{3e^x+5}$.
  \begin{enumerate}
    \item
      Find all vertical asymptotes of $f$.  \textbf{EXPLAIN} and justify your answer by using appropriate limits.
\WkstHop
 \begin{freeResponse}
       Since the exponential function, $e^x$, is always positive, it follows that the denominator of the function $f$ is also  always positive.
       Since the exponential function, $e^x$, is continuous on $(-\infty,\infty)$, it follows that the functions  $6e^x+1$ and $3e^x+5$ are also continuous on $(-\infty,\infty)$.
       Therefore, the function $f$ is also continuous there, since it is a quotient of two continuous functions, and the function in the denominator is never $0$. \\
       
       Therefore, $f$ has no vertical asymptotes, since it is continuous on $(-\infty,\infty)$.
        \end{freeResponse}
          \item
      Find all horizontal asymptotes of $f$.  \textbf{EXPLAIN} and justify your answer by using appropriate limits.
\WkstHop
       \begin{freeResponse}
       	Check End Behavior as $x \to \infty$:\\[2em]
        $\lim_{x \to \infty} \frac{6e^x+1}{3e^x+5}$ is of the form: $\frac{\infty}{\infty}$\\[2em]
     
  
                 $  \lim_{x \to \infty} \frac{6e^x+1}{3e^x+5}=   \lim_{x \to \infty} \frac{6e^x+1}{3e^x+5}\cdot \frac{\frac{1}{e^x}}{\frac{1}{e^x}}=$ \\
                  
 
        $= \lim_{x \to \infty}  \frac{6+\frac{1}{e^x}}{3+\frac{5}{e^x}}=  \frac{6 + 0}{3 +0}=2$ \\
        
              
       The line $y = 2$ is a horizontal asymptote of $f$ because $\lim_{x\to \infty} f(x) = 2$.\\
  
        Check End Behavior as $x \to -\infty$:\\


         $ \lim_{x \to -\infty}\frac{6e^x+1}{3e^x+5}  =\frac{0+1}{0+5}= \frac{1}{5}$ \\
     
     
      The line $y = \frac{1}{5}$ is a horizontal asymptote of $f$ because $\lim_{x\to -\infty} f(x) = \frac{1}{5}$.
	
      \end{freeResponse}
 \end{enumerate}

\end{problem}

\WkstNew

%problem 8
\begin{problem}
The function $f$ is  defined by $\displaystyle f(x) = \frac{\sqrt{2x^2 + 1}}{3x-5}$.
  \begin{enumerate}
    \item
      Find all vertical asymptotes of $f$.  \textbf{EXPLAIN} and ustify your answer by using appropriate limits.
\WkstHop
 \begin{freeResponse}
        Candidate for vertical asymptotes: $3x - 5 = 0 \implies x = 5/3$.

        Test of candidate $x = 5/3$:
        \begin{align*}
          &\lim_{x \to \frac{5}{3}^+} \frac{\sqrt{2x^2 + 1}}{3x-5} = \infty\\
           \end{align*}
	Note: the limit is of the form $\frac{\#}{0}$, the numerator positive, and the denominator\\
	 $3x-5=3(x-5/3)$ is positive, since $x>5/3$.  \\
	
	The line $x = \frac{5}{3}$ is a vertical asymptote of $f$ since $\lim_{x\to \frac{5}{3}^+} f(x) = \infty$.
       

       \end{freeResponse}

    \item
      Find all horizontal asymptotes of $f$.   \textbf{EXPLAIN} and justify your answer by using appropriate limits..
 
\WkstHop
        \begin{freeResponse}
        Check End Behavior as $x \to \infty$:\\\\
        $\lim_{x \to \infty} \frac{\sqrt{2x^2 + 1}}{3x-5}$ is of the form: $\frac{\infty}{\infty}$
        \begin{align*}
                   \lim_{x \to \infty} \frac{\sqrt{2x^2 + 1}}{3x-5}
          &= \lim_{x \to \infty} \frac{\sqrt{2x^2 + 1}}{3x-5} \cdot \frac{\frac{1}{x}}{\frac{1}{x}} \\
          &= \lim_{x \to \infty}  \frac{\frac{\sqrt{2x^2 + 1}}{\sqrt{x^2}}}{3 - \frac{5}{x}}\\
          &= \lim_{x \to \infty}  \frac{\sqrt{2 + \frac{1}{x^2}}}{3 - \frac{5}{x}} \\
          &= \frac{\sqrt{2}}{3}
        \end{align*}
         The line $y = \frac{\sqrt{2}}{3}$ is a horizontal asymptote of $f$ since $\lim_{x\to \infty} f(x) = \frac{\sqrt{2}}{3}$.


        Check End Behavior as $x \to -\infty$:\\\\
         $\lim_{x \to -\infty} \frac{\sqrt{2x^2 + 1}}{3x-5}$ is of the form: $\frac{\infty}{\infty}$
        \begin{align*}
          \lim_{x \to -\infty} \frac{\sqrt{2x^2 + 1}}{3x-5}
          &= \lim_{x \to -\infty} \frac{\sqrt{2x^2 + 1}}{3x-5} \cdot \frac{\frac{1}{-x}}{\frac{1}{-x}} \\
          &= \lim_{x \to -\infty}  \frac{\frac{\sqrt{2x^2 + 1}}{\sqrt{x^2}}}{-3 + \frac{5}{x}}\\
          &= \lim_{x \to -\infty}  \frac{\sqrt{2 + \frac{1}{x^2}}}{-3 + \frac{5}{x}} \\
          &= \frac{-\sqrt{2}}{3} \\
            \end{align*}
         The line $y = -\frac{\sqrt{2}}{3}$ is a horizontal asymptote of $f$ since $\lim_{x\to -\infty} f(x) = -\frac{\sqrt{2}}{3}$.
      
	
      \end{freeResponse}
 \end{enumerate}
 \end{problem}
\WkstNew

%problem 9
  \begin{problem}
A  piecewise defined function $f$ is given by 
 
	$f(x) =   \left\{ \begin{array}{cl}
	\frac{2 x-3}{x-2}		 	&	\qquad \text{if }\hspace{0.1in}  x <  2					\\ \\
	\frac{x^2-5x +6}{x^2-4}	&	\qquad \text{if } \hspace{0.1in}   x>2 	\\ \\
		
						\end{array} \right.  $\\[1em]

  \begin{enumerate}

     \item
      Find all vertical asymptotes.   \textbf{EXPLAIN} and justify your answer by using appropriate limits.\\[1em]
\WkstHop
     \begin{freeResponse}
   
   Since $ \lim_{x \to 2^-} f(x)= \lim_{x \to 2^-}\frac{2x-3}{x-2}$, the form of the limit is $\frac{\#}{0}$.\\[1em]
     $ \lim_{x \to 2^-} f(x)=  \lim_{x \to 2^-}\frac{2x-3}{x-2}= -\infty$,\\[1em]
      since the numerator  positive, the denominator negative and goes to 0.\\
      Therefore, the line $x=2$ is a vertical asymptote of $f$ since $\lim_{x\to 2^-} f(x) = -\infty$.\\

      Note: The line $x=-2$ is not a vertical asymptote, since for values of $x$ near $-2$, $f(x)=\frac{2 x-3}{x-2}$. So, $f$ is continuous at $-2$.
    \end{freeResponse}
 \item
      Find all horizontal asymptotes. \textbf{EXPLAIN} and justify your answer by using appropriate limits.\\[1em]
\WkstHop
     \begin{freeResponse}
    Check End Behavior as $x \to \infty$:\\[1em]
        $\lim_{x \to \infty}\frac{x^2-5x +6}{x^2-4}$ is of the form: $\frac{\infty}{\infty}$
        
        \begin{align*}
                   \lim_{x \to \infty} \frac{x^2-5x +6}{x^2-4}
          &= \lim_{x \to \infty} \frac{x^2-5x +6}{x^2-4} \cdot \frac{\frac{1}{x^{2}}}{\frac{1}{x^{2}}} \\
          &= \lim_{x \to \infty}  \frac{1-\frac{5}{x}+\frac{6}{x^2}}{1-\frac{4}{x^{2}}}\\
          &= 1 \\
        \end{align*}
        Therefore, the line $y=1$ is a horizontal asymptote of $f$ since $\lim_{x \to \infty} f(x) = 1$.\\[2em]

         Check End Behavior as $x \to -\infty$:\\[1em]
        $\lim_{x \to -\infty}\frac{2 x-3}{x-2}$ is of the form: $\frac{\infty}{\infty}$
        
        \begin{align*}
                   \lim_{x \to -\infty}\frac{2 x-3}{x-2}	
          &= \lim_{x \to -\infty}\frac{2 x-3}{x-2}	 \cdot \frac{\frac{1}{x}}{\frac{1}{x}} \\
          &= \lim_{x \to -\infty}  \frac{2-\frac{3}{x}}{1-\frac{2}{x}}\\
          &= 2 \\
        \end{align*}
        Therefore, the line $y=2$ is a horizontal asymptote of $f$ since $\lim_{x \to -\infty} f(x) = 2$.\\[2em]  
          \end{freeResponse}
      \end{enumerate}
\end{problem}
\WkstNew


 \begin{problem}
 For the piecewise function $f$ defined by 
 
	$f(x) =   \left\{ \begin{array}{cl}
	\sin(x)		 	&	\qquad \text{if }\hspace{0.1in}  x \leq  0					\\ \\
	\frac{x^{2}}{x^{2}-4}	&	\qquad \text{if } \hspace{0.1in}  0< x< 2 	\\ \\
	\frac{x^{2}}{x^{2}+4}	&	\qquad \text{if } \hspace{0.1in}  2  \leq x 	\\ \\
						\end{array} \right.  $
  \begin{enumerate}
    \item
      Find all vertical asymptotes.   \textbf{EXPLAIN} and justify your answer by using appropriate limits.
\WkstHop
      \begin{freeResponse}
        Candidates for vertical asymptote: $x^{2}-4 = 0 \implies x= -2$,  $x=2$ (Each piece of $f$ is continuous at all other points).

       Note:  $x = -2$ is not a vertical asymptote, since $f(x)=\sin{(x)}$ near $-2$, so $\lim_{x\to -2} f(x) = \lim_{x\to -2} \sin(x) = \sin(-2)$.

        \[\lim_{x \to 2^-} \frac{x^{2}}{x^{2}-4}=\lim_{x \to 2^-} \frac{x^{2}}{(x-2)(x+2)}= -\infty \]
        
	Note: the limit is of the form $ \frac{\#}{0}$, the numerator positive, and the denominator negative.  \\

      The line $x=2$ is a vertical asymptote of $f$ because $\lim_{x\to 2^-} f(x) = -\infty$. This is the only vertical asymptote of $f$.
     


      \end{freeResponse}
       \item
      Find all horizontal asymptotes. \textbf{EXPLAIN} and justify your answer by using appropriate limits.
\WkstHop
      \begin{freeResponse}
        Check End Behavior as $x \to \infty$:\\\\
        $\lim_{x \to \infty}\frac{x^{2}}{x^{2}+4}$ is of the form: $\frac{\infty}{\infty}$
        \begin{align*}
                   \lim_{x \to \infty} \frac{x^{2}}{x^{2}+4}
          &= \lim_{x \to \infty} \frac{x^{2}}{x^{2}+4} \cdot \frac{\frac{1}{x^{2}}}{\frac{1}{x^{2}}} \\
          &= \lim_{x \to \infty}  \frac{1}{1+\frac{4}{x^{2}}}\\
          &= 1 \\
        \end{align*}
The line $y = 1$ is a horizontal asymptote of $f$ since $\lim_{x \to \infty} f(x) = 1$.\\[2em]

        Check End Behavior as $x \to -\infty$:\\\\
        $\lim_{x \to -\infty}\sin(x)$  does not exist, since the function oscillates.\\[1em]
        For example, $f(-n\pi)=\sin{(-n\pi)}=0$, while $f(-(2n+1)\frac{\pi}{2})=\sin{\Bigl(-(2n+1)\frac{\pi}{2}\Bigr)}=1$, for all positive integers $n$. 
       
There is no  horizontal asymptote as $x \to -\infty$ since $\lim_{x \to -\infty} f(x)$ does not exist.\\[1em]
	
      \end{freeResponse}

\end{enumerate}
    
\end{problem}
\WkstNew

\begin{problem}

  For the function $g$ defined by 
  \[
    g(t) = \frac{t^2 + 7t + 11}{t-3}
  \]
  \begin{enumerate}
    \item
      Find all vertical asymptotes.   \textbf{EXPLAIN} and justify your answer by using appropriate limits.
\WkstHop
      \begin{freeResponse}
         The function $g$ is a rational function which is continuous on the intervals $(-\infty, 3)$ and $(3, \infty)$. 	
         The only candidates for a vertical asymptote is $t = 3$, where $g$ is not defined.

        Test of candidate $t = 3$:
        \begin{align*}
          &\lim_{t \to 3^+} \frac{t^2 + 7t + 11}{t-3}= \infty \\
             \end{align*}
	Note: the limit is of the form $ \frac{\#}{0}$, the numerator positive, and the denominator also positive.  \\
      The line $t=3$ is a vertical asymptote of $g$ since $\lim_{t\to 3^+} g(t) = \infty$.
     


      \end{freeResponse}


    \item

      Find all horizontal asymptotes. \textbf{EXPLAIN} and justify your answer by using appropriate limits.
\WkstHop
      \begin{freeResponse}
	To find the limits, we need to divide the numerator and denominator by the highest power of $t$ in the denominator.
        \begin{align*} 
         &\lim_{t \to \infty} \frac{t^2 + 7t + 11}{t-3} = \lim_{t \to \infty} \frac{t + 7 + \frac{11}{t}}{1-\frac{3}{t}} = \infty \\
            \end{align*}
	Note:  the last limit is of the form $ \frac{\infty}{\#}$, the numerator positive, the denominator  positive. \\

        Therefore, $g$ has no horizontal asymptotes as $ t \to \infty$.
     

        \begin{align*} 
         &\lim_{t \to -\infty} \frac{t^2 + 7t + 11}{t-3} = \lim_{t \to- \infty} \frac{t +7 + \frac{11}{t}}{1-\frac{3}{t}} = -\infty \\
            \end{align*}
	Note:  the last limit is of the form $ \frac{\infty}{\#}$, the numerator negative, the denominator positive.  \\
	
        Therefore, $g$ has no horizontal asymptotes as $ t \to -\infty$ either.
     

      \end{freeResponse}
  \end{enumerate}
\end{problem}
\WkstNew


\begin{problem}
 Sketch a possible graph of a function that satisfies all of the given properties.
  (You \emph{do not} need to find a formula for the function.)
  \begin{align*}
 	\lim_{x \to -2^-} f(x) &= \infty &   f(-2) &= -5 &  f(1) &= 2 \\
	 \lim_{x \to \infty} f(x) &= -\infty & \lim_{x \to -\infty} f(x) &= 4 & \lim_{x \to 5} f(x) &= \infty \\
	 \lim_{x \to 3} f(x) &= 3 & f(3) &=  1 &\\
	  \lim_{x \to -2^+}f(x)&=-5  & f(4) &\text{ is undefined}& \lim_{x \to 4}f(x)&=3\\
  \end{align*}
\WkstHop
  \begin{freeResponse}
      There are many correct solutions to this problem.
  One possible solution can be constructed as follows:
	First we draw the given points.  We'll also draw open brackets on the x-axis at $x=4$ since $f(x)$ is not defined at $x=4$ (This is seen in green.)
	\begin{image}
   	 \includegraphics[scale = 0.4]{Figure10.png}   
	\end{image} 

	Then we draw the vertical  and horizontal asymptotes (in purple) and arrows indicating how the graph approaches the asymptote (in blue).  We also draw end-behavior as $x$ approaches $\infty$ and $-\infty$ (in blue).  We can also draw tails where we know limits at particular x-values.  Such as $ \lim_{x \to -2^+}f(x)=-5$.
	\begin{image}
	\includegraphics[scale = 0.4]{figure11.png}
	\end{image}

	Finally we connect our graph together (seen in blue)
	\begin{image}
    \includegraphics[scale = 0.4]{figure12.png}  
	\end{image}

  \end{freeResponse}
\end{problem}

\end{document} 



	
 % \item
      $\displaystyle \lim_{x \to 3} \frac{x^2 - 2x - 3}{\sqrt{x+1} - 2}$
      \begin{freeResponse}
        We have
        \begin{align*}
          \lim_{x \to 3} \frac{x^2 - 2x - 3}{\sqrt{x+1} - 2} &= \lim_{x \to 3} \frac{(x- 3)(x + 1)}{\sqrt{x+1} - 2}\ \text{This is of the form $\frac{0}{0}$} \\
          &= \lim_{x \to 3} \frac{(x- 3)(x + 1)}{\sqrt{x+1} - 2} \cdot \frac{\sqrt{x+1} + 2}{\sqrt{x+1} + 2}\\
          &= \lim_{x \to 3} \frac{(x- 3)(x + 1)(\sqrt{x+1} + 2)}{x+1 - 4}\\
          &= \lim_{x \to 3} \frac{(x- 3)(x + 1)(\sqrt{x+1} + 2)}{x-3}\\
          &= \lim_{x \to 3} (x + 1)(\sqrt{x+1} + 2) = 4 \cdot 4 = 16
        \end{align*}
      \end{freeResponse}


    %\item
      $\displaystyle \lim_{x \to 0^+} x  \sin(\ln(x))$.
      \begin{freeResponse}
	We know sine of any number is always between $-1$ and $1$, so we have $-1 \le  \sin(\ln(x)) \le 1$.
	\begin{align*}
	-1 \le& \sin(\ln(x)) \le 1  \\
       	-x \le x \cdot& \sin(\ln(x)) \le x\ \text{for}\ x\ \text{postive and near}\ 0\\
	\text{Since}\ \lim_{x \to 0^+} (-x) = 0,\ &\text{and}\ \lim_{x \to 0^+} x = 0\ \text{by the Squeeze Theorem we get:}\\
          &\lim_{x \to 0^+} x \cdot \sin(\ln(x)) = 0
 		\end{align*}
      \end{freeResponse}















