\documentclass{ximera}

\newcommand{\RR}{\mathbb R}
\renewcommand{\d}{\,d}
\newcommand{\dd}[2][]{\frac{d #1}{d #2}}
\renewcommand{\l}{\ell}
\newcommand{\ddx}{\frac{d}{dx}}
\newcommand{\dfn}{\textbf}
\newcommand{\eval}[1]{\bigg[ #1 \bigg]}
\renewcommand{\theenumii}{\textup{(\roman{enumii})}}
\renewcommand{\labelenumii}{\theenumii}

\usepackage{graphicx}
\usepackage{multicol}
\usepackage{tkz-euclide}
%\usepackage{unicode-math}

\usepackage{pgfplots}   % <- for graphics
\pgfplotsset{compat=newest}


\renewenvironment{freeResponse}{
\ifhandout\setbox0\vbox\bgroup\else
\begin{trivlist}\item[\hskip \labelsep\bfseries Solution:\hspace{2ex}]
\fi}
{\ifhandout\egroup\else
\end{trivlist}
\fi}

\newcommand*{\ZeroOverZero}{\ensuremath{\dfrac{0}{0}}}

\providecommand{\HCCondition}{0}
\newcommand{\WkstHop}[1][1]{\if\HCCondition 0
	\vspace*{\stretch{#1}} \fi} 
\newcommand{\WkstNew}{\if\HCCondition 0
	\newpage
	 \fi} 


\title[Problem 10]{Problem 10}

\begin{document}
\begin{abstract} \end{abstract}
\maketitle


% Extracted from usingLimitsToDetectAsymptotes.tex, problem #10
\begin{problem}
 For the piecewise function $f$ defined by 
 
	$f(x) =   \left\{ \begin{array}{cl}
	\sin(x)		 	&	\qquad \text{if }\hspace{0.1in}  x \leq  0					\\ \\
	\frac{x^{2}}{x^{2}-4}	&	\qquad \text{if } \hspace{0.1in}  0< x< 2 	\\ \\
	\frac{x^{2}}{x^{2}+4}	&	\qquad \text{if } \hspace{0.1in}  2  \leq x 	\\ \\
						\end{array} \right.  $
  \begin{enumerate}
    \item
      Find all vertical asymptotes.   \textbf{EXPLAIN} and justify your answer by using appropriate limits.
\begin{explanation}
        Candidates for vertical asymptote: $x^{2}-4 = 0 \implies x= -2$,  $x=2$ (Each piece of $f$ is continuous at all other points).

       Note:  $x = -2$ is not a vertical asymptote, since $f(x)=\sin{(x)}$ near $-2$, so $\lim_{x\to -2} f(x) = \lim_{x\to -2} \sin(x) = \sin(-2)$.

        \[\lim_{x \to 2^-} \frac{x^{2}}{x^{2}-4}=\lim_{x \to 2^-} \frac{x^{2}}{(x-2)(x+2)}= -\infty \]
        
	Note: the limit is of the form $ \frac{\#}{0}$, the numerator positive, and the denominator negative.  \\

      The line $x=2$ is a vertical asymptote of $f$ because $\lim_{x\to 2^-} f(x) = -\infty$. This is the only vertical asymptote of $f$.
     


      \end{explanation}
       \item
      Find all horizontal asymptotes. \textbf{EXPLAIN} and justify your answer by using appropriate limits.
\begin{explanation}
        Check End Behavior as $x \to \infty$:\\\\
        $\lim_{x \to \infty}\frac{x^{2}}{x^{2}+4}$ is of the form: $\frac{\infty}{\infty}$
        \begin{align*}
                   \lim_{x \to \infty} \frac{x^{2}}{x^{2}+4}
          &= \lim_{x \to \infty} \frac{x^{2}}{x^{2}+4} \cdot \frac{\frac{1}{x^{2}}}{\frac{1}{x^{2}}} \\
          &= \lim_{x \to \infty}  \frac{1}{1+\frac{4}{x^{2}}}\\
          &= 1 \\
        \end{align*}
The line $y = 1$ is a horizontal asymptote of $f$ since $\lim_{x \to \infty} f(x) = 1$.\\[2em]

        Check End Behavior as $x \to -\infty$:\\\\
        $\lim_{x \to -\infty}\sin(x)$  does not exist, since the function oscillates.\\[1em]
        For example, $f(-n\pi)=\sin{(-n\pi)}=0$, while $f(-(2n+1)\frac{\pi}{2})=\sin{\Bigl(-(2n+1)\frac{\pi}{2}\Bigr)}=1$, for all positive integers $n$. 
       
There is no  horizontal asymptote as $x \to -\infty$ since $\lim_{x \to -\infty} f(x)$ does not exist.\\[1em]
	
      \end{explanation}

\end{enumerate}
    
\end{problem}



\end{document}
