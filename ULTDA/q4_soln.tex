\documentclass{ximera}

\newcommand{\RR}{\mathbb R}
\renewcommand{\d}{\,d}
\newcommand{\dd}[2][]{\frac{d #1}{d #2}}
\renewcommand{\l}{\ell}
\newcommand{\ddx}{\frac{d}{dx}}
\newcommand{\dfn}{\textbf}
\newcommand{\eval}[1]{\bigg[ #1 \bigg]}
\renewcommand{\theenumii}{\textup{(\roman{enumii})}}
\renewcommand{\labelenumii}{\theenumii}

\usepackage{graphicx}
\usepackage{multicol}
\usepackage{tkz-euclide}
%\usepackage{unicode-math}

\usepackage{pgfplots}   % <- for graphics
\pgfplotsset{compat=newest}


\renewenvironment{freeResponse}{
\ifhandout\setbox0\vbox\bgroup\else
\begin{trivlist}\item[\hskip \labelsep\bfseries Solution:\hspace{2ex}]
\fi}
{\ifhandout\egroup\else
\end{trivlist}
\fi}

\newcommand*{\ZeroOverZero}{\ensuremath{\dfrac{0}{0}}}

\providecommand{\HCCondition}{0}
\newcommand{\WkstHop}[1][1]{\if\HCCondition 0
	\vspace*{\stretch{#1}} \fi} 
\newcommand{\WkstNew}{\if\HCCondition 0
	\newpage
	 \fi} 


\title[Problem 4]{Problem 4}

\begin{document}
\begin{abstract} \end{abstract}
\maketitle


% Extracted from usingLimitsToDetectAsymptotes.tex, problem #4
\begin{problem}
Let $f(x)=\frac{\ln(x)}{x - 2}$. \\
\begin{enumerate}
    \item Evaluate the limit.\\[1em]
      $$ \lim_{x \to 2^-} \frac{\ln(x)}{x - 2}$$
\begin{explanation}
     This limit  is of the form  $\frac{\#}{0}$, the numerator  positive, and the denominator is  negative.
Therefore, $ \lim_{x \to 2^-} \frac{\ln{x}}{x - 2} = -\infty$.
     
      \end{explanation}
\item Find the vertical asymptotes of $f$. \textbf{EXPLAIN} and justify your answer.\\
\begin{explanation}
  	$x=2$ is a vertical asymptote of $f$ since $\lim_{x \to 2^-} \frac{\ln{x}}{x - 2} = -\infty$ calculated in part (a). \\
     On the other hand, we know that \\
      $\displaystyle \lim_{x \to 0^+} \ln(x)=-\infty$.\\
      So, it follows that\\
       $\displaystyle \lim_{x \to 0^+} \frac{\ln(x)}{x - 2}=\infty$,\\
       since the limit is of the form $\frac{\infty}{\#}$, the numerator negative, and  the denominator negative.
       That means $x=0$ is a vertical asymptote of $f$ since $\lim_{x \to 0^+} \frac{\ln{x}}{x - 2} = \infty$
      \end{explanation}
	\end{enumerate}
\end{problem}



\end{document}
