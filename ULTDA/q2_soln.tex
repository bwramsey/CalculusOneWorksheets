\documentclass{ximera}

\newcommand{\RR}{\mathbb R}
\renewcommand{\d}{\,d}
\newcommand{\dd}[2][]{\frac{d #1}{d #2}}
\renewcommand{\l}{\ell}
\newcommand{\ddx}{\frac{d}{dx}}
\newcommand{\dfn}{\textbf}
\newcommand{\eval}[1]{\bigg[ #1 \bigg]}
\renewcommand{\theenumii}{\textup{(\roman{enumii})}}
\renewcommand{\labelenumii}{\theenumii}

\usepackage{graphicx}
\usepackage{multicol}
\usepackage{tkz-euclide}
%\usepackage{unicode-math}

\usepackage{pgfplots}   % <- for graphics
\pgfplotsset{compat=newest}


\renewenvironment{freeResponse}{
\ifhandout\setbox0\vbox\bgroup\else
\begin{trivlist}\item[\hskip \labelsep\bfseries Solution:\hspace{2ex}]
\fi}
{\ifhandout\egroup\else
\end{trivlist}
\fi}

\newcommand*{\ZeroOverZero}{\ensuremath{\dfrac{0}{0}}}

\providecommand{\HCCondition}{0}
\newcommand{\WkstHop}[1][1]{\if\HCCondition 0
	\vspace*{\stretch{#1}} \fi} 
\newcommand{\WkstNew}{\if\HCCondition 0
	\newpage
	 \fi} 


\title[Problem 2]{Problem 2}

\begin{document}
\begin{abstract} \end{abstract}
\maketitle


% Extracted from usingLimitsToDetectAsymptotes.tex, problem #2
\begin{problem}
	Sketch a possible graph of a function $g$ that satisfies the following conditions:
	\begin{align*}
	&\textrm{Domain:} [-5,-2) \cup (-2,3) \cup (3,5)
	& \lim_{x \to -2} g(x) = 3\\
	& g(1)=1
	& \lim_{x \to 1^+} g(x) = -\infty\\
	& \lim_{x \to 3} g(x) = \infty
	& \lim_{x \to 5^-} g(x) = \infty\\
	& \lim_{x \to 1^-} g(x) = 1
	& g(-5)=-1.8
	\end{align*}
\begin{explanation}
	First we will draw our domain on the x-axis using green.  We will use a bracket when a point is in the domain and curved parenthesis when it is not.
  \begin{center}
    \includegraphics[scale=.5]{figure2.pdf}
  \end{center}
	Next, we draw and open circle at $(-2,3)$ because we know $\lim_{x \to -2} g(x) = 3$.  We also draw a closed circle at (1,1) and (-5,-1.8) because we know $g(1)=1$ and $g(-5)=-1.8$.  Finally we draw in our vertical asymptotes in purple as indicated by the infinite limits at $x=1, x=3, x=5$.  We draw the tails of the asmyptotes in blue to indicate whether the function $g$ approaches $ -\infty$ or $\infty$.
  \begin{center}
    \includegraphics[scale=.5]{figure3.pdf}
  \end{center}
	Finally, we connect our graph together.
  \begin{center}
    \includegraphics[scale=.5]{figure4.pdf}
  \end{center}
	\end{explanation}
\end{problem}



\end{document}
