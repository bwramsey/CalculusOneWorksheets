\documentclass{ximera}

\newcommand{\RR}{\mathbb R}
\renewcommand{\d}{\,d}
\newcommand{\dd}[2][]{\frac{d #1}{d #2}}
\renewcommand{\l}{\ell}
\newcommand{\ddx}{\frac{d}{dx}}
\newcommand{\dfn}{\textbf}
\newcommand{\eval}[1]{\bigg[ #1 \bigg]}
\renewcommand{\theenumii}{\textup{(\roman{enumii})}}
\renewcommand{\labelenumii}{\theenumii}

\usepackage{graphicx}
\usepackage{multicol}
\usepackage{tkz-euclide}
%\usepackage{unicode-math}

\usepackage{pgfplots}   % <- for graphics
\pgfplotsset{compat=newest}


\renewenvironment{freeResponse}{
\ifhandout\setbox0\vbox\bgroup\else
\begin{trivlist}\item[\hskip \labelsep\bfseries Solution:\hspace{2ex}]
\fi}
{\ifhandout\egroup\else
\end{trivlist}
\fi}

\newcommand*{\ZeroOverZero}{\ensuremath{\dfrac{0}{0}}}

\providecommand{\HCCondition}{0}
\newcommand{\WkstHop}[1][1]{\if\HCCondition 0
	\vspace*{\stretch{#1}} \fi} 
\newcommand{\WkstNew}{\if\HCCondition 0
	\newpage
	 \fi} 


\title[Problem 6]{Problem 6}

\begin{document}
\begin{abstract} \end{abstract}
\maketitle


% Extracted from definiteIntegrals.tex, problem #6
\begin{problem}

  \begin{enumerate}
    \item
      Find the following definite integral:
	\begin{equation*}
          \int_{-4}^4 \frac{x^2 \sin^3(x)}{\sqrt{x^4 + 1}} \d x
	\end{equation*}
	\begin{explanation}
          Let $f(x) = \frac{x^2 \sin^3(x)}{\sqrt{x^4 + 1}}$.  
	  Then notice that
          \begin{align*}
            f(-x) &= \frac{(-x)^2 \sin^3(-x)}{\sqrt{(-x)^4 + 1}}  \\
                  &= \frac{x^2 (- \sin(x))^3}{\sqrt{x^4 + 1}}  \quad \text{(since } \sin(x) \text{ is an odd function)}  \\
                  &= \frac{- x^2 \sin^3(x)}{\sqrt{x^4 + 1}}  \\
                  &= -f(x).
          \end{align*}
          Thus, $f$ is an odd function and therefore
          \begin{equation*}
            \int_{-4}^4 \frac{x^2 \sin^3(x)}{\sqrt{x^4 + 1}} \d x = 0.
          \end{equation*}
          We can illustrate our computation with the graph of the function $f$, where \\[2em]
           $f(x)=\frac{x^2 \sin^3(x)}{\sqrt{x^4 + 1}}$, for $-4\leq x\leq 4$.
             \begin{image}
       \includegraphics[scale=.5]{DIimage003.png}
  \end{image}
	\end{explanation}

\item
          Suppose that $f$ is an even function.
          Given that $\int_0^6 f(x) \d x = 13$, find $\int_{-6}^6 (5f(x) + 14) \d x$.
          \begin{explanation}
            First notice that
            \begin{equation}\label{linear integral}
              \int_{-6}^6 (5f(x) + 14) \d x = 5 \int_{-6}^6 f(x) \d x + \int_{-6}^6 14 \d x.
            \end{equation}
            
            Since $f$ is even, we know that
            \begin{equation}\label{int of f}
              \int_{-6}^6 f(x) \d x = 2 \int_0^6 f(x) \d x = 2 (13) = 26.
            \end{equation}
            
            We also know that $g(x)=14$ is also an even function.  We have:
             $\int_{-6}^6 14 \d x = 2\int_{0}^6 14 \d x$
            \begin{equation}\label{int of 14}
              2\int_{0}^6 14 \d x =  2(14)(6) = 168.
            \end{equation}
            
            Then substituting equations \eqref{int of f} and \eqref{int of 14} into equation \eqref{linear integral} gives:
            \begin{equation*}
              \int_{-6}^6 (5f(x) + 14) \d x = 5(26) + 168 = 130 + 168 = 298.
            \end{equation*}
          \end{explanation}
\end{enumerate}
\end{problem}



\end{document}
