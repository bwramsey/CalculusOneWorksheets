\documentclass{ximera}

\newcommand{\RR}{\mathbb R}
\renewcommand{\d}{\,d}
\newcommand{\dd}[2][]{\frac{d #1}{d #2}}
\renewcommand{\l}{\ell}
\newcommand{\ddx}{\frac{d}{dx}}
\newcommand{\dfn}{\textbf}
\newcommand{\eval}[1]{\bigg[ #1 \bigg]}
\renewcommand{\theenumii}{\textup{(\roman{enumii})}}
\renewcommand{\labelenumii}{\theenumii}

\usepackage{graphicx}
\usepackage{multicol}
\usepackage{tkz-euclide}
%\usepackage{unicode-math}

\usepackage{pgfplots}   % <- for graphics
\pgfplotsset{compat=newest}


\renewenvironment{freeResponse}{
\ifhandout\setbox0\vbox\bgroup\else
\begin{trivlist}\item[\hskip \labelsep\bfseries Solution:\hspace{2ex}]
\fi}
{\ifhandout\egroup\else
\end{trivlist}
\fi}

\newcommand*{\ZeroOverZero}{\ensuremath{\dfrac{0}{0}}}

\providecommand{\HCCondition}{0}
\newcommand{\WkstHop}[1][1]{\if\HCCondition 0
	\vspace*{\stretch{#1}} \fi} 
\newcommand{\WkstNew}{\if\HCCondition 0
	\newpage
	 \fi} 

\title[Summary]{Summary}

\begin{document}
\begin{abstract} \end{abstract}
\maketitle


\textbf{Definition} \\
If function $f$ is continuous on interval $[a,b]$ then
\begin{center}
 {\large{$ \int_a^b f(x)\d x =  \lim_{n \to \infty} \sum_{k = 1}^n f(x_k^*) \Delta x$}}\\[2em]
 \end{center}
\textbf{Note 1: } When $f(x)\ge0$, on the interval $[a,b]$, the definite integral, $ \int_a^b f(x)\d x$, gives \\the \textbf{area} of the region between the graph of $f$ and the interval $[a,b]$ on the $x-$axis. \\[1em]
\textbf{Note 2: } When $f(x)<0$, on some interval in $[a,b]$, then the definite integral, $ \int_a^b f(x)\d x$,\\ gives the \textbf{net area} of the region between the graph of $f$ and the interval $[a,b]$ on the $x-$axis. \\[2em]
\textbf{Properties of Definite Integrals} \\[1em]
(a) $ \int_a^a f(x)\d x =0$\\[0.8em]
(b) $ \int_a^c f(x)\d x =\int_a^b f(x)\d x +  \int_b^c f(x)\d x$\\[0.8em]
(c) $ \int_a^b f(x)\d x = - \int_b^a f(x)\d x $\\[0.8em]
(d) $ \int_a^b k\cdot f(x)\d x = k\cdot \int_a^b f(x)\d x $\\[0.8em]
(e) $ \int_a^b ( f(x)+g(x))\d x = \int_a^b f(x)\d x +  \int_a^b g(x)\d x$; 
      $ \int_a^b ( f(x)-g(x))\d x = \int_a^b f(x)\d x -  \int_a^b g(x)\d x$\\[2em]
      \textbf{Note 3: } If the function $f$ is \textbf{odd}, i.e, if $f(-x)=-f(x)$, for all $x$ in $[-a,a]$, then
      \begin{center}
      {\large{ $ \int_{-a}^a f(x)\d x =0$}}
        \end{center}
           \textbf{Note 4: } If the function $f$ is \textbf{even}, i.e, if $f(-x)=f(x)$, for all $x$ in $[-a,a]$, then
      \begin{center}
      {\large{ $ \int_{-a}^a f(x)\d x =2\cdot \int_{0}^a f(x)\d x$}}
        \end{center}



\end{document}
