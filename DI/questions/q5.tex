% Extracted from definiteIntegrals.tex, problem #5
\begin{problem}

  \begin{enumerate}
    \item
    If $f$ is an odd function, why is it true that $\int_{-a}^a f(x)
    \d x = 0$?  
    Support your reasoning with a picture.
    \begin{freeResponse}
      If $f$ is odd, then the regions between the graph of $f$ and the $x$-axis from $[-a,0]$ and $[0,a]$ are reflections of each other through the origin.
      Thus, these two regions will have the same area but with opposite signs since they are on opposite sides of the $x$-axis.
      They will therefore cancel each other out.
		
     % \begin{image}
     %   \includegraphics[scale=.7]{Images/Figure1.png}
   %   \end{image}
    \end{freeResponse}

\WkstHop		

    \item
      If $f$ is an even function, why is it true that $\int_{-a}^a f(x) \d x = 2 \int_0^a f(x) \d x$?
      Support your reasoning with a picture.
      \begin{freeResponse}
        If $f$ is even, then the regions between the graph of $f$ and the $x$-axis from $[-a,0]$ and $[0,a]$ are reflections of each other through the $y$-axis.
        Thus, these two regions will have the same area with the same sign since they are on the same sides of the $x$-axis.
        So you can only find one of these areas and then double it.
     % \begin{image}
       % \includegraphics[scale=.7]{figureD2.png}
    %  \end{image}
    \end{freeResponse}
\WkstHop

    \end{enumerate}
\end{problem}
