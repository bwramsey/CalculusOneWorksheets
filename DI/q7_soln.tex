\documentclass{ximera}

\newcommand{\RR}{\mathbb R}
\renewcommand{\d}{\,d}
\newcommand{\dd}[2][]{\frac{d #1}{d #2}}
\renewcommand{\l}{\ell}
\newcommand{\ddx}{\frac{d}{dx}}
\newcommand{\dfn}{\textbf}
\newcommand{\eval}[1]{\bigg[ #1 \bigg]}
\renewcommand{\theenumii}{\textup{(\roman{enumii})}}
\renewcommand{\labelenumii}{\theenumii}

\usepackage{graphicx}
\usepackage{multicol}
\usepackage{tkz-euclide}
%\usepackage{unicode-math}

\usepackage{pgfplots}   % <- for graphics
\pgfplotsset{compat=newest}


\renewenvironment{freeResponse}{
\ifhandout\setbox0\vbox\bgroup\else
\begin{trivlist}\item[\hskip \labelsep\bfseries Solution:\hspace{2ex}]
\fi}
{\ifhandout\egroup\else
\end{trivlist}
\fi}

\newcommand*{\ZeroOverZero}{\ensuremath{\dfrac{0}{0}}}

\providecommand{\HCCondition}{0}
\newcommand{\WkstHop}[1][1]{\if\HCCondition 0
	\vspace*{\stretch{#1}} \fi} 
\newcommand{\WkstNew}{\if\HCCondition 0
	\newpage
	 \fi} 


\title[Problem 7]{Problem 7}

\begin{document}
\begin{abstract} \end{abstract}
\maketitle


% Extracted from definiteIntegrals.tex, problem #7
\begin{problem}
Evaluate the following integrals using symmetry arguments.

\begin{enumerate}

   \item    $\int_{-\pi/4}^{\pi/4} \sin(t) \d t$

      \begin{explanation}
        Since the function $f(t) = \sin(t)$ is odd,    
       $ \int_{-\pi/4}^{\pi/4} \sin(t) \d t=0$
       
         We can illustrate our computation with the graph of the function $f$, where \\[2em]
           $f(t)=\sin{t}$, for $-\dfrac{\pi}{4}\leq x\leq \dfrac{\pi}{4}$.
          \begin{image}
       \includegraphics[scale=.5]{diimage004.png}
  \end{image}

      \end{explanation}

\item $\int_{-2}^2 (1+x+3x^7-x^9) \d x$
      \begin{explanation}
	In order to take advantage of symmetry, we will split the given polynomial into a sum of its even and odd parts.  We will then integrate each part separately:
        $$\int_{-2}^2 (1+x+3x^7-x^9) \d x= \int_{-2}^{2}1 \d x + \int_{-2}^{2} (x+3x^7-x^9) \d x$$
        
        We will first integrate an even function:\\
$\int_{-2}^{2} 1 \d x =2\int_{0}^{2}1\d x=2(2)=4$

Then an odd function:\\
$\int_{-2}^{2} (x+3x^7-x^9) \d x=0$

Therefore:$\int_{-2}^2 (1+x+3x^2-x^9) \d x=4$
      \end{explanation}

\item
     $\int_{-\pi}^{\pi} x\cos(x) \d x$ 
      \begin{explanation}
         
        $\int_{-\pi}^{\pi} x\cos(x) \d x=0$ , since the product of an odd and even function is odd.\\[1em]
           We can illustrate our computation with the graph of the function $f$, where \\[2em]
           $f(x)=x\cos{x}$, for $-\pi\leq x\leq \pi$.
          \begin{image}
       \includegraphics[scale=.5]{diimage005.png}
  \end{image}

      \end{explanation}
\end{enumerate}
\end{problem}



\end{document}
