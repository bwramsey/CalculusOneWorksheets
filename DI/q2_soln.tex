\documentclass{ximera}

\newcommand{\RR}{\mathbb R}
\renewcommand{\d}{\,d}
\newcommand{\dd}[2][]{\frac{d #1}{d #2}}
\renewcommand{\l}{\ell}
\newcommand{\ddx}{\frac{d}{dx}}
\newcommand{\dfn}{\textbf}
\newcommand{\eval}[1]{\bigg[ #1 \bigg]}
\renewcommand{\theenumii}{\textup{(\roman{enumii})}}
\renewcommand{\labelenumii}{\theenumii}

\usepackage{graphicx}
\usepackage{multicol}
\usepackage{tkz-euclide}
%\usepackage{unicode-math}

\usepackage{pgfplots}   % <- for graphics
\pgfplotsset{compat=newest}


\renewenvironment{freeResponse}{
\ifhandout\setbox0\vbox\bgroup\else
\begin{trivlist}\item[\hskip \labelsep\bfseries Solution:\hspace{2ex}]
\fi}
{\ifhandout\egroup\else
\end{trivlist}
\fi}

\newcommand*{\ZeroOverZero}{\ensuremath{\dfrac{0}{0}}}

\providecommand{\HCCondition}{0}
\newcommand{\WkstHop}[1][1]{\if\HCCondition 0
	\vspace*{\stretch{#1}} \fi} 
\newcommand{\WkstNew}{\if\HCCondition 0
	\newpage
	 \fi} 


\title[Problem 2]{Problem 2}

\begin{document}
\begin{abstract} \end{abstract}
\maketitle


% Extracted from definiteIntegrals.tex, problem #2
\begin{problem}
Let $f(x)$ and $g(x)$ be functions for which we only know the following:
$$ \int_1^4 f(x)\d x = 7	\qquad	\int_2^4 f(x)\d x = 5	\qquad	\int_1^4 g(x)\d x = 2 $$
Compute the following integrals, if possible.  If it is not possible, give examples explaining why not.
	\begin{enumerate}
	
	%part a 
	\item  $\int_1^4 (8f(x) - 7g(x))\d x $
		\begin{explanation}
			\begin{align*}
			\int_1^4 (8f(x) - 7g(x))\d x &= 8 \int_1^4 f(x) \d x - 7 \int_1^4 g(x) \d x  \\
			&= 8(7) - 7(2) \\
			&= 56 - 14 = 42
			\end{align*}
		\end{explanation}
		
%part b
	\item  $\int_1^2 (-f(x)) \d x $
		\begin{explanation}
		First notice that
			\begin{equation*}
			\int_1^4 f(x) \d x = \int_1^2 f(x) \d x + \int_2^4 f(x) \d x
			\end{equation*}
			Therefore,
			\begin{equation*}
			\int_1^4 f(x) \d x- \int_2^4 f(x) \d x = \int_1^2 f(x) \d x 
			\end{equation*}
			
		So
			\begin{align*}
			\int_1^2 (-f(x)) \d x &= - \int_1^2 f(x) \d x  \\
			&= - \left( \int_1^4 f(x)\d x - \int_2^4 f(x)\d x \right)  \\
			&= - (7 - 5) = -2.
			\end{align*}
			
			%\begin{align*}
			%\int_1^2 (-f(x)) \d x &= - \int_1^2 f(x) \d x  \\
			%&= - \left( \int_1^4 f(x)\d x + \int_4^2 f(x)\d x \right)  \\
			%&= - \left( \int_1^4 f(x)\d x - \int_2^4 f(x)\d x \right)  \\
			%&= - (7 - 5) = -2
			%\end{align*}
		\end{explanation}
		
		
%part c
	\item  $\int_1^4 \left| f(x) \right| \d x$
		\begin{explanation}
		We are not given enough information to solve this integral because we do not know the regions where $f$ is positive or negative.
		Consider the following two functions $f_1(x)$ and $f_2(x)$:  
		\begin{image}
\includegraphics[scale=.4]{diimage002.png}
\includegraphics[scale=.6]{diimage001.png}
\end{image}

			
	Just using geometry, one can check that
	$$ \int_1^4 f_1(x)\d x = 7	\qquad	\int_2^4 f_1(x)\d x = 5	\qquad	\int_1^4 f_2(x)\d x = 7	\qquad	\int_2^4 f_2(x)\d x = 5 $$
	and so both $f_1$ and $f_2$ satisfy the assumptions of $f$.  But notice that
	$$\int_1^4 \left| f_1(x) \right| \d x =11	\qquad	\text{and}		\qquad	\int_1^4 \left| f_2(x) \right| \d x = 7  $$
	These two examples demonstrate that we were not given enough information to solve this problem.
		
		\end{explanation}

%part d
	\item  $\int_1^4 \left( 2 - x + f(x) \right) \d x$
		\begin{explanation}
		First notice that since the integral is linear over addition:
			\begin{equation}\label{3d}
			\int_1^4 \left( 2 - x + f(x) \right) \d x = \int_1^4 2 \d x - \int_1^4 x \d x + \int_1^4 f(x) \d x = \int_1^4 2 \d x - \int_1^4 x \d x + 7.
			\end{equation}
		By using geometry, we can see that
			\begin{equation*}
			\int_1^4 2 \d x = 2(4-1) = 6
			\end{equation*}
			\begin{equation*}
			\int_1^4 x \d x = 1(4-1) + \frac{1}{2} (4-1)(4-1) = 3 + \frac{9}{2} = \frac{15}{2}.
			\end{equation*}
		Then substituting into equation \eqref{3d} gives:
			\begin{equation*}
			\int_1^4 \left( 2 - x + f(x) \right) \d x = 6 - \frac{15}{2} + 7 = \frac{11}{2}.
			\end{equation*}
		\end{explanation}
\end{enumerate}
\end{problem}



\end{document}
