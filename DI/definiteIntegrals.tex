%Add code to compile both versions from makefile at same time
\providecommand{\HCCondition}{0}
%Define each of the conditions
\ifcase\HCCondition
	%\condition=0 -> handout
	\documentclass[nooutcomes,noauthor,space,handout]{ximera}
	\title{Definite integrals (DI)}
\or	%\condition=1 -> Soln
	\documentclass[nooutcomes,noauthor]{ximera}
	\title{Definite integrals (DI) - Solutions}  
\fi


\usepackage{fullpage}
\newcommand{\RR}{\mathbb R}
\renewcommand{\d}{\,d}
\newcommand{\dd}[2][]{\frac{d #1}{d #2}}
\renewcommand{\l}{\ell}
\newcommand{\ddx}{\frac{d}{dx}}
\newcommand{\dfn}{\textbf}
\newcommand{\eval}[1]{\bigg[ #1 \bigg]}
\renewcommand{\theenumii}{\textup{(\roman{enumii})}}
\renewcommand{\labelenumii}{\theenumii}

\usepackage{graphicx}
\usepackage{multicol}
\usepackage{tkz-euclide}
%\usepackage{unicode-math}

\usepackage{pgfplots}   % <- for graphics
\pgfplotsset{compat=newest}


\renewenvironment{freeResponse}{
\ifhandout\setbox0\vbox\bgroup\else
\begin{trivlist}\item[\hskip \labelsep\bfseries Solution:\hspace{2ex}]
\fi}
{\ifhandout\egroup\else
\end{trivlist}
\fi}

\newcommand*{\ZeroOverZero}{\ensuremath{\dfrac{0}{0}}}

\providecommand{\HCCondition}{0}
\newcommand{\WkstHop}[1][1]{\if\HCCondition 0
	\vspace*{\stretch{#1}} \fi} 
\newcommand{\WkstNew}{\if\HCCondition 0
	\newpage
	 \fi}  


\begin{document}
\begin{abstract}

\end{abstract}
\maketitle
\ifcase\HCCondition
%summary in here
\section*{SUMMARY of Definite Integrals:}

\textbf{Definition} \\
If function $f$ is continuous on interval $[a,b]$ then
\begin{center}
 {\large{$ \int_a^b f(x)\d x =  \lim_{n \to \infty} \sum_{k = 1}^n f(x_k^*) \Delta x$}}\\[2em]
 \end{center}
\textbf{Note 1: } When $f(x)\ge0$, on the interval $[a,b]$, the definite integral, $ \int_a^b f(x)\d x$, gives \\the \textbf{area} of the region between the graph of $f$ and the interval $[a,b]$ on the $x-$axis. \\[1em]
\textbf{Note 2: } When $f(x)<0$, on some interval in $[a,b]$, then the definite integral, $ \int_a^b f(x)\d x$,\\ gives the \textbf{net area} of the region between the graph of $f$ and the interval $[a,b]$ on the $x-$axis. \\[2em]
\textbf{Properties of Definite Integrals} \\[1em]
(a) $ \int_a^a f(x)\d x =0$\\[0.8em]
(b) $ \int_a^c f(x)\d x =\int_a^b f(x)\d x +  \int_b^c f(x)\d x$\\[0.8em]
(c) $ \int_a^b f(x)\d x = - \int_b^a f(x)\d x $\\[0.8em]
(d) $ \int_a^b k\cdot f(x)\d x = k\cdot \int_a^b f(x)\d x $\\[0.8em]
(e) $ \int_a^b ( f(x)+g(x))\d x = \int_a^b f(x)\d x +  \int_a^b g(x)\d x$; 
      $ \int_a^b ( f(x)-g(x))\d x = \int_a^b f(x)\d x -  \int_a^b g(x)\d x$\\[2em]
      \textbf{Note 3: } If the function $f$ is \textbf{odd}, i.e, if $f(-x)=-f(x)$, for all $x$ in $[-a,a]$, then
      \begin{center}
      {\large{ $ \int_{-a}^a f(x)\d x =0$}}
        \end{center}
           \textbf{Note 4: } If the function $f$ is \textbf{even}, i.e, if $f(-x)=f(x)$, for all $x$ in $[-a,a]$, then
      \begin{center}
      {\large{ $ \int_{-a}^a f(x)\d x =2\cdot \int_{0}^a f(x)\d x$}}
        \end{center}
\WkstNew

\section*{Recitation Questions}
\fi



%problem 1
%problem 2
\begin{problem}
  Consider the following limit of Riemann sums of a function $g$ on $[a, b]$:
  \[
    \lim_{n \to \infty} \sum_{k = 1}^n (x_k^* + \cos(x_k^*)) \Delta x\mbox{ , $[0, \pi]$.}
  \]
  Express the limit as a definite integral. Use geometry to evaluate the resulting definite integral.
  \begin{freeResponse}
    We have 
    \[
      \int_0^\pi(x + \cos(x))\d x = \lim_{n \to \infty} \sum_{k = 1}^n (x_k^* + \cos(x_k^*)) \Delta x.
    \]
    
    To evaluate this definite integral, we'll start by writing $\displaystyle \int_0^\pi(x + \cos(x))\d x = \int_0^\pi x \d x + \int_0^\pi \cos(x) \d x$.
    
    The first term is the definite integral $\displaystyle \int_0^\pi x \d x$ which is the net area under the graph of $y = x$ on the interval $[0,\pi]$. This is a triangle of base $\pi$ and height $\pi$, so it has area $\dfrac{1}{2} \pi ^ 2$.
 
    The second term $\displaystyle \int_0^\pi \cos(x) \d x$ is the net area between the graph of $y = \cos(x)$ and the $x$-axis on the interval $[0, \pi]$. In the interval $[0,\frac{\pi}{2}]$ the function is positive, and in the interval $[\frac{\pi}{2}, \pi]$ the function is negative. By the symmetry of the cosine function, these two regions have the same geometric area, so the net area is zero.
 
    $\displaystyle \int_0^\pi(x + \cos(x))\d x = \int_0^\pi x \d x + \int_0^\pi \cos(x) \d x = \dfrac{\pi^2}{2}$.
  \end{freeResponse}
\WkstHop
\end{problem}

\WkstNew

%problem 3
\begin{problem}
Let $f(x)$ and $g(x)$ be functions for which we only know the following:
$$ \int_1^4 f(x)\d x = 7	\qquad	\int_2^4 f(x)\d x = 5	\qquad	\int_1^4 g(x)\d x = 2 $$
Compute the following integrals, if possible.  If it is not possible, give examples explaining why not.
	\begin{enumerate}
	
	%part a 
	\item  $\int_1^4 (8f(x) - 7g(x))\d x $
		\begin{freeResponse}
			\begin{align*}
			\int_1^4 (8f(x) - 7g(x))\d x &= 8 \int_1^4 f(x) \d x - 7 \int_1^4 g(x) \d x  \\
			&= 8(7) - 7(2) \\
			&= 56 - 14 = 42
			\end{align*}
		\end{freeResponse}
		
\WkstHop		
		
	%part b
	\item  $\int_1^2 (-f(x)) \d x $
		\begin{freeResponse}
		First notice that
			\begin{equation*}
			\int_1^4 f(x) \d x = \int_1^2 f(x) \d x + \int_2^4 f(x) \d x
			\end{equation*}
			Therefore,
			\begin{equation*}
			\int_1^4 f(x) \d x- \int_2^4 f(x) \d x = \int_1^2 f(x) \d x 
			\end{equation*}
			
		So
			\begin{align*}
			\int_1^2 (-f(x)) \d x &= - \int_1^2 f(x) \d x  \\
			&= - \left( \int_1^4 f(x)\d x - \int_2^4 f(x)\d x \right)  \\
			&= - (7 - 5) = -2.
			\end{align*}
			
			%\begin{align*}
			%\int_1^2 (-f(x)) \d x &= - \int_1^2 f(x) \d x  \\
			%&= - \left( \int_1^4 f(x)\d x + \int_4^2 f(x)\d x \right)  \\
			%&= - \left( \int_1^4 f(x)\d x - \int_2^4 f(x)\d x \right)  \\
			%&= - (7 - 5) = -2
			%\end{align*}
		\end{freeResponse}
		
		
\WkstHop		
	%part c
	\item  $\int_1^4 \left| f(x) \right| \d x$
		\begin{freeResponse}
		We are not given enough information to solve this integral because we do not know the regions where $f$ is positive or negative.
		Consider the following two functions $f_1(x)$ and $f_2(x)$:  
		\begin{image}
\includegraphics[scale=.4]{DIimage002.png}
\includegraphics[scale=.6]{DIimage001.png}
\end{image}

			
	Just using geometry, one can check that
	$$ \int_1^4 f_1(x)\d x = 7	\qquad	\int_2^4 f_1(x)\d x = 5	\qquad	\int_1^4 f_2(x)\d x = 7	\qquad	\int_2^4 f_2(x)\d x = 5 $$
	and so both $f_1$ and $f_2$ satisfy the assumptions of $f$.  But notice that
	$$\int_1^4 \left| f_1(x) \right| \d x =11	\qquad	\text{and}		\qquad	\int_1^4 \left| f_2(x) \right| \d x = 7  $$
	These two examples demonstrate that we were not given enough information to solve this problem.
		
		\end{freeResponse}

\WkstHop
	%part d
	\item  $\int_1^4 \left( 2 - x + f(x) \right) \d x$
		\begin{freeResponse}
		First notice that since the integral is linear over addition:
			\begin{equation}\label{3d}
			\int_1^4 \left( 2 - x + f(x) \right) \d x = \int_1^4 2 \d x - \int_1^4 x \d x + \int_1^4 f(x) \d x = \int_1^4 2 \d x - \int_1^4 x \d x + 7.
			\end{equation}
		By using geometry, we can see that
			\begin{equation*}
			\int_1^4 2 \d x = 2(4-1) = 6
			\end{equation*}
			\begin{equation*}
			\int_1^4 x \d x = 1(4-1) + \frac{1}{2} (4-1)(4-1) = 3 + \frac{9}{2} = \frac{15}{2}.
			\end{equation*}
		Then substituting into equation \eqref{3d} gives:
			\begin{equation*}
			\int_1^4 \left( 2 - x + f(x) \right) \d x = 6 - \frac{15}{2} + 7 = \frac{11}{2}.
			\end{equation*}
		\end{freeResponse}
\WkstHop
	\end{enumerate}
\end{problem}

\WkstNew
%problem 4
\begin{problem}
  Evaluate the following sums:
  \begin{enumerate}
  \item $\sum_{k=1}^{4} k^5 $
    \begin{freeResponse}
      $\sum_{k=1}^{4}k^5 = 1^5 + 2^5 + 3^5 + 4^5 = 1 + 32 + 243 +
      1024 = 1300$.
    \end{freeResponse}

\WkstHop

  \item $\sum_{k=1}^{400} (5(k+1)^2 + 3) $
    \begin{freeResponse}
      \begin{align*}
        \sum_{k=1}^{400} (5(k+1)^2 + 3) 
        &= \sum_{k=1}^{400} (5(k^2 + 2k + 1) + 3) \\
        &= \sum_{k=1}^{400} (5k^2 + 10k + 8) \\
        &= \sum_{k=1}^{400} 5k^2 + \sum_{k=1}^{400} 10k + \sum_{k=1}^{400} 8  \\
        &= 5\sum_{k=1}^{400}k^2 + 10 \sum_{k=1}^{400} k + 8(400)  \\
        &= 5 \left( \frac{400(400+1)(2(400) + 1)}{6} \right) + 10 \left( \frac{400(400+1)}{2} \right) + 3,200  \\
        &= 5 (200)(401)(267) + 10(200)(401) + 3,200  \\
        &= 107,067,000 + 802,000 + 3,200 = 107,872,200
      \end{align*}
    \end{freeResponse}
\WkstHop

  \end{enumerate}
\end{problem}

%problem 5
\WkstNew

%problem 6
\begin{problem}
Use geometry to evaluate the definite integral.  Sketch the graph of the function and shade the relevant regions.
\begin{enumerate}


	\item $\int_1^3 (2x-4) \d x$

		\begin{freeResponse}
		\begin{image}
		\includegraphics[scale=.4]{DIimage007.png}
		
		\end{image}
		
		We want to find the net area of the shaded region.  We have two identical triangles, each with base $1$ and height $2$.  Since the triangles will have identical area and one lies above and the other below the $x$-axis, the total area will be $0$.
		
		Therefore, $\int_1^3 (2x-4) \d x=0$
		
		\end{freeResponse}

\WkstHop

	\item $\int_1^3 |2x-4| \d x$

		\begin{freeResponse}
		
		The functions changes the sign at $x=2$. Therefore, we will split the integral into two integrals:
		
		 $\int_1^3 |2x-4| \d x=\int_1^2 |2x-4| \d x+\int_2^3 |2x-4| \d x$.
		 
		 Since the function is negative for $x<2$ and positive for $x>2$, we write
		 
		$\int_1^3 |2x-4| \d x=-\int_1^2 (2x-4) \d x+\int_2^3 (2x-4) \d x=1+1=2$.
		
		\end{freeResponse}

\WkstHop

	\item  $\int_0^1 (2x-4) \d x$

		\begin{freeResponse}
		\begin{image}
		\includegraphics[scale=.35]{DIimage008.png}
		\end{image}
		
		Here we need to find the area of a trapezoid.  The trapezoid is made up of a rectangular top with area $(2)(1)=2$ and triangle bottom with area $(1/2)(1)(2)=1$.  This gives us an area of $3$, but, since $f$ is negative on the interval $(0,1)$, the value of the integral is the negative of this area.
		
		Therefore,$\int_0^1 (2x-4) \d x=-3$
		
		
		\end{freeResponse}

\WkstHop

	\item  $\int_{-1}^3 \sqrt{4-(x-1)^2} \d x$
		\begin{freeResponse}
		
		\begin{image}
		\includegraphics[scale=.4]{DIimage006.png}
		\end{image}
		
		
		We are looking for the area of the entire semi-circle above the $x$-axis.  This is a circle with radius $2$ so we have $(1/2)\pi 2^2=2 \pi$
		
		
		$\int_{-1}^3 \sqrt{4-(x-1)^2} \d x= 2\pi$
		
		
		\end{freeResponse}
\WkstHop

\end{enumerate}

\end{problem}

\WkstNew

\begin{problem}

  \begin{enumerate}
    \item
    If $f$ is an odd function, why is it true that $\int_{-a}^a f(x)
    \d x = 0$?  
    Support your reasoning with a picture.
    \begin{freeResponse}
      If $f$ is odd, then the regions between the graph of $f$ and the $x$-axis from $[-a,0]$ and $[0,a]$ are reflections of each other through the origin.
      Thus, these two regions will have the same area but with opposite signs since they are on opposite sides of the $x$-axis.
      They will therefore cancel each other out.
		
     % \begin{image}
     %   \includegraphics[scale=.7]{Images/Figure1.png}
   %   \end{image}
    \end{freeResponse}

\WkstHop		

    \item
      If $f$ is an even function, why is it true that $\int_{-a}^a f(x) \d x = 2 \int_0^a f(x) \d x$?
      Support your reasoning with a picture.
      \begin{freeResponse}
        If $f$ is even, then the regions between the graph of $f$ and the $x$-axis from $[-a,0]$ and $[0,a]$ are reflections of each other through the $y$-axis.
        Thus, these two regions will have the same area with the same sign since they are on the same sides of the $x$-axis.
        So you can only find one of these areas and then double it.
     % \begin{image}
       % \includegraphics[scale=.7]{figureD2.png}
    %  \end{image}
    \end{freeResponse}
\WkstHop

    \end{enumerate}
\end{problem}

\WkstNew
%problem 2
\begin{problem}

  \begin{enumerate}
    \item
      Find the following definite integral:
	\begin{equation*}
          \int_{-4}^4 \frac{x^2 \sin^3(x)}{\sqrt{x^4 + 1}} \d x
	\end{equation*}
	\begin{freeResponse}
          Let $f(x) = \frac{x^2 \sin^3(x)}{\sqrt{x^4 + 1}}$.  
	  Then notice that
          \begin{align*}
            f(-x) &= \frac{(-x)^2 \sin^3(-x)}{\sqrt{(-x)^4 + 1}}  \\
                  &= \frac{x^2 (- \sin(x))^3}{\sqrt{x^4 + 1}}  \quad \text{(since } \sin(x) \text{ is an odd function)}  \\
                  &= \frac{- x^2 \sin^3(x)}{\sqrt{x^4 + 1}}  \\
                  &= -f(x).
          \end{align*}
          Thus, $f$ is an odd function and therefore
          \begin{equation*}
            \int_{-4}^4 \frac{x^2 \sin^3(x)}{\sqrt{x^4 + 1}} \d x = 0.
          \end{equation*}
          We can illustrate our computation with the graph of the function $f$, where \\[2em]
           $f(x)=\frac{x^2 \sin^3(x)}{\sqrt{x^4 + 1}}$, for $-4\leq x\leq 4$.
             \begin{image}
       \includegraphics[scale=.5]{DIimage003.png}
  \end{image}
	\end{freeResponse}

\WkstHop
		
	\item
          Suppose that $f$ is an even function.
          Given that $\int_0^6 f(x) \d x = 13$, find $\int_{-6}^6 (5f(x) + 14) \d x$.
          \begin{freeResponse}
            First notice that
            \begin{equation}\label{linear integral}
              \int_{-6}^6 (5f(x) + 14) \d x = 5 \int_{-6}^6 f(x) \d x + \int_{-6}^6 14 \d x.
            \end{equation}
            
            Since $f$ is even, we know that
            \begin{equation}\label{int of f}
              \int_{-6}^6 f(x) \d x = 2 \int_0^6 f(x) \d x = 2 (13) = 26.
            \end{equation}
            
            We also know that $g(x)=14$ is also an even function.  We have:
             $\int_{-6}^6 14 \d x = 2\int_{0}^6 14 \d x$
            \begin{equation}\label{int of 14}
              2\int_{0}^6 14 \d x =  2(14)(6) = 168.
            \end{equation}
            
            Then substituting equations \eqref{int of f} and \eqref{int of 14} into equation \eqref{linear integral} gives:
            \begin{equation*}
              \int_{-6}^6 (5f(x) + 14) \d x = 5(26) + 168 = 130 + 168 = 298.
            \end{equation*}
          \end{freeResponse}
\WkstHop
	\end{enumerate}
\end{problem}

\WkstNew

%problem 3
\begin{problem}
Evaluate the following integrals using symmetry arguments.

\begin{enumerate}

   \item    $\int_{-\pi/4}^{\pi/4} \sin(t) \d t$

      \begin{freeResponse}
        Since the function $f(t) = \sin(t)$ is odd,    
       $ \int_{-\pi/4}^{\pi/4} \sin(t) \d t=0$
       
         We can illustrate our computation with the graph of the function $f$, where \\[2em]
           $f(t)=\sin{t}$, for $-\dfrac{\pi}{4}\leq x\leq \dfrac{\pi}{4}$.
          \begin{image}
       \includegraphics[scale=.5]{DIimage004.png}
  \end{image}

      \end{freeResponse}

\WkstHop
      
      \item $\int_{-2}^2 (1+x+3x^7-x^9) \d x$
      \begin{freeResponse}
	In order to take advantage of symmetry, we will split the given polynomial into a sum of its even and odd parts.  We will then integrate each part separately:
        $$\int_{-2}^2 (1+x+3x^7-x^9) \d x= \int_{-2}^{2}1 \d x + \int_{-2}^{2} (x+3x^7-x^9) \d x$$
        
        We will first integrate an even function:\\
$\int_{-2}^{2} 1 \d x =2\int_{0}^{2}1\d x=2(2)=4$

Then an odd function:\\
$\int_{-2}^{2} (x+3x^7-x^9) \d x=0$

Therefore:$\int_{-2}^2 (1+x+3x^2-x^9) \d x=4$
      \end{freeResponse}

\WkstHop
      
      \item
     $\int_{-\pi}^{\pi} x\cos(x) \d x$ 
      \begin{freeResponse}
         
        $\int_{-\pi}^{\pi} x\cos(x) \d x=0$ , since the product of an odd and even function is odd.\\[1em]
           We can illustrate our computation with the graph of the function $f$, where \\[2em]
           $f(x)=x\cos{x}$, for $-\pi\leq x\leq \pi$.
          \begin{image}
       \includegraphics[scale=.5]{DIimage005.png}
  \end{image}

      \end{freeResponse}
\WkstHop
      
      
\end{enumerate}
\end{problem}

%problem 4



\end{document} 
\begin{problem}
Snow is starting to fall with a rate at any time $t$ after the start being 
$$ f'(t) = \frac{3}{2} t - \frac{1}{4} t^2 + \frac{3}{10} $$
inches per hour for $t$ in $[0,4]$ (i.e., the snow falls for 4 hours - from noon until 4pm).  
There were already $5$ inches of snow on the ground when the storm started.  
	\begin{enumerate}
	
	%part a 
	\item  Use the formula for a right Riemann sum to estimate how much snow fell during the storm using $n$ rectangles.
		\begin{freeResponse}
		$\Delta x = \frac{b-a}{n} = \frac{4-0}{n} = \frac{4}{n}$.
		
		$x_i = a + i \Delta x = 0 + i \frac{4}{n} = \frac{4i}{n}$.
			\begin{align*}
			f'(x_i) = f' \left( \frac{4i}{n} \right) &= \frac{3}{10} + \frac{3}{2} \left( \frac{4i}{n} \right) - \frac{1}{4} \left( \frac{4i}{n} \right)^2  \\
			&= \frac{3}{10} + \frac{6i}{n} - \frac{4i^2}{n^2}
			\end{align*}
			
		So our approximate area is:
			\begin{align*}
			\sum_{i=1}^n f'(x_i) \Delta x &= \sum_{i=1}^n \left[ \left( \frac{3}{10} + \frac{6i}{n} - \frac{4i^2}{n^2} \right) \left( \frac{4}{n} \right) \right]  \\
			&= \frac{4}{n} \sum_{i=1}^n \left( \frac{3}{10} + \frac{6i}{n} - \frac{4i^2}{n^2} \right)  \\
			&= \frac{4}{n} \sum_{i=1}^n \left( \frac{3}{10} \right) + \frac{4}{n} \sum_{i=1}^n \left( \frac{6i}{n} \right) - \frac{4}{n} \sum_{i=1}^n \left( \frac{4i^2}{n^2} \right)  \\
			&= \frac{6}{5n} \sum_{i=1}^n 1 + \frac{24}{n^2} \sum_{i=1}^n i - \frac{16}{n^3} \sum_{i=1}^n i^2  \\
			&= \frac{6}{5n} (n) + \frac{24}{n^2} \left( \frac{n(n+1)}{2} \right) - \frac{16}{n^3} \left( \frac{n(n+1)(2n+1)}{6} \right)  \\
			&= \frac{6}{5} + \frac{12(n+1)}{n} - \frac{8(n+1)(2n+1)}{3n^2}.
			\end{align*}
		\end{freeResponse}
		
		
		
	%part b
	\item  Take the limit as $n$ goes to infinity to find the exact amount of snow that fell.
		\begin{freeResponse}
			\begin{align*}
			&  \lim_{n \to \infty} \left( \frac{6}{5} + \frac{12(n+1)}{n} - \frac{8(n+1)(2n+1)}{3n^2} \right)  \\
			&= \lim_{n \to \infty} \left( \frac{6}{5} + 12 \left( 1 + \frac{1}{n} \right) - \frac{8(1 + \frac{1}{n})(2 + \frac{1}{n})}{3} \right)  \\
			&= \frac{6}{5} + 12(1 + 0) - \frac{8(1+0)(2+0)}{3}  \\
			&= \frac{6}{5} + 12 - \frac{16}{3} = \frac{18 + 180 - 80}{15} = \frac{118}{15}.
			\end{align*}
		\end{freeResponse}
		
		
		
	\end{enumerate}
	
\end{problem}
\begin{problem}
  The \dfn{velocity} function for a man walking along a straight road which runs east and west is given by $v(t) = -t^2 + 4t - 3$ ft/min.
  \begin{enumerate}
    
  \item  Set up a definite integral for the man's \dfn{displacement} during the time interval from $2$ minutes to $6$ minutes after he began running.
    \begin{freeResponse}
      \begin{align*}
        \int_2^6 v(t) \d t &= \lim_{n \to \infty} \sum_{i=1}^n v(t_i) \Delta t
      \end{align*}
      Where:  \\	
      $\Delta t = \frac{b-a}{n} = \frac{6-2}{n} = \frac{4}{n}$.
      
      $t_i = a + i \Delta t = 2 + i \frac{4}{n} = 2 + \frac{4i}{n}$.
    \end{freeResponse}
    
  \item  \dfn{At home:}  Evaluate the definite integral using the limit of a right Riemann sum.
    \begin{freeResponse}
      \begin{align*}
        v(t_i) &= -\left(2 + \frac{4i}{n} \right)^2 + 4 \left( 2 + \frac{4i}{n} \right) - 3  \\
               &= - \left( 4 + \frac{16i}{n} + \frac{16i^2}{n^2} \right) + 8 + \frac{16i}{n} - 3  \\
               &= 1 - \frac{16i^2}{n^2}
      \end{align*}
      
      So we compute:
      \begin{align*}
        \int_2^6 v(t) \d t &= \lim_{n \to \infty} \sum_{i=1}^n \left[ \left( 1 - \frac{16i^2}{n^2} \right) \left( \frac{4}{n} \right) \right]  \\
                           &= \lim_{n \to \infty} \sum_{i=1}^n \left( \frac{4}{n} - \frac{64 i^2}{n^3} \right)  \\
                           &= \lim_{n \to \infty} \left[ \frac{4}{n} \sum_{i=1}^n 1 - \frac{64}{n^3} \sum_{i=1}^n i^2 \right]  \\
                           &= \lim_{n \to \infty} \left[ \frac{4}{n} (n) - \frac{64}{n^3} \left( \frac{n(n+1)(2n+1)}{6} \right) \right]  \\
                           &= 4 - \frac{64}{3} = \frac{12-64}{3} = - \frac{52}{3}.
      \end{align*}
    \end{freeResponse}
    
  \item  Is this the same as the total \dfn{distance} the man walked from $2$ minutes to $6$ minutes?
    Why or why not?
    \begin{freeResponse}
      This number is not the same as the total distance.
      The man starts his walk by going east (the positive direction) but eventually ends his walk west of where he started.
      
      The total distance that the man walks would be measured by computing 
      $$\int_2^6 \left| v(t) \right| \d t$$  
      \begin{image}
      \includegraphics[scale=.7]{figure1.png}
            \end{image}
    \end{freeResponse}
  \end{enumerate}
\end{problem}
\end{document}

