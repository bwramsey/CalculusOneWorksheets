\documentclass{ximera}

\newcommand{\RR}{\mathbb R}
\renewcommand{\d}{\,d}
\newcommand{\dd}[2][]{\frac{d #1}{d #2}}
\renewcommand{\l}{\ell}
\newcommand{\ddx}{\frac{d}{dx}}
\newcommand{\dfn}{\textbf}
\newcommand{\eval}[1]{\bigg[ #1 \bigg]}
\renewcommand{\theenumii}{\textup{(\roman{enumii})}}
\renewcommand{\labelenumii}{\theenumii}

\usepackage{graphicx}
\usepackage{multicol}
\usepackage{tkz-euclide}
%\usepackage{unicode-math}

\usepackage{pgfplots}   % <- for graphics
\pgfplotsset{compat=newest}


\renewenvironment{freeResponse}{
\ifhandout\setbox0\vbox\bgroup\else
\begin{trivlist}\item[\hskip \labelsep\bfseries Solution:\hspace{2ex}]
\fi}
{\ifhandout\egroup\else
\end{trivlist}
\fi}

\newcommand*{\ZeroOverZero}{\ensuremath{\dfrac{0}{0}}}

\providecommand{\HCCondition}{0}
\newcommand{\WkstHop}[1][1]{\if\HCCondition 0
	\vspace*{\stretch{#1}} \fi} 
\newcommand{\WkstNew}{\if\HCCondition 0
	\newpage
	 \fi} 


\title[Problem 7]{Problem 7}

\begin{document}
\begin{abstract} \end{abstract}
\maketitle


% Extracted from indeterminateForms.tex, problem #7
\begin{problem}
The graph of a function $g(x)$ with domain $(-4,4)$ is given in the figure.  This portions of this graph in the intervals $(-2,0)$ and $(2, 4)$ are straight lines. The portion on the interval $(0,2)$ is not parabolic.
		\begin{center}
				\begin{tikzpicture}
					\begin{axis}[
						xmin=-4.2, xmax=4.2, ymin=-2.3,ymax=2.3,    
						axis lines =middle, 
						every axis y label/.style={at=(current axis.above origin),anchor=south},
						every axis x label/.style={at=(current axis.right of origin),anchor=west},
						xtick={-5,...,5}, ytick={-2,...,2},
						grid=major, width=4in, height=2in,
						grid style={dashed, gray!40}
						]						
						\addplot[color=blue, very thick, smooth, domain=-2:0]{-2*(x+1)};						
						\addplot[color=blue, very thick, smooth, domain=0:2]{0.5*x^3-2};
						\addplot[color=blue, very thick, smooth, domain=2:4]{-2*(x-3)};
						
						\addplot[color=blue, very thick, smooth, samples=100, domain=-3:-2]{(3+x)^(1/2)+1};
						\addplot[color=blue, very thick, smooth, samples=100,domain=-4:-3]{-(-x-3)^(1/2)+1};

						
						\draw[fill=blue] (axis cs:-4,0) circle [color=blue,radius=3pt];
						\draw[fill=white] (axis cs:-4,0) circle [color=white,radius=2pt];
						\draw[fill=blue] (axis cs:4,-2) circle [color=blue,radius=3pt];
						\draw[fill=white] (axis cs:4,-2) circle [color=white,radius=2pt];
					\end{axis}
					
				\end{tikzpicture}
				\end{center}
		For each of the limits below, give the form of the limit, then evaluate the limit. \\
		  \begin{enumerate}
				
			\item $\displaystyle\lim_{x\to -3} \dfrac{g(x)+\sin\left(\dfrac{\pi}{4}x\right)}{x\sqrt{x+4} }$. \\
\begin{explanation}
			The numerator and denominator are each continuous at $x=-3$, and at $x=-3$ the denominator is nonzero. The entire fraction is continuous at $x=-3$. To find the limit, we can plug in the value. (We would call this form either ``$\dfrac{\#}{\#}$'' or ``Continuous'').
			\begin{align*}
				\lim_{x\to -3} \dfrac{g(x)+\sin\left(\dfrac{\pi}{4}x\right)}{x\sqrt{x+4}} &= \dfrac{g(-3)+\sin\left(\dfrac{-3 \pi}{4}\right)}{-3\sqrt{-3+4}} \\
	&=\dfrac{1-\frac{1}{\sqrt{2}}}{-3}. 
			\end{align*}    
		\end{explanation}
			

			\item $\displaystyle \lim_{x\to 0^+}  \dfrac{e^x}{\left|g(x)+2\right|}$. \\
\begin{explanation}
			$\lim_{x\to 0^+} e^x = e^0 = 1$ and $\lim_{x\to 0^+} |g(x)+2| = 0$, so this limit has form $\dfrac{\#}{0}$. One-sided limits with form $\dfrac{\#}{0}$ are either $+\infty$ or $-\infty$, so we need to check the sign of the fraction.
			
			We know the range of the famous function $e^x$ is $(0, \infty)$, so the numerator here is always positive. The denominator is an absolute value, so it is not negative either. As long as $x$ is near, but not equal, to $0$, $g(x)+2$ will be close, but not equal, to $0$, so $|g(x)+2|$ will be positive.
		
		$\displaystyle \lim_{x\to 0^+}  \dfrac{e^x}{\left|g(x)+2\right|} = \infty$.	
		\end{explanation}															
					%  
			\item	$\displaystyle \lim_{x\to 3} \dfrac{x^2+2x-15}{x + g(x) - 3}$
\begin{explanation}
			$\lim_{x\to 3} (x^2+2x-15) = 0$ and $\lim_{x\to 3} (x + g(x) - 3) = 0$, so this limit has form $\dfrac{0}{0}$. In order to use algebra to simplify this fraction, we need a formula for $g(x)$ for $x$ near $3$. From the graph we see that in the interval $(2, 4)$, the graph of $g$ is a straight line with slope $-2$. The equation of that line is $y = -2x+6$, so for $x$ close to $3$, we know $g(x) = -2x+6$.
			

			\begin{align*}
				\lim_{x\to 3} \dfrac{x^2+2x-15}{x + g(x) - 3} &= \lim_{x\to 3} \dfrac{x^2+2x-15}{x + (-2x+6) - 3} \\
	&=\lim_{x\to 3} \dfrac{x^2+2x-15}{-x+3} \\ 
	&=\lim_{x\to 3} \dfrac{(x-3)(x+5)}{-(x-3)} \\ 
	&=\lim_{x\to 3} -(x+5) \\ 
	&= -8.
			\end{align*}    
		\end{explanation}									
		
		\end{enumerate}
\end{problem}



\end{document}
