\documentclass{ximera}

\newcommand{\RR}{\mathbb R}
\renewcommand{\d}{\,d}
\newcommand{\dd}[2][]{\frac{d #1}{d #2}}
\renewcommand{\l}{\ell}
\newcommand{\ddx}{\frac{d}{dx}}
\newcommand{\dfn}{\textbf}
\newcommand{\eval}[1]{\bigg[ #1 \bigg]}
\renewcommand{\theenumii}{\textup{(\roman{enumii})}}
\renewcommand{\labelenumii}{\theenumii}

\usepackage{graphicx}
\usepackage{multicol}
\usepackage{tkz-euclide}
%\usepackage{unicode-math}

\usepackage{pgfplots}   % <- for graphics
\pgfplotsset{compat=newest}


\renewenvironment{freeResponse}{
\ifhandout\setbox0\vbox\bgroup\else
\begin{trivlist}\item[\hskip \labelsep\bfseries Solution:\hspace{2ex}]
\fi}
{\ifhandout\egroup\else
\end{trivlist}
\fi}

\newcommand*{\ZeroOverZero}{\ensuremath{\dfrac{0}{0}}}

\providecommand{\HCCondition}{0}
\newcommand{\WkstHop}[1][1]{\if\HCCondition 0
	\vspace*{\stretch{#1}} \fi} 
\newcommand{\WkstNew}{\if\HCCondition 0
	\newpage
	 \fi} 


\title[Problem 5]{Problem 5}

\begin{document}
\begin{abstract} \end{abstract}
\maketitle


% Extracted from indeterminateForms.tex, problem #5
\begin{problem}
  Evaluate each of the following limits. Possible answers include a number, $+\infty$,  $-\infty$ and ``Does Not Exist'' (DNE).  Make sure to state the form of the limit.
 Justify your answer.

  \begin{enumerate}
	\item
      $\displaystyle \lim_{x \to 3^-} \frac{x^2 - 3}{x^2 - x - 6}$
\begin{explanation} 

         $ \lim_{x \to 3^-} \frac{x^2 - 3}{x^2 - x - 6} = \lim_{x \to 3^-} \frac{x^2 - 3}{(x-3)(x+2)}$  is of the form $\frac{\#}{0}$.\\[1em]
Since  $ \lim_{x \to 3^-} (x^2 - 3)=6$, and $ \lim_{x \to 3^-} (x^2 - x - 6)= \lim_{x \to 3^-} (x-3)(x+2)=0$, and since \\[1em]$x^2 - x - 6=(x-3)(x+2)<0$, if $x<3$ and $x$ close to 3.
	$$\lim_{x \to 3^-} \frac{x^2 - 3}{x^2 - x - 6} =-\infty$$
      \end{explanation}

    \item
      $\displaystyle \lim_{x \to 5^+} \frac{x^2 + 6}{x^2 - 3x - 10}$
\begin{explanation}
   The limit is of the form $\frac{\#}{0}$.\\[1em]
  
         $ \lim_{x \to 5^+} \frac{x^2 + 6}{x^2 - 3x - 10} = \lim_{x \to 5^+} \frac{x^2 + 6}{(x-5)(x+2)}  = \infty $,  because\\[1em]
 $\lim_{x \to 5^+} (x^2 + 6)=31$ and  $ x^2 - 3x - 10= (x-5)(x+2)>0 $, for $x>5$ and $x$ close to 5.
 
      \end{explanation}

    \item
      $\displaystyle \lim_{x \to 1} \frac{4-x}{x^2 - 2x + 1}$
\begin{explanation}
       The limit is of the form $\frac{\#}{0}$.\\[1em]
        Checking left and right sided limits we see:
        \begin{align*}
          \lim_{x \to 1^+} \frac{4-x}{x^2 - 2x + 1} &= \lim_{x \to 1^+}\frac{4-x}{(x-1)^2} = \infty\\
        \end{align*}
 	 and
        \begin{align*}
 	\lim_{x \to 1^-} \frac{4-x}{x^2 - 2x + 1} = \lim_{x \to 1^-}\frac{4-x}{(x-1)^2} = \infty\\
	\end{align*}
	Since $\lim_{x \to 1^-} \frac{4-x}{x^2 - 2x + 1}=\lim_{x \to 1^+} \frac{4-x}{x^2 - 2x + 1} \implies  \lim_{x \to 1} \frac{4-x}{x^2 - 2x + 1} = \infty$
      \end{explanation}
\item
      $\displaystyle \lim_{x \to 2} \frac{-e^x}{(2-x)^3}$
\begin{explanation}
            The limit is of the form $\frac{\#}{0}$.\\[1em]
        Checking right limit we have:
 \begin{align*}
          \lim_{x \to 2^+} \frac{-e^x}{(2-x)^3} &= \infty\\
        \end{align*}
 	 and checking the left limit we have:
        \begin{align*}
 	\lim_{x \to 2^-} \frac{-e^x}{(2-x)^3} &= - \infty\\
	\end{align*}
	Since $ \lim_{x \to 2^+} \frac{-e^x}{(2-x)^3} \ne \lim_{x \to 2^-} \frac{-e^x}{(2-x)^3} \implies   \lim_{x \to 2} \frac{-e^x}{(2-x)^3}$ Does not exist
      \end{explanation}
         \item
      $\displaystyle \lim_{x \to 1} \frac{\sin{x}}{\sqrt{2-x^{2}}-1}$
\begin{explanation}
            The limit is of the form $\frac{\#}{0}$.\\[1em]
        Checking right limit we have:

         $ \lim_{x \to 1^+} \frac{\sin{x}}{\sqrt{2-x^{2}}-1}= -\infty$,\\[1em]
        $ \lim_{x \to 1^+} \sin{x}=\sin{1}>0$, since $1<\frac{\pi}{2}$.
     Explanation: Note that for x near 1 and such that $x>1$, we have that \\[1em]
      $x^{2}>1$,    and by multiplying by $(-1)$, we get\\[1em]
  $ - x^{2}<-1$,   and by adding 2 on both sides, we get\\[1em]
   $2 - x^{2}<2-1$, and by taking the square root, we get\\[1em]
   $ \sqrt{2 - x^{2}}<\sqrt{1}$, or\\[1em]
      $\sqrt{2 - x^{2}}<1$, and by subtracting 1 from both sides, we get\\[1em]
        $ \sqrt{2 - x^{2}}-1<0$. \\[1em] Therefore the numerator is positive, and denominator is \textbf{negative} and goes to 0.\\[1em]

 	 Checking the left limit we have:
      $\displaystyle \lim_{x \to 1^{-}} \frac{\sin{x}}{\sqrt{2-x^{2}}-1}=+\infty$\\[1em]
      
     Explanation: Note that and for x near 1 such that  $x<1$, we have that\\[1em]
       
      $x^{2}<1$,    and by multiplying by $(-1)$, we get\\[1em]
  $ - x^{2}>-1$,   and by adding 2 on both sides, we get\\[1em]
   $2 - x^{2}>2-1$, and by taking the square root, we get\\[1em]
   $ \sqrt{2 - x^{2}}>\sqrt{1}$, or\\[1em]
      $\sqrt{2 - x^{2}}>1$, and by subtracting 1 from both sides, we get\\[1em]
        $ \sqrt{2 - x^{2}}-1>0$.\\[1em] Therefore the numerator is positive, and denominator is \textbf{positive} and goes to 0.\\[1em]
        
      	Since $ \lim_{x \to 1^+}  \frac{\sin{x}}{\sqrt{2-x^{2}}-1} \ne \lim_{x \to 1^-} \frac{\sin{x}}{\sqrt{2-x^{2}}-1} \implies   \lim_{x \to 1}  \frac{\sin{x}}{\sqrt{2-x^{2}}-1}$ does not exist.
      \end{explanation}
\end{enumerate}
\end{problem}



\end{document}
