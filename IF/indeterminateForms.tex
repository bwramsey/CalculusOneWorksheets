%Add code to compile both versions from makefile at same time
\providecommand{\HCCondition}{0}
%Define each of the conditions
\ifcase\HCCondition
	%\condition=0 -> handout
	\documentclass[nooutcomes,noauthor,space,handout]{ximera}
	\title{(In)determinate Forms (IF)}
\or	%\condition=1 -> Soln
	\documentclass[nooutcomes,noauthor]{ximera}
	\title{(In)determinate Forms (IF) - Solutions} 
\fi
\newcommand{\RR}{\mathbb R}
\renewcommand{\d}{\,d}
\newcommand{\dd}[2][]{\frac{d #1}{d #2}}
\renewcommand{\l}{\ell}
\newcommand{\ddx}{\frac{d}{dx}}
\newcommand{\dfn}{\textbf}
\newcommand{\eval}[1]{\bigg[ #1 \bigg]}
\renewcommand{\theenumii}{\textup{(\roman{enumii})}}
\renewcommand{\labelenumii}{\theenumii}

\usepackage{graphicx}
\usepackage{multicol}
\usepackage{tkz-euclide}
%\usepackage{unicode-math}

\usepackage{pgfplots}   % <- for graphics
\pgfplotsset{compat=newest}


\renewenvironment{freeResponse}{
\ifhandout\setbox0\vbox\bgroup\else
\begin{trivlist}\item[\hskip \labelsep\bfseries Solution:\hspace{2ex}]
\fi}
{\ifhandout\egroup\else
\end{trivlist}
\fi}

\newcommand*{\ZeroOverZero}{\ensuremath{\dfrac{0}{0}}}

\providecommand{\HCCondition}{0}
\newcommand{\WkstHop}[1][1]{\if\HCCondition 0
	\vspace*{\stretch{#1}} \fi} 
\newcommand{\WkstNew}{\if\HCCondition 0
	\newpage
	 \fi}  %% we can turn off input when making a master document
\usepackage{fullpage}


\begin{document}
\begin{abstract}		\end{abstract}
\maketitle

\ifcase\HCCondition
\section{Some notes on notation for finding limits:}
\begin{enumerate}
	\item {\bf The first step in evaluating any limit is to find the form.} This is true, even if the question does not specifically ask for the form.

	\item To find the form of $\lim_{x \to a}\dfrac{f(x)}{g(x)}$, take the limits of the numerator and denominator SEPARATELY. Do not just ``plug in $x=a$''. ``Plugging in'' can not occur until after the function is noticed as continuous.
	
	\item When writing a form of a limit, we NEVER write $\lim_{x \to a}\dfrac{f(x)}{g(x)}=\frac{0}{0}$.  It is the $=$ that makes this a mathematically incorrect statement.  Instead, write $\lim_{x \to a}\dfrac{f(x)}{g(x)}$ is of the form $\frac{0}{0}$. 



	\item When evaluating limits, do NOT drop the $\lim_{x \to a}$ until you have evaluated the limit.  For example, the following is INCORRECT:
	\begin{center}
	$\lim_{x \to 2}\frac{2x^2-4x}{x-2}=\frac{2x(x-2)}{x-2}=2x=4$\\	
	\end{center}

	Instead, we write: 
		\begin{center}
	$\lim_{x \to 2}\frac{2x^2-4x}{x-2}=\lim_{x \to 2}\frac{2x(x-2)}{x-2}=\lim_{x \to 2}2x=4$\\	
	\end{center}

\end{enumerate}
\newpage
\section*{Recitation Questions}
\fi

	%problem 4
\begin{problem}
	Which of the following limits are of the form \ZeroOverZero?	
	\begin{enumerate}
		\item $ \lim_{x \to 7} \frac{\sin{(x-7)}}{|x-7|}  $
		\item $ \lim_{x \to \frac{\pi}{2}} \frac{\sin{(x)}}{x}  $
		\item $ \lim_{x \to \frac{\pi}{2}} \frac{|\cos{(x)}|}{x}  $
		\item $ \lim_{x \to \frac{\pi}{2}} \frac{\pi-2x}{\cos{(x)}}$
		\item	$\lim_{x \to 2} \frac{x^2 - 4}{x-4} $\\
		\item	$\lim_{x \to 4} \frac{x^2 - 4}{x-4} $\\
		\item	$\lim_{x \to 4} \frac{x^2 - 16}{x-4} $\\
		\item	$\lim_{x \to 3} \frac{x-4}{\sqrt{25-x^2 }-4} $\\
		\item	$\lim_{x \to 3} \frac{x-3}{\sqrt{25-x^2 }-4} $\\
	\end{enumerate}
\WkstHop
	\begin{freeResponse}
	(a), (d), (g) and (i)
	\end{freeResponse}
\end{problem}
%problem 5
\WkstNew


%problem 2
\begin{problem}
State the form of the limit and then evaluate the limit. 
	
	\begin{enumerate}
	
	%part a		
	\item $ \lim_{x \to 6} \frac{4x^2 - 144}{x-6}  $
\WkstHop
	\begin{freeResponse}
		$\lim_{x \to 6} \frac{4x^2 - 144}{x-6} = \lim_{x \to 6}\dfrac{4(x^2 - 36)}{x-6}$.\\
		 \hspace{0.5in}\text{Form:} \ZeroOverZero \\ 
	\begin{align*}
	\lim_{x \to 6} \frac{4x^2 - 144}{x-6} &= \lim_{x \to 6} \frac{4(x-6)(x+6)}{x-6} \\
	&= \lim_{x \to 6} 4(x+6) \\
	&= 4(12) = 48  
	\end{align*}
	\end{freeResponse}
	
	
	%part b		
	\item  $ \lim_{x \to 6} \frac{x-6}{\sqrt{2x-8} - 2}  $
\WkstHop
	\begin{freeResponse}
	\begin{align*}
	\text{Form:} \ZeroOverZero\\
	\lim_{x \to 6} \frac{x-6}{\sqrt{2x-8} - 2} &= \lim_{x \to 6} \frac{x-6}{\sqrt{2x-8} - 2} \cdot \frac{\sqrt{2x-8} + 2}{\sqrt{2x-8}+2} \\
	&= \lim_{x \to 6} \frac{(x-6)(\sqrt{2x-8} + 2)}{2x - 8 - 4} \\
	&= \lim_{x \to 6} \frac{(x-6)(\sqrt{2x-8} + 2)}{2(x-6)} \\
	&= \lim_{x \to 6} \frac{\sqrt{2x-8}+2}{2} \\
	&= \frac{\sqrt{12-8}+2}{2} \\
	&= \frac{4}{2} = 2
	\end{align*}
			
	\end{freeResponse}
	
	
	%part c		
	\item  $ \lim_{x \to 2} \frac{(3x-2)^2 - 16}{x-2}  $
\WkstHop
	\begin{freeResponse}
	\begin{align*}
	\text{Form:} \ZeroOverZero\\
	\lim_{x \to 2} \frac{(3x-2)^2-16}{x-2} &=\lim_{x \to 2} \frac{((3x-2)-4)((3x-2)+4)}{x-2} \\
	&= \lim_{x \to 2} \frac{(3x-6)(3x+2)}{x-2} \\
	&= \lim_{x \to 2} \frac{3(x-2)(3x+2)}{x-2} \\
	&= \lim_{x \to 2} 3(3x+2)  \\
	&= 3(6+2) = 24   
	\end{align*}
	\end{freeResponse}
	

	%part d		
	\item $ \lim_{x \to 6} \frac{\frac{x}{x-2} - \frac{3}{2}}{x-6}  $
\WkstHop
	\begin{freeResponse}
	 \hspace{0.5in}\text{Form:} \ZeroOverZero\\ 
	 \begin{align*}
		\lim_{x \to 6} \frac{\frac{x}{x-2} - \frac{3}{2}}{x-6} &= \lim_{x \to 6} \frac{\frac{2x}{2(x-2)} - \frac{3(x-2)}{2(x-2}}{x-6}\\
		&=\lim_{x \to 6} \frac{\frac{2x-3x+6}{2(x-2)} }{x-6} \\
		&=\lim_{x \to 6} \frac{\frac{-x+6}{2(x-2)} }{x-6}\\
		&=\lim_{x \to 6}\frac{-(x-6)}{2(x-2)(x-6)} \\
		&=\lim_{x \to 6}\frac{-1}{2(x-2)} \\
		&=\frac{-1}{2(6-2)}= \frac{-1}{2(4)}=\frac{-1}{8}\\
			\end{align*}
	\end{freeResponse}
	
	
	%part f		
	\item  $ \lim_{x \to 1} \frac{\sqrt{5x-2} - \sqrt{3}}{x-1} $
\WkstHop
	\begin{freeResponse}
	\begin{align*}
	\text{Form:} \ZeroOverZero\\
	\lim_{x \to 1} \frac{\sqrt{5x-2} - \sqrt{3}}{x-1} &= \lim_{x \to 1} \frac{\sqrt{5x-2} - \sqrt{3}}{x-1} \cdot \frac{\sqrt{5x-2} + \sqrt{3}}{\sqrt{5x-2} + \sqrt{3}} \\
	&= \lim_{x \to 1} \frac{(5x-2)-3}{(x-1)(\sqrt{5x-2} + \sqrt{3})} \\
	&= \lim_{x \to 1} \frac{5(x-1)}{(x-1)(\sqrt{5x-2} + \sqrt{3})} \\
	&= \lim_{x \to 1} \frac{5}{\sqrt{5x-2} + \sqrt{3}} \\
	&=   \frac{5}{\sqrt{5(1)-2} + \sqrt{3}} \\
	&= \frac{5}{2 \sqrt{3}} 
	\end{align*}
	\end{freeResponse}
	\end{enumerate}
\end{problem}
	
\WkstNew	
	



\begin{problem}
Evaluate the following limits.
	\begin{enumerate}
		\item $ \lim_{x \to \frac{\pi}{2}} \frac{\cos{(x)}}{x}  $
	\WkstHop
	\begin{freeResponse} Apply the Quotient Law and use continuity of the numerator and denominator.\\[1em]
	 $ \lim_{x \to \frac{\pi}{2}} \frac{\cos{(x)}}{x}=\frac{0}{\frac{\pi}{2}}=0  $
	\end{freeResponse} 
		
	\item	$\lim_{x \to 3} \frac{\sqrt{25-x^2 }-4}{x-4} $\\
	\WkstHop
	\begin{freeResponse}Apply the Quotient Law and use continuity.	\\[1em]
	 $ \lim_{x \to 3} \frac{\sqrt{25-x^2 }-4}{x-4}=  \frac{\sqrt{25-(3)^2 }-4}{3-4}= \frac{4-4}{-1}=0$\\
	
	\end{freeResponse} 
	\item	$\lim_{x \to 3} \frac{\sqrt{25-x^2 }-4}{x-3} $\\
	\WkstHop
	\begin{freeResponse}
	Since $\lim_{x \to 3} (\sqrt{25-x^2 }-4) = 0$ and $\lim_{x \to 3} {x-3} = 0$, this limit has form $\frac{0}{0}$.\\[1em]
	 $ \lim_{x \to 3} \frac{\sqrt{25-x^2 }-4}{x-3}=  \lim_{x \to 3} \frac{\sqrt{25-x^2 }-4}{x-3}\frac{\sqrt{25-x^2 }+4}{\sqrt{25-x^2 }+4}=$\\
	 $ \lim_{x \to 3} \frac{25-x^2-16}{(x-3)(\sqrt{25-x^2 }+4)}=  \lim_{x \to 3} \frac{9-x^2}{(x-3)(\sqrt{25-x^2 }+4)}= \lim_{x \to 3} \frac{(3+x)(3-x)}{(x-3)(\sqrt{25-x^2 }+4)}=\\   $
	 $\lim_{x \to 3} \frac{-(3+x)}{\sqrt{25-x^2 }+4}=\frac{-6}{\sqrt{25-9 }+4}=\frac{-6}{\sqrt{16 }+4}=\frac{-6}{8}=\frac{-3}{4}$
	\end{freeResponse}
\end{enumerate}
\end{problem}
\WkstNew

%problem 6
\begin{problem}
	Two polynomials, $h$ and $g$, are given
	\[ h(x)=\frac{x^2-4}{4} \]
	\[ g(x)=x-2 \]
	 State the form of the limit, evaluate the limit, or state that it does not exist.  Justify your answer.
		
		\begin{enumerate}
		
		\item $\lim_{x \to 2}\frac{h(x)}{g(x)}$

\WkstHop
		\begin{freeResponse}
			$\lim_{x \to 2}\frac{h(x)}{g(x)}=\lim_{x \to 2}\frac{\frac{x^2-4}{4}}{x-2}$\\
		Form: $\frac{{0}}{0}$ \\
		\begin{align*}
		\lim_{x \to 2}\frac{\frac{x^2-4}{4}}{x-2}&=\lim_{x \to 2}\frac{(x-2)(x+2)}{4(x-2)}\\
		&= \lim_{x \to 2} \frac{x+2}{4}\\
		&= \frac{4}{4}\\
		&= 1
		\end{align*}
		\end{freeResponse}	
\item $\lim_{x \to 3}\frac{g(x)-g(3)}{x-3}$
\WkstHop
			\begin{freeResponse}
			$\lim_{x \to 3}\frac{g(x)-g(3)}{x-3}=\lim_{x \to 3}\frac{x-2-1}{x-3}$\\
		Form: $\frac{{0}}{0}$ \\
		\begin{align*}
		\lim_{x \to 3}\frac{g(x)-g(3)}{x-3}&=\lim_{x \to 3}\frac{x-2-1}{x-3}\\
		&=\lim_{x \to 3}1\\	
		&=1		
		\end{align*}
		\end{freeResponse}	
\item  $\lim_{x \to4}\frac{h(x)-h(4)}{x-4}$
\WkstHop
			\begin{freeResponse}
			$\lim_{x \to 4}\frac{h(x)-h(4)}{x-4}=\lim_{x \to 4}\frac{\frac{x^2-4}{4}-3}{x-4}$\\
		Form: $\frac{{0}}{0}$ \\
		\begin{align*}
		\lim_{x \to 4}\frac{h(x)-h(4)}{x-4}&=\lim_{x \to 4}\frac{\frac{x^2-4}{4}-3}{x-4}\\
		&=\lim_{x \to 4}\dfrac{x^2-4-12}{4(x-4)}\\	
		&=\lim_{x \to 4}\dfrac{x^2-16}{4(x-4)}\\
		&=\lim_{x \to 4}\dfrac{(x-4)(x+4)}{4(x-4)}\\
		&=\lim_{x \to 4}\dfrac{x+4}{4}\\	
		&=2		
		\end{align*}
		\end{freeResponse}	
\end{enumerate}
\end{problem}

\WkstNew

\begin{problem}
  Evaluate each of the following limits. Possible answers include a number, $+\infty$,  $-\infty$ and ``Does Not Exist'' (DNE).  Make sure to state the form of the limit.
 Justify your answer.

  \begin{enumerate}
	\item
      $\displaystyle \lim_{x \to 3^-} \frac{x^2 - 3}{x^2 - x - 6}$
\WkstHop
      \begin{freeResponse} 

         $ \lim_{x \to 3^-} \frac{x^2 - 3}{x^2 - x - 6} = \lim_{x \to 3^-} \frac{x^2 - 3}{(x-3)(x+2)}$  is of the form $\frac{\#}{0}$.\\[1em]
Since  $ \lim_{x \to 3^-} (x^2 - 3)=6$, and $ \lim_{x \to 3^-} (x^2 - x - 6)= \lim_{x \to 3^-} (x-3)(x+2)=0$, and since \\[1em]$x^2 - x - 6=(x-3)(x+2)<0$, if $x<3$ and $x$ close to 3.
	$$\lim_{x \to 3^-} \frac{x^2 - 3}{x^2 - x - 6} =-\infty$$
      \end{freeResponse}

    \item
      $\displaystyle \lim_{x \to 5^+} \frac{x^2 + 6}{x^2 - 3x - 10}$
\WkstHop
      \begin{freeResponse}
   The limit is of the form $\frac{\#}{0}$.\\[1em]
  
         $ \lim_{x \to 5^+} \frac{x^2 + 6}{x^2 - 3x - 10} = \lim_{x \to 5^+} \frac{x^2 + 6}{(x-5)(x+2)}  = \infty $,  because\\[1em]
 $\lim_{x \to 5^+} (x^2 + 6)=31$ and  $ x^2 - 3x - 10= (x-5)(x+2)>0 $, for $x>5$ and $x$ close to 5.
 
      \end{freeResponse}

    \item
      $\displaystyle \lim_{x \to 1} \frac{4-x}{x^2 - 2x + 1}$
\WkstHop
      \begin{freeResponse}
       The limit is of the form $\frac{\#}{0}$.\\[1em]
        Checking left and right sided limits we see:
        \begin{align*}
          \lim_{x \to 1^+} \frac{4-x}{x^2 - 2x + 1} &= \lim_{x \to 1^+}\frac{4-x}{(x-1)^2} = \infty\\
        \end{align*}
 	 and
        \begin{align*}
 	\lim_{x \to 1^-} \frac{4-x}{x^2 - 2x + 1} = \lim_{x \to 1^-}\frac{4-x}{(x-1)^2} = \infty\\
	\end{align*}
	Since $\lim_{x \to 1^-} \frac{4-x}{x^2 - 2x + 1}=\lim_{x \to 1^+} \frac{4-x}{x^2 - 2x + 1} \implies  \lim_{x \to 1} \frac{4-x}{x^2 - 2x + 1} = \infty$
      \end{freeResponse}
\WkstNew

    \item
      $\displaystyle \lim_{x \to 2} \frac{-e^x}{(2-x)^3}$
\WkstHop
      \begin{freeResponse}
            The limit is of the form $\frac{\#}{0}$.\\[1em]
        Checking right limit we have:
 \begin{align*}
          \lim_{x \to 2^+} \frac{-e^x}{(2-x)^3} &= \infty\\
        \end{align*}
 	 and checking the left limit we have:
        \begin{align*}
 	\lim_{x \to 2^-} \frac{-e^x}{(2-x)^3} &= - \infty\\
	\end{align*}
	Since $ \lim_{x \to 2^+} \frac{-e^x}{(2-x)^3} \ne \lim_{x \to 2^-} \frac{-e^x}{(2-x)^3} \implies   \lim_{x \to 2} \frac{-e^x}{(2-x)^3}$ Does not exist
      \end{freeResponse}
         \item
      $\displaystyle \lim_{x \to 1} \frac{\sin{x}}{\sqrt{2-x^{2}}-1}$
\WkstHop
      \begin{freeResponse}
            The limit is of the form $\frac{\#}{0}$.\\[1em]
        Checking right limit we have:

         $ \lim_{x \to 1^+} \frac{\sin{x}}{\sqrt{2-x^{2}}-1}= -\infty$,\\[1em]
        $ \lim_{x \to 1^+} \sin{x}=\sin{1}>0$, since $1<\frac{\pi}{2}$.
     Explanation: Note that for x near 1 and such that $x>1$, we have that \\[1em]
      $x^{2}>1$,    and by multiplying by $(-1)$, we get\\[1em]
  $ - x^{2}<-1$,   and by adding 2 on both sides, we get\\[1em]
   $2 - x^{2}<2-1$, and by taking the square root, we get\\[1em]
   $ \sqrt{2 - x^{2}}<\sqrt{1}$, or\\[1em]
      $\sqrt{2 - x^{2}}<1$, and by subtracting 1 from both sides, we get\\[1em]
        $ \sqrt{2 - x^{2}}-1<0$. \\[1em] Therefore the numerator is positive, and denominator is \textbf{negative} and goes to 0.\\[1em]

 	 Checking the left limit we have:
      $\displaystyle \lim_{x \to 1^{-}} \frac{\sin{x}}{\sqrt{2-x^{2}}-1}=+\infty$\\[1em]
      
     Explanation: Note that and for x near 1 such that  $x<1$, we have that\\[1em]
       
      $x^{2}<1$,    and by multiplying by $(-1)$, we get\\[1em]
  $ - x^{2}>-1$,   and by adding 2 on both sides, we get\\[1em]
   $2 - x^{2}>2-1$, and by taking the square root, we get\\[1em]
   $ \sqrt{2 - x^{2}}>\sqrt{1}$, or\\[1em]
      $\sqrt{2 - x^{2}}>1$, and by subtracting 1 from both sides, we get\\[1em]
        $ \sqrt{2 - x^{2}}-1>0$.\\[1em] Therefore the numerator is positive, and denominator is \textbf{positive} and goes to 0.\\[1em]
        
      	Since $ \lim_{x \to 1^+}  \frac{\sin{x}}{\sqrt{2-x^{2}}-1} \ne \lim_{x \to 1^-} \frac{\sin{x}}{\sqrt{2-x^{2}}-1} \implies   \lim_{x \to 1}  \frac{\sin{x}}{\sqrt{2-x^{2}}-1}$ does not exist.
      \end{freeResponse}
\end{enumerate}
\end{problem}
\WkstNew


 \begin{problem}
A  piecewise defined function $f$ is given by 
 
	$f(x) =   \left\{ \begin{array}{cl}
	\frac{2 x-3}{x-2}		 	&	\qquad \text{if }\hspace{0.1in}  x <  2					\\ \\
	\frac{x^2-5x +6}{x^2-4}	&	\qquad \text{if } \hspace{0.1in}   x>2 	\\ \\
		
						\end{array} \right.  $\\[1em]
	Determine the form of the limit, then find the limit.
  \begin{enumerate}

    \item $ \lim_{x \to 2^+} f(x)$
\WkstHop
     \begin{freeResponse}
   
   Since $ \lim_{x \to 2^+} f(x)= \lim_{x \to 2^+}\frac{x^2-5x +6}{x^2-4}$, the form of the limit is $\frac{0}{0}$.\\[1em]
     $ \lim_{x \to 2^+} f(x)= \lim_{x \to 2^+}\frac{x^2-5x +6}{x^2-4}= \lim_{x \to 2^+}\frac{(x-3)(x-2)}{(x-2)(x+2)}= \lim_{x \to 2^+}\frac{x-3}{x+2}=-\frac{1}{4}$
    \end{freeResponse}
  \item   $ \lim_{x \to 2^-} f(x)$
\WkstHop
     \begin{freeResponse}
   
   Since $ \lim_{x \to 2^-} f(x)= \lim_{x \to 2^-}\frac{2x-3}{x-2}$, the form of the limit is $\frac{\#}{0}$.\\[1em]
     $ \lim_{x \to 2^-} f(x)=  \lim_{x \to 2^-}\frac{2x-3}{x-2}= -\infty$,\\[1em]
      since the numerator  positive, the denominator negative and goes to 0.
    \end{freeResponse}
 \item $ \lim_{x \to 2} f(x)$
\WkstHop
   \begin{freeResponse}
   The limit does not exist, because  $ \lim_{x \to 2^+} f(x)\ne \lim_{x \to 2^-} f(x)$.
    \end{freeResponse}
      \end{enumerate}
\end{problem}
\WkstNew


\begin{problem}
The graph of a function $g(x)$ with domain $(-4,4)$ is given in the figure.  This portions of this graph in the intervals $(-2,0)$ and $(2, 4)$ are straight lines. The portion on the interval $(0,2)$ is not parabolic.
		\begin{center}
				\begin{tikzpicture}
					\begin{axis}[
						xmin=-4.2, xmax=4.2, ymin=-2.3,ymax=2.3,    
						axis lines =middle, 
						every axis y label/.style={at=(current axis.above origin),anchor=south},
						every axis x label/.style={at=(current axis.right of origin),anchor=west},
						xtick={-5,...,5}, ytick={-2,...,2},
						grid=major, width=4in, height=2in,
						grid style={dashed, gray!40}
						]						
						\addplot[color=blue, very thick, smooth, domain=-2:0]{-2*(x+1)};						
						\addplot[color=blue, very thick, smooth, domain=0:2]{0.5*x^3-2};
						\addplot[color=blue, very thick, smooth, domain=2:4]{-2*(x-3)};
						
						\addplot[color=blue, very thick, smooth, samples=100, domain=-3:-2]{(3+x)^(1/2)+1};
						\addplot[color=blue, very thick, smooth, samples=100,domain=-4:-3]{-(-x-3)^(1/2)+1};

						
						\draw[fill=blue] (axis cs:-4,0) circle [color=blue,radius=3pt];
						\draw[fill=white] (axis cs:-4,0) circle [color=white,radius=2pt];
						\draw[fill=blue] (axis cs:4,-2) circle [color=blue,radius=3pt];
						\draw[fill=white] (axis cs:4,-2) circle [color=white,radius=2pt];
					\end{axis}
					
				\end{tikzpicture}
				\end{center}
		For each of the limits below, give the form of the limit, then evaluate the limit. \\
		  \begin{enumerate}
				
			\item $\displaystyle\lim_{x\to -3} \dfrac{g(x)+\sin\left(\dfrac{\pi}{4}x\right)}{x\sqrt{x+4} }$. \\
\WkstHop						
		 \begin{freeResponse}
			The numerator and denominator are each continuous at $x=-3$, and at $x=-3$ the denominator is nonzero. The entire fraction is continuous at $x=-3$. To find the limit, we can plug in the value. (We would call this form either ``$\dfrac{\#}{\#}$'' or ``Continuous'').
			\begin{align*}
				\lim_{x\to -3} \dfrac{g(x)+\sin\left(\dfrac{\pi}{4}x\right)}{x\sqrt{x+4}} &= \dfrac{g(-3)+\sin\left(\dfrac{-3 \pi}{4}\right)}{-3\sqrt{-3+4}} \\
	&=\dfrac{1-\frac{1}{\sqrt{2}}}{-3}. 
			\end{align*}    
		\end{freeResponse}
			

			\item $\displaystyle \lim_{x\to 0^+}  \dfrac{e^x}{\left|g(x)+2\right|}$. \\
\WkstHop
		 \begin{freeResponse}
			$\lim_{x\to 0^+} e^x = e^0 = 1$ and $\lim_{x\to 0^+} |g(x)+2| = 0$, so this limit has form $\dfrac{\#}{0}$. One-sided limits with form $\dfrac{\#}{0}$ are either $+\infty$ or $-\infty$, so we need to check the sign of the fraction.
			
			We know the range of the famous function $e^x$ is $(0, \infty)$, so the numerator here is always positive. The denominator is an absolute value, so it is not negative either. As long as $x$ is near, but not equal, to $0$, $g(x)+2$ will be close, but not equal, to $0$, so $|g(x)+2|$ will be positive.
		
		$\displaystyle \lim_{x\to 0^+}  \dfrac{e^x}{\left|g(x)+2\right|} = \infty$.	
		\end{freeResponse}															
					%  
			\item	$\displaystyle \lim_{x\to 3} \dfrac{x^2+2x-15}{x + g(x) - 3}$
\WkstHop						
								 \begin{freeResponse}
			$\lim_{x\to 3} (x^2+2x-15) = 0$ and $\lim_{x\to 3} (x + g(x) - 3) = 0$, so this limit has form $\dfrac{0}{0}$. In order to use algebra to simplify this fraction, we need a formula for $g(x)$ for $x$ near $3$. From the graph we see that in the interval $(2, 4)$, the graph of $g$ is a straight line with slope $-2$. The equation of that line is $y = -2x+6$, so for $x$ close to $3$, we know $g(x) = -2x+6$.
			

			\begin{align*}
				\lim_{x\to 3} \dfrac{x^2+2x-15}{x + g(x) - 3} &= \lim_{x\to 3} \dfrac{x^2+2x-15}{x + (-2x+6) - 3} \\
	&=\lim_{x\to 3} \dfrac{x^2+2x-15}{-x+3} \\ 
	&=\lim_{x\to 3} \dfrac{(x-3)(x+5)}{-(x-3)} \\ 
	&=\lim_{x\to 3} -(x+5) \\ 
	&= -8.
			\end{align*}    
		\end{freeResponse}									
		
		\end{enumerate}
\end{problem}

\end{document} 


















