\documentclass{ximera}

\newcommand{\RR}{\mathbb R}
\renewcommand{\d}{\,d}
\newcommand{\dd}[2][]{\frac{d #1}{d #2}}
\renewcommand{\l}{\ell}
\newcommand{\ddx}{\frac{d}{dx}}
\newcommand{\dfn}{\textbf}
\newcommand{\eval}[1]{\bigg[ #1 \bigg]}
\renewcommand{\theenumii}{\textup{(\roman{enumii})}}
\renewcommand{\labelenumii}{\theenumii}

\usepackage{graphicx}
\usepackage{multicol}
\usepackage{tkz-euclide}
%\usepackage{unicode-math}

\usepackage{pgfplots}   % <- for graphics
\pgfplotsset{compat=newest}


\renewenvironment{freeResponse}{
\ifhandout\setbox0\vbox\bgroup\else
\begin{trivlist}\item[\hskip \labelsep\bfseries Solution:\hspace{2ex}]
\fi}
{\ifhandout\egroup\else
\end{trivlist}
\fi}

\newcommand*{\ZeroOverZero}{\ensuremath{\dfrac{0}{0}}}

\providecommand{\HCCondition}{0}
\newcommand{\WkstHop}[1][1]{\if\HCCondition 0
	\vspace*{\stretch{#1}} \fi} 
\newcommand{\WkstNew}{\if\HCCondition 0
	\newpage
	 \fi} 

\title[Summary]{Summary}

\begin{document}
\begin{abstract} \end{abstract}
\maketitle


\textbf{Some notes on notation for finding limits:}
\begin{enumerate}
	\item {\bf The first step in evaluating any limit is to find the form.} This is true, even if the question does not specifically ask for the form.

	\item To find the form of $\lim_{x \to a}\dfrac{f(x)}{g(x)}$, take the limits of the numerator and denominator SEPARATELY. Do not just ``plug in $x=a$''. ``Plugging in'' can not occur until after the function is noticed as continuous.
	
	\item When writing a form of a limit, we NEVER write $\lim_{x \to a}\dfrac{f(x)}{g(x)}=\frac{0}{0}$.  It is the $=$ that makes this a mathematically incorrect statement.  Instead, write $\lim_{x \to a}\dfrac{f(x)}{g(x)}$ is of the form $\frac{0}{0}$. 



	\item When evaluating limits, do NOT drop the $\lim_{x \to a}$ until you have evaluated the limit.  For example, the following is INCORRECT:
	\begin{center}
		$\lim_{x \to 2}\frac{2x^2-4x}{x-2}=\frac{2x(x-2)}{x-2}=2x=4$\\	
	\end{center}

	Instead, we write: 
	\begin{center}
		$\lim_{x \to 2}\frac{2x^2-4x}{x-2}=\lim_{x \to 2}\frac{2x(x-2)}{x-2}=\lim_{x \to 2}2x=4$\\	
	\end{center}

\end{enumerate}




\end{document}
