\documentclass{ximera}

\newcommand{\RR}{\mathbb R}
\renewcommand{\d}{\,d}
\newcommand{\dd}[2][]{\frac{d #1}{d #2}}
\renewcommand{\l}{\ell}
\newcommand{\ddx}{\frac{d}{dx}}
\newcommand{\dfn}{\textbf}
\newcommand{\eval}[1]{\bigg[ #1 \bigg]}
\renewcommand{\theenumii}{\textup{(\roman{enumii})}}
\renewcommand{\labelenumii}{\theenumii}

\usepackage{graphicx}
\usepackage{multicol}
\usepackage{tkz-euclide}
%\usepackage{unicode-math}

\usepackage{pgfplots}   % <- for graphics
\pgfplotsset{compat=newest}


\renewenvironment{freeResponse}{
\ifhandout\setbox0\vbox\bgroup\else
\begin{trivlist}\item[\hskip \labelsep\bfseries Solution:\hspace{2ex}]
\fi}
{\ifhandout\egroup\else
\end{trivlist}
\fi}

\newcommand*{\ZeroOverZero}{\ensuremath{\dfrac{0}{0}}}

\providecommand{\HCCondition}{0}
\newcommand{\WkstHop}[1][1]{\if\HCCondition 0
	\vspace*{\stretch{#1}} \fi} 
\newcommand{\WkstNew}{\if\HCCondition 0
	\newpage
	 \fi} 


\title[Problem 6]{Problem 6}

\begin{document}
\begin{abstract} \end{abstract}
\maketitle


% Extracted from indeterminateForms.tex, problem #6
\begin{problem}
A  piecewise defined function $f$ is given by 
 
	$f(x) =   \left\{ \begin{array}{cl}
	\frac{2 x-3}{x-2}		 	&	\qquad \text{if }\hspace{0.1in}  x <  2					\\ \\
	\frac{x^2-5x +6}{x^2-4}	&	\qquad \text{if } \hspace{0.1in}   x>2 	\\ \\
		
						\end{array} \right.  $\\[1em]
	Determine the form of the limit, then find the limit.
  \begin{enumerate}

    \item $ \lim_{x \to 2^+} f(x)$
\begin{explanation}
   
   Since $ \lim_{x \to 2^+} f(x)= \lim_{x \to 2^+}\frac{x^2-5x +6}{x^2-4}$, the form of the limit is $\frac{0}{0}$.\\[1em]
     $ \lim_{x \to 2^+} f(x)= \lim_{x \to 2^+}\frac{x^2-5x +6}{x^2-4}= \lim_{x \to 2^+}\frac{(x-3)(x-2)}{(x-2)(x+2)}= \lim_{x \to 2^+}\frac{x-3}{x+2}=-\frac{1}{4}$
    \end{explanation}
  \item   $ \lim_{x \to 2^-} f(x)$
\begin{explanation}
   
   Since $ \lim_{x \to 2^-} f(x)= \lim_{x \to 2^-}\frac{2x-3}{x-2}$, the form of the limit is $\frac{\#}{0}$.\\[1em]
     $ \lim_{x \to 2^-} f(x)=  \lim_{x \to 2^-}\frac{2x-3}{x-2}= -\infty$,\\[1em]
      since the numerator  positive, the denominator negative and goes to 0.
    \end{explanation}
 \item $ \lim_{x \to 2} f(x)$
\begin{explanation}
   The limit does not exist, because  $ \lim_{x \to 2^+} f(x)\ne \lim_{x \to 2^-} f(x)$.
    \end{explanation}
      \end{enumerate}
\end{problem}



\end{document}
