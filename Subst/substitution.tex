%Add code to compile both versions from makefile at same time
\providecommand{\HCCondition}{0}
%Define each of the conditions
\ifcase\HCCondition
	%\condition=0 -> handout
	\documentclass[nooutcomes,noauthor,space,handout]{ximera}
	\title{Substitution (TIOS, WWS)}
\or	%\condition=1 -> Soln
	\documentclass[nooutcomes,noauthor]{ximera}
	\title{Substitution (TIOS, WWS) - Solutions}  
\fi
\usepackage{fullpage}

\newcommand{\RR}{\mathbb R}
\renewcommand{\d}{\,d}
\newcommand{\dd}[2][]{\frac{d #1}{d #2}}
\renewcommand{\l}{\ell}
\newcommand{\ddx}{\frac{d}{dx}}
\newcommand{\dfn}{\textbf}
\newcommand{\eval}[1]{\bigg[ #1 \bigg]}
\renewcommand{\theenumii}{\textup{(\roman{enumii})}}
\renewcommand{\labelenumii}{\theenumii}

\usepackage{graphicx}
\usepackage{multicol}
\usepackage{tkz-euclide}
%\usepackage{unicode-math}

\usepackage{pgfplots}   % <- for graphics
\pgfplotsset{compat=newest}


\renewenvironment{freeResponse}{
\ifhandout\setbox0\vbox\bgroup\else
\begin{trivlist}\item[\hskip \labelsep\bfseries Solution:\hspace{2ex}]
\fi}
{\ifhandout\egroup\else
\end{trivlist}
\fi}

\newcommand*{\ZeroOverZero}{\ensuremath{\dfrac{0}{0}}}

\providecommand{\HCCondition}{0}
\newcommand{\WkstHop}[1][1]{\if\HCCondition 0
	\vspace*{\stretch{#1}} \fi} 
\newcommand{\WkstNew}{\if\HCCondition 0
	\newpage
	 \fi} 


\title{Substitution (TIOS, WWS) }

\begin{document}
\begin{abstract}
\end{abstract}
\maketitle
\ifcase\HCCondition
\section*{Recitation Questions}
\fi

\begin{problem}
	Compute the following indefinite integrals.
	\begin{enumerate}
		\item $\displaystyle \int 2t \sin(t^2) \, \d t$.
			\WkstHop
        	\begin{freeResponse}
        		Let $w=t^2$. Then $\d w = 2t \, \d t$ so
        		\begin{align*}
        			\int 2t \sin(t^2) \, \d t &= \int \sin(2) \, \d w\\
        				&= -\cos(w) + C\\
        				&= -\cos(t^2) + C.
        		\end{align*}
        	\end{freeResponse}
	
		\item $\displaystyle \int \cos(x) \sqrt{\sin(x)} \, \d x$.
	\WkstHop
	\begin{freeResponse}
		Set $v = \sin(x)$. Then $dv = \cos(x) \, dx$, so
		\begin{align*}
			\int \cos(x) \sqrt{\sin(x)} \, \d x &= \int \sqrt{v} \, dv\\
				&= \dfrac{2}{3} v^{3/2} + C\\
				&= \dfrac{2}{3} (\sin(x))^{3/2} + C
		\end{align*}
	\end{freeResponse}
		
	\item $\displaystyle \int \left( 3t^2-4+\dfrac{1}{t} \right) e^{t^3-4t+\ln(t)-9}\, \d t$.
	\WkstHop
	\begin{freeResponse}
		This problem looks a lot more difficult than it really is. Let $u = t^3-4t+\ln(t)-9$. Then $\d u = \left( 3t^2- 4+\dfrac{1}{t} \right) \, \d t$, 
		which is exactly the factor in front of the exponential.
		\begin{align*}
			\int \left( 3t^2-4+\dfrac{1}{t} \right) e^{t^3-4t+\ln(t)-9}\, \d t &= \int e^u \, \d u\\
				&= e^u + C\\
				&= e^{t^3-4t+\ln(t)-9}+C.
		\end{align*}
	\end{freeResponse}

	\end{enumerate}
	
\end{problem}


\WkstNew


\begin{problem}
	Evaluate the following definite integrals:
	\begin{enumerate}
		\item $\displaystyle \int_{-2}^1 t^2 \sin(t^3) \, \d t$
\WkstHop
			\begin{freeResponse}
			Set $v = v(t) = t^3$, where ``v(t)'' is meant to indicate that the variable $v$ is really a function of $t$. Then $dv = 3t^2 dt$. But notice that
		\begin{align*}
			v(-2) &= (-2)^3 = -8\\
			v(1) &= (1)^3 = 1.
		\end{align*}
			Then
				\begin{align*}
					 \int_{-2}^1 t^2 \sin(t^3) \, dt &= \dfrac{1}{3} \int_{-2}^{1} 3t^2 \sin(t^3) dt\\
					 	&= \dfrac{1}{3} \int_{-8}^1 \sin(v) dv\\
					 	&= -\dfrac{1}{3}\eval{\cos(v)}_{-8}^1\\
					 	&= -\dfrac{1}{3}\left( \cos(1)-\cos(-8)\right)
				\end{align*}
			\end{freeResponse}
		
		\item $\displaystyle \int_{0}^{1/2} \dfrac{13e^u}{3e^u-5} \, \d u$
\WkstHop
			\begin{freeResponse}
				Let $w = 3e^u - 5$. Then
				\begin{align*}
					dw &= 3e^u du\\
					w(0) &= 3e^0 - 5 = 3-5 = -2\\
					w\left(\frac{1}{2}\right) &= 3e^{\frac{1}{2}} - 5.
				\end{align*}
				Substituting this into the original integral gives
				\begin{align*}
					\int_{0}^{1/2} \dfrac{13e^u}{3e^u-5} \, du &= \dfrac{13}{3} \int_{0}^{1/2} \dfrac{3e^u}{3e^u-5} \, du\\
						&= \dfrac{13}{3} \int_{-2}^{3e^{\frac{1}{2}} - 5} \dfrac{1}{w} dw\\
						&= \dfrac{13}{3} \eval{\ln|w|}_{-2}^{3e^{\frac{1}{2}} - 5} \\
						&= \dfrac{13}{3} \left( \ln|3e^{\frac{1}{2}} - 5|- \ln|-2|\right)\\ 
						&= \dfrac{13}{3} \left( \ln\left(5-3e^{\frac{1}{2}} \right)- \ln(2)\right). 
				\end{align*}
				NOTE: $3e^u -5$ is always negative on our interval $\left[0, \frac{1}{2}\right]$. Therefore, the denominator is never $0$ 
				and the integrand (function) is continuous on $\left[0, \frac{1}{2}\right]$.
			\end{freeResponse}

		\item $\displaystyle \int_{0}^{\pi/2} \dfrac{\cos(x)}{4+\sin(x)} \, \d x$
\WkstHop
			\begin{freeResponse}
				Let $w = 4+\sin(x)$, so that $dw = \cos(x) dx$. When $x=0$, $w = 4$ and when $x=\frac{\pi}{2}$, $w=5$.
				\begin{align*}
					\int_{0}^{\pi/2} \dfrac{\cos(x)}{4+\sin(x)} \, dx &= \int_{4}^5 \dfrac{dw}{w}\\
						&= \eval{ \ln|w| }_4^5\\
						&= \ln(5) - \ln(4)\\
						&= \ln\left( \frac{5}{4} \right).
				\end{align*}
			\end{freeResponse}
\WkstNew


		\item $\displaystyle \int_{1}^{4} \dfrac{e^{\sqrt{x}}}{3\sqrt{x}} \, \d x$
\WkstHop
			\begin{freeResponse}
				Set $u = \sqrt{x}$. Then
				\begin{align*}
					du &= \dfrac{1}{2\sqrt{x}} dx\\
					u(1) &= \sqrt{1} = 1\\
					u(4) &= \sqrt{4} = 2.
				\end{align*}
				Substituting these into the original integral gives:
				\begin{align*}
					\int_{1}^{4} \dfrac{e^{\sqrt{x}}}{3\sqrt{x}} \, \d x &= \dfrac{2}{3}\int_{1}^{4} \dfrac{e^{\sqrt{x}}}{2\sqrt{x}} \, \d x \\
						&= \dfrac{2}{3}\int_{1}^{2} e^u du\\
						&= \dfrac{2}{3} \eval{e^u}_1^2\\
						&= \dfrac{2}{3} \left( e^2-e\right)
				\end{align*}

			\end{freeResponse}

		\item $\displaystyle \int_{\pi/3}^{\pi/2} \sin(x)\sec^2(\cos(x)) \, \d x$
\WkstHop
			\begin{freeResponse}
				Let $v = \cos(x)$. Then
				\begin{align*}
					dv &= -\sin(x) dx\\
					v\left(\frac{\pi}{3} \right) &= \frac{1}{2}\\
					v\left(\frac{\pi}{2} \right) &= 0.
				\end{align*}
				So
				\begin{align*}
					\int_{\pi/3}^{\pi/2} \sin(x)\sec^2(\cos(x)) \, \d x &= -\int_{1/2}^0 \sec^2(v) dv\\
						&= -\eval{\tan(v)}_{1/2}^0\\
						&= -\left( 0 - \tan\left( \frac{1}{2}\right) \right)\\
						&= \tan\left(\frac{1}{2}\right).
				\end{align*}
			\end{freeResponse}
	\end{enumerate}
\end{problem}



\WkstNew

\begin{problem}
Suppose that 
\(
\int_{1}^{3} f(x)\d x=4
\)
\begin{enumerate}
\item Evaluate the following integrals.
	\begin{enumerate}
	\item $\int_{0}^{2} f(x+1)\d x$
\WkstHop
	  \begin{freeResponse}
	  Let $u=x+1$. Then $\d u=\d x$; when $x=0$, $u=1$; when $x=2$, $u=3$.
	   Therefore,\\[1em]
	$\int_{0}^{2} f(x+1)\d x=\int_{1}^{3} f(u)\d u=4$\\[1em]
	Notice that the region whose net area is given by the integral  $\int_{0}^{2} f(x+1)\d x$  is just the region whose net area is given by the integral  $\int_{1}^{3} f(x)\d x$, only ``shifted to the left by 1".
	Obviously, their net areas  have to be equal. We illustrate this in the figure below. \\
	 \includegraphics[scale = 0.5]{Substitutionimage002.png}
	    \end{freeResponse}
	\item $\int_{1}^{9} 3\dfrac{f(\sqrt{x})}{\sqrt{x}}\d x$
\WkstHop
	  \begin{freeResponse}
	  Let $u=\sqrt{x}$. Then $\d u=\dfrac{1}{2\sqrt{x}} \d x$; when $x=1$, $u=1$; when $x=9$, $u=3$.
	  Therefore,\\[1em]
	  $\int_{1}^{9} 3\dfrac{f(\sqrt{x})}{\sqrt{x}}\d x=\int_{1}^{9} 3\cdot 2\cdot\dfrac{f(\sqrt{x})}{2\cdot\sqrt{x}}\d x=\int_{1}^{3} 6f(u)\d u=6\int_{1}^{3} f(u)\d u=6\cdot 4=24  $
	    \end{freeResponse}
	\item  $\int_{\frac{1}{3}}^{1} f(3x)\d x$
\WkstHop
	  \begin{freeResponse}
	  Let $u=3x$. Then $\d u=3\d x$; when $x=\frac{1}{3}$, $u=1$; when $x=1$, $u=3$.
	  Therefore,\\[1em]
	  $\int_{\frac{1}{3}}^{1} f(3x)\d x=\int_{\frac{1}{3}}^{1}\dfrac{3}{3} f(3x)\d x=\int_{1}^{3} \dfrac{1}{3}f(u)\d u= \dfrac{1}{3}\int_{1}^{3}f(u)\d u= \dfrac{1}{3}\cdot4= \dfrac{4}{3}$
	    \end{freeResponse}
	\item $\int_{2}^{4} 3f(x-1)\d x$
\WkstHop
	  \begin{freeResponse}
	  Let $u=x-1$. Then $\d u=\d x$; when $x=1$, $u=1$;  when $x=4$, $u=3$.
	    Therefore,\\[1em]
    	$\int_{2}^{4} 3f(x-1)\d x= \int_{1}^{3} 3f(u)\d u=3\int_{1}^{3} f(u)\d u=3\cdot4=12$
	    \end{freeResponse}
	\item $\int_{0}^{\sqrt{2}} 3x f(x^2+1)\d x$
\WkstHop
	  \begin{freeResponse}
	  Let $u=x^2+1$. Then $\d u= 2x\d x$; when $x=0$, $u=1$; when $x=\sqrt{2}$, $u=3$.
	   Therefore,\\[1em]
	   $\int_{0}^{\sqrt{2}} 3x f(x^2+1)\d x=\int_{0}^{\sqrt{2}} 3\cdot \dfrac{2}{2}\cdot x f(x^2+1)\d x=\int_{1}^{3}  \dfrac{3}{2}\cdot  f(u)\d u= \dfrac{3}{2}\cdot \int_{1}^{3}  f(u)\d u= \dfrac{3}{2}\cdot 4=6$
	    \end{freeResponse}
	
	\end{enumerate}
\WkstNew

\item Assume that $f$ is odd. Evaluate   $\int_{-3}^{-1} f(x)\d x$
\WkstHop
  \begin{freeResponse}
  Since $f$ is odd, then $f(-x)=-f(x)$ or $f(x)=-f(-x)$, for all $x$ in its domain.\\
   Let $u=-x$. Then $\d u=-\d x$; when $x=-3$, $u=3$; when $x=-1$, $u=1$. \\[1em]
       Therefore,\\[1em]
  $\int_{-3}^{-1} f(x)\d x=\int_{-3}^{-1}[ -f(-x)]\d x=\int_{3}^{1}f(u)\d u=-\int_{1}^{3}f(u)\d u=-4$
  
    \end{freeResponse}
\item Assume that $f$ is even. Evaluate   $\int_{-3}^{-1} f(x)\d x$
\WkstHop
  \begin{freeResponse}
   Since $f$ is even, then $f(-x)=f(x)$, for all $x$ in its domain.\\
   Let $u=-x$. Then $\d u=-\d x$; when $x=-3$, $u=3$; when $x=-1$, $u=1$. \\[1em]
       Therefore,\\[1em]
  $\int_{-3}^{-1} f(x)\d x=\int_{-3}^{-1}[ f(-x)]\d x=-\int_{-3}^{-1}[ f(-x)](-\d x)=-\int_{3}^{1}f(u)\d u=\int_{1}^{3}f(u)\d u=4$
  

    \end{freeResponse}
\end{enumerate}
\end{problem}

\WkstNew



\begin{problem}
Show that the areas of the  three shaded regions in the figure below are equal. Find this area.\\
 \includegraphics[scale = 0.7]{Substitutionimage001.png}
\WkstHop
  \begin{freeResponse}
  The area of the first shaded region is given by
  $\int_{0}^{1} \sqrt{3x+1}\d x$.\\
  We can evaluate this integral by substituting:
  $u=3x+1$, $\d u=3 \d x$; when $x=0$, $u=1$; when $x=1$, $u=4$.\\
  Therefore 
   $\int_{0}^{1} \sqrt{3x+1}\d x= \int_{1}^{4} \dfrac{1}{3}\sqrt{u}\d u$\\
   and the last integral gives the area of the region in the middle.
   
   The same integral can also  be evaluated by substituting
    $u=\sqrt{3x+1}$, $\d u=\dfrac{3}{2\sqrt{3x+1}} \d x=\dfrac{3}{2u} \d x$;\\ when $x=0$, $u=1$; when $x=1$, $u=2$.\\[1em]
     In this case we have 

 $\int_{0}^{1} \sqrt{3x+1}\d x= \int_{1}^{2} u\cdot\dfrac{2}{3}u \d u=\int_{1}^{2} \dfrac{2}{3}u^2 \d u$.\\
 
 The last integral gives the area of the third shaded region.\\
 Therefore, the areas are equal.\\[1em]
 In order to compute this area, let's  evaluate the last integral.\\
$ \int_{1}^{2} \dfrac{2}{3}x^2 \d x=\dfrac{2}{3}\Bigl[\dfrac{x^3}{3}\Bigr]_{1}^{2}=\dfrac{2}{3}\Bigl[\dfrac{8-1}{3}\Bigr]=\dfrac{14}{9}$
   \end{freeResponse}
\end{problem}


\WkstNew



\begin{problem}
Find the error in the following ``solution'':

Find $\displaystyle \int_{-2}^{2} \dfrac{1}{x^8-1} \, \d x$.
	\begin{center}
	 \includegraphics[scale = 0.7]{Images/figure1.PNG}
	\end{center}
\WkstHop
	\begin{freeResponse}
		The error is that the definite integral is not even defined. Specifically, the function $\dfrac{1}{x^8-1}$ is not defined at $x=1$, and the limit
		$\displaystyle \lim_{x\to 1^+} \frac{1}{x^8-1} = \infty$. Other errors were also made. For example, when $-2 \leq x \leq 0$ we know that
		$u^{\frac{1}{4}} = -x$, not $x$, which makes the rest of this ``solution'' invalid.
	\end{freeResponse}
\end{problem}


\WkstNew




\begin{problem}
Find the error in the following ``solution", correct the error (or errors) and evaluate the integral.

Find $\int_{-\frac{\pi}{2}}^{\frac{\pi}{2}} \cos(x)\sqrt{1-\cos^2(x)}\d x$.\\[1em]
"Solution":\\[1em]
Since $\sin(x)=\sqrt{1-\cos^2(x)}$, let $u=\sin(x)$. Then $du=\cos(x)dx$; \\[1em]
when $x=-\frac{\pi}{2}$, $u=\sin\Bigl(-\frac{\pi}{2}\Bigr)=-1$; when $x=\frac{\pi}{2}$, $u=\sin\Bigl(\frac{\pi}{2}\Bigr)=1$. Therefore\\[1em]
	$\int_{-\frac{\pi}{2}}^{\frac{\pi}{2}} \cos(x)\sqrt{1-\cos^2(x)}\d x=\int_{-1}^{1}u \cdot du=0$,\\[1em] due to symmetry (integrating an odd function over a symmetric interval, $[-1,1]$).

\WkstHop
	\begin{freeResponse}
	[Recall: $\sin(x)\le 0$ in the fourth quadrant! ]\\[1em]
	The error is that   for $-\frac{\pi}{2}\le x\le 0$  \\
	$\sqrt{1-\cos^2(x)}=\sqrt{\sin^2(x)}=|\sin(x)|=-\sin(x)$,\\
	 not`` $\sin(x)$", which makes the rest of the "solution" completely wrong.  \\[1em]
	In order to correct this error, we can split the integral into two integrals \\[1em]
	$\int_{-\frac{\pi}{2}}^{\frac{\pi}{2}} \cos(x)\sqrt{1-\cos^2(x)}\d x=\int_{-\frac{\pi}{2}}^{0} \cos(x)\sqrt{1-\cos^2(x)}\d x+\int_{0}^{\frac{\pi}{2}} \cos(x)\sqrt{1-\cos^2(x)}\d x$\\\\[1em]
	and evaluate them separately, i.e., in the first integral we have that $\sqrt{1-\cos^2(x)}=-\sin(x)$  and in the second that $\sqrt{1-\cos^2(x)}=\sin(x)$.\\[1em]
	Or, we can use symmetry to make the computation easier.\\[1em] Namely, we ``integrate" an even function over a symmetric interval $\Bigl[-\dfrac{\pi}{2},\dfrac{\pi}{2}\Bigr]$. Therefore\\[1em]
	$\int_{-\frac{\pi}{2}}^{\frac{\pi}{2}} \cos(x)\sqrt{1-\cos^2(x)}\d x=2\int_{0}^{\frac{\pi}{2}} \cos(x)\sqrt{1-\cos^2(x)}\d x$.\\[1em]
	Now we use the same substitution as in the ``solution" above.\\[1em]
	$\int_{-\frac{\pi}{2}}^{\frac{\pi}{2}} \cos(x)\sqrt{1-\cos^2(x)}\d x=2\int_{0}^{\frac{\pi}{2}} \cos(x)\sqrt{1-\cos^2(x)}\d x=2\int_{0}^{1}u \cdot du=2\Bigl[\dfrac{u^2}{2}\Bigr]_{0}^{1}=2\cdot \dfrac{1}{2}=1$
	\end{freeResponse}
\end{problem}

\WkstNew



\begin{problem}
	Compute the integral: $\displaystyle \int \dfrac{1+3x}{4+4x^2} \, dx$.
\WkstHop
	\begin{freeResponse}
		First notice that
		\[ \int \dfrac{1+3x}{4+4x^2} \, dx = \dfrac{1}{4}\int \dfrac{1+3x}{1+x^2} \, dx = \dfrac{1}{4} \int \left(\dfrac{1}{1+x^2} + \dfrac{3x}{1+x^2} \right) \, dx.\]
		The first integral is $\arctan(x)$, so we have
		\[ \int \dfrac{1+3x}{4+4x^2} \, dx = \dfrac{1}{4}\arctan(x) + \dfrac{3}{4}\int \dfrac{x}{1+x^2} \, dx.\]
		We can evaluate the second integral by substitution. If we set $u = 1+x^2$, then $du = 2x dx$ and $\dfrac{1}{2}du = x dx$.
		\[ \int \dfrac{x}{1+x^2} dx = \frac{1}{2} \int \frac{1}{u} du = \frac{1}{2} \ln|u| + C = \frac{1}{2}\ln\left(1+x^2\right)+C.\]
		
		Plugging this into our calculation from above gives
		\[ \int \dfrac{1+3x}{4+4x^2} \, dx = \dfrac{1}{4}\arctan(x) + \dfrac{3}{8}\ln\left(1+x^2\right)+C.\]
	\end{freeResponse}
\end{problem}


\begin{problem}
	Evaluate the integral $\displaystyle \int \dfrac{x^2}{1+x^2} \, \d x$.
\WkstHop
	\begin{freeResponse}
	
		Evaluating this integral does not involve substitution. Determining whether substitution is useful for calculating a particular integral takes practice.
		
		Even though it looks similar to others in this worksheet, we can rewrite the function $\dfrac{x^2}{1+x^2}$
		into a form that we can integrate from our derivative shortcut formulas by just a bit of algebra.
		
		Typically, when integrating a rational function where the degree of the numerator is greater than or equal to the degree of the denominator, 
		you will perform long division to get the smallest possible degree in the numerator. 

		Using long division: $\displaystyle \frac{x^2}{1+x^2} = 1 - \frac{1}{1+x^2}$.
		
		But watch this trick:
		\[ \frac{x^2}{1+x^2} = \frac{x^2+1-1}{1+x^2} = \frac{(1+x^2)-1}{1+x^2} = \frac{1+x^2}{1+x^2}-\frac{1}{1+x^2} = 1 - \frac{1}{1+x^2}.\]

		Then
		\begin{align*}
			\int \frac{x^2}{1+x^2} dx &= \int \left( 1 - \frac{1}{1+x^2} \right) dx \\
				&= x - \arctan(x) + C.
		\end{align*}
	\end{freeResponse}
\end{problem}


\WkstNew

\begin{problem}
	Compute the integral: $\displaystyle \int \sec^2(x) \tan(x) \, \d x$.
\WkstHop
	\begin{freeResponse}
		Let $u = \tan(x)$. Then $du = \sec^2(x) dx$, and so
		\begin{align*}
			\int sec^2(x) \tan(x) \, \d x &= \int u du\\
				&= \dfrac{1}{2}u^2 + C\\
				&= \dfrac{1}{2}\tan^2(x) + C.
		\end{align*}
		NOTE: The substitution $v = \sec(x)$ would also work to solve this problem, as would the substitution $w = \sec^2(x)$. 
		It is a good exercise to work these out!
	\end{freeResponse}
\end{problem}

\WkstNew

\begin{problem}
	What are two substitutions that can be used to evaluate the integral
	\[ \int x \sqrt{x+8} \, \d x \]
\WkstHop
	\begin{freeResponse}
		Two substitutions which would work are $w = x+8$ and $v = \sqrt{x+8}$. 
		
		The $w = x+8$ is the more obvious choice, so let's work through that
		one first. If $w = x+8$, then $\d w = \d x$ and $x = w-8$. Substituting into the original integral:
		\begin{align*}
			\int x \sqrt{x+8} \, \d x &= \int (w-8) \sqrt{w} \, dw\\
				&= \int (w-8) w^{1/2} \, dw\\
				&= \int \left( w^{3/2} - 8 w^{1/2} \right) \, dw\\
				&= \dfrac{2}{5}w^{5/2} - 8 \cdot \dfrac{2}{3} w^{3/2} + C\\
				&= \dfrac{2}{5}(x+8)^{5/2} - \dfrac{16}{3} (x+8)^{3/2} + C
		\end{align*}


		If $v = \sqrt{x+8}$, then 
		\[dv = \dfrac{1}{2\sqrt{x+8}} dx = \dfrac{1}{2v} dx \implies dx = 2v dv.\] 
		Also \[ v = \sqrt{x+8} \implies v^2=x+8 \implies x = v^2-8. \]
		Substituting these into the original integral gives:
		\begin{align*}
			\int x \sqrt{x+8} \, \d x &= \int (v^2-8) (v)(2v) \, dv\\
				&= \int (2v^4-16v^2) \, dv\\
				&= \dfrac{2}{5} v^5 - \dfrac{16}{3}v^3 + C\\			
				&= \dfrac{2}{5}(\sqrt{x+8})^5 - \dfrac{16}{3}(\sqrt{x+8})^3 + C.			
		\end{align*}
	\end{freeResponse}
\end{problem}

\WkstNew


\begin{problem}
	Compute the integral: $\displaystyle \int \dfrac{x}{\sqrt{x-4}} \, \d x$.
\WkstHop
	\begin{freeResponse}
		There are two possible solutions: Let $w = \sqrt{x-4}$. Then
		\[ dw = \dfrac{dx}{2\sqrt{x-4}} \implies 2 dw = \dfrac{dx}{\sqrt{x-4}} \]
		and
		\[ w = \sqrt{x-4} \implies x = w^2+4. \]
		Substituting these into the original integral and evaluating:
		\begin{align*}
			\int \dfrac{x}{\sqrt{x-4}} \, dx &= \int 2(w^2+4) \, dw\\
				&= 2\left( \dfrac{w^3}{3} + 4w \right) + C\\
				&= 2\left( \dfrac{(\sqrt{x-4})^3}{3} + 4\sqrt{x-4} \right) + C.
		\end{align*}
		
		The other solution: Let $v = x-4$. Then $dv = dx$ and $x=v+4$.
		\begin{align*}
			\int \dfrac{x}{\sqrt{x-4}} \, dx &= \int \dfrac{v+4}{\sqrt{v}} \, dv\\
				&= \int \left( v^{1/2} + 4v^{-1/2}\right) \, dv\\
				&= \dfrac{2}{3}v^{3/2} + 8v^{1/2} + C\\
				&= \dfrac{2}{3}(x-4)^{3/2} + 8(x-4)^{1/2} + C\\
		\end{align*}
	\end{freeResponse}
\end{problem}

\WkstNew

\begin{problem}
	Evaluate the following indefinite integrals:
	\begin{enumerate}
		
		\item $\displaystyle \int \dfrac{13x^7}{\sqrt{3x^4-5}} \, \d x$
\WkstHop
			\begin{freeResponse}
				Set $v = 3x^4-5$. Then $dv = 12x^3 dx$ and $x^4 = \frac{1}{3}(v+5)$. Substituting into the integral gives:
				\begin{align*}
					 \int \dfrac{13x^7}{\sqrt{3x^4-5}} \, \d x &= \dfrac{13}{12} \int \dfrac{(x^4)(12x^3)}{\sqrt{3x^4-5}} dx\\
					 	&= \dfrac{13}{12} \int \dfrac{\frac{1}{3}(v+5)}{\sqrt{v}} dv\\
					 	&= \dfrac{13}{36} \int \left( v^{\frac{1}{2}} + 5v^{-\frac{1}{2}}\right) dv\\
					 	&= \dfrac{13}{36} \left( \frac{2}{3}v^{\frac{3}{2}} + 10 v^{\frac{1}{2}} \right) + C\\
					 	&= \dfrac{13}{36} \left( \frac{2}{3}(3x^4-5)^{\frac{3}{2}} + 10 (3x^4-5)^{\frac{1}{2}} \right) + C.
				\end{align*}

			\end{freeResponse}

		\item $\displaystyle \int \dfrac{u^3}{u^2-3} \, \d u$
\WkstHop
			\begin{freeResponse}
				Set $w = u^2-3$. Then $dw = 2u du$ and $u^2=w+3$. This gives
				\begin{align*}
					\int \dfrac{u^3}{u^2-3} \, \d u &= \frac{1}{2} \int\frac{(u^2)(2u)}{u^2-3} du\\
						&= \frac{1}{2} \int\frac{w+3}{w} du\\
						&= \frac{1}{2} \int \left( 1 + \frac{3}{w} \right) du\\
						&= \frac{1}{2} \left( w + 3\ln|w| \right) + C\\ 
						&= \frac{1}{2} \left(  u^2-3 + 3\ln| u^2-3| \right) + C\\ 
				\end{align*}

			\end{freeResponse}

		\item $\displaystyle \int e^{t^2+\ln(t)} \, \d t$
\WkstHop
			\begin{freeResponse}
				First notice that $\displaystyle \int e^{t^2+\ln(t)} dt =\int e^{t^2}e^{\ln(t)} \, \d t = \int t e^{t^2} \, \d t $.

				Let $u = t^2$. Then $du = 2t dt$ and $\frac{1}{2} \d u = t \d t$.
				\begin{align*}
					\int e^{t^2+\ln(t)} \, \d t &= \int t e^{t^2} \, \d t\\
						&= \frac{1}{2} \int e^u du\\
						&= \frac{1}{2} e^u + C\\						
						&= \frac{1}{2} e^{t^2} + C.
				\end{align*}

			\end{freeResponse}
	\end{enumerate}
\end{problem}






%problem 12


\end{document}

