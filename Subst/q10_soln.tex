\documentclass{ximera}

\newcommand{\RR}{\mathbb R}
\renewcommand{\d}{\,d}
\newcommand{\dd}[2][]{\frac{d #1}{d #2}}
\renewcommand{\l}{\ell}
\newcommand{\ddx}{\frac{d}{dx}}
\newcommand{\dfn}{\textbf}
\newcommand{\eval}[1]{\bigg[ #1 \bigg]}
\renewcommand{\theenumii}{\textup{(\roman{enumii})}}
\renewcommand{\labelenumii}{\theenumii}

\usepackage{graphicx}
\usepackage{multicol}
\usepackage{tkz-euclide}
%\usepackage{unicode-math}

\usepackage{pgfplots}   % <- for graphics
\pgfplotsset{compat=newest}


\renewenvironment{freeResponse}{
\ifhandout\setbox0\vbox\bgroup\else
\begin{trivlist}\item[\hskip \labelsep\bfseries Solution:\hspace{2ex}]
\fi}
{\ifhandout\egroup\else
\end{trivlist}
\fi}

\newcommand*{\ZeroOverZero}{\ensuremath{\dfrac{0}{0}}}

\providecommand{\HCCondition}{0}
\newcommand{\WkstHop}[1][1]{\if\HCCondition 0
	\vspace*{\stretch{#1}} \fi} 
\newcommand{\WkstNew}{\if\HCCondition 0
	\newpage
	 \fi} 


\title[Problem 10]{Problem 10}

\begin{document}
\begin{abstract} \end{abstract}
\maketitle


% Extracted from substitution.tex, problem #10
\begin{problem}
	What are two substitutions that can be used to evaluate the integral
	\[ \int x \sqrt{x+8} \, \d x \]
\begin{explanation}
		Two substitutions which would work are $w = x+8$ and $v = \sqrt{x+8}$. 
		
		The $w = x+8$ is the more obvious choice, so let's work through that
		one first. If $w = x+8$, then $\d w = \d x$ and $x = w-8$. Substituting into the original integral:
		\begin{align*}
			\int x \sqrt{x+8} \, \d x &= \int (w-8) \sqrt{w} \, dw\\
				&= \int (w-8) w^{1/2} \, dw\\
				&= \int \left( w^{3/2} - 8 w^{1/2} \right) \, dw\\
				&= \dfrac{2}{5}w^{5/2} - 8 \cdot \dfrac{2}{3} w^{3/2} + C\\
				&= \dfrac{2}{5}(x+8)^{5/2} - \dfrac{16}{3} (x+8)^{3/2} + C
		\end{align*}


		If $v = \sqrt{x+8}$, then 
		\[dv = \dfrac{1}{2\sqrt{x+8}} dx = \dfrac{1}{2v} dx \implies dx = 2v dv.\] 
		Also \[ v = \sqrt{x+8} \implies v^2=x+8 \implies x = v^2-8. \]
		Substituting these into the original integral gives:
		\begin{align*}
			\int x \sqrt{x+8} \, \d x &= \int (v^2-8) (v)(2v) \, dv\\
				&= \int (2v^4-16v^2) \, dv\\
				&= \dfrac{2}{5} v^5 - \dfrac{16}{3}v^3 + C\\			
				&= \dfrac{2}{5}(\sqrt{x+8})^5 - \dfrac{16}{3}(\sqrt{x+8})^3 + C.			
		\end{align*}
	\end{explanation}
\end{problem}



\end{document}
