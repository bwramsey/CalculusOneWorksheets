% Extracted from substitution.tex, problem #8
\begin{problem}
	Evaluate the integral $\displaystyle \int \dfrac{x^2}{1+x^2} \, \d x$.
\WkstHop
	\begin{freeResponse}
	
		Evaluating this integral does not involve substitution. Determining whether substitution is useful for calculating a particular integral takes practice.
		
		Even though it looks similar to others in this worksheet, we can rewrite the function $\dfrac{x^2}{1+x^2}$
		into a form that we can integrate from our derivative shortcut formulas by just a bit of algebra.
		
		Typically, when integrating a rational function where the degree of the numerator is greater than or equal to the degree of the denominator, 
		you will perform long division to get the smallest possible degree in the numerator. 

		Using long division: $\displaystyle \frac{x^2}{1+x^2} = 1 - \frac{1}{1+x^2}$.
		
		But watch this trick:
		\[ \frac{x^2}{1+x^2} = \frac{x^2+1-1}{1+x^2} = \frac{(1+x^2)-1}{1+x^2} = \frac{1+x^2}{1+x^2}-\frac{1}{1+x^2} = 1 - \frac{1}{1+x^2}.\]

		Then
		\begin{align*}
			\int \frac{x^2}{1+x^2} dx &= \int \left( 1 - \frac{1}{1+x^2} \right) dx \\
				&= x - \arctan(x) + C.
		\end{align*}
	\end{freeResponse}
\end{problem}
