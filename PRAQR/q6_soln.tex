\documentclass{ximera}

\newcommand{\RR}{\mathbb R}
\renewcommand{\d}{\,d}
\newcommand{\dd}[2][]{\frac{d #1}{d #2}}
\renewcommand{\l}{\ell}
\newcommand{\ddx}{\frac{d}{dx}}
\newcommand{\dfn}{\textbf}
\newcommand{\eval}[1]{\bigg[ #1 \bigg]}
\renewcommand{\theenumii}{\textup{(\roman{enumii})}}
\renewcommand{\labelenumii}{\theenumii}

\usepackage{graphicx}
\usepackage{multicol}
\usepackage{tkz-euclide}
%\usepackage{unicode-math}

\usepackage{pgfplots}   % <- for graphics
\pgfplotsset{compat=newest}


\renewenvironment{freeResponse}{
\ifhandout\setbox0\vbox\bgroup\else
\begin{trivlist}\item[\hskip \labelsep\bfseries Solution:\hspace{2ex}]
\fi}
{\ifhandout\egroup\else
\end{trivlist}
\fi}

\newcommand*{\ZeroOverZero}{\ensuremath{\dfrac{0}{0}}}

\providecommand{\HCCondition}{0}
\newcommand{\WkstHop}[1][1]{\if\HCCondition 0
	\vspace*{\stretch{#1}} \fi} 
\newcommand{\WkstNew}{\if\HCCondition 0
	\newpage
	 \fi} 


\title[Problem 6]{Problem 6}

\begin{document}
\begin{abstract} \end{abstract}
\maketitle


% Extracted from productRuleAndQuotientRule.tex, problem #6
\begin{problem}
	The graph of a function $g$ is given below.  Using the graph, estimate the derivative at the given point: \\
	$\ddx \left({\frac{xg(x)}{x+3}}\right)\ \text{at}\ x=1$

   \begin{image}
     \includegraphics[scale = 0.5]{Figure1.png}
   \end{image}
\begin{explanation}
		From the graph it appears that $g(1)=2$.  We can draw the line tangent to the curve $y=g(x)$ at the point $(1,g(1))$
  		 \begin{image}
    			 \includegraphics[scale = 0.5]{Figure2.png}
 		  \end{image}
		
		Since $g'(1)$ is the slope of the line tangent to the curve $y=g(x)$ at the point $(1,g(1))$, we can estimate it from the figure above.  The tangent line seems to pass through the points $(1,2)$ and $(2,3)$, so its slope is $\frac{3-2}{2-1}=1$.

	\begin{align*}
	\eval{ \ddx \left( \frac{xg(x)}{x+3}\right)}_{x=1}&= \eval{\frac{(g(x)+xg'(x))(x+3)-xg(x)}{(x+3)^2}}_{x=1}\\
	&=\frac{(g(1)+1g'(1))(1+3)-(1)g(1)}{(1+3)^2} \\
	& \approx \frac{(2+1(1))(4)-1(2)}{4^2}\\
	&=\frac{(2+1)(4)-2}{16}\\
	&= \frac{3(4)-2}{16}\\
	&= \frac{12-2}{16}\\
	&= \frac{10}{16}\\
	&=\frac{5}{8}
	\end{align*}



	\end{explanation}

\end{problem}



\end{document}
