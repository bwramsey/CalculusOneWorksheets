%Add code to compile both versions from makefile at same time
\providecommand{\HCCondition}{0}
%Define each of the conditions
\ifcase\HCCondition
	%\condition=0 -> handout
	\documentclass[nooutcomes,noauthor,space,handout]{ximera}
	\title{Limit Laws (LL)}
\or	%\condition=1 -> Soln
	\documentclass[nooutcomes,noauthor]{ximera}
	\title{Limit Laws (LL) - Solutions} 
\fi

\newcommand{\RR}{\mathbb R}
\renewcommand{\d}{\,d}
\newcommand{\dd}[2][]{\frac{d #1}{d #2}}
\renewcommand{\l}{\ell}
\newcommand{\ddx}{\frac{d}{dx}}
\newcommand{\dfn}{\textbf}
\newcommand{\eval}[1]{\bigg[ #1 \bigg]}
\renewcommand{\theenumii}{\textup{(\roman{enumii})}}
\renewcommand{\labelenumii}{\theenumii}

\usepackage{graphicx}
\usepackage{multicol}
\usepackage{tkz-euclide}
%\usepackage{unicode-math}

\usepackage{pgfplots}   % <- for graphics
\pgfplotsset{compat=newest}


\renewenvironment{freeResponse}{
\ifhandout\setbox0\vbox\bgroup\else
\begin{trivlist}\item[\hskip \labelsep\bfseries Solution:\hspace{2ex}]
\fi}
{\ifhandout\egroup\else
\end{trivlist}
\fi}

\newcommand*{\ZeroOverZero}{\ensuremath{\dfrac{0}{0}}}

\providecommand{\HCCondition}{0}
\newcommand{\WkstHop}[1][1]{\if\HCCondition 0
	\vspace*{\stretch{#1}} \fi} 
\newcommand{\WkstNew}{\if\HCCondition 0
	\newpage
	 \fi}  %% we can turn off input when making a master document
\usepackage{fullpage}


\begin{document}
\begin{abstract}		\end{abstract}
\maketitle

\ifcase\HCCondition

 \section*{The Limit Laws:} 
 
  \begin{description}
    \item [Continuity] 
  
 
    If a function $f$ is \textbf{continuous at $a$}, then 
     \[
      \lim_{x \to a} f(x)  = f(a)
  \]
     \end{description}

\begin{description}
    \item [Composition Limit Law] 
  
 
    If a function $f$ is \textbf{continuous at $b= \lim_{x \to a} g(x)$}, then 
     \[
      \lim_{x \to a} f(g(x))  = f\Bigl( \lim_{x \to a}g(x)\Bigr)
  \]
     \end{description}

  For the limit laws below, assume $\lim_{x \to a} f(x)$ and $\lim_{x \to a} g(x)$ both exist. The following properties hold.
    \begin{description}
      \item[Sum Limit Law]
        \[
          \lim_{x \to a} \bigl(f(x) + g(x)\bigr) = \lim_{x \to a} f(x) + \lim_{x \to a} g(x)
        \]

      \item[Difference Limit Law]
        \[
          \lim_{x \to a} \bigl(f(x) - g(x)\bigr) = \lim_{x \to a} f(x) - \lim_{x \to a} g(x)
        \]

    %  \item[Constant multiple]
      %  \[
        %  \lim_{x \to a} \left[c \cdot f(x) \right] = c \cdot \lim_{x \to a} f(x)
       % \]

      \item[Product Limit Law]
        \[
          \lim_{x \to a} \bigl(f(x) \cdot g(x)\bigr) = \bigl(\lim_{x \to a} f(x)\bigr) \cdot \bigl(\lim_{x \to a} g(x) \bigr)
        \]

      \item[Quotient Limit Law]
        \[
          \lim_{x \to a} \left(\frac{f(x)}{g(x)}\right) = \frac{\lim_{x \to a} f(x)}{\lim_{x \to a} g(x)},\ \text{provided} \lim_{x \to a} g(x) \ne 0
        \]   

     % \item[ Power]
       % \[
        %  \lim_{x \to a} \bigl(f(x)\bigr)^n = \bigl(\lim_{x \to a} f(x)\bigr)^n
       % \]

     % \item[Fractional power]
      %  \[
      %    \lim_{x \to a} \bigl(f(x)\bigr)^{n/m} = \bigl(\lim_{x \to a} f(x)\bigr)^{n/m},
       % \]
      %  provided $f(x) \ge 0$, for $x$ near $a$, if $m$ is even and $n/m$ is reduced to lowest terms
    \end{description}
Limit laws also hold for one-sided limits.

\newpage
\section*{Recitation Questions}
\fi
%Problem 1
\begin{problem}

  The following argument shows 
  \[
    \lim_{x \to 3} \frac{5x^3 - 4 \sqrt{x}}{\sqrt{x^5 - 87}} = \frac{135 - 4\sqrt{3}}{\sqrt{156}}.
  \]
  State which limit law is used to justify each step.
  (A particular step may have more than one limit law as a justification.)

  \begin{align*}
    \lim_{x \to 3} \frac{5x^3 - 4 \sqrt{x}}{\sqrt{x^5 - 87}}
    &= \frac{\lim_{x \to 3} \left(5x^3 - 4 \sqrt{x}\right)}{\lim_{x \to 3}\sqrt{x^5 - 87}}\\
     &= \frac{\Big(\lim_{x \to 3}5\Big)\Big(\lim_{x \to 3}(x^3)\Big) - \Big(\lim_{x \to 3}4 \Big)\Big(\lim_{x \to 3}\sqrt{x}\Big)}{\sqrt{\lim_{x \to 3}(x^5 - 87)}}\\
    &= \frac{5\lim_{x \to 3}(x^3) - 4 \lim_{x \to 3}\sqrt{x}}{\sqrt{\lim_{x \to 3}(x^5 - 87)}}\\
    &= \frac{5(\lim_{x \to 3}x)^3 - 4 \sqrt{3}}{\sqrt{\lim_{x \to 3}(x^5) - \lim_{x \to 3} (87)}}\\
    &= \frac{5(3)^3 - 4 \sqrt{3}}{\sqrt{3^5 - 87}}\\
    &= \frac{135 - 4\sqrt{3}}{\sqrt{156}}
  \end{align*}
\WkstHop
  \begin{freeResponse}
      \begin{align*}
    \lim_{x \to 3} \frac{5x^3 - 4 \sqrt{x}}{\sqrt{x^5 - 87}}
    &= \frac{\lim_{x \to 3} \left(5x^3 - 4 \sqrt{x}\right)}{\lim_{x \to 3}\sqrt{x^5 - 87}}   \hspace{0.2in}Quotient \hspace{0.06in}law
\\ 
         &= \frac{\Big(\lim_{x \to 3}5\Big)\Big(\lim_{x \to 3}(x^3)\Big) - \Big(\lim_{x \to 3}4 \Big)\Big(\lim_{x \to 3}\sqrt{x}\Big)}{\sqrt{\lim_{x \to 3}(x^5 - 87)}} \hspace{0.2in}Product \hspace{0.06in}law, \hspace{0.06in}Composition\\
    &= \frac{5\lim_{x \to 3}(x^3) - 4 \lim_{x \to 3}\sqrt{x}}{\sqrt{\lim_{x \to 3}(x^5 - 87)}}\hspace{0.2in}Continuity\\
    &= \frac{5(\lim_{x \to 3}x)^3 - 4 \sqrt{3}}{\sqrt{\lim_{x \to 3}(x^5) - \lim_{x \to 3} (87)}}\hspace{0.2in} Difference, \hspace{0.2in}Continuity\\
    &= \frac{5(3)^3 - 4 \sqrt{3}}{\sqrt{3^5 - 87}}\hspace{0.2in}Continuity\\
    &= \frac{135 - 4\sqrt{3}}{\sqrt{156}}
  \end{align*}

  \end{freeResponse}



\end{problem}
\WkstNew

	%problem 2		
\begin{problem}
Find the limit and justify your answer.
\begin{enumerate}
	\item $\displaystyle \lim_{x \to 0}|x|$
\WkstHop	
	\begin{freeResponse}
		
Remember
\hspace{0.2in}$|x| =   \left\{ \begin{array}{lr}
	-x 	&	\text{if } x < 0	\\
	x	&	\text{if } x \ge 0	\end{array} \right.  $\\	
We first compute the one-sided limit: 
	$ \lim_{x \to 0^{+}} |x|=  \lim_{x \to 0^{+}} x= 0$.\\	
Next, we compute the other one-sided limit:\\
	$ \lim_{x \to 0^{-}} |x|=  \lim_{x \to 0^{-}} (-x)= \Big(\lim_{x \to 0^{-}} (-1)\Bigr)\Bigl(\lim_{x \to 0^{-}} x\Bigr)=(-1)(0)=0$.\\[1em]	
Therefore	
	$ \lim_{x \to 0^{-}} |x|=0=\lim_{x \to 0^{+}} |x|$.
Hence,
$ \lim_{x \to 0} |x| =0 $.
\end{freeResponse}


\item $\displaystyle \lim_{x \to 2} \ln{\Bigl(\sin{(x-2)}+e^{x}\sin\left({\dfrac{\pi x}{4}}\right)\Bigr)}$

\WkstHop
\begin{freeResponse}
	\begin{align*}
		\lim_{x \to 2} \ln\left(\sin(x-2)+e^{x}\sin\left(\dfrac{\pi x}{4}\right) \right) 
			&= \ln\left( \lim_{x \to 2} \left( \sin(x-2)+e^{x}\sin\left(\dfrac{\pi x}{4}\right) \right) \right) \\
			&= \ln\left( \lim_{x \to 2} \sin(x-2)+\lim_{x \to 2} \left(e^{x}\sin\left(\dfrac{\pi x}{4}\right) \right) \right) \\
			&= \ln\left( \sin\left(\lim_{x \to 2}(x-2)\right)+\left(\lim_{x \to 2} e^{x}\right)\left(\lim_{x \to 2}\sin\left(\dfrac{\pi x}{4}\right) \right) \right) \\
			&= \ln\left( \sin(0)+\left(\lim_{x \to 2} e^{x}\right)\sin\left(\lim_{x \to 2}\dfrac{\pi x}{4}\right) \right) \\
			&= \ln\left( 0+e^2 \sin\left(\dfrac{2\pi}{4}\right) \right) \\
			&= \ln\left( e^2 \sin\left(\dfrac{\pi}{2}\right) \right) = \ln\left( e^2 (1) \right) \\
			&= \ln\left( e^2 \right) = 2
	\end{align*}
\end{freeResponse}

\item $\displaystyle \lim_{x \to 1} \dfrac{3+2\cos\left(\frac{\pi x}{3}\right)}{4x-2x^{3}}$
\WkstHop
\begin{freeResponse}
	\begin{align*}
		\lim_{x\to 1} \frac{3+2\cos\left(\frac{\pi x}{3}\right)}{4x-2x^3} 
			&= \frac{\lim_{x\to 1}\left(3+2\cos\left(\frac{\pi x}{3}\right)\right)}{\lim_{x\to 1}\left(4x-2x^3\right)}\\
			&= \frac{\lim_{x\to 1}(3) +\lim_{x\to 1}\left(2\cos\left(\frac{\pi x}{3}\right)\right)}{4(1)-2(1)^3}\\	
			&= \frac{3 +\left(\lim_{x\to 1}2\right)\left(\lim_{x\to 1}\cos\left(\frac{\pi x}{3}\right)\right)}{4-2(1)}\\	
			&= \frac{3 +2\cos\left(\lim_{x\to 1}\frac{\pi x}{3}\right)}{4-2}\\	
			&= \frac{3 +2\cos\left(\frac{\pi}{3}\right)}{2} = \frac{3 +2\left(\frac{1}{2}\right)}{2}\\	
			&= \frac{3+1}{2} = 2.
	\end{align*}

\end{freeResponse}
\end{enumerate}
\end{problem}

\WkstNew
%problem 3		
\begin{problem}
Suppose
	$f(x) =   \left\{ \begin{array}{lr}
	x^2 - ax 	&	\text{if } x < 3	\\ \\
	a 2^x + 7 + a	&	\text{if } x > 3	\end{array} \right.  $
	
	Find $a$ so that $ \lim_{x \to 3} f(x)$ exists. 
\WkstHop
	\begin{freeResponse}
	Remember, it's not enough to find a value. We must also justify that the limit exists for the value chosen. 
	
	 To find a number $a$ for which $\lim_{x \to 3} f(x)$ exists we find $a$ such that $\lim_{x \to 3^-} f(x) = \lim_{x \to 3^+} f(x)$.

    To find the left-sided limit:
    \[  \lim_{x \to 3^-} f(x) = \lim_{x \to 3^-} (x^2 - ax) = 9 - 3a. \]
  
    To find the right-sided limit:
    \[  \lim_{x \to 3^+} f(x) = \lim_{x \to 3^+} (a2^x + 7 + a) = a2^3 + 7 + a = 9a + 7. \]

    In order for $\lim_{x \to 3} f(x)$ to exists we must have the two one-sided limits equal. That is,
    \begin{align*}
    	\lim_{x\to 3^-} f(x) &= \lim_{x\to 3^+} f(x)\\
      9 - 3a &= 9a + 7 \\
      	2 &= 12a\\
                      a &= 1/6.
    \end{align*}

	\end{freeResponse}
\end{problem}
	
\WkstNew

%problem 4	
\begin{problem}


  Determine the value of $\lim_{x \to 0} \left(x^2 \cos\left(\frac{1}{x}\right) \right)$. \textbf{EXPLAIN}.
\WkstHop
  \begin{freeResponse}
	$\lim_{x \to 0} \cos\left(\frac{1}{x}\right)$ does not exist, so we cannot apply the Product law. 

	We know that for any $x\neq 0$ that $ -1\leq \cos\left( \frac{1}{x}\right) \leq 1$. Multiplying this inequality by $x^2$ (which is positive) we find that for all $x$ near $0$,

		\[ -x^2 \leq x^2 \cdot \cos\left( \frac{1}{x}\right) \leq x^2 \] 
		The limits of the functions on the left and right sides of this compound inequality are:
		\begin{align*}
			\lim_{x \to 0} (-x^2) &= 0\\
			\lim_{x \to 0} x^2 &= 0\\
		\end{align*}

		Our explanation:
		
		$\lim_{x \to 0} \left(x^2 \cos\left(\frac{1}{x}\right) \right) =0$ by the Squeeze Theorem since 
		$-x^2 \leq x^2 \cdot \cos\left( \frac{1}{x}\right) \leq x^2$ and
		$\lim_{x \to 0} (-x^2) = \lim_{x \to 0} x^2 = 0$.


  \end{freeResponse}

\end{problem}
	
	
	
	
%problem 5
\begin{problem}
 For all $x$ near 0, the inequalities $1 - \dfrac{x^2}{6} \leq \dfrac{\sin(x)}{x} \leq 1$ are true.
 Use these inequalities to find $\displaystyle\lim_{x \to 0} \frac{\sin(x)}{x}$. \textbf{EXPLAIN}.
\WkstHop
  \begin{freeResponse}
	$1-x^2/6$ is a polynomial, so it is continuous at $x=0$. 
	That means $\lim_{x\to 0} \left(1 - \frac{x^2}{6}\right) = 1$. The function $1$ is continuous since it is a constant, so $\lim_{x \to 0} 1 = 1$.

	That means $\lim_{x\to 0} \frac{\sin(x)}{x} = 1$ by the Squeeze Theorem, since
	$1 - \dfrac{x^2}{6} \leq \dfrac{\sin(x)}{x} \leq 1$ and
	$\lim_{x \to 0} \left(1 - \frac{x^2}{6}\right)  = \lim_{x \to 0} 1 = 1$.

  \end{freeResponse}



\end{problem}
\WkstNew
	

%problem 6
\begin{problem}
	Two functions, $h$ and $g$, are given
	\[
	h(x)=\frac{x^2-4}{4},\  1 <x < 3
	\]
	\[
	g(x)=x-2,\  1 < x < 3
	\]
The graph of the function $h$ is given in the figure below.  
Let $f$ be a function defined on the interval $(1,3)$ that satisfies the following inequalities
	\[
	g(x) \le f(x) \le h(x),\ 1 < x < 3
	\]

\begin{enumerate}
	
	\item In the figure below, sketch and label the graph of $g$ and a possible graph of $f$.  (All three functions have a common domain $(1,3)$.)

% replaced by tikzpic below
%		\begin{image}
%			\includegraphics{Figure1}
%		\end{image}

	  \begin{center}
		\begin{tikzpicture}
			\begin{axis}[
				xmin=-0.3, xmax=3.3, ymin=-1.8,ymax=1.8,    
				axis lines =middle, 
				every axis y label/.style={at=(current axis.above origin),anchor=south},
				every axis x label/.style={at=(current axis.right of origin),anchor=west},
				xtick={0,...,3}, ytick={-1.5,-1,...,1.5},
				grid=major, width=3in, height = 2.5in,
				grid style={dashed, gray!40}
				]
				\addplot[color=blue, ultra thick, domain=1:3]{0.25*x^2-1}node[pos=0.75, above left]{\large{$y=h(x)$}};
			\end{axis}
		\end{tikzpicture}
	\end{center}
\WkstHop	
		\begin{freeResponse}	
%	replaced by tikzpic below
%			\begin{image}
%				\includegraphics{Figure2}
%			\end{image}
	  \begin{center}
		\begin{tikzpicture}
			\begin{axis}[
				xmin=-0.3, xmax=3.3, ymin=-1.8,ymax=1.8,    
				axis lines =middle, 
				every axis y label/.style={at=(current axis.above origin),anchor=south},
				every axis x label/.style={at=(current axis.right of origin),anchor=west},
				xtick={0,...,3}, ytick={-1.5,-1,...,1.5},
				grid=major, width=5in, height = 5in,
				grid style={dashed, gray!40}
				]
				\addplot[color=blue, ultra thick, domain=1:3]{0.25*x^2-1}node[pos=0.75, above left]{\large{$y=h(x)$}};
				\addplot[color=red, ultra thick, domain=1:3]{x-2}node[pos=0.15, below right]{\large{$y=g(x)$}};
				\addplot[color=green, ultra thick, domain=1:3]{0.1*(x+3)^2-2.5}node[pos=0, left]{\large{$y=f(x)$}};
			\end{axis}
		\end{tikzpicture}
	\end{center}
		\end{freeResponse}

	\item	Evaluate the limit, or state that it does not exist.  Justify your answer.
		\begin{enumerate}
			\item $\displaystyle \lim_{x \to 2} f(x)$
		\WkstHop
			\begin{freeResponse}
				$g$ and $h$ are both continuous function since they are both polynomials so we have:
				\[
					\lim_{x \to 2} g(x)=g(2)=0
				\]
				\[
					\lim_{x \to 2} h(x)=h(2)=0
				\]

				We have $g(x) \le f(x) \le h(x)$ so by the squeeze theorem, $\lim_{x \to 2} f(x)=0$ since
				$g(x)\leq f(x) \leq h(x)$ with $\lim_{x\to 2}g(x) = \lim_{x\to 2}h(x) = 0$.

			\end{freeResponse}

			\item $\displaystyle \lim_{x \to 2} \frac{f(x)+2}{x-1}$
		\WkstHop
			\begin{freeResponse}
				\[
					\lim_{x \to 2}\frac{f(x)+2}{x-1}=\frac{\lim_{x \to 2}(f(x)+2)}{\lim_{x \to 2}(x-1)}\\
						= \frac{0+2}{2-1}\\
						= \frac{2}{1}\\
						= 2
				\]
			\end{freeResponse}
			\item $\displaystyle \lim_{x \to 2} g(1+e^{f(x)})$
		\WkstHop
				\begin{freeResponse}
					\[
						\lim_{x \to 2} g\Bigl(1+e^{f(x)}\Bigr)= g \Bigl(\lim_{x \to 2} (1+e^{f(x)})\Bigr)= g \Bigl(\lim_{x \to 2} 1+\lim_{x \to 2}e^{f(x)})\Bigr)
							=g \Bigl(1+e^{\lim_{x \to 2}f(x)}\Bigr)
					\]
					\[
						=g\Bigl(1+e^0\Bigr)=g(2)=0
					\]
			\end{freeResponse}

		\end{enumerate}
	\end{enumerate}
	\end{problem}
\end{document}	
	%\item State the form of the limit, evaluate the limit, or state that it does not exist.  Justify your answer.
		%\begin{enumerate}
		
		
		%\item $\lim_{x \to 2}\frac{h(x)}{g(x)}$

		%\begin{freeResponse}
			%$\lim_{x \to 2}\frac{h(x)}{g(x)}=\lim_{x \to 2}\frac{\frac{x^2-4}{4}}{x-2}$
		%Form: $\frac{{0}}{0}$ \\
		%\begin{align*}
		%\lim_{x \to 2}\frac{\frac{x^2-4}{4}}{x-2}&=\lim_{x \to 2}\frac{(x-2)(x+2)}{4}\cdot \frac{1}{x-2}\\
		%&= \lim_{x \to 2} \frac{x+2}{4}\\
		%&= \frac{4}{4}\\
		%&= 1
		%\end{align*}
		%\end{freeResponse}	

%	\end{enumerate}
%	\end{problem}
			
 


















