\documentclass{ximera}

\newcommand{\RR}{\mathbb R}
\renewcommand{\d}{\,d}
\newcommand{\dd}[2][]{\frac{d #1}{d #2}}
\renewcommand{\l}{\ell}
\newcommand{\ddx}{\frac{d}{dx}}
\newcommand{\dfn}{\textbf}
\newcommand{\eval}[1]{\bigg[ #1 \bigg]}
\renewcommand{\theenumii}{\textup{(\roman{enumii})}}
\renewcommand{\labelenumii}{\theenumii}

\usepackage{graphicx}
\usepackage{multicol}
\usepackage{tkz-euclide}
%\usepackage{unicode-math}

\usepackage{pgfplots}   % <- for graphics
\pgfplotsset{compat=newest}


\renewenvironment{freeResponse}{
\ifhandout\setbox0\vbox\bgroup\else
\begin{trivlist}\item[\hskip \labelsep\bfseries Solution:\hspace{2ex}]
\fi}
{\ifhandout\egroup\else
\end{trivlist}
\fi}

\newcommand*{\ZeroOverZero}{\ensuremath{\dfrac{0}{0}}}

\providecommand{\HCCondition}{0}
\newcommand{\WkstHop}[1][1]{\if\HCCondition 0
	\vspace*{\stretch{#1}} \fi} 
\newcommand{\WkstNew}{\if\HCCondition 0
	\newpage
	 \fi} 


\title[Problem 6]{Problem 6}

\begin{document}
\begin{abstract} \end{abstract}
\maketitle


% Extracted from limitLaws.tex, problem #6
\begin{problem}
	Two functions, $h$ and $g$, are given
	\[
	h(x)=\frac{x^2-4}{4},\  1 <x < 3
	\]
	\[
	g(x)=x-2,\  1 < x < 3
	\]
The graph of the function $h$ is given in the figure below.  
Let $f$ be a function defined on the interval $(1,3)$ that satisfies the following inequalities
	\[
	g(x) \le f(x) \le h(x),\ 1 < x < 3
	\]

\begin{enumerate}
	
	\item In the figure below, sketch and label the graph of $g$ and a possible graph of $f$.  (All three functions have a common domain $(1,3)$.)

% replaced by tikzpic below
%		\begin{image}
%			\includegraphics{figure1}
%		\end{image}

	  \begin{center}
		\begin{tikzpicture}
			\begin{axis}[
				xmin=-0.3, xmax=3.3, ymin=-1.8,ymax=1.8,    
				axis lines =middle, 
				every axis y label/.style={at=(current axis.above origin),anchor=south},
				every axis x label/.style={at=(current axis.right of origin),anchor=west},
				xtick={0,...,3}, ytick={-1.5,-1,...,1.5},
				grid=major, width=3in, height = 2.5in,
				grid style={dashed, gray!40}
				]
				\addplot[color=blue, ultra thick, domain=1:3]{0.25*x^2-1}node[pos=0.75, above left]{\large{$y=h(x)$}};
			\end{axis}
		\end{tikzpicture}
	\end{center}
\begin{explanation}	
%	replaced by tikzpic below
%			\begin{image}
%				\includegraphics{figure2}
%			\end{image}
	  \begin{center}
		\begin{tikzpicture}
			\begin{axis}[
				xmin=-0.3, xmax=3.3, ymin=-1.8,ymax=1.8,    
				axis lines =middle, 
				every axis y label/.style={at=(current axis.above origin),anchor=south},
				every axis x label/.style={at=(current axis.right of origin),anchor=west},
				xtick={0,...,3}, ytick={-1.5,-1,...,1.5},
				grid=major, width=5in, height = 5in,
				grid style={dashed, gray!40}
				]
				\addplot[color=blue, ultra thick, domain=1:3]{0.25*x^2-1}node[pos=0.75, above left]{\large{$y=h(x)$}};
				\addplot[color=red, ultra thick, domain=1:3]{x-2}node[pos=0.15, below right]{\large{$y=g(x)$}};
				\addplot[color=green, ultra thick, domain=1:3]{0.1*(x+3)^2-2.5}node[pos=0, left]{\large{$y=f(x)$}};
			\end{axis}
		\end{tikzpicture}
	\end{center}
		\end{explanation}

	\item	Evaluate the limit, or state that it does not exist.  Justify your answer.
		\begin{enumerate}
			\item $\displaystyle \lim_{x \to 2} f(x)$
		\begin{explanation}
				$g$ and $h$ are both continuous function since they are both polynomials so we have:
				\[
					\lim_{x \to 2} g(x)=g(2)=0
				\]
				\[
					\lim_{x \to 2} h(x)=h(2)=0
				\]

				We have $g(x) \le f(x) \le h(x)$ so by the squeeze theorem, $\lim_{x \to 2} f(x)=0$ since
				$g(x)\leq f(x) \leq h(x)$ with $\lim_{x\to 2}g(x) = \lim_{x\to 2}h(x) = 0$.

			\end{explanation}

			\item $\displaystyle \lim_{x \to 2} \frac{f(x)+2}{x-1}$
		\begin{explanation}
				\[
					\lim_{x \to 2}\frac{f(x)+2}{x-1}=\frac{\lim_{x \to 2}(f(x)+2)}{\lim_{x \to 2}(x-1)}\\
						= \frac{0+2}{2-1}\\
						= \frac{2}{1}\\
						= 2
				\]
			\end{explanation}
			\item $\displaystyle \lim_{x \to 2} g(1+e^{f(x)})$
		\begin{explanation}
					\[
						\lim_{x \to 2} g\Bigl(1+e^{f(x)}\Bigr)= g \Bigl(\lim_{x \to 2} (1+e^{f(x)})\Bigr)= g \Bigl(\lim_{x \to 2} 1+\lim_{x \to 2}e^{f(x)})\Bigr)
							=g \Bigl(1+e^{\lim_{x \to 2}f(x)}\Bigr)
					\]
					\[
						=g\Bigl(1+e^0\Bigr)=g(2)=0
					\]
			\end{explanation}

		\end{enumerate}
	\end{enumerate}
	\end{problem}



\end{document}
