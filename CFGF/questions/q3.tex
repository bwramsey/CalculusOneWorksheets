% Extracted from computationsForGraphingFunctions.tex, problem #3
\begin{problem}
  Given that:
  \begin{align*}
    \lim_{x \to - \infty} f(x) &= 0 & \lim_{x \to -3^-} f(x) &= \infty  & \lim_{x \to -3^+} f(x) &= - \infty & \lim_{x \to 4^-} f(x) &= \infty\\
  \lim_{x \to 4^+} f(x) &= -\infty & f(1) &= 1 & f(5) &= -2 & \lim_{x \to 9} f(x) &= 3& \\
    f'(1) &\ne 0 & f'(7) &= 0 & f'(x) &= 2 \text{ for } x > 9 & f''(1) &= 0 &
 \end{align*}
     the domain of $f$ is: $(-\infty,-3) \cup (-3,4) \cup (4,9)\cup (9,\infty)$ and $f$ is continuous on its domain.
The following sign chart for the first and second derivatives of $f$:
  \begin{image}
    \includegraphics[scale=.8]{Figure8.png}
  \end{image}
	
find the following:
  \begin{enumerate}
     \item  Critical points.
     \item  Intervals where $f$ is increasing/decreasing.
     \item  Local extrema.
     \item  Inflection points.
     \item  Intervals of concavity.
     \item  Sketch the graph of $f$.
	
  \end{enumerate}
\WkstHop

  \begin{freeResponse}
    \begin{enumerate}
      \item
        Critical points. \\
	The critical points of $f$ occur at points in the domain of $f$ where either $f'(x)=0$ or where $f'(x)$ does not exist.  Even though $f'(-3), f'(4),$ and $f'(9)$ do not exist, all three of those points are not in the domain of $f$ and therefore are not critical points.  
        We are given that $f'(7)=0$, and so $x=7$ is a critical point of $f$.
                Therefore, $x=7$ is the only critical point of $f$.  
			
      \item
        Intervals where $f$ is increasing/decreasing.  \\
        $f$ is increasing when $f'(x)>0$.
        From the sign chart and our critical points, these are the intervals $(-\infty ,-3)$, $(-3,4)$, $(4,7)$, and $(9,\infty )$.
        $f(x)$ is decreasing when $f'(x)<0$.
        From the sign chart, this is on the interval $(7,9)$.
			
      \item
        Local extrema.  \\
        Using the first derivative test, $f(7)$ is a local maximum because the derivative changes sign from positive to negative.  So, $x=7$ is the only local extremum of $f$, and it is a local maximum.
			
      \item
        Inflection points.  \\
        Possible inflection points occur where $f''(x)=0$ or where $f''(x)$ does not exist.
        We are given that $f''(1)=0$.
        In addition, $f''(x)$ does not exist at $x=-3,4,9$.
        However, these are not inflection points because $f$ is not defined at these points.
        Since $f''$ changes sign at $x=1$, this is in fact an inflection point of $f$.
        We are given that $f(1) = 1$, and so the only inflection point of $f$ is the point $(1,1)$.  
			
      \item
        Intervals of concavity.  \\
        $f(x)$ is concave up when $f''(x)>0$.
        From the sign chart, this is on the intervals $(-\infty ,-3)$ and $(1,4)$.
        $f(x)$ is concave down when $f''(x)<0$.
        From the sign chart, this is on the interval $(-3,1)$ and $(4,9)$.
        
			
      \item
        Sketch the graph of $f$.
        \begin{image}
          \includegraphics[scale=.55]{Figure5.png}
	\end{image}
    \end{enumerate}
  \end{freeResponse}
\end{problem}
