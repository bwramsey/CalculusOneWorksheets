\documentclass{ximera}

\newcommand{\RR}{\mathbb R}
\renewcommand{\d}{\,d}
\newcommand{\dd}[2][]{\frac{d #1}{d #2}}
\renewcommand{\l}{\ell}
\newcommand{\ddx}{\frac{d}{dx}}
\newcommand{\dfn}{\textbf}
\newcommand{\eval}[1]{\bigg[ #1 \bigg]}
\renewcommand{\theenumii}{\textup{(\roman{enumii})}}
\renewcommand{\labelenumii}{\theenumii}

\usepackage{graphicx}
\usepackage{multicol}
\usepackage{tkz-euclide}
%\usepackage{unicode-math}

\usepackage{pgfplots}   % <- for graphics
\pgfplotsset{compat=newest}


\renewenvironment{freeResponse}{
\ifhandout\setbox0\vbox\bgroup\else
\begin{trivlist}\item[\hskip \labelsep\bfseries Solution:\hspace{2ex}]
\fi}
{\ifhandout\egroup\else
\end{trivlist}
\fi}

\newcommand*{\ZeroOverZero}{\ensuremath{\dfrac{0}{0}}}

\providecommand{\HCCondition}{0}
\newcommand{\WkstHop}[1][1]{\if\HCCondition 0
	\vspace*{\stretch{#1}} \fi} 
\newcommand{\WkstNew}{\if\HCCondition 0
	\newpage
	 \fi} 


\title[Problem 4]{Problem 4}

\begin{document}
\begin{abstract} \end{abstract}
\maketitle


% Extracted from computationsForGraphingFunctions.tex, problem #4
\begin{problem}
  Determine the following information about the given function and then graph the function:
  \[
    f(x) = 3\sin\left(\frac{x}{2}\right), [-2\pi,2\pi]
  \]

 Domain\\
  $x,y$-intercepts\\
  Symmetry\\
  Asympotes\\
  Intervals of increasing/decreasing\\
  Maxima/Minima\\
  Inflection points\\
    Intervals of concavity\\
    Sketch a graph of $f$

\begin{explanation}
\begin{enumerate}

\item Domain: This was given as $[-2\pi,2\pi]$

\item $x,y$-intercepts:\\
x-interpects occur when $f(x) = 3\sin\left(\frac{x}{2}\right)=0 \implies \frac{x}{2}=0+k\pi$ where $k$ is an integer, $-1\le k \le 1$.  On the domain this is $x=-2\pi,0,2\pi$.
y-intercepts: $f(0) = 3\sin\left(\frac{0}{2}\right)=0$  The y-intercept is the point $(0,0)$

\item Symmetry:\\
$f(-x)=3\sin\left(\frac{-x}{2}\right)=-3\sin\left(\frac{x}{2}\right) \ne f(x)$ so $f$ is not even
$-f(x)=-3\sin\left(\frac{x}{2}\right)=f(-x)$ so $f$ is odd

\item Intervals of increasing/decreasing:
            \[
              f'(x) = 3 \cos\left(\frac{x}{2}\right) \cdot \frac{1}{2}
              = \frac{3}{2} \cos\left(\frac{x}{2}\right)
            \]
                  Since $f'$ is defined everywhere on $(-2\pi, 2\pi)$ to find
        the critical points we just solve the equation $f'(x) = 0$
        where $-2\pi < x < 2\pi$:
        \begin{align*}
          f'(x) = 0 &\iff \frac{3}{2} \cos\left(\frac{x}{2}\right) = 0
          \\
           &\iff \cos\left(\frac{x}{2}\right) = 0 \\
          &\iff \mbox{$-2\pi < x < 2\pi$ and $x/2 = \pi/2 + n\pi$
            with $n$ an integer}\\
          &\iff \mbox{$-2\pi < x < 2\pi$ and $x = \pi + n2\pi$
            with $n$ an integer}\\
          &\iff \mbox{$x = -\pi$ and $x = \pi$.}
        \end{align*}
        So the only critical points are $x = -\pi$ and $x = \pi$.\\
        
        If $-2\pi <x< -\pi$, it follows that $-\pi<x/2<-\pi /2$ so $x/2$ lies in the third quadrant and $\cos(x/2)<0$. \\
         If $-\pi <x< \pi$, it follows that $-\pi /2<x/2<\pi /2$ so $x/2$ lies in the fourth and first quadrants and $\cos(x/2)>0$.  \\
         If $\pi <x< 2\pi$, it follows that $\pi /2<x/2<\pi$ so $x/2$ lies in the second quadrant and $\cos(x/2)<0$


         $\implies f$ is increasing on the interval $(-\pi, \pi)$, and $f$ is
        decreasing on the intervals $(-2\pi, -\pi)$ and $(\pi, 2\pi)$.
     
\item Maxima/minima:
 
      The first derivative test states
        we have a local maximum if the sign of $f'$ changes from
        positive to negative, and a local minimum if the sign of $f'$
        changes from negative to positive.

        Therefore there is a local maximum at $x = \pi$ and local
        minimum at $x = -\pi$ by the results in part (d).
        
        
     \item Intervals of concavity:
      Concavity can change at  interior points in $[-2\pi, 2\pi]$
            where $f''$ is undefined or equal to 0 are only
            \emph{candidates} for the inflection points.  
             \[
          f''(x) = \frac{-3}{2} \sin\left(\frac{x}{2}\right) \cdot
          \frac{1}{2} = \frac{-3}{4} \sin\left(\frac{x}{2}\right)
        \]
$\frac{-3}{4} \sin\left(\frac{x}{2}\right)=0$ when $\sin\left(\frac{x}{2}\right)=0$ which in the domain of $-2\pi < x < 2\pi$ occurs at $x/2 = n\pi$ with $n$ an integer.
$-2\pi < x < 2\pi$ and $x =  n2\pi$   with $n$ an integer $\implies x = 0$

We have a candidate for an inflection point at $x=0$. \\
 If $-2\pi <x< 0$, it follows that $-\pi<x/2<0$ so $x/2$ lies in the third and fourth quadrant and $\sin(x/2)<0$. \\
  If $0 <x<2 \pi$, it follows that $0<x/2<\pi$ so $x/2$ lies in the first and second quadrant and $\sin(x/2)>0$. \\


            So $f$ is concave up on the interval $(-2\pi, 0)$, and $f$
            is concave down on the interval $(0, 2\pi)$.   
    \item Inflection points: 
             Inflection points occur where the concavity of a
            function changes. Those interior points in $[-2\pi, 2\pi]$
            where $f''$ is undefined or equal to 0 are only
            \emph{candidates} for the inflection points.  We found in part f that we had a candidate for an inflection point at $x=0$.  We verified in part f that concavity changes at $x=0$.
            


            Since $f(0)=3\sin\left(\frac{0}{2}\right)=0$, the inflection point is $(0,0)$
            
            

   

  \item Sketch the graph: 
        \begin{image}
        \includegraphics[scale = .8]{graphOfFunction.png}
      \end{image}
  
  
\end{enumerate}
\end{explanation}
\end{problem}



\end{document}
