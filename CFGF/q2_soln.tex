\documentclass{ximera}

\newcommand{\RR}{\mathbb R}
\renewcommand{\d}{\,d}
\newcommand{\dd}[2][]{\frac{d #1}{d #2}}
\renewcommand{\l}{\ell}
\newcommand{\ddx}{\frac{d}{dx}}
\newcommand{\dfn}{\textbf}
\newcommand{\eval}[1]{\bigg[ #1 \bigg]}
\renewcommand{\theenumii}{\textup{(\roman{enumii})}}
\renewcommand{\labelenumii}{\theenumii}

\usepackage{graphicx}
\usepackage{multicol}
\usepackage{tkz-euclide}
%\usepackage{unicode-math}

\usepackage{pgfplots}   % <- for graphics
\pgfplotsset{compat=newest}


\renewenvironment{freeResponse}{
\ifhandout\setbox0\vbox\bgroup\else
\begin{trivlist}\item[\hskip \labelsep\bfseries Solution:\hspace{2ex}]
\fi}
{\ifhandout\egroup\else
\end{trivlist}
\fi}

\newcommand*{\ZeroOverZero}{\ensuremath{\dfrac{0}{0}}}

\providecommand{\HCCondition}{0}
\newcommand{\WkstHop}[1][1]{\if\HCCondition 0
	\vspace*{\stretch{#1}} \fi} 
\newcommand{\WkstNew}{\if\HCCondition 0
	\newpage
	 \fi} 


\title[Problem 2]{Problem 2}

\begin{document}
\begin{abstract} \end{abstract}
\maketitle


% Extracted from computationsForGraphingFunctions.tex, problem #2
\begin{problem}

  Determine the following information about the given function and then graph the function:
  \[ 
    f(x) = \frac{x^2 + x + 1}{x^2}
  \]
  
  Domain\\
  $x,y$-intercepts\\
  Symmetry\\
  Asympotes\\
  Intervals of increasing/decreasing\\
  Maxima/Minima\\
  Intervals of concavity\\
  Inflection points
\begin{explanation}
    \mbox{}
    \begin{enumerate}
      \item  
        \dfn{Domain}  \\
        
        The function is a rational function, and so the domain of the function is all real numbers except where the denominator equals zero.
        \[
          x^2 = 0 \quad \Longrightarrow \quad x=0
        \]
	So the domain of $f$ is $(-\infty ,0)\cup (0,\infty )$.


      \item
        \dfn{$x,\, y$-intercepts}  \\
        To find any $x$-intercept(s), set $y=0$ and solve:
        \begin{align*}
          \frac{x^2 + x + 1}{x^2} = 0 &\implies x^2 + x + 1 = 0 \\
          &\implies x = \frac{-1 \pm \sqrt{1-4(1)(1)}}{2(1)}
        \end{align*}
	which has no real solutions.
        Thus, $f$ has no $x$-intercepts.
			
	Since $x=0$ is not in the domain of $f$, $f$ has no $y$-intercepts as well.
 
     \item 
       \dfn{Symmetry}  \\

       Note that $f(1) = 3$ and $f(-1) = 1$.
       So it cannot be for all values of $x$ that either $f(-x) = f(x)$ or $f(-x) = -f(x)$.
       So $f$ is neither even nor odd, and therefore $f$ has no symmetry.
			
     \item
       \dfn{Asymptotes}  \\

       \dfn{Vertical Asymptotes:}  Our only candidate is $x=0$, and so we compute the two one-sided limits:
       \[
         \lim_{x \to 0^-} \frac{x^2+x+1}{x^2} = \infty 
       \]
       \[
         \lim_{x \to 0^+} \frac{x^2+x+1}{x^2} = \infty
       \]
       Therefore, $x=0$ is the only vertical asymptote of $f$.

       \dfn{Horizontal Asymptotes:}  We compute the following limits:
       \[
         \lim_{x \to \infty} \frac{x^2+x+1}{x^2} = 1
       \]
       \[
         \lim_{x \to -\infty} \frac{x^2+x+1}{x^2} = 1
       \]
       we checked both ends and so the only horizontal asymptote of $f$ is $y=1$.
			

			
     \item
       \dfn{Increasing/Decreasing}  \\

       \begin{align*}
         f'(x) &= \frac{x^2(2x+1) - (x^2+x+1)(2x)}{x^4} \\
               &= \frac{2x^3 + x^2 - 2x^3 - 2x^2 - 2x}{x^4} \\
               &= \frac{-x^2 - 2x}{x^4} \\
               &= \frac{-x-2}{x^3}
       \end{align*}
			
       To find where $f'$ is positive and where $f'$ is negative, we need to find where $f'(x) = 0$ and where $f'(x)$ does not exist.
       Clearly, $f'(x)$ does not exist when $x=0$.
       To find when $f'(x) = 0$, we solve:
       \begin{align*}
         \frac{-x-2}{x^3} = 0 &\implies -x-2 = 0 \\
         &\implies -x = 2\\
         &\implies x = -2
       \end{align*}
       Since $x=0$ is not in the domain of $f$, $x=-2$ is the only critical point of $f$.
       To see where $f$ is increasing and decreasing, consider the following sign chart for $f'$:
       \begin{center}
         \begin{image}
           \begin{tikzpicture}
             \draw [<->] (-4,0) -- (2,0);
             \draw (0,0.1) -- (0,-0.1);
             \draw (-2,0.1) -- (-2,-0.1);
             \draw (-2,-0.3)node[below]{$-2$};
             \draw (0,-0.3)node[below]{$0$};
             \draw (-3.5,-0.8)node[below]{$f'(-3) = \frac{-1}{27}$};
             \draw (-1,-1)node[below]{$f'(-1) = 1$};
             \draw (1.2,-1)node[below]{$f'(1) = -3$};
             \draw[red] (-1,1)node[below]{(+)};
             \draw[blue] (1,1)node[below]{(-)};
             \draw[blue] (-3,1)node[below]{(-)};
             \draw (2.5,0)node[above]{$f'$};
           \end{tikzpicture}
         \end{image}
       \end{center}

       So we see that $f$ is increasing on $(-2,0)$, and $f$ is decreasing on $(-\infty, -2)$ and $(0,\infty)$.
       
     \item
       \dfn{Local Extrema}  \\
       $f'$ changes from negative to positive at $x=-2$, so this is the location of a local minimum.
       $f'$ also changes from positive to negative at $x=0$, but $f$ is not defined at $x=0$ and so this is not a local extreme value.
       $f$ has a local minimum at $\left( -2,\frac{3}{4} \right)$.
			
			
			
     \item
       \dfn{Concavity}
       \begin{align*}
         f''(x) &= \frac{x^3(-1) - (-x-2)(3x^2)}{x^6} \\
                &= \frac{-x^3 + 3x^3 + 6x^2}{x^6} \\
		&= \frac{2x^3 + 6x^2}{x^6} \\
		&= \frac{2x+6}{x^4} \\
		&= \frac{2(x+3)}{x^4}
       \end{align*}
			
       To find where $f''$ is positive and where $f''$ is negative, we need to find where $f''(x) = 0$ and where $f''(x)$ does not exist.
       Clearly, $f''(x)$ does not exist when $x=0$.
       To find when $f''(x) = 0$, we solve:
       \begin{align*}
         \frac{2(x+3)}{x^4} = 0 &\implies 2(x+3) = 0\\
         &\implies x=-3
       \end{align*}

	The denominator of $f''$ is always positive so the sign of $f''$ depends on the numerator.  When $x<-3, f''<0$ and when $x>-3, f''>0$.		
									
       To see where $f$ is concave up and concave down, consider the following sign chart for $f''$:
       \begin{center}
         \begin{image}
           \begin{tikzpicture}
             \draw [<->] (-5,0) -- (2,0);
             \draw (0,0.1) -- (0,-0.1);
             \draw (-3,0.1) -- (-3,-0.1);
             \draw (-3,-0.3)node[below]{$-3$};
             \draw (0,-0.3)node[below]{$0$};
             \draw[red] (-1.5,1)node[below]{(+)};
             \draw[red] (1,1)node[below]{(+)};
             \draw[blue] (-4,1)node[below]{(-)};
             \draw (2.5,0)node[above]{$f''$};
           \end{tikzpicture}
         \end{image}
       \end{center}

       So we see that $f$ is concave up on $(-3,0)$ and $(0,\infty)$, and $f$ is concave down on $(-\infty, -3)$.
     \item
       \dfn{Inflection Points}  \\
       $f''(x)$ changes sign from negative to positive at $x=-3$ so $f$ has an inflection point at $\left( -3, \frac{7}{9} \right)$
			
     \item
       \dfn{The graph of $f$}
       \begin{image}
         \includegraphics[scale=.5]{Figure6.png}
       \end{image}

    \end{enumerate}
  \end{explanation}
\end{problem}



\end{document}
