% Extracted from understandingFunctions.tex, problem #6
\begin{problem}
The entire graph of $f(x)$ is given below.

% Replaced by the tikzpic below
%	\begin{image}
%	\includegraphics[scale=.5]{UFFigure2.png}
%	\end{image}

	  \begin{figure}[H] 
	  	\centering
		
		\begin{tikzpicture}[alt={Graph of piecewise-defined function for reading domains, ranges, and intervals of increasing and decreasing.}]
			\begin{axis}[
				xmin=-4.3, xmax=4.3, ymin=-4.3,ymax=4.3,    
				axis lines =middle, 
				every axis y label/.style={at=(current axis.above origin),anchor=south},
				every axis x label/.style={at=(current axis.right of origin),anchor=west},
				xtick={-4,...,4}, ytick={-4,...,4},
				grid=major, width=3.5in, height = 3.5in,
				grid style={dashed, gray!40}
				]
				\addplot[color=blue, ultra thick, smooth, samples=200, domain=-1:2]{x^2-1}node[pos=0.75, right]{\large{$f$}};;				
				\path[draw, color=blue, very thick] (axis cs:-4,-4) --  (axis cs:-3,-3);
				\path[draw, color=blue, very thick] (axis cs:2,-1) --  (axis cs:4,0); 
				\draw[fill=blue] (axis cs:-4,-4) circle [color=blue,radius=3pt];		
				\draw[fill=blue] (axis cs:-3,0) circle [color=blue,radius=3pt];	
				\draw[fill=blue] (axis cs:-2,-2) circle [color=blue,radius=3pt];
				\draw[fill=blue] (axis cs:2,-1) circle [color=blue,radius=3pt];
				\draw[fill=blue] (axis cs:4,0) circle [color=blue,radius=3pt];
	
				\draw[fill=blue] (axis cs:-3,-3) circle [color=blue,radius=3pt];
				\draw[fill=white] (axis cs:-3,-3) circle [color=white,radius=2pt];

				\draw[fill=blue] (axis cs:-1,0) circle [color=blue,radius=3pt];
				\draw[fill=white] (axis cs:-1,0) circle [color=white,radius=2pt];

				\draw[fill=blue] (axis cs:2,3) circle [color=blue,radius=3pt];
				\draw[fill=white] (axis cs:2,3) circle [color=white,radius=2pt];
			\end{axis}
		\end{tikzpicture}

	\end{figure}
	
\begin{enumerate}	
	\item  Find the domain and range of $f$
		\WkstHop

		\begin{freeResponse}
			Domain: $[-4,-3]\cup\left\{-2\right\}\cup(-1,4]$
			Range: $[-4,-3)\cup\left\{-2\right\}\cup[-1,3)$
		\end{freeResponse}	

	\item  Find the values of $f(-3),f(-2), f(-1),f(2)$
		\WkstHop
		
		\begin{freeResponse}
		$f(-3)=0, f(-2)=-2, f(-1)\ \text{does not exist}, f(2)=-1$
		\end{freeResponse}	

	\item  Find the intervals on which $f(x)$ is positive.  Find the intervals on which $f(x)$ is negative.
		\WkstHop
		
		\begin{freeResponse}
		 $f(x)$ is postive on $(1,2)$. $f(x)$ is negative on $[-4,-3),\left\{-2\right\},(-1,1),[2,4)$
		\end{freeResponse}
	
	\item Find the intervals on which $f$ is increasing.  Find the intervals on which $f$ is decreasing.
		\WkstHop
		
		\begin{freeResponse}
			$f$ is increasing on $(-4,-3),(0,2), (2,4)$.  $f$ is decreasing on $(-1,0)$. It would also be OK to say that		
			$f$ is increasing on $[-4,-3],[0,2), [2,4]$ and $f$ is decreasing on $(-1,0]$. Typically we will not be including endpoints of intervals
			when we talk about increasing/decreasing.
		\end{freeResponse}
	
	\item True or False: $f(1.5) < f(2)$
		\WkstHop
		
		\begin{freeResponse}
			False, $f(2) < f(1.5)$
		\end{freeResponse}	
	
	\end{enumerate}

	
\end{problem}
