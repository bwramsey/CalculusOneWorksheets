\documentclass[nooutcomes,handout]{ximera}
%handout:  for handout version with no solutions or instructor notes
%handout,instructornotes:  for instructor version with just problems and notes, no solutions
%noinstructornotes:  shows only problem and solutions

%% handout
%% space
%% newpage
%% numbers
%% nooutcomes



%\begin{image}
%\includegraphics{Figure1.pdf}
%\end{image}

%add a ``.'' below when used in a specific directory.
\newcommand{\RR}{\mathbb R}
\renewcommand{\d}{\,d}
\newcommand{\dd}[2][]{\frac{d #1}{d #2}}
\renewcommand{\l}{\ell}
\newcommand{\ddx}{\frac{d}{dx}}
\newcommand{\dfn}{\textbf}
\newcommand{\eval}[1]{\bigg[ #1 \bigg]}
\renewcommand{\theenumii}{\textup{(\roman{enumii})}}
\renewcommand{\labelenumii}{\theenumii}

\usepackage{graphicx}
\usepackage{multicol}
\usepackage{tkz-euclide}
%\usepackage{unicode-math}

\usepackage{pgfplots}   % <- for graphics
\pgfplotsset{compat=newest}


\renewenvironment{freeResponse}{
\ifhandout\setbox0\vbox\bgroup\else
\begin{trivlist}\item[\hskip \labelsep\bfseries Solution:\hspace{2ex}]
\fi}
{\ifhandout\egroup\else
\end{trivlist}
\fi}

\newcommand*{\ZeroOverZero}{\ensuremath{\dfrac{0}{0}}}

\providecommand{\HCCondition}{0}
\newcommand{\WkstHop}[1][1]{\if\HCCondition 0
	\vspace*{\stretch{#1}} \fi} 
\newcommand{\WkstNew}{\if\HCCondition 0
	\newpage
	 \fi}  %% we can turn off input when making a master document

\usepackage{fullpage}

\title{Review of famous functions (ROFF)}  

\begin{document}
\begin{abstract}		\end{abstract}
\maketitle
{\large{\textbf{SUMMARY: Polynomial, rational, exponential, logarithmic,}}}\\
 {\large{\textbf{trigonometric and inverse trigonometric functions }}}\\[0.8em]
\begin{itemize}
	\item Know the graphs and properties of ``famous'' functions
	 \item Understand the definition of a polynomial function
        \item Understand the definition of a rational function
        \item Know and use the properties of exponential and logarithmic functions
	\item Understand the relationship between exponential and logarithmic functions
           \item Evaluate expressions and solve equations involving\\
          exponential and logarithmic functions
	\item Understand the properties of trigonometric functions
	\item Evaluate expressions and solve equations involving\\
          trigonometric functions and inverse trigonometric functions
\end{itemize}
%problem 1
\begin{problem}
The graph of $g(x)=e^x$ is given below.

	\begin{image}		
	\includegraphics[scale=.6]{Figure1.pdf}
	\end{image}

\begin{enumerate}	
	\item  Find the domain and range of $g$.
		\begin{freeResponse}
		Domain: $(-\infty,\infty)$, Range: $(0,\infty)$
		\end{freeResponse}


	
	\item  Find the values of $g(1), g(0), g(-1)$ and plot the points $(1,g(1)), (0,g(0)),$ and $(-1,g(-1))$  on the graph below.
		\begin{freeResponse}
	
			$$g(1)=e^1=e$$
			$$ g(0)=e^0=1$$ 
			$$g(-1)=e^{-1}=\frac{1}{e}$$
			 These are values of $g(x)=e^x$ that you should know.

		\begin{image}		
		\includegraphics[scale=.6]{Figure5.pdf}
		\end{image}


		\end{freeResponse}
	\item Graph $h(x)=ln(x)$ on the same axis below.
		\begin{freeResponse}
		Recall:  $ln(x)$ is the inverse of $e^x$.  To find the graph of $ln(x)$ we reflect the graph of $e^x$ over the line $y=x$.  Since the points $\left(-1,\frac{1}{e}\right),(0,1),(1,e)$ are on $y=e^x=g(x)$, the points $\left(\frac{1}{e}, -1\right),(1,0),(e,1)$ are on the graph of $y=\ln(x)=g^{-1}(x)=h(x)$ 

		\begin{image}		
		\includegraphics[scale=.7]{Figure4.pdf}
		\end{image}
		\end{freeResponse}

	\item Find the domain and range of $h$.
		\begin{freeResponse}
		The domain of $h(x)=ln(x)$ is $(0,\infty)$.  The range is $(-\infty,\infty)$.
		\end{freeResponse}

	\item Find the values of $h(1), h(0), h(-1), h(e), h\left(\frac{1}{e}\right)$, or say $x$ not in the domain.
			\begin{freeResponse}
			 $$h(1)=ln(1)=0$$
			$$ h(0)\ \text{is not defined}, 0\ \text{is not in the domain}$$
			$$ h(-1)\ \text{is not defined}, -1\ \text{is not in the domain}$$
			$$h(e)=ln(e)=1$$
			$$h\left(\frac{1}{e}\right)=\ln\left(\frac{1}{e}\right)=\ln\left(e^{-1}\right)= -1\ln(e)=-1$$
			 These are values of $h(x)=ln(x)$ that you should know.
		\end{freeResponse}
	\end{enumerate}
	
 		
		
	
\end{problem}




%problem 3
\begin{problem} \hfil
	\begin{enumerate}
	\item Suppose we're given the right triangle below.  Express $\sin(\theta)$ and $\cos(\theta)$ in terms of the sides of the triangle.

	\begin{image}
	\includegraphics[scale=.3]{figure11l.png}
	\end{image}
	\begin{freeResponse}
	$\sin(\theta)=\frac{B}{C}=B$ and $\cos(\theta)=\frac{A}{C}=A$ 
	\end{freeResponse}

	\item Suppose we are given the triangle below.  
		\begin{image}
		\includegraphics[scale=.3]{figure22l.png}
		\end{image}

		\begin{enumerate}
	\item Find the length of the sides A and B.
	\begin{freeResponse}
	This is an isosceles triangle. $A=B$  Using the Pythagorean Theorem:
	\begin{align*}
	A^2+B^2&=1^2\\
	2A^2&=1 \\ 	
	A^2&=\frac{1}{2}\\
	A&=\sqrt{\frac{1}{2}}=\frac{\sqrt{2}}{2}\\
	B&=\sqrt{\frac{1}{2}}=\frac{\sqrt{2}}{2}
	\end{align*}
	\end{freeResponse}
	\item Express $\sin\left(\frac{\pi}{4}\right)$ and $\cos\left(\frac{\pi}{4}\right)$ in terms of the sides of the triangle.

	\begin{freeResponse}
	$\sin\left(\frac{\pi}{4}\right)=\frac{B}{C}=\frac{\sqrt{2}}{2}$ and $\cos\left(\frac{\pi}{4}\right)=\frac{A}{C}=\frac{\sqrt{2}}{2}$

	\end{freeResponse}
	\end{enumerate}

	\item Suppose we are given the triangle below.  
		\begin{image}
		\includegraphics[scale=.5]{figure33l.png}
		\end{image}

		\begin{enumerate}
	\item Find the length of the sides A and B.
	\begin{freeResponse}
	You might remember this as a 30/60/90 triangle.  To find the lengths of the sides of the triangle, we create a triangle as in the figure below.  All the angles in this triangle are of 60 degrees, therefore, this is an equilateral triangle 
		\begin{image}
		\includegraphics[scale=.5]{figure44l.png}
		\end{image}
	Now that we have an equilateral triangle, we have $C=C=2A$.  Thus, $1=2A \implies A=\frac{1}{2}$ \\
	To find B, we use the Pythagorean Theorem.
	\begin{align*}
	\left(\frac{1}{2}\right)^2+B^2&=1^2\\
	\frac{1}{4}+B^2&=1 \\ 	
	B^2&=\frac{3}{4}\\
	B&=\sqrt{\frac{3}{4}}=\frac{\sqrt{3}}{2}
	\end{align*}
	\end{freeResponse}

	\item Express $\sin\left(\frac{\pi}{3}\right)$ and $\cos\left(\frac{\pi}{3}\right)$ in terms of the sides of the triangle.

	\begin{freeResponse}
	$\sin\left(\frac{\pi}{3}\right)=\frac{B}{C}=\frac{\sqrt{3}}{2}$ and $\cos\left(\frac{\pi}{3}\right)=\frac{A}{C}=\frac{1}{2}$

	\end{freeResponse}
	\end{enumerate}

\item Suppose we are given the triangle below.  
		\begin{image}
		\includegraphics[scale=.5]{figure55l.png}
		\end{image}

		\begin{enumerate}
	\item Find the length of the sides A and B.
	\begin{freeResponse}
	We've actually already found the lengths of the sides for this type of triangle in part c.  $B=\sqrt{\frac{3}{4}}=\frac{\sqrt{3}}{2}$ and $A=\frac{1}{2}$ 
	\end{freeResponse}

	\item Write $\sin\left(\frac{\pi}{6}\right)$ and $\cos\left(\frac{\pi}{6}\right)$ in terms of the sides of the triangle.

	\begin{freeResponse}
	$\sin\left(\frac{\pi}{6}\right)=\frac{A}{C}=\frac{1}{2}$ and $\cos\left(\frac{\pi}{6}\right)=\frac{B}{C}=\frac{\sqrt{3}}{2}$

	\end{freeResponse}
	\end{enumerate}

\item 
For any point P(x,y) on the unit circle, we can express its coordinates in terms of $\sin(\theta)$ and $\cos(\theta)$.
Here $\theta$ is the radian measure of the angle in standard position whose terminal side is the line through the origin and the point $P(x,y)$.
\begin{image}
		\includegraphics[scale=.8]{figure1313l.png}
		\end{image}
		\begin{freeResponse}
	$(x,y)=(\cos(\theta),\sin(\theta))$
	\end{freeResponse}

	\item  Use all of the above information to label the given points on the unit circle.  That is, for each point on the unit circle, provide the angle measure in radians and degrees, and give the (x,y) coordinate for the point.
		\begin{image}
		\includegraphics{figure66l.png}
		\end{image}

		\begin{freeResponse} \hfil
		\begin{image}
		\includegraphics[scale=.7]{figure77l.png}
		\end{image}
		\end{freeResponse}
	\end{enumerate}

\end{problem}


%problem2
\begin{problem}

  Find all real numbers which satisfy each of the equations.  In the previous problem, we used $\theta$ to denote the radian measure of the angle.  However, we can use any variable to represent the angle measure.  For example, in part a, $x$ is the variable representing the radian measure of the angle.
  \begin{enumerate}
    \item
      $\cos(x) = 1$
      \begin{freeResponse}
        This is asking for the collection of all angles such that cosine of that angle equals 1.

        The unit circle shows that one such angle is $0$ (since $\cos(0) = 1$).
        There is a slight trick here: since cosine has period $2\pi$ we actually have $\cos(0 + 2\pi n) = 1$ for every integer $n$.
        In summary, $x = 2\pi n$, where $n$ is any integer, gives all the solutions to this equation.
      \end{freeResponse}

    \item
      $\sin(3 \theta) = \sqrt{3}/2$ for $0 \leq \theta \leq 2\pi$
      \begin{freeResponse}
        Finding all numbers $\theta$ with $0 \leq \theta \leq 2\pi$ that satisfy $\sin(3 \theta) = \sqrt{3}/2$ is a bit tricky.
        We first perform a useful trick from algebra~---~variable substitution.

        Let $x = 3\theta$.
        So, we are trying to find all numbers $x$ such that $\sin(x) = \sqrt{3}/2$ for $0 \leq x/3 \leq 2\pi$.
        Then $ x= \frac{\pi}{3} + 2 \pi n$ or $ x = \frac{2 \pi }{3} + 2 \pi n $ for $n$ any integer as long as $0 \leq \theta \leq 2\pi$
        Since $x = 3 \theta$, we can solve for $\theta$ to obtain $\theta = \pi/9 + (2 \pi n)/3$ or $\theta = (2\pi)/9 + (2\pi n)/3$, where $n$ is again any integer as long as $0 \leq \theta \leq 2\pi$.
        We are only looking for solutions of $\theta$ in $[0, 2\pi ]$, and so our solutions are
        \[
        \theta = \frac{\pi}{9}, \frac{2\pi}{9}, \frac{7\pi}{9}, \frac{8\pi}{9}, \frac{13\pi}{9}, \frac{14\pi}{9}. 
        \]
      \end{freeResponse}
  \end{enumerate}
\end{problem}


\begin{problem} \hfil

\begin{enumerate}
	\item Graph $f(\theta)=\sin(\theta)$ and $g(\theta)=\cos(\theta)$ from $[-\frac{\pi}{8},2\pi+\frac{\pi}{8}]$
		\begin{freeResponse} \hfil
		\begin{image}
		\includegraphics{figure88l.png}
		\end{image}
		\begin{image}
		\includegraphics{figure99l.png}
		\end{image}
		\end{freeResponse}
  \end{enumerate}
\end{problem}
%problem 4
\begin{problem}
 Without using a calculator, determine if the statement
  \[
    \cos^{-1}\bigl(\cos(7\pi/6)\bigr) = 7\pi/6
  \]
  is true or false.
  \begin{freeResponse}
    This statement is \textbf{false}: the correct statement is $\cos^{-1}\bigl(\cos(7\pi/6)\bigr) = 5\pi/6$. (Why?)

    \textbf{Spoiler Alert}: the cosine function is \emph{not} invertible since its graph fails the horizontal line test.
    \begin{image}
      \includegraphics[scale = 0.8]{figure12l.png}
    \end{image}
    To produce the inverse cosine we must first restrict the domain of cosine, to the interval $[0, \pi]$, to produce an invertible function:
    \begin{image}
      \includegraphics[scale = 0.8]{figure23l.png}
    \end{image}
    So, the range of $\cos^{-1}$ is $[0, \pi]$.
    Since $7\pi/6$ is not in this range, $7\pi/6$ is \emph{never} a possible output of $\cos^{-1}$.
  \end{freeResponse}
\end{problem}

%problem 5
\begin{problem}
A boat sails directly toward a 100-meter skyscraper that stands on the edge of a harbor.  The angular size $\theta$ of the building is the angle formed by lines from the top and bottom of the building to the observer on the boat (see figure below).
    \begin{image}
      \includegraphics[scale = 0.6]{figure67l.png}
    \end{image}
\begin{enumerate}
 \item Express the angle $\theta$ as the function of $x$, the distance of the boat from the building.
	\begin{freeResponse}
	$\tan\theta=\frac{100}{x} \implies \theta=\tan^{-1}\left( \frac{100}{x} \right)$
	
	\end{freeResponse}
	\item Find the angular size, $\theta$, when the boat is $x=100\sqrt{3}$m from the building.
\begin{freeResponse}
$\theta=\tan^{-1}\left( \frac{100}{100\sqrt{3}} \right)=\tan^{-1}\left( \frac{1}{\sqrt{3}} \right)=\frac{\pi}{6}$
	\end{freeResponse}
\end{enumerate}
\end{problem}

%problem 6

\begin{problem}

  \textbf{True or False:}
  $\sin^{-1}(0) = \pi$.
  \begin{freeResponse}
    This statement is \textbf{false}: the correct statement is $\sin^{-1}(0) = 0$. (Why?)

    \textbf{Spoiler Alert}: the sine function is \emph{not} invertible since its graph fails the horizontal line test.
    \begin{image}
      \includegraphics[scale = 0.4]{figure34l.png}
    \end{image}
    To produce the inverse sine we first restrict the domain of sine, to the interval $[-\pi/2, \pi/2]$, to produce an invertible function:
    \begin{image}
      \includegraphics[scale = 0.4]{figure45l.png}
    \end{image}
    So, the range of $\sin^{-1}$ is $[-\pi/2, \pi/2]$.
    Since $\pi$ is not in this range, $\pi$, is \emph{never} a possible output of $\sin^{-1}$.
  \end{freeResponse}
\end{problem}


%problem7
  \begin{problem}
  Simplify each of the following expressions.
  \begin{enumerate}
    \item
      $\cos^{-1} \bigl( \sin(\pi/2) \bigr)$
      \begin{freeResponse}
        By the unit circle, $\sin(\pi/2) = 1$, and so we are looking for $\cos^{-1}(1)$.
        The range of $\cos^{-1}$ is $[0, \pi]$, and so, by properties of inverse functions, $\cos^{-1}(1) = 0$.
      \end{freeResponse}

    \item
      $\tan \bigl( \sin^{-1}(x/4) \bigr)$
      \begin{freeResponse}
        Let $\theta = \sin^{-1}(x/4)$, then $\sin(\theta) = x/4$.
        We can then draw the corresponding right triangle:
        \begin{image}
          \includegraphics[scale = 0.4]{figure56l.png}
        \end{image}
        Calling the adjacent side $y$, by the Pythagorean Theorem we obtain

          $$4^2 = x^2 + y^2 \implies y = \sqrt{16-x^2}$$
Remark: Since $\theta$ is in the range of $\sin^{-1}$, it follows that $-\pi /2 \le \theta \le \pi /2$. 
On the other hand, the expression $ \tan \left( \sin^{-1} \left(4/x \right) \right)$ is defined only for $-4<x<4$ (Note: $\sin^{-1} \left(4/4 \right)=\frac{\pi}{2}$ and  $\tan$ is not defined at $\frac{\pi}{2}$; similarly for $x=-4$.)

 Therefore, $-\pi /2 < \theta < \pi /2$, which implies that  $\cos(\theta)=y/4> 0$.
  Therefore, $y> 0$.\\
 
        Then
        \begin{align*}
          \tan \left( \sin^{-1} \left(4/x \right) \right) &= \tan( \theta), \\
                                                                   &= \frac{x}{y}, \\
                                                                 &= \frac{x}{\sqrt{16-x^2}}.
        \end{align*}
Note:  $\tan\left(\theta\right)$  has the same sign as $x$, since $y>0$. 
      \end{freeResponse}
  \end{enumerate}
\end{problem}  

%problem8


%problem9
\end{document} 









