\documentclass{ximera}

\newcommand{\RR}{\mathbb R}
\renewcommand{\d}{\,d}
\newcommand{\dd}[2][]{\frac{d #1}{d #2}}
\renewcommand{\l}{\ell}
\newcommand{\ddx}{\frac{d}{dx}}
\newcommand{\dfn}{\textbf}
\newcommand{\eval}[1]{\bigg[ #1 \bigg]}
\renewcommand{\theenumii}{\textup{(\roman{enumii})}}
\renewcommand{\labelenumii}{\theenumii}

\usepackage{graphicx}
\usepackage{multicol}
\usepackage{tkz-euclide}
%\usepackage{unicode-math}

\usepackage{pgfplots}   % <- for graphics
\pgfplotsset{compat=newest}


\renewenvironment{freeResponse}{
\ifhandout\setbox0\vbox\bgroup\else
\begin{trivlist}\item[\hskip \labelsep\bfseries Solution:\hspace{2ex}]
\fi}
{\ifhandout\egroup\else
\end{trivlist}
\fi}

\newcommand*{\ZeroOverZero}{\ensuremath{\dfrac{0}{0}}}

\providecommand{\HCCondition}{0}
\newcommand{\WkstHop}[1][1]{\if\HCCondition 0
	\vspace*{\stretch{#1}} \fi} 
\newcommand{\WkstNew}{\if\HCCondition 0
	\newpage
	 \fi} 


\title[Problem 14]{Problem 14}

\begin{document}
\begin{abstract} \end{abstract}
\maketitle


% Extracted from understandingFunctions.tex, problem #14
\begin{problem}

  We're given the following graph of a function:

	  \begin{image} 
		
		\begin{tikzpicture}
			\begin{axis}[
				xmin=-3.3, xmax=3.3, ymin=-2.3,ymax=4.3,    
				axis lines =middle, 
				every axis y label/.style={at=(current axis.above origin),anchor=south},
				every axis x label/.style={at=(current axis.right of origin),anchor=west},
				xtick={-3,...,3}, ytick={-2,...,4},
				grid=major, width=3.5in, height = 3.5in,
				grid style={dashed, gray!40}
				]
				\addplot[color=blue, ultra thick, smooth, samples=200, domain=-2.3:0]{x^2}node[pos=0.75, left]{\large{$f$}};;				
				\path[draw, color=blue, very thick] (axis cs:0,1) --  (axis cs:1,0);
				\path[draw, color=blue, very thick] (axis cs:1,-1) --  (axis cs:3,-3);
				 
				\draw[fill=blue] (axis cs:0,3) circle [color=blue,radius=3pt];		
	
				\draw[fill=blue] (axis cs:0,0) circle [color=blue,radius=3pt];
				\draw[fill=white] (axis cs:0,0) circle [color=white,radius=2pt];

				\draw[fill=blue] (axis cs:1,0) circle [color=blue,radius=3pt];
				\draw[fill=white] (axis cs:1,0) circle [color=white,radius=2pt];

				\draw[fill=blue] (axis cs:0,1) circle [color=blue,radius=3pt];
				\draw[fill=white] (axis cs:0,1) circle [color=white,radius=2pt];

				\draw[fill=blue] (axis cs:1,-1) circle [color=blue,radius=3pt];
				\draw[fill=white] (axis cs:1,-1) circle [color=white,radius=2pt];
			\end{axis}
		\end{tikzpicture}
	\end{image}
	
  Use this graph to answer the following questions:
  \begin{enumerate}
  %a
    \item What is the domain of this function?
	\begin{explanation}
        $(-\infty, 1) \cup (1, \infty)$
      \end{explanation}
      
    %b
    \item What is the range of this function?
	\begin{explanation}
        $(-\infty, -1) \cup (0, \infty)$
      \end{explanation}

    %c
    \item What is the value of $f(0)$, $f(1)$, and $f(2)$?
\begin{explanation}
        $f(0) = 3$, $f(1)$ does not exist, $f(2) = -2$
      \end{explanation}

   %d
    \item Does this function have an inverse? (Why or why not?)
	\begin{explanation}
        No, the function does not have an inverse.
        It is not one-to-one (that is, it does not pass the horizontal line test).
      \end{explanation}
      
     %e
    \item Find at least two intervals on which the function is one-to-one.
	\begin{explanation}
        The function becomes one-to-one when we restrict its domain to $(-\infty, 0)$:
        
        \begin{image}
      	    \includegraphics[scale = 0.4, alt={Graph of restriction of the function to x<0, showing that the graph is one-to-one here.}]{UFFigure122.png}
        \end{image}
        The function also becomes one-to-one when we restrict its domain to $[0, 1) \cup (1,\infty)$:
       
        \begin{image}
          \includegraphics[scale = 0.4, alt={Graph of restriction to [0,1) union (1, infinity), showing that the graph is one-to-one here.}]{UFFigure222.png}
          \end{image}
      \end{explanation}

   %f
    \item Find $f^{-1}(3)$ on a restricted domain of $f$.
	\begin{explanation}
        In this case restrict the domain of $f$ to $[0, 1)\cup(1,\infty)$.
        By definition we have $f^{-1}(3) = y \iff 3 = f(y)$.
        Looking at the second graph of the restricted function we see that $f(0) = 3$, that is, $f^{-1}(3) = 0$.

        (If we restrict the domain of $f$ to $(-\infty, 0)$, then $f^{-1}(3) = y \iff 3 = f(y)$ implies that $f(-1.7) \approx 3$.
        Hence $-1.7 \approx f^{-1}(3)$.)
      \end{explanation}
%g
	\item Using the restricted domain for $f$ of $(-\infty,0)$, sketch a graph of $f^{-1}$.  
	\begin{explanation}
	To graph the inverse, we can think of the graph being reflected over the line $y=x$.  Another way to obtain the graph is to remember than if $(x,y)$ is a point on the graph of $f$, then $(y,x)$ is a point on the graph of $f^{-1}$.  For example, $(-1,1)$ is on the graph of $f$ so $(1,-1)$ is on the graph of $f^{-1}$.
	
	 \begin{image}
          \includegraphics[scale = 0.4, alt={Graph showing the restriction of the function, the line y=x, and the reflection of the restriction across that line.}]{UFFigure223.png}
        \end{image}
	\end{explanation}	
	 %h
	\item Using the restricted domain for $f$ of $[0,1)\cup(1,\infty)$, sketch a graph of $f^{-1}$.  
	\begin{explanation}
	We can graph the inverse of $f$, when restricted to $[0,1)\cup(1,\infty)$, in pieces from left to right.  Since $(0,3)$ is a point on the graph of $f$, the point $(3,0)$ in on the graph of $f^{-1}$.  Then, we see on the graph of $f$, there is a linear piece going from $(0,1)$ to $(1,0)$.  When we imagine reflecting this part of the graph of $f$ over the line $y=x$, it reflects onto itself.  The last piece of the graph of $f$ is a line from $(1,-1)$ which appears to also contain the point $(-2,2)$.  Reflecting this part of $f$ over the line $y=x$, we obtain a line starting at $(-1,1)$ and through the point $(-2,2)$.
	
	 \begin{image}
          \includegraphics[scale = 0.4, alt={Graph showing the restriction of the function, the line y=x, and the reflection of the restriction across that line.}]{UFFigure333.png}
        \end{image}
	
	
	 \end{explanation} 

  \end{enumerate}
\end{problem}



\end{document}
