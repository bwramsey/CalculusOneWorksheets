

\newcommand{\wkstTitle}{Understanding functions (UF)}
\newcommand{\solnTitle}{\wkstTitle \, - Solutions}

%Add code to compile both versions from makefile at same time
\providecommand{\HCCondition}{0}
%Define each of the conditions
\ifcase\HCCondition
	%\condition=0 -> handout
	\documentclass[nooutcomes,noauthor,space,handout]{ximera}
	\title{\wkstTitle}
\or	%\condition=1 -> Soln
	\documentclass[nooutcomes,noauthor]{ximera}
	\title{\solnTitle} 
\fi

\newcommand{\RR}{\mathbb R}
\renewcommand{\d}{\,d}
\newcommand{\dd}[2][]{\frac{d #1}{d #2}}
\renewcommand{\l}{\ell}
\newcommand{\ddx}{\frac{d}{dx}}
\newcommand{\dfn}{\textbf}
\newcommand{\eval}[1]{\bigg[ #1 \bigg]}
\renewcommand{\theenumii}{\textup{(\roman{enumii})}}
\renewcommand{\labelenumii}{\theenumii}

\usepackage{graphicx}
\usepackage{multicol}
\usepackage{tkz-euclide}
%\usepackage{unicode-math}

\usepackage{pgfplots}   % <- for graphics
\pgfplotsset{compat=newest}


\renewenvironment{freeResponse}{
\ifhandout\setbox0\vbox\bgroup\else
\begin{trivlist}\item[\hskip \labelsep\bfseries Solution:\hspace{2ex}]
\fi}
{\ifhandout\egroup\else
\end{trivlist}
\fi}

\newcommand*{\ZeroOverZero}{\ensuremath{\dfrac{0}{0}}}

\providecommand{\HCCondition}{0}
\newcommand{\WkstHop}[1][1]{\if\HCCondition 0
	\vspace*{\stretch{#1}} \fi} 
\newcommand{\WkstNew}{\if\HCCondition 0
	\newpage
	 \fi}  %% we can turn off input when making a master document




\begin{document}
\begin{abstract}		\end{abstract}

\makeTagTitle

\ifcase\HCCondition

\subsection*{SUMMARY: What we need to know about functions}


\begin{itemize}
	\item State the definition of a function.
	\item Find the domain and range of a function.
	\item Distinguish between functions by considering their domains.
	\item Recognize different representations of the same function.
	\item Determine where a function is positive or negative.
	\item Determine algebraically whether a function is even, odd, or neither.
	\item Use symmetry  when  graphing  even or odd functions.
	\item Plot basic functions.
	\item Apply appropriate transformations to graphs of basic functions\\
	         (vertical and horizontal shifts, vertical and horizontal stretching and reflecting)
        \item Perform basic operations and compositions on functions.
        \item Work with piecewise defined functions.
	\item Determine if a function is one-to-one.
        \item Define and work with inverse functions.
        \item Plot inverses of basic functions.
	\item Find inverse functions (algebraically and graphically).
        \item Find the largest interval containing a given point
          where the function is invertible.
	\item Determine the intervals on which a function has an inverse.
         \item Relate the domain and range of $f$ and the domain and range of $f^{-1}$.  
	\item Use a sign-chart to determine where the function is positive/negative.
\end{itemize}

\WkstNew

\subsection*{Recitation Questions}

\fi





%problem 1
\begin{problem}

	Find a formula for the function $f$ whose graph is given in the figure below.\\ What is the domain of $f$?
What is the range of $f$? Is the function $f$ linear?
% replaced by tikzpic below
%	\begin{image}		
%		\includegraphics[scale=.4]{figure11a.png}
%	\end{image}	

	  \begin{figure}[H]
	  	\centering
		
		\begin{tikzpicture}[alt={Graph of piecewise-linear function for questions below.}]
			\begin{axis}[
				xmin=-0.3, xmax=3.3, ymin=-0.3,ymax=6.5,    
				axis lines =middle, 
				every axis y label/.style={at=(current axis.above origin),anchor=south},
				every axis x label/.style={at=(current axis.right of origin),anchor=west},
				xtick={0,...,3}, ytick={0,...,6},
				grid=major, width=2in, height = 3in,
				grid style={dashed, gray!40}
				]
				\path[draw, color=blue, very thick] (axis cs:0,2) --  (axis cs:1,2) -- (axis cs:3,6) node[pos=0.75, below right]{\large{$f$}};
				\draw[fill=blue] (axis cs:0,2) circle [color=blue,radius=3pt];
				\draw[fill=white] (axis cs:0,2) circle [color=white,radius=2pt];
				\draw[fill=blue] (axis cs:3,6) circle [color=blue,radius=3pt];
				\draw[fill=white] (axis cs:3,6) circle [color=white,radius=2pt];
			\end{axis}
		\end{tikzpicture}
	\end{figure}


\WkstHop

	\begin{freeResponse} 
$f(x) =   \left\{ \begin{array}{cl}
	2	 	&	\qquad \text{if } 0< x \leq{ 1}					\\
	
2+2(x-1)			&	\qquad \text{if } 1<x<3 	\\		\end{array} \right.  $
			

	Domain of $f$$=(0,3)$, range of $f$$=[2,6)$.\\
	$f$ is not linear; because its graph is NOT a line; it is a piecewise defined function.	
	\end{freeResponse}

\end{problem}


\WkstNew
\begin{problem}

	A mirror has the shape of a rectangle surmounted by a semicircle (see figure).
The area of the mirror is $32$ $in^{2}$. Let $x$ be the radius of the semicircle (that lies on top of the rectangle).
Express the perimeter of the mirror P (in inches) as a function of $x$ (in inches). Is P a polynomial, a rational function or a transcendental function?
% replaced by the tikzpic below
%	\begin{image}		 
%		\includegraphics[scale=.4]{figure11b.png}
%	\end{image}

	\begin{figure}[H]
		
		\centering
		
		\begin{tikzpicture}[alt={A rectangle of height y with a semicircle of radius x on it.}]

			\draw[very thick] (2,0) arc(0:180:2) -- (-2,-3) -- (2, -3) -- cycle node[pos=0.5, right]{\large{$y$}};;
			\draw[dashed] (0,0) -- (2,0) node[pos=0.5, above]{\large{$x$}};

		\end{tikzpicture}

	\end{figure}

\WkstHop

	\begin{freeResponse} 
Since the area $A=32=2xy+\dfrac{x^{2}\pi}{2}$, it follows that $y=\dfrac{16}{x}-\dfrac{x\pi}{4}$ and that\
$P(x) =  2x + 2y +x\pi= 2x +\dfrac{32}{x}-\dfrac{x\pi}{2} +x\pi=2x+\dfrac{x\pi}{2}+\dfrac{32}{x}$
					
	\end{freeResponse}
	
\end{problem}



\WkstNew
%problem 1
\begin{problem}
Define 
	$f(x) =   \left\{ \begin{array}{cl}
	x^2-1		 	&	\qquad \text{if } x < 0					\\
	\text{? }			&	\qquad \text{if }  x > 0 	\\		\end{array} \right.  $
\begin{enumerate}	
	\item  Find an expression for "?" such that $f$ will be even.

\WkstHop	

	\begin{freeResponse} 
		If $f$ is even then $f(x)=f(-x)$.  
		\begin{align*}
			f(x)&=f(-x)\ \text{for}\ x>0\\
				&=(-x)^2-1\\
				&=x^2-1	
		\end{align*}

% replaced by the tikzpic below
%	\begin{image}		
%	\includegraphics[scale=.4]{UFFigure1.png}
%	\end{image}

	  \begin{figure}[H]
	  	\centering
	  	
		\begin{tikzpicture}[alt={The graph of the parabola, with vertex at the point (0,-1) removed.}]
			\begin{axis}[
				xmin=-4.3, xmax=4.3, ymin=-4.3,ymax=4.3,    
				axis lines =middle, 
				every axis y label/.style={at=(current axis.above origin),anchor=south},
				every axis x label/.style={at=(current axis.right of origin),anchor=west},
				xtick={-4,...,4}, ytick={-4,...,4},
				grid=major, width=3in, height = 3in,
				grid style={dashed, gray!40}
				]
				\addplot[color=blue, ultra thick, smooth, samples=200, domain=-2.3:2.3]{x^2-1}node[pos=0.55, below right]{\large{$y=x^2-1$}};;				

				\draw[fill=blue] (axis cs:0,-1) circle [color=blue,radius=3pt];
				\draw[fill=white] (axis cs:0,-1) circle [color=white,radius=2pt];
			\end{axis}
		\end{tikzpicture}

	\end{figure}
		
	\end{freeResponse}
	
	\item  Find an expression for "?" such that $f$ will be odd.

\WkstHop	

	\begin{freeResponse}
	   If $f$ is odd then $f(-x)=-f(x)$. 
			\begin{align*}
			-f(x)&=f(-x)\ \text{for}\ x>0\\
			&=(-x)^2-1\\
			&=x^2-1\\
			& \implies f(x)=-x^2+1	
			\end{align*}
	
	% replaced by tikzpic below
	%\begin{image}		
	%\includegraphics[scale=.4]{UFFigure11.png}
	%\end{image}
	  \begin{figure}[H] 
	  	\centering
		
		\begin{tikzpicture}[alt={The graph whose left-side coincides with the parabola y=x^2-1, and whose right-side coincides with y=1-x^2.}]
			\begin{axis}[
				xmin=-4.3, xmax=4.3, ymin=-4.3,ymax=4.3,    
				axis lines =middle, 
				every axis y label/.style={at=(current axis.above origin),anchor=south},
				every axis x label/.style={at=(current axis.right of origin),anchor=west},
				xtick={-4,...,4}, ytick={-4,...,4},
				grid=major, width=3in, height = 3in,
				grid style={dashed, gray!40}
				]
				\addplot[color=blue, ultra thick, smooth, samples=200, domain=-2.3:0]{x^2-1};
				\addplot[color=blue, ultra thick, smooth, samples=200, domain=0:2.3]{1-x^2}node[pos=0.1, above right]{\large{$y=1-x^2$}};		

				\draw[fill=blue] (axis cs:0,-1) circle [color=blue,radius=3pt];
				\draw[fill=white] (axis cs:0,-1) circle [color=white,radius=2pt];

				\draw[fill=blue] (axis cs:0,1) circle [color=blue,radius=3pt];
				\draw[fill=white] (axis cs:0,1) circle [color=white,radius=2pt];
			\end{axis}
		\end{tikzpicture}
	\end{figure}

	\end{freeResponse}


	\end{enumerate}
	
	
	
	
\end{problem}

\WkstNew



%problem 2
\begin{problem}
Given $y(t)=t- \frac{\pi}{3}$ and $w(t)=\sin(t)$.  Find:
\begin{enumerate}	
	\item  $y(w(t))$
		\WkstHop

		\begin{freeResponse}
			$y(w(t))=y\left( \sin(t) \right)=\sin(t)-\frac{\pi}{3}$
		\end{freeResponse}	


	\item  $w(y(t))$
		\WkstHop

		\begin{freeResponse}
		$w(y(t))=w\left( t-\frac{\pi}{3}\right)=\sin\left( t-\frac{\pi}{3}\right)$
		\end{freeResponse}	


	\item  $w \left(y \left(\frac{4\pi}{3} \right)\right)$
		\WkstHop
		
		\begin{freeResponse}
		$w \left(y \left(\frac{4\pi}{3} \right)\right)=\sin \left(\frac{4\pi}{3}-\frac{\pi}{3}\right)=sin(\pi)=0$
		\end{freeResponse}	


	\item  $y(w(\frac{4\pi}{3}))$
		\WkstHop
		
		\begin{freeResponse}
		$y \left(w \left(\frac{4\pi}{3} \right)\right)=\sin \left(\frac{4\pi}{3}\right)-\frac{\pi}{3}=\frac{-\sqrt{3}}{2}-\frac{\pi}{3}$

		You should know values of $sin(x)$ and $cos(x)$ for all values found on the standard unit circle.
		\end{freeResponse}	
	
	\end{enumerate}
	
	
\end{problem}
\WkstNew


%problem 4
\begin{problem}
Define 
	$g(x) =   \left\{ \begin{array}{cl}
	|x-2|		 	&	\qquad \text{if } x < 0					\\
	\cos(x)			&	\qquad \text{if }  x \geq 0  	\\		\end{array} \right.  $
\begin{enumerate}	
	\item  Sketch a graph of $g$
		\WkstHop
		\WkstHop

		\begin{freeResponse} \hfil

			  \begin{figure}[H] 
			  	\centering
				
				\begin{tikzpicture}[alt={A  sketch of the graph. The left-side is the line y=-x+2, and the right-side is the curve y=cos(x).}]
					\begin{axis}[
						xmin=-4.3, xmax=9.3, ymin=-1.3,ymax=7.3,    
						axis lines =middle, 
						every axis y label/.style={at=(current axis.above origin),anchor=south},
						every axis x label/.style={at=(current axis.right of origin),anchor=west},
						xtick={-4,...,9}, ytick={-1,...,7},
						grid=major, width=4in, height = 2.5in,
						grid style={dashed, gray!40}
						]
						\addplot[color=blue, ultra thick, domain=-4.3:0]{2-x}node[pos=0.25, above right]{\large{$y=g(x)$}};
						\addplot[color=blue, ultra thick, smooth, domain=0:9.3]{cos(deg(x))};%node[pos=0.1, above right]{\large{$y=1-x^2$}};		
		
						\draw[fill=blue] (axis cs:0,2) circle [color=blue,radius=3pt];
						\draw[fill=white] (axis cs:0,2) circle [color=white,radius=2pt];
		
						\draw[fill=blue] (axis cs:0,1) circle [color=blue,radius=3pt];
					\end{axis}
				\end{tikzpicture}

			\end{figure}
	
		\end{freeResponse}

	\item  Find the domain and range of $g$
		\WkstHop
		
		\begin{freeResponse}	
			Domain: $(-\infty,\infty)$, Range: $[-1,1]\cup(2,\infty)$
		\end{freeResponse}		
	
	\item  Find the values of $g(\pi)$ and $g(-\pi)$
		\WkstHop
		
		\begin{freeResponse}	
			$g(\pi)=\cos (\pi)=-1$ and $g(-\pi)=|-\pi-2|=\pi+2$
		\end{freeResponse}
	
	\end{enumerate}

\end{problem}
\WkstNew

%problem 5
\begin{problem}
The entire graph of $f(x)$ is given below.

% Replaced by the tikzpic below
%	\begin{image}
%	\includegraphics[scale=.5]{UFFigure2.png}
%	\end{image}

	  \begin{figure}[H] 
	  	\centering
		
		\begin{tikzpicture}[alt={Graph of piecewise-defined function for reading domains, ranges, and intervals of increasing and decreasing.}]
			\begin{axis}[
				xmin=-4.3, xmax=4.3, ymin=-4.3,ymax=4.3,    
				axis lines =middle, 
				every axis y label/.style={at=(current axis.above origin),anchor=south},
				every axis x label/.style={at=(current axis.right of origin),anchor=west},
				xtick={-4,...,4}, ytick={-4,...,4},
				grid=major, width=3.5in, height = 3.5in,
				grid style={dashed, gray!40}
				]
				\addplot[color=blue, ultra thick, smooth, samples=200, domain=-1:2]{x^2-1}node[pos=0.75, right]{\large{$f$}};;				
				\path[draw, color=blue, very thick] (axis cs:-4,-4) --  (axis cs:-3,-3);
				\path[draw, color=blue, very thick] (axis cs:2,-1) --  (axis cs:4,0); 
				\draw[fill=blue] (axis cs:-4,-4) circle [color=blue,radius=3pt];		
				\draw[fill=blue] (axis cs:-3,0) circle [color=blue,radius=3pt];	
				\draw[fill=blue] (axis cs:-2,-2) circle [color=blue,radius=3pt];
				\draw[fill=blue] (axis cs:2,-1) circle [color=blue,radius=3pt];
				\draw[fill=blue] (axis cs:4,0) circle [color=blue,radius=3pt];
	
				\draw[fill=blue] (axis cs:-3,-3) circle [color=blue,radius=3pt];
				\draw[fill=white] (axis cs:-3,-3) circle [color=white,radius=2pt];

				\draw[fill=blue] (axis cs:-1,0) circle [color=blue,radius=3pt];
				\draw[fill=white] (axis cs:-1,0) circle [color=white,radius=2pt];

				\draw[fill=blue] (axis cs:2,3) circle [color=blue,radius=3pt];
				\draw[fill=white] (axis cs:2,3) circle [color=white,radius=2pt];
			\end{axis}
		\end{tikzpicture}

	\end{figure}
	
\begin{enumerate}	
	\item  Find the domain and range of $f$
		\WkstHop

		\begin{freeResponse}
			Domain: $[-4,-3]\cup\left\{-2\right\}\cup(-1,4]$
			Range: $[-4,-3)\cup\left\{-2\right\}\cup[-1,3)$
		\end{freeResponse}	

	\item  Find the values of $f(-3),f(-2), f(-1),f(2)$
		\WkstHop
		
		\begin{freeResponse}
		$f(-3)=0, f(-2)=-2, f(-1)\ \text{does not exist}, f(2)=-1$
		\end{freeResponse}	

	\item  Find the intervals on which $f(x)$ is positive.  Find the intervals on which $f(x)$ is negative.
		\WkstHop
		
		\begin{freeResponse}
		 $f(x)$ is postive on $(1,2)$. $f(x)$ is negative on $[-4,-3),\left\{-2\right\},(-1,1),[2,4)$
		\end{freeResponse}
	
	\item Find the intervals on which $f$ is increasing.  Find the intervals on which $f$ is decreasing.
		\WkstHop
		
		\begin{freeResponse}
			$f$ is increasing on $(-4,-3),(0,2), (2,4)$.  $f$ is decreasing on $(-1,0)$. It would also be OK to say that		
			$f$ is increasing on $[-4,-3],[0,2), [2,4]$ and $f$ is decreasing on $(-1,0]$. Typically we will not be including endpoints of intervals
			when we talk about increasing/decreasing.
		\end{freeResponse}
	
	\item True or False: $f(1.5) < f(2)$
		\WkstHop
		
		\begin{freeResponse}
			False, $f(2) < f(1.5)$
		\end{freeResponse}	
	
	\end{enumerate}

	
\end{problem}


\WkstNew

%problem 6
\begin{problem}
Determine if the function is even, odd, or neither.


\begin{enumerate}	
	\item  $h(x)=x^4+x^2-3$
		\WkstHop
		
		\begin{freeResponse}

			A function is even if $f(x)=f(-x)$, for all $x$ in the domain, which means its graph is symmetic about the y-axis.  A function is odd if $f(-x)=-f(x)$, for all $x$ in the domain, which means its graph is symmetic about the origin. 
			 $$h(x)=x^4+x^2-3$$
				$$h(-x)=(-x)^4+(-x)^2-3=x^4+x^2-3$$
				$h(x)=h(-x)$.  Hence $h$ is even.  This can be verified by graphing $h$ and seeing that its graph is symmetric about the y-axis.
		\end{freeResponse}

	\item  $s(t)=t^2-t$
		\WkstHop
		
		\begin{freeResponse}
			$$s(t)=t^2-t$$
			$$s(-t)=(-t)^2-(-t)=t^2+t$$ This does not equal $s(t)$ so $s(t)$ is not even.
			$$-s(t)=-(t^2-t)=-t^2+t$$  This does not equal $s(-t)$ so $s$ is not odd.  Hence, $s(t)$ is neither even, nor odd.
		\end{freeResponse}

	\item  We know that $\sin(\theta)$ is odd and $\cos(\theta)$ is even.  Is $g(\theta)=\tan(\theta)$ even, odd, or neither?
		\WkstHop
		
		\begin{freeResponse}
			 $$g(\theta)=\tan(\theta)=\frac{\sin(\theta)}{\cos(\theta)}$$
				\begin{align*}
				g(-\theta)&=\frac{\sin(-\theta)}{\cos(-\theta)}\\
				&=\frac{-\sin(\theta)}{\cos(\theta)}\\
				&=-\frac{\sin(\theta)}{\cos(\theta)}
				\end{align*}
				\begin{center}
				$$-g(\theta)=-\frac{\sin(\theta)}{\cos(\theta)}$$
				$g(-\theta)=-g(\theta) \implies g$ odd
				\end{center}
		\end{freeResponse}

	\end{enumerate}
	

	
\end{problem}


\WkstNew

%problem7
\begin{problem}
Using the known graphs of $y=\sqrt{x}$ and $y=\frac{1}{x}$, sketch the graphs of the following using transformations.

\begin{enumerate}
 	\item $y=\sqrt{x+2}-3$
	\WkstHop
	
	\begin{freeResponse}

		This is a shift of $y=\sqrt{x}$ moved left two units and down three units.
	
		\begin{image}		
			\includegraphics[scale = 0.5, alt={Graph of square root function and of it shifted 2 units down and 3 units to the left.}]{UFFigure8.png }
		\end{image}

	\end{freeResponse}
	

	\item $y=\frac{1}{x-4}+1$
		\WkstHop
	
		\begin{freeResponse}
			This is a shift of $y=\frac{1}{x}$ moved right four units and up one unit.
			\begin{image}		
				\includegraphics[scale=0.5, alt={Graph of 1 over x function and of it shifted 4 units right and 1 unit up.}]{UFFigure4.png}
			\end{image}
		\end{freeResponse}

\end{enumerate}

\end{problem}

\WkstNew

%problem8
\begin{problem}
Find the domain of the function.  Determine whether the function is odd, even, or neither.

\begin{enumerate}
	\item $f(x)=\frac{x}{\sqrt{x^2-9}}$
		\WkstHop
		
		\begin{freeResponse}
			To find the domain, recall that a rational expression cannot have $0$ in the denominator and a square root expression cannot have a negative number under the square root.  Thus, $x^2-9>0$.
			$\implies (x-3)(x+3)>0$  The zeros are located at $x=-3,3$.  From this we can draw a sign chart  for the expression, $x^2-9$, and test values.
			\begin{image}		
				\includegraphics[scale=0.6, alt={Sign-chart for x^2-9 showing that it is positive when x < -3 or when x>3, but is negative if -3<x<3.}]{UFFigure21.png}
			\end{image}
			We see that $x^2-9>0$ on the interval $(-\infty,-3)\cup (3,\infty)$.  Thus our domain is $(-\infty,-3)\cup (3,\infty)$.
	
			Next we check for even/odd/neither.\\
			$f(-x)=\frac{-x}{\sqrt{(-x)^2-9}}$ which does not equal $f(x)$ so $f$ is not even\\
			$-f(x)=-\frac{x}{\sqrt{x^2-9}}$ which is equal to $f(-x)$ so $f$ is odd.
		
		\end{freeResponse}

	\item $g(x)= \frac{\sin(x)}{x}$
		\WkstHop
		
		\begin{freeResponse}
		
			To find the domain, recall that a rational expression cannot have $0$ in the denominator.  Therefore, our domain is $(-\infty,0) \cup (0,\infty)$
			
			Next we check for even/odd/neither.\\
				$f(-x)= \frac{\sin(-x)}{-x}=\frac{-\sin(x)}{-x}=\frac{\sin(x)}{x}$ which equals $f(x)$ so $f$ is even

		\end{freeResponse}

	\item $h(t)= \ln(t^3-1)$
		\WkstHop
		
		\begin{freeResponse}
		
			To find the domain, recall that we cannot take the natural logarithm of $0$ or a negative number.  Therefore, $t^3-1>0$.
			$\implies (t-1)(t^2+t+1)>0$.  The zero is located at $t=1$.  From this we can draw a sign chart and test values.

			\begin{image}		
				\includegraphics[scale=0.2, alt={Sign-chart showing t^3-1 is negative for t < 1 and positive for t>1.}]{Figure10.png}
			\end{image}

			We see that $t^3-1>0$ on the interval $(1,\infty)$.  Thus our domain is $ (1,\infty)$.\\ \\
		
			Next we check for even/odd/neither.\\
			$h(t)$ is neither even nor odd because if $t$ is in the domain, then $-t$ is not in the domain.

		\end{freeResponse}

\end{enumerate}

\end{problem}
\WkstNew

%problem9
\begin{problem}

  Let $g$ be a one-to-one function and let $g^{-1}$ be its inverse.
  \textbf{True or False:}
  If the point $(2, 1/5)$ lies on the graph of $g$, then the point $(2, 5)$ lies on the graph of $g^{-1}$.
\WkstHop

	 \begin{freeResponse}
	    This statement is \textbf{false}: we have $g(2) = 1/5 \iff 2 = g^{-1}(1/5)$.
	    The notation $g^{-1}$ never, in this course, means $1/g$.
	  \end{freeResponse}

\end{problem}




\begin{problem}
  Each of the following functions are invertible on their given domains.
  For each one find a formula for its inverse and give the domain and range of the inverse.
  \begin{enumerate}
	\item The function $f$ defined by $f(x)=x^2-4x-5$ for every $x \ge 2$.
		\WkstHop
	
		\begin{freeResponse}
			To help find the a formula for $f^{-1}$ we will first ``complete the square'':
		        
		        \begin{align*}
		          x^2-4x-5 &= x^2 - 4x + 4 - 4 - 5,\\
		                   &= (x - 2)^2 - 9.
		        \end{align*}
		    
		        Setting $y = f(x) = x^2-4x-5 = (x - 2)^2 - 9$, we can follow the procedures outlined for algebracially finding the formula for an inverse function.
		        
		        \begin{align*}
			          &\mbox{} y = (x - 2)^2 - 9\\
			          &\implies y + 9 = (x - 2)^2\\
			          &\implies \sqrt{y + 9} = |x - 2| \\
			          &\implies \sqrt{y + 9} = x - 2 \hspace{1em} \textrm{(since $x \ge 2$)}\\
			          &\implies \sqrt{y + 9} + 2 = x \\
			          &\implies 2 + \sqrt{x + 9} = y \hspace{1em} \textrm{(interchange $x$ and $y$ along with rewriting)}
		        \end{align*}
		        
		        Therefore we have that $f^{-1}$ is defined by $f^{-1}(x) = 2 + \sqrt{x + 9}$.
		        The domain of $f^{-1}(x)$ is $[-9, \infty)$ and the range is $[2, \infty)$.
		      \end{freeResponse}
\WkstNew

	\item The function $g$ defined by $g(u) = \sqrt[4]{u + 2}$.
		\WkstHop

		 \begin{freeResponse}
		        Following the procedure to algebraically find the formula for the inverse function we have
		        
		        \begin{align*}
		          &\mbox{} z = \sqrt[4]{u + 2}\\
		          &\implies z = (u+2)^{1/4}\\
		          &\implies z^4 = u + 2\\
		          &\implies z^4 - 2 = u\\
		          &\implies u^4 - 2 = z \hspace{1em} \mbox{(interchange $u$ and $z$)}
		        \end{align*}
		        Therefore we have that $g^{-1}$ is defined by $g^{-1}(u) = u^4 - 2$.
		        The domain of $g^{-1}$ is $[0, \infty)$ and the range is $[-2, \infty)$.
	      \end{freeResponse}

    \item The function $h$ defined by $h(t) = 1/(t+2)^2$ for every $t > -2$.
	\WkstHop
      
      	\begin{freeResponse}
        Following the procedure to algebraically find  the formula for the inverse function we have

        \begin{align*}
          &\mbox{} s = \frac{1}{(t+2)^2}\\
          &\implies (t+2)^2 = \frac{1}{s}\\
          &\implies |t + 2| = \sqrt{\frac{1}{s}} \\
          &\implies t + 2 = \sqrt{\frac{1}{s}} \hspace{1em} \mbox{(since $t > -2$)}\\
          &\implies t = \sqrt{\frac{1}{s}} - 2 \\
          &\implies s = \frac{1}{\sqrt{t}} - 2 \hspace{1em} \mbox{(interchange $s$ and $t$)}
        \end{align*}
        Therefore we have $h^{-1}$ is defined by $h^{-1} = \frac{1}{\sqrt{t}} - 2$.
        The domain of $h^{-1}$ is $(0, \infty)$ and the range is $(-2, \infty)$.
      \end{freeResponse}
  
  \end{enumerate}
\end{problem}



\WkstNew
\begin{problem}
Explain what each of the following means:

  \begin{enumerate}

    % part a
  \item $f^{-1}(x)$
\WkstHop

    \begin{freeResponse}
      This denotes the inverse function of $f$, $f^{-1}$, evauluated at $x$.
    \end{freeResponse}
		
    % part b
  \item $f(x^{-1})$
\WkstHop

    \begin{freeResponse}
      This means $f \left( \frac{1}{x} \right)$.
    \end{freeResponse}
		
    % part c
  \item $\left( f(x) \right)^{-1}$
\WkstHop

    \begin{freeResponse}
      This means $f(x)$ raised to the $-1$ power,
      i.e. $\frac{1}{f(x)}$.
    \end{freeResponse}
  \end{enumerate}
\end{problem}

%problem2
\begin{problem}
If $f(x)$ represents the number of packages of buns needed for $x$ packages of hotdogs, what does $f^{-1}(x)$ represent?
\WkstHop

\begin{freeResponse}
$f^{-1}(x)$ represents the numbers of packages of hotdogs needed for $x$ packages of buns.
\end{freeResponse}

\end{problem}


\WkstNew



\begin{problem}

  We're given the following graph of a function:
% replaced by the tikzpic below
%  \begin{image}
%    \includegraphics[scale = 0.4]{UFFigure112.png}
%  \end{image}

	  \begin{figure}[H] 
	  	\centering
		
		\begin{tikzpicture}[alt={Graph piecewise-defined function to read info about domain, range, and one-to-one information.}]
			\begin{axis}[
				xmin=-3.3, xmax=3.3, ymin=-2.3,ymax=4.3,    
				axis lines =middle, 
				every axis y label/.style={at=(current axis.above origin),anchor=south},
				every axis x label/.style={at=(current axis.right of origin),anchor=west},
				xtick={-3,...,3}, ytick={-2,...,4},
				grid=major, width=3.5in, height = 3.5in,
				grid style={dashed, gray!40}
				]
				\addplot[color=blue, ultra thick, smooth, samples=200, domain=-2.3:0]{x^2}node[pos=0.75, left]{\large{$f$}};;				
				\path[draw, color=blue, very thick] (axis cs:0,1) --  (axis cs:1,0);
				\path[draw, color=blue, very thick] (axis cs:1,-1) --  (axis cs:3,-3);
				 
				\draw[fill=blue] (axis cs:0,3) circle [color=blue,radius=3pt];		
	
				\draw[fill=blue] (axis cs:0,0) circle [color=blue,radius=3pt];
				\draw[fill=white] (axis cs:0,0) circle [color=white,radius=2pt];

				\draw[fill=blue] (axis cs:1,0) circle [color=blue,radius=3pt];
				\draw[fill=white] (axis cs:1,0) circle [color=white,radius=2pt];

				\draw[fill=blue] (axis cs:0,1) circle [color=blue,radius=3pt];
				\draw[fill=white] (axis cs:0,1) circle [color=white,radius=2pt];

				\draw[fill=blue] (axis cs:1,-1) circle [color=blue,radius=3pt];
				\draw[fill=white] (axis cs:1,-1) circle [color=white,radius=2pt];
			\end{axis}
		\end{tikzpicture}
	\end{figure} 
	
  Use this graph to answer the following questions:
  \begin{enumerate}
  %a
    \item What is the domain of this function?
	\WkstHop

      \begin{freeResponse}
        $(-\infty, 1) \cup (1, \infty)$
      \end{freeResponse}
      
    %b
    \item What is the range of this function?
	\WkstHop
      
      \begin{freeResponse}
        $(-\infty, -1) \cup (0, \infty)$
      \end{freeResponse}

    %c
    \item What is the value of $f(0)$, $f(1)$, and $f(2)$?
\WkstHop
      
      \begin{freeResponse}
        $f(0) = 3$, $f(1)$ does not exist, $f(2) = -2$
      \end{freeResponse}

   %d
    \item Does this function have an inverse? (Why or why not?)
	\WkstHop
      
      	\begin{freeResponse}
        No, the function does not have an inverse.
        It is not one-to-one (that is, it does not pass the horizontal line test).
      \end{freeResponse}
      
     %e
    \item Find at least two intervals on which the function is one-to-one.
	\WkstHop
      
      \begin{freeResponse}
        The function becomes one-to-one when we restrict its domain to $(-\infty, 0)$:
        
        \begin{image}
      	    \includegraphics[scale = 0.4, alt={Graph of restriction of the function to x<0, showing that the graph is one-to-one here.}]{UFFigure122.png}
        \end{image}
        The function also becomes one-to-one when we restrict its domain to $[0, 1) \cup (1,\infty)$:
       
        \begin{image}
          \includegraphics[scale = 0.4, alt={Graph of restriction to [0,1) union (1, infinity), showing that the graph is one-to-one here.}]{UFFigure222.png}
          \end{image}
      \end{freeResponse}

   %f
    \item Find $f^{-1}(3)$ on a restricted domain of $f$.
	\WkstHop
      
      \begin{freeResponse}
        In this case restrict the domain of $f$ to $[0, 1)\cup(1,\infty)$.
        By definition we have $f^{-1}(3) = y \iff 3 = f(y)$.
        Looking at the second graph of the restricted function we see that $f(0) = 3$, that is, $f^{-1}(3) = 0$.

        (If we restrict the domain of $f$ to $(-\infty, 0)$, then $f^{-1}(3) = y \iff 3 = f(y)$ implies that $f(-1.7) \approx 3$.
        Hence $-1.7 \approx f^{-1}(3)$.)
      \end{freeResponse}
\WkstNew      
      %g
	\item Using the restricted domain for $f$ of $(-\infty,0)$, sketch a graph of $f^{-1}$.  
	\WkstHop
	
	\begin{freeResponse}
	To graph the inverse, we can think of the graph being reflected over the line $y=x$.  Another way to obtain the graph is to remember than if $(x,y)$ is a point on the graph of $f$, then $(y,x)$ is a point on the graph of $f^{-1}$.  For example, $(-1,1)$ is on the graph of $f$ so $(1,-1)$ is on the graph of $f^{-1}$.
	
	 \begin{image}
          \includegraphics[scale = 0.4, alt={Graph showing the restriction of the function, the line y=x, and the reflection of the restriction across that line.}]{UFFigure223.png}
        \end{image}
	\end{freeResponse}	
	 %h
	\item Using the restricted domain for $f$ of $[0,1)\cup(1,\infty)$, sketch a graph of $f^{-1}$.  
	\WkstHop
	
	\begin{freeResponse}
	We can graph the inverse of $f$, when restricted to $[0,1)\cup(1,\infty)$, in pieces from left to right.  Since $(0,3)$ is a point on the graph of $f$, the point $(3,0)$ in on the graph of $f^{-1}$.  Then, we see on the graph of $f$, there is a linear piece going from $(0,1)$ to $(1,0)$.  When we imagine reflecting this part of the graph of $f$ over the line $y=x$, it reflects onto itself.  The last piece of the graph of $f$ is a line from $(1,-1)$ which appears to also contain the point $(-2,2)$.  Reflecting this part of $f$ over the line $y=x$, we obtain a line starting at $(-1,1)$ and through the point $(-2,2)$.
	
	 \begin{image}
          \includegraphics[scale = 0.4, alt={Graph showing the restriction of the function, the line y=x, and the reflection of the restriction across that line.}]{UFFigure333.png}
        \end{image}
	
	
	 \end{freeResponse} 

  \end{enumerate}
\end{problem}
		
\end{document} 









