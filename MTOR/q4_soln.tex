\documentclass{ximera}

\newcommand{\RR}{\mathbb R}
\renewcommand{\d}{\,d}
\newcommand{\dd}[2][]{\frac{d #1}{d #2}}
\renewcommand{\l}{\ell}
\newcommand{\ddx}{\frac{d}{dx}}
\newcommand{\dfn}{\textbf}
\newcommand{\eval}[1]{\bigg[ #1 \bigg]}
\renewcommand{\theenumii}{\textup{(\roman{enumii})}}
\renewcommand{\labelenumii}{\theenumii}

\usepackage{graphicx}
\usepackage{multicol}
\usepackage{tkz-euclide}
%\usepackage{unicode-math}

\usepackage{pgfplots}   % <- for graphics
\pgfplotsset{compat=newest}


\renewenvironment{freeResponse}{
\ifhandout\setbox0\vbox\bgroup\else
\begin{trivlist}\item[\hskip \labelsep\bfseries Solution:\hspace{2ex}]
\fi}
{\ifhandout\egroup\else
\end{trivlist}
\fi}

\newcommand*{\ZeroOverZero}{\ensuremath{\dfrac{0}{0}}}

\providecommand{\HCCondition}{0}
\newcommand{\WkstHop}[1][1]{\if\HCCondition 0
	\vspace*{\stretch{#1}} \fi} 
\newcommand{\WkstNew}{\if\HCCondition 0
	\newpage
	 \fi} 


\title[Problem 4]{Problem 4}

\begin{document}
\begin{abstract} \end{abstract}
\maketitle


% Extracted from moreThanOneRate.tex, problem #4
\begin{problem}
	A \textbf{rectangle in the first quadrant} is constructed by taking a point $(x,y)$ on the graph of the function $f(x)=9-x^2$, drawing a line segment vertically downward 
		to the $x$-axis and a line segment horizontally leftward to the $y$-axis, as in the picture below. 
		Denote the \textbf{length of the base} (along the $x$-axis) of the rectangle by $x$, the 
		\textbf{height of the rectangle} (along the $y$-axis) by $y$ (in meters), and the \textbf{PERIMETER} of the rectangle by $P$. 
		When $x = 2$ m (and only at that moment), the height $y$ is shrinking at a rate of $\dfrac{1}{5}$ m/s.  
		Find the value of $\displaystyle \eval{\dd[P]{t}}_{x=2 m}$ \textbf{by performing the steps below}.\\
		\begin{center}
			  \begin{tikzpicture}
						\begin{axis}[
							xmin=-0.2, xmax=3.2, ymin=-0.5,ymax=9.5,    
							axis lines =middle, 
							every axis y label/.style={at=(current axis.above origin),anchor=south},
							every axis x label/.style={at=(current axis.right of origin),anchor=west},
							xtick={0,...,3}, ytick={0,...,9},
							grid=major, width=2.5in, height = 2.5in,
							grid style={dashed, gray!40}
							]
							
							\addplot[color=blue, very thick, smooth, samples=200, domain=0:3.2]{9-x^(2)};
							\draw[fill=gray!10] (0,0) -- (1.8, 0) -- (1.8, 5.76) -- (0, 5.76) --  cycle;
							\closedcircle{(1.8, 5.76)};
							\node[above right] at (1.8, 5.76){$(x,y)$};
						\end{axis}
	
			  \end{tikzpicture}
		\end{center}
		


		\begin{enumerate}			
			\item\label{partA} \textbf{Find a formula} for the height, $y$, of the rectangle as a function of $x$. (This is denoted as $y(x)$.) \\

\begin{explanation}
					The height of the rectangle is the $y$-coordinate of the point in the
					 upper-right corner. Since this point lies on the graph of $f(x)=9-x^2$, 
					 we know $y = 9-x^2$.
				\end{explanation}


			\item\label{partB} \textbf{Find the value} of $\displaystyle \eval{\dd[x]{t}}_{x=2 m}$.

\begin{explanation}
					Since $y = 9-x^2$, we know $\displaystyle \dd[y]{t} = -2x \dd[x]{t}$.
					Solving for $\dd[x]{t}$ gives:
					\begin{align*}
						\dd[x]{t} &= -\dfrac{ \dd[y]{t} }{2x} \\
					\eval{\dd[x]{t}}_{x=2 m} &= -\dfrac{ \eval{\dd[y]{t}}_{x=2 m} }{2(2)} \\
						&= -\dfrac{ (-1/5) }{4} \\
						&= \dfrac{ 1 }{20}
					\end{align*}
				\end{explanation}

				
			\item\label{partC} \textbf{Find a formula} for the perimeter, $P$, of the rectangle as a function of $x$. (This is denoted as $P(x)$.)\\[1em]
\begin{explanation}
					\begin{align*}
						P(x) &= 2x + 2y \\
						&= 2x + 2( 9-x^2)\\
						&= -2x^2 + 2x + 18
					\end{align*}
				\end{explanation}			


			\item\label{partD} \textbf{Find the value} of $\displaystyle \eval{\dd[P]{t}}_{x=2 m}$.
\begin{explanation}
					\begin{align*}
						P(x) &= -2x^2 + 2x + 18\\
						\dd[P]{t} &= (-4x+2) \dd[x]{t} \\
						\\eval{\dd[P]{t}}_{x=2 m} &= \eval{(-4x+2) \dd[x]{t}}_{x=2 m} \\
							&= (-4(2)+2) \left(\dfrac{ 1 }{20}\right)\\
							&= -\dfrac{3}{10}
					\end{align*}
				\end{explanation}						
			
			
			

			\item \textbf{Write a sentence to explain} what the value found in \ref{partD} means about the rectangle. (Don't forget UNITS.)
\begin{explanation}
					At the instant when $x = 2$ m, the perimeter of the rectangle is shrinking at the rate of $\dfrac{3}{10}$ m/s.
				\end{explanation}
		\end{enumerate}
		
\end{problem}



\end{document}
