%Add code to compile both versions from makefile at same time
\providecommand{\HCCondition}{0}
%Define each of the conditions
\ifcase\HCCondition
	%\condition=0 -> handout
	\documentclass[nooutcomes,noauthor,space,handout]{ximera}
	\title{More Than One Rate} 
\or	%\condition=1 -> Soln
	\documentclass[nooutcomes,noauthor]{ximera}
	\title{More Than One Rate- Solutions} 
\fi



\usepackage{fullpage}
\newcommand{\RR}{\mathbb R}
\renewcommand{\d}{\,d}
\newcommand{\dd}[2][]{\frac{d #1}{d #2}}
\renewcommand{\l}{\ell}
\newcommand{\ddx}{\frac{d}{dx}}
\newcommand{\dfn}{\textbf}
\newcommand{\eval}[1]{\bigg[ #1 \bigg]}
\renewcommand{\theenumii}{\textup{(\roman{enumii})}}
\renewcommand{\labelenumii}{\theenumii}

\usepackage{graphicx}
\usepackage{multicol}
\usepackage{tkz-euclide}
%\usepackage{unicode-math}

\usepackage{pgfplots}   % <- for graphics
\pgfplotsset{compat=newest}


\renewenvironment{freeResponse}{
\ifhandout\setbox0\vbox\bgroup\else
\begin{trivlist}\item[\hskip \labelsep\bfseries Solution:\hspace{2ex}]
\fi}
{\ifhandout\egroup\else
\end{trivlist}
\fi}

\newcommand*{\ZeroOverZero}{\ensuremath{\dfrac{0}{0}}}

\providecommand{\HCCondition}{0}
\newcommand{\WkstHop}[1][1]{\if\HCCondition 0
	\vspace*{\stretch{#1}} \fi} 
\newcommand{\WkstNew}{\if\HCCondition 0
	\newpage
	 \fi}  %% we can turn off input when making a master document

   % For points in graphs
  \newcommand{\opencircle}[1]{	
 						\draw[fill=blue] #1 circle [color=blue,radius=3pt];
						\draw[fill=white] #1 circle [color=white,radius=2pt] }
						
 \newcommand{\closedcircle}[1]{	
 						\draw[fill=blue] #1 circle [color=blue,radius=3pt]
						}  

\begin{document}
\begin{abstract}		\end{abstract}
\maketitle
\ifcase\HCCondition
\section*{General steps for solving Related Rates problems.}
	\begin{itemize}
		\item Introduce variables, identify the given rate and unknown rate.
		\item Draw a picture.
		\item Find equations.
		\item Differentiate with respect to $t$.
		\item Evaluate and solve.
	\end{itemize}
\section*{Recitation Questions}
\fi

%problem1
\begin{problem} The radius of a circle is increasing at a rate of 2 inches per minute.  \\

	\begin{image}
		 
		\includegraphics[scale=0.5]{Figure2.png}
	\end{image}

\begin{enumerate}
	\item At what rate is the circumference of the circle changing when the radius is 10 inches?
	\begin{freeResponse}
	We know: ${\frac{dr}{dt}}=2$ inches per minute and we want to find ${\frac{dc}{dt}}$ when $r=10$.\\\\
	\begin{align*}
	c&=2\pi r\\
	{\frac{dc}{dt}}&=2\pi {\frac{dr}{dt}}\\
	\eval{{\frac{dc}{dt}}}_{r=10}&=2 \pi (2)\\
	\eval{{\frac{dc}{dt}}}_{r=10}&=4 \pi\ \text{inches per minute}
	\end{align*}
	\end{freeResponse}
\WkstHop
	\item At what rate is the area of the circle changing when the radius is 12 inches?
		\begin{freeResponse}
	We know: ${\frac{dr}{dt}}=2$ inches per minute and we want to find ${\frac{dA}{dt}}$ when $r=12$.\\\\
	\begin{align*}
	A&=\pi r^2\\
	{\frac{dA}{dt}}&=2\pi r{\frac{dr}{dt}}\\
	\eval{{\frac{dA}{dt}}}_{r=12}&=2 \pi (12)(2)\\
	\eval{{\frac{dA}{dt}}}_{r=12}&=48 \pi\ \text{inches squared per minute}
	\end{align*}
	\end{freeResponse}
\WkstHop
\end{enumerate}

\end{problem}
\WkstNew

\begin{problem}
A right cone has a fixed slant height (see figure below) of 9 ft.  The cone's height is shrinking at a rate of 0.5 ft/sec.  At what rate is the \textbf{volume} of the cone changing when the height is 6 ft?  Be sure to label the picture.
	
	\begin{image}
		\includegraphics[scale=.5]{Figure8.png}
	\end{image}
\WkstHop
\begin{freeResponse} \hfil
\vspace*{\stretch{1}}	
	We introduce the variables $V$, for the volume, $r$ for the radius and $h$ for the height of the cone.
	
	The given rate is $\frac{dh}{dt}=-0.5$ ft/s;
	
	the rate to be determined is $\left[\frac{dV}{dt}\right]_{h=6}$.
	
	We label the picture.
	
	\begin{image}
		 
		\includegraphics[scale=.4]{Figure9.png}
	\end{image}

The equation relating all relevant variables is

\[
V=\dfrac{1}{3} \pi r^2\cdot h
\]

Before we differentiate, we will express $r^2$ in terms of $h$.
\[
r^2=81-h^2
\]

and we substitute this into the equation above

\[
V=\dfrac{1}{3} \pi(81-h^2)\cdot h
\]
 Then we simplify it.
 
 \[
V=\dfrac{1}{3}\pi (81h-h^3)
\]
 
 Now, we differentiate with respect to $t$.
	\[
	\frac{dV}{dt}=\dfrac{\pi}{3}(81-3h^2) \cdot \frac{dh}{dt} 
\]
	Now, we evaluate.
	\begin{align*}
	\eval{\frac{dV}{dt}}_{h=6}&=\frac{\pi}{3}(81- 3(6)^2)(-0.5)\\
	\eval{\frac{dV}{dt}}_{h=6}&=\frac{\pi}{3}(-27)(-0.5)\\
	\eval{\frac{dV}{dt}}_{h=6}&=4.5 \pi\text{   f$t^3$/s}
	\end{align*}

\end{freeResponse}

\end{problem}

\WkstNew

\begin{problem}
A part of a circle centered at the origin with radius $r=7$ cm is given in the figure (A) below.  A right triangle is formed in the first quadrant (see Figure (A)).  One of its sides lies on the $x$-axis.  Its hypotenuse runs from the origin to a point on the circle.  The hypotenuse makes an angle $\theta$ with the $x$-axis.  Assume that the angle $\theta$ changes at the rate $\frac{d\theta}{dt}=0.2$\ radians per second.

	\begin{image}
		 
		\includegraphics[scale=.5]{Figure10.png}
	\end{image}
	
\begin{enumerate}
	\item Label Figure A.
		\begin{freeResponse} \hfil

		\begin{image}
		 
			\includegraphics[scale=.5]{Figure11.png}
		\end{image}
		
		\end{freeResponse}
		
	\item In Figure B, draw the the triangle twice; once when $\theta$ is small and once more, when $\theta$ is close to $\frac{\pi}{2}$.

		\begin{image}
		 
			\includegraphics[scale=.4]{Figure12.png}
		\end{image}
	
		\begin{freeResponse} \hfil
	
			\begin{image}
				 
				\includegraphics[scale=.5]{Figure13.png}
			\end{image}
	
		\end{freeResponse}
\WkstNew
	\item Find the rate of change of the height of the triangle when $\theta=\frac{\pi}{3}$
\WkstHop
		\begin{freeResponse}
		$y=7\sin{\theta}$
		\begin{align*}
		\frac{dy}{dt}&=\frac{dy}{d\theta}\cdot \frac{d\theta}{dt}\\
		\frac{dy}{dt}&=7\cos{\theta} \cdot \frac{d\theta}{dt}\\
		\eval{\frac{dy}{dt}}_{\theta=\frac{\pi}{3}}&=7(1/2)(0.2)\\
		&=0.7 \text{cm/sec}
		\end{align*}
				\end{freeResponse}
	\item Find the rate of change of the area of the triangle when $\theta=\frac{\pi}{3}$
\WkstHop
		\begin{freeResponse}
		First, we need to find $x$: $x=7 \cos{\theta}$.  From part c, we know $y=7\sin{\theta}$
		\begin{align*}
		A&=\frac{1}{2}xy\\
		A&=\frac{49}{2}\cos(\theta)\sin(\theta)\\
		\frac{dA}{dt}&=\frac{49}{2}(-\sin^2(\theta)+\cos^2(\theta)) \cdot \frac{d\theta}{dt}\\
		\eval{\frac{dA}{dt}}_{\theta=\frac{\pi}{3}}&=\frac{49}{2}\left(-\left(\frac{\sqrt{3}}{2}\right)^2+\left(\frac{1}{2}\right)^2 \right) \cdot 0.2\\\\
		&=\frac{49}{2}\cdot \frac{-1}{2} \cdot 0.2\\
		&=\frac{-4.9}{2} cm^2/sec
		\end{align*}
		\end{freeResponse}

\end{enumerate}
\end{problem}
\WkstNew


\begin{problem}
	A \textbf{rectangle in the first quadrant} is constructed by taking a point $(x,y)$ on the graph of the function $f(x)=9-x^2$, drawing a line segment vertically downward 
		to the $x$-axis and a line segment horizontally leftward to the $y$-axis, as in the picture below. 
		Denote the \textbf{length of the base} (along the $x$-axis) of the rectangle by $x$, the 
		\textbf{height of the rectangle} (along the $y$-axis) by $y$ (in meters), and the \textbf{PERIMETER} of the rectangle by $P$. 
		When $x = 2$ m (and only at that moment), the height $y$ is shrinking at a rate of $\dfrac{1}{5}$ m/s.  
		Find the value of $\displaystyle \eval{\dd[P]{t}}_{x=2 m}$ \textbf{by performing the steps below}.\\
		\begin{center}
			  \begin{tikzpicture}
						\begin{axis}[
							xmin=-0.2, xmax=3.2, ymin=-0.5,ymax=9.5,    
							axis lines =middle, 
							every axis y label/.style={at=(current axis.above origin),anchor=south},
							every axis x label/.style={at=(current axis.right of origin),anchor=west},
							xtick={0,...,3}, ytick={0,...,9},
							grid=major, width=2.5in, height = 2.5in,
							grid style={dashed, gray!40}
							]
							
							\addplot[color=blue, very thick, smooth, samples=200, domain=0:3.2]{9-x^(2)};
							\draw[fill=gray!10] (0,0) -- (1.8, 0) -- (1.8, 5.76) -- (0, 5.76) --  cycle;
							\closedcircle{(1.8, 5.76)};
							\node[above right] at (1.8, 5.76){$(x,y)$};
						\end{axis}
	
			  \end{tikzpicture}
		\end{center}
		


		\begin{enumerate}			
			\item\label{partA} \textbf{Find a formula} for the height, $y$, of the rectangle as a function of $x$. (This is denoted as $y(x)$.) \\

\WkstHop
				\begin{freeResponse}
					The height of the rectangle is the $y$-coordinate of the point in the
					 upper-right corner. Since this point lies on the graph of $f(x)=9-x^2$, 
					 we know $y = 9-x^2$.
				\end{freeResponse}


			\item\label{partB} \textbf{Find the value} of $\displaystyle \eval{\dd[x]{t}}_{x=2 m}$.

\WkstHop
				\begin{freeResponse}
					Since $y = 9-x^2$, we know $\displaystyle \dd[y]{t} = -2x \dd[x]{t}$.
					Solving for $\dd[x]{t}$ gives:
					\begin{align*}
						\dd[x]{t} &= -\dfrac{ \dd[y]{t} }{2x} \\
					\eval{\dd[x]{t}}_{x=2 m} &= -\dfrac{ \eval{\dd[y]{t}}_{x=2 m} }{2(2)} \\
						&= -\dfrac{ (-1/5) }{4} \\
						&= \dfrac{ 1 }{20}
					\end{align*}
				\end{freeResponse}

				
			\item\label{partC} \textbf{Find a formula} for the perimeter, $P$, of the rectangle as a function of $x$. (This is denoted as $P(x)$.)\\[1em]
\WkstHop			
				\begin{freeResponse}
					\begin{align*}
						P(x) &= 2x + 2y \\
						&= 2x + 2( 9-x^2)\\
						&= -2x^2 + 2x + 18
					\end{align*}
				\end{freeResponse}			


			\item\label{partD} \textbf{Find the value} of $\displaystyle \eval{\dd[P]{t}}_{x=2 m}$.
\WkstHop
				\begin{freeResponse}
					\begin{align*}
						P(x) &= -2x^2 + 2x + 18\\
						\dd[P]{t} &= (-4x+2) \dd[x]{t} \\
						\\eval{\dd[P]{t}}_{x=2 m} &= \eval{(-4x+2) \dd[x]{t}}_{x=2 m} \\
							&= (-4(2)+2) \left(\dfrac{ 1 }{20}\right)\\
							&= -\dfrac{3}{10}
					\end{align*}
				\end{freeResponse}						
			
			
			

			\item \textbf{Write a sentence to explain} what the value found in \ref{partD} means about the rectangle. (Don't forget UNITS.)
\WkstHop
				\begin{freeResponse}
					At the instant when $x = 2$ m, the perimeter of the rectangle is shrinking at the rate of $\dfrac{3}{10}$ m/s.
				\end{freeResponse}
		\end{enumerate}
		
\end{problem}		



\end{document} 


















