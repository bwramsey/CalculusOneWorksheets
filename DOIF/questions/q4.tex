% Extracted from derivativesOfInverseFunctions.tex, problem #4
\begin{problem}
  Simplify each of the following expressions.
  \begin{enumerate}
    \item
      $\cos^{-1} \bigl( \sin(\pi/2) \bigr)$
\WkstHop

      \begin{freeResponse}
        By the unit circle, $\sin(\pi/2) = 1$, and so we are looking for $\cos^{-1}(1)$.
        The range of $\cos^{-1}$ is $[0, \pi]$, and so, by properties of inverse functions, $\cos^{-1}(1) = 0$.
      \end{freeResponse}

    \item
      $\tan \bigl( \sin^{-1}(x/4) \bigr)$
\WkstHop

      \begin{freeResponse}
        Let $\theta = \sin^{-1}(x/4)$, then $\sin(\theta) = x/4$.
        We can then draw the corresponding right triangle:
        \begin{image}
          \includegraphics[scale = 0.4]{figure5.png}
        \end{image}
        Calling the adjacent side $y$, by the Pythagorean Theorem we obtain

          $$4^2 = x^2 + y^2 \implies y = \sqrt{16-x^2}$$
Remark: Since $\theta$ is in the range of $\sin^{-1}$, it follows that $-\pi /2 <\theta <\pi /2$.  Therefore, $\cos(\theta)=y/4>0$.  Therefore, $y>0$.\\
        Then
        \begin{align*}
          \tan \left( \sin^{-1} \left(4/x \right) \right) &= \tan( \theta), \\
                                                                   &= \frac{x}{y}, \\
                                                                 &= \frac{x}{\sqrt{16-x^2}}.
        \end{align*}
Note: $\tan\theta$ has the same sign as $x$, since $y>0$.
      \end{freeResponse}
  \end{enumerate}
\end{problem}
