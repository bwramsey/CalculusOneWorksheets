\documentclass{ximera}

\newcommand{\RR}{\mathbb R}
\renewcommand{\d}{\,d}
\newcommand{\dd}[2][]{\frac{d #1}{d #2}}
\renewcommand{\l}{\ell}
\newcommand{\ddx}{\frac{d}{dx}}
\newcommand{\dfn}{\textbf}
\newcommand{\eval}[1]{\bigg[ #1 \bigg]}
\renewcommand{\theenumii}{\textup{(\roman{enumii})}}
\renewcommand{\labelenumii}{\theenumii}

\usepackage{graphicx}
\usepackage{multicol}
\usepackage{tkz-euclide}
%\usepackage{unicode-math}

\usepackage{pgfplots}   % <- for graphics
\pgfplotsset{compat=newest}


\renewenvironment{freeResponse}{
\ifhandout\setbox0\vbox\bgroup\else
\begin{trivlist}\item[\hskip \labelsep\bfseries Solution:\hspace{2ex}]
\fi}
{\ifhandout\egroup\else
\end{trivlist}
\fi}

\newcommand*{\ZeroOverZero}{\ensuremath{\dfrac{0}{0}}}

\providecommand{\HCCondition}{0}
\newcommand{\WkstHop}[1][1]{\if\HCCondition 0
	\vspace*{\stretch{#1}} \fi} 
\newcommand{\WkstNew}{\if\HCCondition 0
	\newpage
	 \fi} 


\title[Problem 5]{Problem 5}

\begin{document}
\begin{abstract} \end{abstract}
\maketitle


% Extracted from derivativesOfInverseFunctions.tex, problem #5
\begin{problem}
  A table of values for $f$ and $f'$ is shown below.
  Suppose that $f$ is a one-to-one function and $f^{-1}$ is its inverse.
  \begin{center}
    \begin{tabular}{ccc}
\hline
      $x$ & $f(x)$ & $f'(x)$\\
\hline \hline
      1 & 3 & 4\\

      3 & 4 & 5\\

      4 & 6 & 3\\
\hline
    \end{tabular}
  \end{center}

  \begin{enumerate}
    \item Evaluate $f^{-1}(f(x))$ at $x = 3$.
\begin{explanation}
         $ f^{-1}(f(3)) = f^{-1}(4) = 3$
      \end{explanation}


    \item Evaluate $\ddx f(f(x))$ at $x = 3$.
\begin{explanation}
        \begin{align*}
          \ddx f(f(x)) = f'(f(x)) \cdot f'(x) &\implies \eval{\ddx f(f(x))}_{x = 3} = f'(f(3)) \cdot f'(3)\\
          &= f'(4) \cdot 5 = 3 \cdot 5 = 15
        \end{align*}
      \end{explanation}
    \item Evaluate $\ddx\ln((f(x))$ at $x = 3$.
[2]
      \begin{explanation}
        \begin{align*}
          \ddx\ln((f(x)) = \frac{f'(x)}{f(x)} \\
          &\implies \eval{\ddx \ln(f(x))}_{x = 3} = \frac{f'(3)}{f(3)} = \frac{5}{4}
        \end{align*}
      \end{explanation}
    \item Evaluate $f^{-1}(x)$ at $x = 3$.
[2]
      \begin{explanation}
	$ f^{-1}(3) = 1 \iff 3 = f(1)$
      \end{explanation}
    \item Evaluate $\ddx f^{-1}(x)$ at $x = 3$.
[2]
      \begin{explanation}
        \begin{align*}
          \ddx f^{-1}(x) &= \frac{1}{f'(f^{-1}(x))} \\
          &\implies \eval{\ddx f^{-1}(x)}_{x = 3} = \frac{1}{f'(f^{-1}(3))} \\
          &= \frac{1}{f'(1)} = \frac{1}{4}
        \end{align*}
      \end{explanation}
      \item Evaluate $\lim_{x \to 4} \frac{f(x)-f(4)}{x-4}$
[2]
      \begin{explanation}
$\lim_{x \to 4} \frac{f(x)-f(4)}{x-4} = f'(4)=3$
      \end{explanation}
  \end{enumerate}
\end{problem}



\end{document}
