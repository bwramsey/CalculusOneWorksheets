\documentclass{ximera}

\newcommand{\RR}{\mathbb R}
\renewcommand{\d}{\,d}
\newcommand{\dd}[2][]{\frac{d #1}{d #2}}
\renewcommand{\l}{\ell}
\newcommand{\ddx}{\frac{d}{dx}}
\newcommand{\dfn}{\textbf}
\newcommand{\eval}[1]{\bigg[ #1 \bigg]}
\renewcommand{\theenumii}{\textup{(\roman{enumii})}}
\renewcommand{\labelenumii}{\theenumii}

\usepackage{graphicx}
\usepackage{multicol}
\usepackage{tkz-euclide}
%\usepackage{unicode-math}

\usepackage{pgfplots}   % <- for graphics
\pgfplotsset{compat=newest}


\renewenvironment{freeResponse}{
\ifhandout\setbox0\vbox\bgroup\else
\begin{trivlist}\item[\hskip \labelsep\bfseries Solution:\hspace{2ex}]
\fi}
{\ifhandout\egroup\else
\end{trivlist}
\fi}

\newcommand*{\ZeroOverZero}{\ensuremath{\dfrac{0}{0}}}

\providecommand{\HCCondition}{0}
\newcommand{\WkstHop}[1][1]{\if\HCCondition 0
	\vspace*{\stretch{#1}} \fi} 
\newcommand{\WkstNew}{\if\HCCondition 0
	\newpage
	 \fi} 


\title[Problem 8]{Problem 8}

\begin{document}
\begin{abstract} \end{abstract}
\maketitle


% Extracted from derivativesOfInverseFunctions.tex, problem #8
\begin{problem}

  Find the slope of the tangent line to the curve $y = f^{-1}(x)$ at $(4,7)$ if the slope of the tangent line to the curve $y=f(x)$ at $(7,4)$ is $\frac{2}{3}$.
\begin{explanation}
    Note that the statement ``the slope of the tangent line to the curve $y=f(x)$ at $(7,4)$ is $\frac{2}{3}$" specifically means that $f'(7) = \frac{2}{3}$.
    The slope of the tangent line to the curve $y = f^{-1}(x)$ at $(4,7)$ is $(f^{-1})'(4)$, and so we use the formula for the derivative of the inverse function to compute: $(f^{-1})'(4) = \frac{1}{f'(7)} = \frac{1}{\frac{2}{3}} = \frac{3}{2}$.
  \end{explanation}
\end{problem}



\end{document}
