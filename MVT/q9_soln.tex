\documentclass{ximera}

\newcommand{\RR}{\mathbb R}
\renewcommand{\d}{\,d}
\newcommand{\dd}[2][]{\frac{d #1}{d #2}}
\renewcommand{\l}{\ell}
\newcommand{\ddx}{\frac{d}{dx}}
\newcommand{\dfn}{\textbf}
\newcommand{\eval}[1]{\bigg[ #1 \bigg]}
\renewcommand{\theenumii}{\textup{(\roman{enumii})}}
\renewcommand{\labelenumii}{\theenumii}

\usepackage{graphicx}
\usepackage{multicol}
\usepackage{tkz-euclide}
%\usepackage{unicode-math}

\usepackage{pgfplots}   % <- for graphics
\pgfplotsset{compat=newest}


\renewenvironment{freeResponse}{
\ifhandout\setbox0\vbox\bgroup\else
\begin{trivlist}\item[\hskip \labelsep\bfseries Solution:\hspace{2ex}]
\fi}
{\ifhandout\egroup\else
\end{trivlist}
\fi}

\newcommand*{\ZeroOverZero}{\ensuremath{\dfrac{0}{0}}}

\providecommand{\HCCondition}{0}
\newcommand{\WkstHop}[1][1]{\if\HCCondition 0
	\vspace*{\stretch{#1}} \fi} 
\newcommand{\WkstNew}{\if\HCCondition 0
	\newpage
	 \fi} 


\title[Problem 9]{Problem 9}

\begin{document}
\begin{abstract} \end{abstract}
\maketitle


% Extracted from meanValueTheorem.tex, problem #9
\begin{problem}
  Let $f(x) = (x-3)^{-2}$.
  Show that there is no value $c$ in $(1,4)$ such that $f(4) - f(1) = f^{\prime}(c) (4-1)$.
  Why does this not contradict the Mean Value Theorem?
\begin{explanation}
    First notice that 
    $$f(4)-f(1) = 1^{-2} - (-2)^{-2} = 1-\frac{1}{4} = \frac{3}{4}$$
    and so we are looking for a value $c$ such that 
    $$3 f^\prime (c) = \frac{3}{4} \quad \Longrightarrow \quad f^\prime (c) = \frac{1}{4} $$
    Then since $f'(x) = \frac{-2}{(x-3)^3}$, we can compute:
    $$ f'(c) = \frac{-2}{(c-3)^3} := \frac{1}{4}$$
    $$ (c-3)^3 = - 8 $$
    $$ c = 3 - 2 = 1 $$
    But $1$ is not in the interval $(1,4)$.  This does not contradict the Mean Value Theorem since $f$ is not continuous at $x=3$.

    \begin{image}
      \includegraphics[scale=.65]{ifigure5.png}
    \end{image}
  \end{explanation}
\end{problem}



\end{document}
