% Extracted from meanValueTheorem.tex, problem #7
\begin{problem}
  Heidi drives from her house in Columbus, OH to Indianapolis, IN for vacation.
  Google maps says her driving distance is $170$ miles.
  The drive takes her $2.5$ hours.
  The police send her a speeding ticket in the mail, saying she must have sped to arrive so quickly.
  She is fighting the ticket, saying she just never stopped through the whole drive.
  Can you prove she broke the $65$ mph speed limit at some point during her drive? \textbf{EXPLAIN}.
\WkstHop
  \begin{freeResponse}
    Let us define $s(t)$ to be the position function of Heidi after $t$ hours.
    $s(t)$ will be continuous and differentiable because the position of a car is continuous and differentiable.
    Heidi's average velocity for the whole trip is $\frac{170}{2.5}\approx 68$ mph.
    By the Mean Value Theorem, Heidi 's instantaneous velocity, $s'(t)$, was $68$ mph at some time $t=c$: hence, she broke the $65$ mph speed limit.
  \end{freeResponse}
\end{problem}
