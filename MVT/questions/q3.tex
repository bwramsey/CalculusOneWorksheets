% Extracted from meanValueTheorem.tex, problem #3
\begin{problem}
Find the  x-coordinates of global extrema and global extreme values  of $f$  on the given interval.
		\begin{enumerate}
		
			%part a
			\item  $f(x) = x \sqrt{2-x^2}$ on $[ -\sqrt{2}, \sqrt{2} ]$.
\WkstHop
				
				\begin{freeResponse}
				First note that $f$ is continuous on this closed interval so, by the Extreme Value Theorem, it must attain a maximum and minimum value on the interval $[-\sqrt{2}, \sqrt{2}]$. Those global extrema must occur
				either at critical points or at the boundary points.
				\begin{align*}
				f'(x) &= \sqrt{2-x^2} + x \left( \frac{1}{2 \sqrt{2-x^2}} (-2x) \right) \\
				&= \sqrt{2-x^2} - \frac{x^2}{\sqrt{2-x^2}} \\
				&= \frac{2-x^2-x^2}{\sqrt{2-x^2}} \\
				&= \frac{2(1-x^2)}{\sqrt{2-x^2}}
				\end{align*}
				
				Critical points of $f$ occur where $f'(x) = 0$ or where $f'(x)$ does not exist.  Solving $f'(x) = 0$ yields that $2(1-x^2) = 0$, or $x = \pm 1$.  $f'(x)$ does not exist when $2-x^2 \leq 0$.  But since we are restricting to the interval $[-\sqrt{2}, \sqrt{2}]$, the only points where $f'(x)$ does not exist are the endpoints $\pm \sqrt{2}$.  
				
				The critical points of $f$ in the interval $[-\sqrt{2}, \sqrt{2}]$ are $x = \pm 1$. (The points $x = \pm \sqrt{2}$ technically are not critical points since they are the endpoints of the domain of $f$.)  The functin $f$ must attain its maximum and minimum values in this set $x =\{ \pm 1, \pm \sqrt{2}\}$.  
				We compute
				\begin{align*}
					f(-\sqrt{2}) &= 0 \\
					f(\sqrt{2}) &= 0\\
					f(1) &= 1 \sqrt{2-1} = 1\\
					f(-1) &= -1 \sqrt{2-(-1)^2} = -1
				\end{align*}
				
				That means $f$ has a maximum value of $1$ (at $x= 1$) and a minimum value of -1 (at $x= -1 $) over the interval $[-\sqrt{2}, \sqrt{2}]$.
				\end{freeResponse}
				
				
				
			%part b
			\item  $f(x) = x^3 e^{-x}$ on $[-1,5]$.
\WkstHop
			
				\begin{freeResponse}
				First note that $f$ is continuous on this closed interval so, by the Extreme Value Theorem, it must attain a maximum and minimum value on the interval $[-1,5]$. Those global extrema must occur
				either at critical points or at the boundary points.
				\begin{align*}
					f'(x) &= 3x^2 e^{-x} + x^3(-e^{-x}) \\
					&= x^2 e^{-x} (3-x)
				\end{align*}
				
				Notice that $f'(x)$ always exists so any critical point of must $f$ occur when $f'(x)=0$.  Solving this equation:
				\begin{align*}
					x^2 e^{-x} (3-x) &= 0 \\
					x^2 (3-x) & = 0 \\
				 	x & = 0 \qquad \text{or} \qquad x=3	
				 \end{align*}
				
				Since both critical points are in the given interval, the global extrema must occur at the points $x=-1,0,3,5$.
				\begin{align*}
					f(-1) &= -e \\
					f(0) &= 0 \\
				 	f(3) &= 27e^{-3} \\
				 	f(5) &= 125 e^{-5}
				\end{align*}
				
				Since $-e$ is the only negative value, the minimum value of $f$ over the interval is $-e$.  Since $e^3 < 27$, $27e^{-3} > 1$.  But $e^5 > 125$, and so $125e^{-5} < 1$.  Thus the maximum value of $f$ over the interval is $27e^{-3}$.  
		
				\end{freeResponse}
\WkstNew
				
				
				
			%part c
			\item  $f(x) = x \ln \left( \frac{x}{5} \right)$ on $[0.1, 5]$.
\WkstHop
			
				\begin{freeResponse}
				First note that $f$ is continuous on this closed interval so, by the Extreme Value Theorem, it must attain a maximum and minimum value on the interval $[0.1,5]$.
				\begin{align*}
					f'(x) &= \ln \left( \frac{x}{5} \right) + x \cdot \frac{5}{x} \cdot \frac{1}{5} \\
					&= \ln \left( \frac{x}{5} \right) + 1
				\end{align*}
				
				Notice that $f'(x)$ exists for all values in $[0.1,5]$, and so all of the critical points of $f$ occur when $f'(x)=0$.  Solving this equation:
				$$ \ln \left( \frac{x}{5} \right) + 1 = 0 $$
				$$ \ln \left( \frac{x}{5} \right) = -1 $$
				$$ \frac{x}{5} = e^{-1} $$
				$$ x = 5e^{-1} = \frac{5}{e} $$
				
				Since $1 < \frac{5}{e} < 2$, $\frac{5}{e}$ is in the given interval.  So we need to consider the points $x = \frac{1}{10}, \frac{5}{e}, 5$.
				$$ f \left( \frac{1}{10} \right) = \frac{1}{10} \ln \left( \frac{1}{50} \right)  = \frac{1}{10} \left( \ln 1 - \ln 50 \right) = -\frac{1}{10} \ln 50 $$
				$$ f \left( \frac{5}{e} \right) = \frac{5}{e} \ln \left( \frac{1}{e} \right) = - \frac{5}{e} $$
				$$ f(5) = 5 \ln 1 = 0 $$
				
				Since the first two values are negative, $0$ is the maximum value of $f$ on $[0.1,5]$.  Also, since $e^{10} > 50$, $\ln 50 < 10$ and therefore $-1 < -\frac{1}{10} \ln 50$.  But clearly $\frac{-5}{e} < -1$, and so $- \frac{5}{e}$ is the minimum value of $f$ on $[0.1, 5]$.  
				
				\end{freeResponse}
				
				
				
			\end{enumerate}

		
		
		

\end{problem}
