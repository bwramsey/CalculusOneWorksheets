\documentclass{ximera}

\newcommand{\RR}{\mathbb R}
\renewcommand{\d}{\,d}
\newcommand{\dd}[2][]{\frac{d #1}{d #2}}
\renewcommand{\l}{\ell}
\newcommand{\ddx}{\frac{d}{dx}}
\newcommand{\dfn}{\textbf}
\newcommand{\eval}[1]{\bigg[ #1 \bigg]}
\renewcommand{\theenumii}{\textup{(\roman{enumii})}}
\renewcommand{\labelenumii}{\theenumii}

\usepackage{graphicx}
\usepackage{multicol}
\usepackage{tkz-euclide}
%\usepackage{unicode-math}

\usepackage{pgfplots}   % <- for graphics
\pgfplotsset{compat=newest}


\renewenvironment{freeResponse}{
\ifhandout\setbox0\vbox\bgroup\else
\begin{trivlist}\item[\hskip \labelsep\bfseries Solution:\hspace{2ex}]
\fi}
{\ifhandout\egroup\else
\end{trivlist}
\fi}

\newcommand*{\ZeroOverZero}{\ensuremath{\dfrac{0}{0}}}

\providecommand{\HCCondition}{0}
\newcommand{\WkstHop}[1][1]{\if\HCCondition 0
	\vspace*{\stretch{#1}} \fi} 
\newcommand{\WkstNew}{\if\HCCondition 0
	\newpage
	 \fi} 


\title[Problem 6]{Problem 6}

\begin{document}
\begin{abstract} \end{abstract}
\maketitle


% Extracted from meanValueTheorem.tex, problem #6
\begin{problem}

  Given the four functions on the interval $[1, 5]$, answer the questions below.
  \begin{image}
    \includegraphics[scale = 0.4]{MVTimage002.png}
  \end{image}
  \begin{enumerate}
	\item List the functions that satisfy the hypothesis of the Extreme Value Theorem on $[1,5]$.
	\begin{explanation}
			The Extreme Value Theorem only requires that the function be continuous on $[1,5]$. That means the functions in figures (A) and (C).
		\end{explanation}
	\item For each of the functions, determine if they have a global maximum, a global minimum, both, or neither.
	\begin{explanation}
			The function in figures (A) and (C) satisfy the hypothesis of the Extreme Value Theorem, so they are guaranteed to have both a global
			maximum and a global minimum. The points where these global extrema are attained can be seen from the graph. (A) has a maximum at $x=4$ and minimum at $x=1$. (C) has a maximum at $x=1$ and $x=4$, and minimum around $x=\frac{5}{2}$.
			
			The function in (B) has a maximum at $x=3$, but no global minimum. The function in (D) has a minimum at $x=1$, but no global maximum.
		\end{explanation}
    \item
      List the functions that satisfy  the hypothesis of the Mean Value Theorem on $[1, 5]$.
    \begin{explanation}
        Only the function in figure (C) satisfies the hypothesis of the Mean Value Theorem on $[1,5]$. The function in figure (C) is \textbf{continuous} on $[1,5]$ and \textbf{differentiable} on $(1,5)$.\\
        The function in figure (A) is  \textbf{not differentiable}  at $x=3$, therefore it is  \textbf{not differentiable}  on $(1,5)$.\\
 The functions in figure (B) and (D) are \textbf{not continuous}   on $[1,5]$.
      \end{explanation}
    \item
      List the function (or functions) for which there exists a point $c$ in $(1, 5)$ such that 
      \[
        f'(c) = \frac{f(5) - f(1)}{5 - 1}
      \]
    \begin{explanation}
        Only the graphs (A) and (C). Explanation below.
      Look at these figures.
        \begin{image}
    \includegraphics[scale = 0.4]{MVTimage001.png}
  \end{image}
  In figure(A), the secant line connecting the points $(1,f(1))$ and $(5,f(5))$ is also a tangent line to the curve  at $x$, $1<x<3$.
    
          (A): $\frac{f(5)-f(1)}{5-1}=\frac{1-5}{4}=-1$ and for any $c$ in $(1,5), f'(c)=-1$.\\
          (C):$\frac{f(5)-f(1)}{5-1}$ is the slope of the secant line connecting the points $(1,f(1))$ and $(5,f(5))$. This line is parallel to two tangent lines shown in the figure (C). The lines have the same slope. The slope of either of a  tangent line is $f'(s)$, where is $(s, f(s))$ is the point where the line touches the curve.\\
          (B): The slope of the secant line connecting $(1,f(1))$ and $(5,f(5))$ is negative On the other hand $f'(s)>0$, for $1<s<3$ and $3<s<5$.
          So, $\frac{f(5)-f(1)}{5-1}=\frac{1-5}{4}=-\frac{1}{4}\ne f'(s)$, for all $s$ where $f'(s)$ defined.
           (D): The slope of the secant line connecting $(1,f(1))$ and $(5,f(5))$ is given by $\frac{f(5)-f(1)}{5-1}=\frac{3-1}{4}=\frac{1}{2}$. On the other hand, on the interval $(1,5)$, the graph of the function is a line whose slope is 1.  Therefore, $f'(x)=1$, for $1<x<5$.
         \end{explanation}
  \end{enumerate}
\end{problem}



\end{document}
