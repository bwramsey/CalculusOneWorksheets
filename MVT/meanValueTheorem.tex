%Add code to compile both versions from makefile at same time
\providecommand{\HCCondition}{0}
%Define each of the conditions
\ifcase\HCCondition
	%\condition=0 -> handout
	\documentclass[nooutcomes,noauthor,space,handout]{ximera}
	\title{Extreme and Mean Value Theorems (MVT)}
\or	%\condition=1 -> Soln
	\documentclass[nooutcomes,noauthor]{ximera}
	\title{Extreme and Mean Value Theorems (MVT) - Solutions}
\fi


\newcommand{\RR}{\mathbb R}
\renewcommand{\d}{\,d}
\newcommand{\dd}[2][]{\frac{d #1}{d #2}}
\renewcommand{\l}{\ell}
\newcommand{\ddx}{\frac{d}{dx}}
\newcommand{\dfn}{\textbf}
\newcommand{\eval}[1]{\bigg[ #1 \bigg]}
\renewcommand{\theenumii}{\textup{(\roman{enumii})}}
\renewcommand{\labelenumii}{\theenumii}

\usepackage{graphicx}
\usepackage{multicol}
\usepackage{tkz-euclide}
%\usepackage{unicode-math}

\usepackage{pgfplots}   % <- for graphics
\pgfplotsset{compat=newest}


\renewenvironment{freeResponse}{
\ifhandout\setbox0\vbox\bgroup\else
\begin{trivlist}\item[\hskip \labelsep\bfseries Solution:\hspace{2ex}]
\fi}
{\ifhandout\egroup\else
\end{trivlist}
\fi}

\newcommand*{\ZeroOverZero}{\ensuremath{\dfrac{0}{0}}}

\providecommand{\HCCondition}{0}
\newcommand{\WkstHop}[1][1]{\if\HCCondition 0
	\vspace*{\stretch{#1}} \fi} 
\newcommand{\WkstNew}{\if\HCCondition 0
	\newpage
	 \fi} 
\usepackage{fullpage}




\begin{document}
\begin{abstract}	
	\end{abstract}
\maketitle
\ifcase\HCCondition
%summary in here
\section*{SUMMARY of the Extreme Value Theorem}

A function $f$ has a \textbf{global maximum} at $x=a$, if $f(a)\ge f(x)$, for all $x$ in the domain of $f$.\\[1em]
A function $f$ has a \textbf{global minimum} at $x=a$, if $f(a)\le f(x)$, for all $x$ in the domain of $f$.\\[2em]

\begin{theorem}[Extreme Value Theorem (EVT)]
	If a function $f$ is \textbf{continuous on the closed interval $[a,b]$}, then  $f$ attains  both 
	a  global maximum and a global minimum on  the closed interval $[a,b]$.
\end{theorem}

\textbf{Note: }If a function $f$ has a global extremum (minimum or maximum) at $x=c$, then $c$ is either a boundary point (which means that $c=a$ or $c=b$), or $f$ has a critical point at $x=c$.\\[1em]
So, if we have to find an extreme value of $f$ on $[a,b]$, we should check all the boundary and all the critical points of f and compare the values of $f$ at those points.
 The biggest of those values is the maximum value of $f$ on $[a,b]$, and the smallest one is the minimum value of $f$ on $[a,b]$.\\[2em]

\section*{SUMMARY of The Mean Value Theorem}


\begin{theorem}[The Mean Value Theorem (MVT)]
If a function $f$ is \textbf{continuous on the closed interval $[a,b]$}, and \textbf{differentiable on $(a,b)$}, then there exists a point $c$ in the open interval $(a,b)$ such that
\[ f'(c)=\frac{f(b)-f(a)}{b-a}\]
\end{theorem}
\textbf{Note 1a: }The value $\frac{f(b)-f(a)}{b-a}$ is the slope of the (secant) line through the points $(a,f(a))$ and $(b,f(b))$. \\[1em]
\textbf{Note 1b: }The line tangent to the graph of $f$ at the point where $x=c$ is \textbf{parallel} to the line through the points  $(a,f(a))$ and $(b,f(b))$. \\[1em]
\textbf{Note 2a: }The value $\frac{f(b)-f(a)}{b-a}$ is the average rate of change of $f$ on the interval $[a,b]$. \\[1em]
\textbf{Note 2b: }The instantaneous rate of change $f$ at the point where $x=c$ is equal to the average rate of change of $f$ on $[a,b]$. \\[1em]
 
\WkstNew

\section*{Recitation Questions}
\fi

\begin{problem}
Find the x-coordinates of the points where the  function $f$ has a global max or min. 

	\begin{center}
	\begin{image}
	\includegraphics[trim= 20 430 250 200]{Figure3.pdf}
	\end{image}
	\end{center}	
\WkstHop
	
		\begin{freeResponse}
(a) The function $f$ has a global minimum at $x=a$

(p) The function $f$ has a  absolute maximum/local maximum at $x=p$.  


		\end{freeResponse}	
		
\end{problem}

\WkstNew

\begin{problem}
In each problem, sketch a graph of a function meeting the given critera.
\begin{enumerate}
	%part a
	\item Sketch a possible graph of a function which is continuous on an open interval $(-1,5)$ but does not have an global maximum or minimum.
\WkstHop

	\begin{freeResponse} \hfil
	\begin{image}
	\includegraphics[scale=0.45]{figure5.png}
	\end{image}	
	\end{freeResponse}
	
	%part b
	\item Sketch a possible graph of a function with an global minimum, a local maximum and a local minimum, but no global maximum on the interval $[-4,5]$.
\WkstHop

	\begin{freeResponse} \hfil
	\begin{image}
	\includegraphics[scale=0.45]{figure6.png}
	\end{image}	
	\end{freeResponse}
	
	%part d	
	 \item Sketch a possible graph of a function $f$, continuous on $[1,4]$ with the following properties: $f'(x)=0$ for $x=2$ and $x=3$; $f$ has a global minimum at $x=4$; $f$ has a global maxmimum at $x=3$, and $f$ has a local minimum at $x=2$.
\WkstHop
		\begin{freeResponse} \hfil
	\begin{image}
	\includegraphics[scale=0.45]{figure7.png}
	\end{image}	
	\end{freeResponse} 
	 
	 \end{enumerate}
	 \end{problem}

\WkstNew

\begin{problem}
Find the  x-coordinates of global extrema and global extreme values  of $f$  on the given interval.
		\begin{enumerate}
		
			%part a
			\item  $f(x) = x \sqrt{2-x^2}$ on $[ -\sqrt{2}, \sqrt{2} ]$.
\WkstHop
				
				\begin{freeResponse}
				First note that $f$ is continuous on this closed interval so, by the Extreme Value Theorem, it must attain a maximum and minimum value on the interval $[-\sqrt{2}, \sqrt{2}]$. Those global extrema must occur
				either at critical points or at the boundary points.
				\begin{align*}
				f'(x) &= \sqrt{2-x^2} + x \left( \frac{1}{2 \sqrt{2-x^2}} (-2x) \right) \\
				&= \sqrt{2-x^2} - \frac{x^2}{\sqrt{2-x^2}} \\
				&= \frac{2-x^2-x^2}{\sqrt{2-x^2}} \\
				&= \frac{2(1-x^2)}{\sqrt{2-x^2}}
				\end{align*}
				
				Critical points of $f$ occur where $f'(x) = 0$ or where $f'(x)$ does not exist.  Solving $f'(x) = 0$ yields that $2(1-x^2) = 0$, or $x = \pm 1$.  $f'(x)$ does not exist when $2-x^2 \leq 0$.  But since we are restricting to the interval $[-\sqrt{2}, \sqrt{2}]$, the only points where $f'(x)$ does not exist are the endpoints $\pm \sqrt{2}$.  
				
				The critical points of $f$ in the interval $[-\sqrt{2}, \sqrt{2}]$ are $x = \pm 1$. (The points $x = \pm \sqrt{2}$ technically are not critical points since they are the endpoints of the domain of $f$.)  The functin $f$ must attain its maximum and minimum values in this set $x =\{ \pm 1, \pm \sqrt{2}\}$.  
				We compute
				\begin{align*}
					f(-\sqrt{2}) &= 0 \\
					f(\sqrt{2}) &= 0\\
					f(1) &= 1 \sqrt{2-1} = 1\\
					f(-1) &= -1 \sqrt{2-(-1)^2} = -1
				\end{align*}
				
				That means $f$ has a maximum value of $1$ (at $x= 1$) and a minimum value of -1 (at $x= -1 $) over the interval $[-\sqrt{2}, \sqrt{2}]$.
				\end{freeResponse}
				
				
				
			%part b
			\item  $f(x) = x^3 e^{-x}$ on $[-1,5]$.
\WkstHop
			
				\begin{freeResponse}
				First note that $f$ is continuous on this closed interval so, by the Extreme Value Theorem, it must attain a maximum and minimum value on the interval $[-1,5]$. Those global extrema must occur
				either at critical points or at the boundary points.
				\begin{align*}
					f'(x) &= 3x^2 e^{-x} + x^3(-e^{-x}) \\
					&= x^2 e^{-x} (3-x)
				\end{align*}
				
				Notice that $f'(x)$ always exists so any critical point of must $f$ occur when $f'(x)=0$.  Solving this equation:
				\begin{align*}
					x^2 e^{-x} (3-x) &= 0 \\
					x^2 (3-x) & = 0 \\
				 	x & = 0 \qquad \text{or} \qquad x=3	
				 \end{align*}
				
				Since both critical points are in the given interval, the global extrema must occur at the points $x=-1,0,3,5$.
				\begin{align*}
					f(-1) &= -e \\
					f(0) &= 0 \\
				 	f(3) &= 27e^{-3} \\
				 	f(5) &= 125 e^{-5}
				\end{align*}
				
				Since $-e$ is the only negative value, the minimum value of $f$ over the interval is $-e$.  Since $e^3 < 27$, $27e^{-3} > 1$.  But $e^5 > 125$, and so $125e^{-5} < 1$.  Thus the maximum value of $f$ over the interval is $27e^{-3}$.  
		
				\end{freeResponse}
\WkstNew
				
				
				
			%part c
			\item  $f(x) = x \ln \left( \frac{x}{5} \right)$ on $[0.1, 5]$.
\WkstHop
			
				\begin{freeResponse}
				First note that $f$ is continuous on this closed interval so, by the Extreme Value Theorem, it must attain a maximum and minimum value on the interval $[0.1,5]$.
				\begin{align*}
					f'(x) &= \ln \left( \frac{x}{5} \right) + x \cdot \frac{5}{x} \cdot \frac{1}{5} \\
					&= \ln \left( \frac{x}{5} \right) + 1
				\end{align*}
				
				Notice that $f'(x)$ exists for all values in $[0.1,5]$, and so all of the critical points of $f$ occur when $f'(x)=0$.  Solving this equation:
				$$ \ln \left( \frac{x}{5} \right) + 1 = 0 $$
				$$ \ln \left( \frac{x}{5} \right) = -1 $$
				$$ \frac{x}{5} = e^{-1} $$
				$$ x = 5e^{-1} = \frac{5}{e} $$
				
				Since $1 < \frac{5}{e} < 2$, $\frac{5}{e}$ is in the given interval.  So we need to consider the points $x = \frac{1}{10}, \frac{5}{e}, 5$.
				$$ f \left( \frac{1}{10} \right) = \frac{1}{10} \ln \left( \frac{1}{50} \right)  = \frac{1}{10} \left( \ln 1 - \ln 50 \right) = -\frac{1}{10} \ln 50 $$
				$$ f \left( \frac{5}{e} \right) = \frac{5}{e} \ln \left( \frac{1}{e} \right) = - \frac{5}{e} $$
				$$ f(5) = 5 \ln 1 = 0 $$
				
				Since the first two values are negative, $0$ is the maximum value of $f$ on $[0.1,5]$.  Also, since $e^{10} > 50$, $\ln 50 < 10$ and therefore $-1 < -\frac{1}{10} \ln 50$.  But clearly $\frac{-5}{e} < -1$, and so $- \frac{5}{e}$ is the minimum value of $f$ on $[0.1, 5]$.  
				
				\end{freeResponse}
				
				
				
			\end{enumerate}

		
		
		

\end{problem}
\WkstNew


\begin{problem}
  Verify that the given function satisfies the hypotheses of the Mean Value Theorem in the given interval.
  Then algebraically find all numbers $c$ that satisfy the conclusion of the Mean Value Theorem.
  Using the graph provided, label the point(s) $c$ and sketch the secant line through the points $(1,f(1))$ and $(4,f(4))$ and the tangent line at $c$.
  $$ f(x) = \frac{x}{x+2} \qquad \text{on } [1,4] $$.
  \begin{image}
    \includegraphics[scale=0.45]{Images/Figure3.png}
  \end{image}
\WkstHop
  \begin{freeResponse}
    The function $f(x) = \frac{x}{x+2}$ is continuous on $[1,4]$ and differentiable on $(1,4)$ since it is a rational function, and is therefore continuous and differentiable on its domain.
    Therefore, $f$ satisfies the hypotheses of the Mean Value Theorem.
    We have that
    $$ f'(x) = \frac{(x+2)(1) - x(1)}{(x+2)^2} = \frac{2}{(x+2)^2} $$
    $$ \frac{f(4) - f(1)}{4-1} = \frac{\frac{2}{3} - \frac{1}{3}}{3} = \frac{1}{9} $$
    So we are looking to find all points $c \in (1,4)$ which satisfy that $ f'(c) = \frac{1}{9} $.  So we solve:
    $$ \frac{2}{(c+2)^2} = \frac{1}{9} $$
    $$ (c+2)^2 = 18 $$
    $$ c = \sqrt{18} - 2 = 3\sqrt{2} - 2 $$
    Note that we omitted $-\sqrt{18} - 2$ above because it is not in the interval $(1,4)$.  Therefore, $c = 3\sqrt{2} - 2$.
       \begin{image}
      \includegraphics[scale=0.45]{Images/Figure4.png}
    \end{image}
  \end{freeResponse}
\end{problem}

\WkstNew

\begin{problem}
  A curve is given in the figure below, where $f(x) = 8/x$.
  \begin{image}
     \includegraphics[scale = 0.16]{Images/"Graph of hyperbola".png}
  \end{image}
  \begin{enumerate}

      \item In the figure above, draw a secant line joining the points $A = (1, f(1))$ and $B = (8, f(8))$.
         \begin{freeResponse}
        \begin{image}
     \includegraphics[scale = 0.5]{MVTimage003.png}
  \end{image}
    \end{freeResponse}
     \item
      Find the slope, $m_{\text{sec}}$, of this secant line.
\WkstHop
      \begin{freeResponse}

          $$m_{\text{sec}} = \frac{f(8) - f(1)}{8 - 1}= \frac{1 - 8}{7} = \frac{-7}{7} = -1$$

      \end{freeResponse}

    \item
      Show that the function $f$ satisfies the conditions of the Mean Value Theorem on the interval $[1, 8]$ and find a point (or points) guaranteed to exist by the Mean Value Theorem.
\WkstHop
      \begin{freeResponse}
        $f$ is continuous on the interval $[1, 8]$, $f$ is differentiable on the interval $(1, 8)$.
        By the Mean Value Theorem there exists a(t least one) point $c$ in $(1, 8)$ such that
        \[
          f'(c) = \frac{f(8) - f(1)}{8 - 1} = -1
        \]

        Now $f'(x) = -8/x^2$ hence
        \begin{align*}
          f'(c) = -1 &\iff \frac{-8}{c^2} = -1 \\
                     &\iff c^2 = 8 \\
                     &\iff c= \pm 2\sqrt{2}
        \end{align*}
        Since $c$ must be in $(1, 8)$ we have that $c = 2\sqrt{2}$ is the only point in $(1, 8)$ with $f'(c) = -1$.
          \begin{image}
          \includegraphics[scale = 0.4]{MVTimage004.png}
            \end{image}
      \end{freeResponse}

  \end{enumerate}
\end{problem}

\WkstNew

%problem 1
\begin{problem}

  Given the four functions on the interval $[1, 5]$, answer the questions below.
  \begin{image}
    \includegraphics[scale = 0.4]{MVTimage002.png}
  \end{image}
  \begin{enumerate}
	\item List the functions that satisfy the hypothesis of the Extreme Value Theorem on $[1,5]$.
	\WkstHop
		\begin{freeResponse}
			The Extreme Value Theorem only requires that the function be continuous on $[1,5]$. That means the functions in figures (A) and (C).
		\end{freeResponse}
	\item For each of the functions, determine if they have a global maximum, a global minimum, both, or neither.
	\WkstHop
		\begin{freeResponse}
			The function in figures (A) and (C) satisfy the hypothesis of the Extreme Value Theorem, so they are guaranteed to have both a global
			maximum and a global minimum. The points where these global extrema are attained can be seen from the graph. (A) has a maximum at $x=4$ and minimum at $x=1$. (C) has a maximum at $x=1$ and $x=4$, and minimum around $x=\frac{5}{2}$.
			
			The function in (B) has a maximum at $x=3$, but no global minimum. The function in (D) has a minimum at $x=1$, but no global maximum.
		\end{freeResponse}
    \item
      List the functions that satisfy  the hypothesis of the Mean Value Theorem on $[1, 5]$.
    \WkstHop
      \begin{freeResponse}
        Only the function in figure (C) satisfies the hypothesis of the Mean Value Theorem on $[1,5]$. The function in figure (C) is \textbf{continuous} on $[1,5]$ and \textbf{differentiable} on $(1,5)$.\\
        The function in figure (A) is  \textbf{not differentiable}  at $x=3$, therefore it is  \textbf{not differentiable}  on $(1,5)$.\\
 The functions in figure (B) and (D) are \textbf{not continuous}   on $[1,5]$.
      \end{freeResponse}
    \item
      List the function (or functions) for which there exists a point $c$ in $(1, 5)$ such that 
      \[
        f'(c) = \frac{f(5) - f(1)}{5 - 1}
      \]
    \WkstHop
      \begin{freeResponse}
        Only the graphs (A) and (C). Explanation below.
      Look at these figures.
        \begin{image}
    \includegraphics[scale = 0.4]{MVTimage001.png}
  \end{image}
  In figure(A), the secant line connecting the points $(1,f(1))$ and $(5,f(5))$ is also a tangent line to the curve  at $x$, $1<x<3$.
    
          (A): $\frac{f(5)-f(1)}{5-1}=\frac{1-5}{4}=-1$ and for any $c$ in $(1,5), f'(c)=-1$.\\
          (C):$\frac{f(5)-f(1)}{5-1}$ is the slope of the secant line connecting the points $(1,f(1))$ and $(5,f(5))$. This line is parallel to two tangent lines shown in the figure (C). The lines have the same slope. The slope of either of a  tangent line is $f'(s)$, where is $(s, f(s))$ is the point where the line touches the curve.\\
          (B): The slope of the secant line connecting $(1,f(1))$ and $(5,f(5))$ is negative On the other hand $f'(s)>0$, for $1<s<3$ and $3<s<5$.
          So, $\frac{f(5)-f(1)}{5-1}=\frac{1-5}{4}=-\frac{1}{4}\ne f'(s)$, for all $s$ where $f'(s)$ defined.
           (D): The slope of the secant line connecting $(1,f(1))$ and $(5,f(5))$ is given by $\frac{f(5)-f(1)}{5-1}=\frac{3-1}{4}=\frac{1}{2}$. On the other hand, on the interval $(1,5)$, the graph of the function is a line whose slope is 1.  Therefore, $f'(x)=1$, for $1<x<5$.
         \end{freeResponse}
  \end{enumerate}
\end{problem}

\WkstNew

%problem 2


% problem 3
\begin{problem}
  Heidi drives from her house in Columbus, OH to Indianapolis, IN for vacation.
  Google maps says her driving distance is $170$ miles.
  The drive takes her $2.5$ hours.
  The police send her a speeding ticket in the mail, saying she must have sped to arrive so quickly.
  She is fighting the ticket, saying she just never stopped through the whole drive.
  Can you prove she broke the $65$ mph speed limit at some point during her drive? \textbf{EXPLAIN}.
\WkstHop
  \begin{freeResponse}
    Let us define $s(t)$ to be the position function of Heidi after $t$ hours.
    $s(t)$ will be continuous and differentiable because the position of a car is continuous and differentiable.
    Heidi's average velocity for the whole trip is $\frac{170}{2.5}\approx 68$ mph.
    By the Mean Value Theorem, Heidi 's instantaneous velocity, $s'(t)$, was $68$ mph at some time $t=c$: hence, she broke the $65$ mph speed limit.
  \end{freeResponse}
\end{problem}

\WkstNew

%problem 4
\begin{problem}
  Does the function given in the graph below satisfy the hypotheses of the Mean Value Theorem in the interval $[-1,6]$?
  If so, estimate the values of all numbers $c$ that satisfy the conclusion of the Mean Value Theorem.  
  \begin{image}
    \includegraphics[scale=0.65]{Images/figure1.png}
  \end{image}
\WkstHop
  \begin{freeResponse}
    The function is continuous on $[-1,6]$ and differentiable on $(-1,6)$: \\
    hence it satisfies the hypotheses of the Mean Value Theorem.
    \begin{image}
      \includegraphics[scale=0.6]{Images/Figure2.png}
    \end{image}
  \end{freeResponse}
\end{problem}

\WkstNew


%problem 6
\begin{problem}
  Let $f(x) = (x-3)^{-2}$.
  Show that there is no value $c$ in $(1,4)$ such that $f(4) - f(1) = f^{\prime}(c) (4-1)$.
  Why does this not contradict the Mean Value Theorem?
\WkstHop
  \begin{freeResponse}
    First notice that 
    $$f(4)-f(1) = 1^{-2} - (-2)^{-2} = 1-\frac{1}{4} = \frac{3}{4}$$
    and so we are looking for a value $c$ such that 
    $$3 f^\prime (c) = \frac{3}{4} \quad \Longrightarrow \quad f^\prime (c) = \frac{1}{4} $$
    Then since $f'(x) = \frac{-2}{(x-3)^3}$, we can compute:
    $$ f'(c) = \frac{-2}{(c-3)^3} := \frac{1}{4}$$
    $$ (c-3)^3 = - 8 $$
    $$ c = 3 - 2 = 1 $$
    But $1$ is not in the interval $(1,4)$.  This does not contradict the Mean Value Theorem since $f$ is not continuous at $x=3$.

    \begin{image}
      \includegraphics[scale=.65]{Images/Figure5.png}
    \end{image}
  \end{freeResponse}
\end{problem}


\WkstNew

%problem7
\begin{problem}
  Two runners start a race at the same time and finish in a tie.
  Prove that at some time during the race they have the same speed. 
  (Hint:  Consider the function $h(t)=f(t)-g(t)$, where $f(t)$ and $g(t)$ are the position of the first and second runner at time $t$, respectively.)
\WkstHop
  \begin{freeResponse}
    Let $f(t)$ be the position of the first runner at time $t$, and let $g(t)$ be the position of the second runner at time $t$.
    Let $T$ denote the time that the two runners finish the race (which is the same, since they finish in a tie).
    Also, let $h(t) = f(t) - g(t)$.
    Since $h(0) = 0$ and $h(T) = 0$, by Rolle's Theorem there exists some $c$ with $0 < c < T$ such that $h'(c)=0$.
    But since $h^\prime (t) = f^\prime (t) - g^\prime (t)$, we have that $f^\prime (c) = g^\prime (c)$.
    Therefore, the runners have the same speed at time $c$.  

    \begin{image}
      \includegraphics[scale=.45]{Images/Figure6.png}
    \end{image}
  \end{freeResponse}
\end{problem}

\WkstNew

\begin{problem}
  Verify that the given function satisfies the hypotheses of the Mean Value Theorem in the given interval.
  Then algebraically find all numbers $c$ that satisfy the conclusion of the Mean Value Theorem.
  $$ f(x) = x+\sin{\Bigl(\frac{\pi}{4}\cdot x\Bigr)} \qquad \text{on } [0,2] $$.
\WkstHop

   \begin{freeResponse}
   The function $f$ is continuous on $[0,2]$ and differentiable on $(0,2)$ since it is a sum of a linear polynomial and a composition of a sine function  and another linear polynomial. Therefore, $f$ satisfies the hypotheses of the Mean Value Theorem.
    We have that
    $$ f'(x) = 1+\frac{\pi}{4}\cdot\cos{\Bigl(\frac{\pi}{4}\cdot x\Bigr)}  $$
    $$ \frac{f(2) - f(0)}{2-0} = \frac{3}{2} $$
    So we are looking to find all points $c \in (0,2)$ which satisfy that $ f'(c) = \frac{3}{2} $.  So we solve:
    $$  1+\frac{\pi}{4}\cdot\cos{\Bigl(\frac{\pi}{4}\cdot c\Bigr)} = \frac{3}{2} $$
    Therefore,
    $$  \frac{\pi}{4}\cdot\cos{\Bigl(\frac{\pi}{4}\cdot c\Bigr)} = \frac{1}{2} $$
    and
 $$  \cos{\Bigl(\frac{\pi}{4}\cdot c\Bigr)} = \frac{2}{\pi} $$
  $$ \frac{\pi}{4}\cdot c = \arccos{\Bigl(\frac{2}{\pi}\Bigr) }$$
    $$  c =\frac{4}{\pi}\cdot \arccos{\Bigl(\frac{2}{\pi}\Bigr)\approx 1.12 }$$
    Let's confirm this result with the graph.
     \begin{image}
     \includegraphics{MVTimage005.png}
  \end{image}
     \end{freeResponse}
  \end{problem}
\end{document} 
