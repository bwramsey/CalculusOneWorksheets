\documentclass{ximera}

\newcommand{\RR}{\mathbb R}
\renewcommand{\d}{\,d}
\newcommand{\dd}[2][]{\frac{d #1}{d #2}}
\renewcommand{\l}{\ell}
\newcommand{\ddx}{\frac{d}{dx}}
\newcommand{\dfn}{\textbf}
\newcommand{\eval}[1]{\bigg[ #1 \bigg]}
\renewcommand{\theenumii}{\textup{(\roman{enumii})}}
\renewcommand{\labelenumii}{\theenumii}

\usepackage{graphicx}
\usepackage{multicol}
\usepackage{tkz-euclide}
%\usepackage{unicode-math}

\usepackage{pgfplots}   % <- for graphics
\pgfplotsset{compat=newest}


\renewenvironment{freeResponse}{
\ifhandout\setbox0\vbox\bgroup\else
\begin{trivlist}\item[\hskip \labelsep\bfseries Solution:\hspace{2ex}]
\fi}
{\ifhandout\egroup\else
\end{trivlist}
\fi}

\newcommand*{\ZeroOverZero}{\ensuremath{\dfrac{0}{0}}}

\providecommand{\HCCondition}{0}
\newcommand{\WkstHop}[1][1]{\if\HCCondition 0
	\vspace*{\stretch{#1}} \fi} 
\newcommand{\WkstNew}{\if\HCCondition 0
	\newpage
	 \fi} 


\title[Problem 4]{Problem 4}

\begin{document}
\begin{abstract} \end{abstract}
\maketitle


% Extracted from meanValueTheorem.tex, problem #4
\begin{problem}
  Verify that the given function satisfies the hypotheses of the Mean Value Theorem in the given interval.
  Then algebraically find all numbers $c$ that satisfy the conclusion of the Mean Value Theorem.
  Using the graph provided, label the point(s) $c$ and sketch the secant line through the points $(1,f(1))$ and $(4,f(4))$ and the tangent line at $c$.
  $$ f(x) = \frac{x}{x+2} \qquad \text{on } [1,4] $$.
  \begin{image}
    \includegraphics[scale=0.45]{ifigure3.png}
  \end{image}
\begin{explanation}
    The function $f(x) = \frac{x}{x+2}$ is continuous on $[1,4]$ and differentiable on $(1,4)$ since it is a rational function, and is therefore continuous and differentiable on its domain.
    Therefore, $f$ satisfies the hypotheses of the Mean Value Theorem.
    We have that
    $$ f'(x) = \frac{(x+2)(1) - x(1)}{(x+2)^2} = \frac{2}{(x+2)^2} $$
    $$ \frac{f(4) - f(1)}{4-1} = \frac{\frac{2}{3} - \frac{1}{3}}{3} = \frac{1}{9} $$
    So we are looking to find all points $c \in (1,4)$ which satisfy that $ f'(c) = \frac{1}{9} $.  So we solve:
    $$ \frac{2}{(c+2)^2} = \frac{1}{9} $$
    $$ (c+2)^2 = 18 $$
    $$ c = \sqrt{18} - 2 = 3\sqrt{2} - 2 $$
    Note that we omitted $-\sqrt{18} - 2$ above because it is not in the interval $(1,4)$.  Therefore, $c = 3\sqrt{2} - 2$.
       \begin{image}
      \includegraphics[scale=0.45]{ifigure4.png}
    \end{image}
  \end{explanation}
\end{problem}



\end{document}
