\documentclass{ximera}

\newcommand{\RR}{\mathbb R}
\renewcommand{\d}{\,d}
\newcommand{\dd}[2][]{\frac{d #1}{d #2}}
\renewcommand{\l}{\ell}
\newcommand{\ddx}{\frac{d}{dx}}
\newcommand{\dfn}{\textbf}
\newcommand{\eval}[1]{\bigg[ #1 \bigg]}
\renewcommand{\theenumii}{\textup{(\roman{enumii})}}
\renewcommand{\labelenumii}{\theenumii}

\usepackage{graphicx}
\usepackage{multicol}
\usepackage{tkz-euclide}
%\usepackage{unicode-math}

\usepackage{pgfplots}   % <- for graphics
\pgfplotsset{compat=newest}


\renewenvironment{freeResponse}{
\ifhandout\setbox0\vbox\bgroup\else
\begin{trivlist}\item[\hskip \labelsep\bfseries Solution:\hspace{2ex}]
\fi}
{\ifhandout\egroup\else
\end{trivlist}
\fi}

\newcommand*{\ZeroOverZero}{\ensuremath{\dfrac{0}{0}}}

\providecommand{\HCCondition}{0}
\newcommand{\WkstHop}[1][1]{\if\HCCondition 0
	\vspace*{\stretch{#1}} \fi} 
\newcommand{\WkstNew}{\if\HCCondition 0
	\newpage
	 \fi} 


\title[Problem 4]{Problem 4}

\begin{document}
\begin{abstract} \end{abstract}
\maketitle


% Extracted from antiderivativesAndArea.tex, problem #4
\begin{problem}
  The \dfn{velocity} function for a man walking along a straight road which runs east and west is given by $v(t) = -t^2 + 4t - 3$ ft/min.
  \begin{enumerate}
    
  \item  Set up a definite integral for the man's \dfn{displacement} during the time interval from $2$ minutes to $6$ minutes after he began running.
\begin{explanation}
      \begin{align*}
        \int_2^6 v(t) \d t &= \lim_{n \to \infty} \sum_{k=1}^n v(t_k) \Delta t
      \end{align*}
      Where:  \\	
      $\Delta t = \frac{b-a}{n} = \frac{6-2}{n} = \frac{4}{n}$.
      
      $t_k = a + k \Delta t = 2 + k \frac{4}{n} = 2 + \frac{4k}{n}$.
    \end{explanation}
    
  \item  \dfn{At home:}  Evaluate the definite integral using the limit of a right Riemann sum.
\begin{explanation}
      \begin{align*}
        v(t_k) &= -\left(2 + \frac{4k}{n} \right)^2 + 4 \left( 2 + \frac{4k}{n} \right) - 3  \\
               &= - \left( 4 + \frac{16k}{n} + \frac{16k^2}{n^2} \right) + 8 + \frac{16k}{n} - 3  \\
               &= 1 - \frac{16k^2}{n^2}
      \end{align*}
      
      So we compute:
      \begin{align*}
        \int_2^6 v(t) \d t &= \lim_{n \to \infty} \sum_{k=1}^n \left[ \left( 1 - \frac{16k^2}{n^2} \right) \left( \frac{4}{n} \right) \right]  \\
                           &= \lim_{n \to \infty} \sum_{k=1}^n \left( \frac{4}{n} - \frac{64 k^2}{n^3} \right)  \\
                           &= \lim_{n \to \infty} \left[ \frac{4}{n} \sum_{k=1}^n 1 - \frac{64}{n^3} \sum_{k=1}^n k^2 \right]  \\
                           &= \lim_{n \to \infty} \left[ \frac{4}{n} (n) - \frac{64}{n^3} \left( \frac{n(n+1)(2n+1)}{6} \right) \right]  \\
                           &= 4 - \frac{64}{3} = \frac{12-64}{3} = - \frac{52}{3}.
      \end{align*}
    \end{explanation}
    
  \item  Is this the same as the total \dfn{distance} the man walked from $2$ minutes to $6$ minutes?
    Why or why not?
\begin{explanation}
      This number is not the same as the total distance.
      The man starts his walk by going east (the positive direction) but eventually ends his walk west of where he started.
      
      The total distance that the man walks would be measured by computing 
      $$\int_2^6 \left| v(t) \right| \d t$$  
      \begin{image}
      \includegraphics[scale=.7]{figureaaa3a.png}
            \end{image}
            $\int_2^6 \left| v(t) \right| \d t=\int_2^3 \left| v(t) \right| \d t+\int_3^6 \left| v(t) \right| \d t=\int_2^3  v(t) \d t+\int_3^6(- v(t))  \d t=\int_2^3  v(t) \d t-\int_3^6v(t)  \d t$
    \end{explanation}
  \end{enumerate}
\end{problem}



\end{document}
