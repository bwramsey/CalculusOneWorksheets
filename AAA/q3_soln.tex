\documentclass{ximera}

\newcommand{\RR}{\mathbb R}
\renewcommand{\d}{\,d}
\newcommand{\dd}[2][]{\frac{d #1}{d #2}}
\renewcommand{\l}{\ell}
\newcommand{\ddx}{\frac{d}{dx}}
\newcommand{\dfn}{\textbf}
\newcommand{\eval}[1]{\bigg[ #1 \bigg]}
\renewcommand{\theenumii}{\textup{(\roman{enumii})}}
\renewcommand{\labelenumii}{\theenumii}

\usepackage{graphicx}
\usepackage{multicol}
\usepackage{tkz-euclide}
%\usepackage{unicode-math}

\usepackage{pgfplots}   % <- for graphics
\pgfplotsset{compat=newest}


\renewenvironment{freeResponse}{
\ifhandout\setbox0\vbox\bgroup\else
\begin{trivlist}\item[\hskip \labelsep\bfseries Solution:\hspace{2ex}]
\fi}
{\ifhandout\egroup\else
\end{trivlist}
\fi}

\newcommand*{\ZeroOverZero}{\ensuremath{\dfrac{0}{0}}}

\providecommand{\HCCondition}{0}
\newcommand{\WkstHop}[1][1]{\if\HCCondition 0
	\vspace*{\stretch{#1}} \fi} 
\newcommand{\WkstNew}{\if\HCCondition 0
	\newpage
	 \fi} 


\title[Problem 3]{Problem 3}

\begin{document}
\begin{abstract} \end{abstract}
\maketitle


% Extracted from antiderivativesAndArea.tex, problem #3
\begin{problem}
The velocity of the object moving along a straight line is given by
$v(t)=t-5$, $0\le t\le10$, where $t$ is in seconds and $v$ in ft/s.
\begin{enumerate}
\item Use geometry to evaluate the displacement of the object on the time interval $[0,10]$. 
 Sketch the graph of the velocity and shade the relevant region.
[3]\begin{explanation}
$s(10)-s(0)=\int_{0}^{10} (t-5) \d t=$ area under the curve and the interval $[0,10]$ on the $t-$axis.
Therefore
$s(10)-s(0)=0$ ft.
\begin{image}
\includegraphics[scale=.4]{figureaaa2.png}
\end{image}
\end{explanation}
\item Compute the displacement on the interval $[0,10]$ using antiderivatives of the function $v$.  
\begin{explanation}
    Since
   $ s(t)=\frac{t^2}{2}-5t+s(0)$ ,
   it follows that
    $ s(10)-s(0)=\frac{10^2}{2}-5(10)+s(0)-s(0)=50-50=0$ ft.
    \end{explanation}
    \item Use geometry to evaluate the displacement of the object on the time interval $[0,8]$. 
 Sketch the graph of the velocity and shade the relevant region.
[3]\begin{explanation}
\begin{image}
\includegraphics[scale=.4]{figureaaa3.png}
\end{image}
$s(8)-s(0)=\int_{0}^{8} (t-5) \d t=$ area under the curve and the interval $[0,8]$ on the $t-$axis.
Therefore
$s(8)-s(0)=-\frac{25}{2}+\frac{9}{2}=-8$ ft.

\end{explanation}
\item Compute the displacement on the interval $[0,8]$ using antiderivatives of the function $v$.  
\begin{explanation}
    Since
   $ s(t)=\frac{t^2}{2}-5t+s(0)$,
   it follows that
   
    $ s(8)-s(0)=\frac{8^2}{2}-5(8)+s(0)-s(0)=32-40=-8$ ft.
    \end{explanation}
    \end{enumerate}
\end{problem}



\end{document}
