% Extracted from antiderivativesAndArea.tex, problem #2
\begin{problem}
The velocity of an object moving along a straight line is given by $v(t)$ (in ft/s) and  we only know the following:
$ \int_0^3 v(t)\d t = -3$, and 		$\int_3^4 v(t)\d t = 5$.

Compute the following displacements, if possible.  If it is not possible, give examples explaining why not.
	\begin{enumerate}
	
	%part a 
	\item  $s(3)-s(0)$
\WkstHop		\begin{freeResponse}
		\[
			s(3)-s(0)=\int_0^3 v(t)\d t  =-3	ft		
		\]
		\end{freeResponse}
		
			%part b
	\item  $s(4)-s(0) $
\WkstHop		\begin{freeResponse}
		First notice that
			\begin{equation*}
			s(4)-s(0)= \int_0^4 v(t) \d t 
			\end{equation*}
			Therefore,
			\begin{equation*}
			s(4)-s(0)=\int_0^3v(t) \d t + \int_3^4 v(t) \d t  =-3+5=2 ft
			\end{equation*}
			
		
			%\begin{align*}
			%\int_1^2 (-f(x)) \d x &= - \int_1^2 f(x) \d x  \\
			%&= - \left( \int_1^4 f(x)\d x + \int_4^2 f(x)\d x \right)  \\
			%&= - \left( \int_1^4 f(x)\d x - \int_2^4 f(x)\d x \right)  \\
			%&= - (7 - 5) = -2
			%\end{align*}
		\end{freeResponse}
		
		
		
	%part c
	\item Find the displacement during the interval $[0,4]$ if the velocity  at the time $t$,  $0\le t\le4$, was $v(t)+2$ ft/s instead?
\WkstHop		\begin{freeResponse}
		In that case, we would have that
		
				$s(4)-s(0)=\int_0^4(v(t)+2) \d t =\int_0^4v(t) \d t+\int_0^42\d t$
				
				We use the result in part (b) for the first integral, and geometry for the second interval.
				
				$s(4)-s(0) =2+2(4-0)=2+8=10$ ft
			\end{freeResponse}
		
	%part d
	\item Find the displacement during the interval $[0,4]$ if the velocity  at the time $t$, $0\le t\le 4$, was $5v(t)$ ft/s instead?
\WkstHop			\begin{freeResponse}
		$s(4)-s(0)=\int_0^4 5v(t) \d t =5\int_0^4v(t) \d t=5(2)=10$ ft

		\end{freeResponse}
	\end{enumerate}
\end{problem}
