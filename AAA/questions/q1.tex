% Extracted from antiderivativesAndArea.tex, problem #1
\begin{problem}
The graph of the velocity (t in min, v in ft/min) of a particle moving along a straight line is given in the figure. 
Assume that the particle was at the origin initially.
\begin{image}
\includegraphics[scale=.4]{figureAAA4.png}
\end{image}
\begin{enumerate}
\item Evaluate the displacement of the particle over the following intervals.
\begin{enumerate}
\item $[0,10]$
\WkstHop\begin{freeResponse}
		\[
			s(10)-s(0)=\int_0^{10} v(t)\d t= 3	ft		
		\]
		\end{freeResponse}
		\item $[0,7]$
\WkstHop\begin{freeResponse}
		\[
			s(7)-s(0)= -3	ft		
		\]
		\end{freeResponse}
		\item $[0,5]$
\WkstHop\begin{freeResponse}
		\[
			s(5)-s(0)= -5	ft		
		\]
		\end{freeResponse}
\end{enumerate}
\item  When was the particle farthest from the origin?
\WkstHop[3]\begin{freeResponse}
			
		We check the end points: $s(0)=0$, $s(10)=s(0)+3=3$, and critical points. Since  the only critical point of $s$ is at  $t=5$, since $s'(5)=v(5)=0$ there, we evaluate $s(5)=s(0)-5=-5$. It follows that the particle was farthest from the origin at $t=5$ min.
		\end{freeResponse}
		\item  What was  the total distance the particle has travelled over the time interval $[0,10]$?
\WkstHop
		\begin{freeResponse}The total distance travelled is given by the definite integral
		\[
			\int_0^{10}| v(t)|\d t=\int_0^{5}(- v(t))\d t+  \int_5^{10} v(t)\d t=5+8=13  ft		
		\]
				\end{freeResponse}
\end{enumerate}
\end{problem}
