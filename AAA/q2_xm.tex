\documentclass{ximera}

\newcommand{\RR}{\mathbb R}
\renewcommand{\d}{\,d}
\newcommand{\dd}[2][]{\frac{d #1}{d #2}}
\renewcommand{\l}{\ell}
\newcommand{\ddx}{\frac{d}{dx}}
\newcommand{\dfn}{\textbf}
\newcommand{\eval}[1]{\bigg[ #1 \bigg]}
\renewcommand{\theenumii}{\textup{(\roman{enumii})}}
\renewcommand{\labelenumii}{\theenumii}

\usepackage{graphicx}
\usepackage{multicol}
\usepackage{tkz-euclide}
%\usepackage{unicode-math}

\usepackage{pgfplots}   % <- for graphics
\pgfplotsset{compat=newest}


\renewenvironment{freeResponse}{
\ifhandout\setbox0\vbox\bgroup\else
\begin{trivlist}\item[\hskip \labelsep\bfseries Solution:\hspace{2ex}]
\fi}
{\ifhandout\egroup\else
\end{trivlist}
\fi}

\newcommand*{\ZeroOverZero}{\ensuremath{\dfrac{0}{0}}}

\providecommand{\HCCondition}{0}
\newcommand{\WkstHop}[1][1]{\if\HCCondition 0
	\vspace*{\stretch{#1}} \fi} 
\newcommand{\WkstNew}{\if\HCCondition 0
	\newpage
	 \fi} 

\title[Problem 2]{Problem 2}

\begin{document}
\begin{abstract} \end{abstract}
\maketitle

% Extracted from antiderivativesAndArea.tex, problem #2
\begin{problem}
The velocity of an object moving along a straight line is given by $v(t)$ (in ft/s) and  we only know the following:
$ \int_0^3 v(t)\d t = -3$, and 		$\int_3^4 v(t)\d t = 5$.

Compute the following displacements, if possible.  If it is not possible, give examples explaining why not.
	\begin{enumerate}
	
	%part a 
	\item  $s(3)-s(0)$
%part b
	\item  $s(4)-s(0) $
%part c
	\item Find the displacement during the interval $[0,4]$ if the velocity  at the time $t$,  $0\le t\le4$, was $v(t)+2$ ft/s instead?
%part d
	\item Find the displacement during the interval $[0,4]$ if the velocity  at the time $t$, $0\le t\le 4$, was $5v(t)$ ft/s instead?
\end{enumerate}
\end{problem}

\end{document}
