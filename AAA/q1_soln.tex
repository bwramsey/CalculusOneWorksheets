\documentclass{ximera}

\newcommand{\RR}{\mathbb R}
\renewcommand{\d}{\,d}
\newcommand{\dd}[2][]{\frac{d #1}{d #2}}
\renewcommand{\l}{\ell}
\newcommand{\ddx}{\frac{d}{dx}}
\newcommand{\dfn}{\textbf}
\newcommand{\eval}[1]{\bigg[ #1 \bigg]}
\renewcommand{\theenumii}{\textup{(\roman{enumii})}}
\renewcommand{\labelenumii}{\theenumii}

\usepackage{graphicx}
\usepackage{multicol}
\usepackage{tkz-euclide}
%\usepackage{unicode-math}

\usepackage{pgfplots}   % <- for graphics
\pgfplotsset{compat=newest}


\renewenvironment{freeResponse}{
\ifhandout\setbox0\vbox\bgroup\else
\begin{trivlist}\item[\hskip \labelsep\bfseries Solution:\hspace{2ex}]
\fi}
{\ifhandout\egroup\else
\end{trivlist}
\fi}

\newcommand*{\ZeroOverZero}{\ensuremath{\dfrac{0}{0}}}

\providecommand{\HCCondition}{0}
\newcommand{\WkstHop}[1][1]{\if\HCCondition 0
	\vspace*{\stretch{#1}} \fi} 
\newcommand{\WkstNew}{\if\HCCondition 0
	\newpage
	 \fi} 


\title[Problem 1]{Problem 1}

\begin{document}
\begin{abstract} \end{abstract}
\maketitle


% Extracted from antiderivativesAndArea.tex, problem #1
\begin{problem}
The graph of the velocity (t in min, v in ft/min) of a particle moving along a straight line is given in the figure. 
Assume that the particle was at the origin initially.
\begin{image}
\includegraphics[scale=.4]{figureAAA4.png}
\end{image}
\begin{enumerate}
\item Evaluate the displacement of the particle over the following intervals.
\begin{enumerate}
\item $[0,10]$
\begin{explanation}
		\[
			s(10)-s(0)=\int_0^{10} v(t)\d t= 3	ft		
		\]
		\end{explanation}
		\item $[0,7]$
\begin{explanation}
		\[
			s(7)-s(0)= -3	ft		
		\]
		\end{explanation}
		\item $[0,5]$
\begin{explanation}
		\[
			s(5)-s(0)= -5	ft		
		\]
		\end{explanation}
\end{enumerate}
\item  When was the particle farthest from the origin?
[3]\begin{explanation}
			
		We check the end points: $s(0)=0$, $s(10)=s(0)+3=3$, and critical points. Since  the only critical point of $s$ is at  $t=5$, since $s'(5)=v(5)=0$ there, we evaluate $s(5)=s(0)-5=-5$. It follows that the particle was farthest from the origin at $t=5$ min.
		\end{explanation}
		\item  What was  the total distance the particle has travelled over the time interval $[0,10]$?
\begin{explanation}The total distance travelled is given by the definite integral
		\[
			\int_0^{10}| v(t)|\d t=\int_0^{5}(- v(t))\d t+  \int_5^{10} v(t)\d t=5+8=13  ft		
		\]
				\end{explanation}
\end{enumerate}
\end{problem}



\end{document}
