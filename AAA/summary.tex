\documentclass{ximera}

\newcommand{\RR}{\mathbb R}
\renewcommand{\d}{\,d}
\newcommand{\dd}[2][]{\frac{d #1}{d #2}}
\renewcommand{\l}{\ell}
\newcommand{\ddx}{\frac{d}{dx}}
\newcommand{\dfn}{\textbf}
\newcommand{\eval}[1]{\bigg[ #1 \bigg]}
\renewcommand{\theenumii}{\textup{(\roman{enumii})}}
\renewcommand{\labelenumii}{\theenumii}

\usepackage{graphicx}
\usepackage{multicol}
\usepackage{tkz-euclide}
%\usepackage{unicode-math}

\usepackage{pgfplots}   % <- for graphics
\pgfplotsset{compat=newest}


\renewenvironment{freeResponse}{
\ifhandout\setbox0\vbox\bgroup\else
\begin{trivlist}\item[\hskip \labelsep\bfseries Solution:\hspace{2ex}]
\fi}
{\ifhandout\egroup\else
\end{trivlist}
\fi}

\newcommand*{\ZeroOverZero}{\ensuremath{\dfrac{0}{0}}}

\providecommand{\HCCondition}{0}
\newcommand{\WkstHop}[1][1]{\if\HCCondition 0
	\vspace*{\stretch{#1}} \fi} 
\newcommand{\WkstNew}{\if\HCCondition 0
	\newpage
	 \fi} 

\title[Summary]{Summary}

\begin{document}
\begin{abstract} \end{abstract}
\maketitle

We know, from the Antiderivatives section, that \\
(1)  the position function, s(t), of the object moving along a straight line
 with velocity $v(t)$ is an antiderivative of the velocity, i.e.

\[ \int v(t)\d t=s(t) +C\]

(2) the displacement of the object over the time interval $[a,b]$ is given by \\
\[ \text{Displacement} = \Delta s = s(b)- s(a)\]

 In this section, we learned that \\
 (3) the displacement of the object over the time interval $[a,b]$ can be represented \\[0.5em]as the area under the curve $y=v(t)$, (if $v(t)\ge 0$ on $[a,b]$), which means that
\[ \text{Displacement} = \int_{a}^{b} v(t)\d t\]

(4) Combining the last two equations, we get 
\[s(b)-s(a)= \int_{a}^{b} v(t)\d t\]
 \textbf{Note: } The last formula states that a definite integral of a \emph{velocity} function can be computed by using an antiderivative of the function and evaluating it at the and points of the interval.
 This the first time that we make a connection between definite integrals and antiderivatives.

\end{document}
