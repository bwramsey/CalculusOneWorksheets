%Add code to compile both versions from makefile at same time
\providecommand{\HCCondition}{0}
%Define each of the conditions
\ifcase\HCCondition
	%\condition=0 -> handout
	\documentclass[nooutcomes,noauthor,space,handout]{ximera}
	\title{Antiderivatives and area (AAA)}
\or	%\condition=1 -> Soln
	\documentclass[nooutcomes,noauthor]{ximera}
	\title{Antiderivatives and area (AAA) - Solutions}  
\fi

\usepackage{fullpage}
\newcommand{\RR}{\mathbb R}
\renewcommand{\d}{\,d}
\newcommand{\dd}[2][]{\frac{d #1}{d #2}}
\renewcommand{\l}{\ell}
\newcommand{\ddx}{\frac{d}{dx}}
\newcommand{\dfn}{\textbf}
\newcommand{\eval}[1]{\bigg[ #1 \bigg]}
\renewcommand{\theenumii}{\textup{(\roman{enumii})}}
\renewcommand{\labelenumii}{\theenumii}

\usepackage{graphicx}
\usepackage{multicol}
\usepackage{tkz-euclide}
%\usepackage{unicode-math}

\usepackage{pgfplots}   % <- for graphics
\pgfplotsset{compat=newest}


\renewenvironment{freeResponse}{
\ifhandout\setbox0\vbox\bgroup\else
\begin{trivlist}\item[\hskip \labelsep\bfseries Solution:\hspace{2ex}]
\fi}
{\ifhandout\egroup\else
\end{trivlist}
\fi}

\newcommand*{\ZeroOverZero}{\ensuremath{\dfrac{0}{0}}}

\providecommand{\HCCondition}{0}
\newcommand{\WkstHop}[1][1]{\if\HCCondition 0
	\vspace*{\stretch{#1}} \fi} 
\newcommand{\WkstNew}{\if\HCCondition 0
	\newpage
	 \fi}  


\begin{document}
\begin{abstract}

\end{abstract}
\maketitle

\ifcase\HCCondition
\section*{SUMMARY: Antiderivatives and Area}
We know, from the Antiderivatives section, that \\
(1)  the position function, s(t), of the object moving along a straight line
 with velocity $v(t)$ is an antiderivative of the velocity, i.e.

\[ \int v(t)\d t=s(t) +C\]

(2) the displacement of the object over the time interval $[a,b]$ is given by \\
\[ \text{Displacement} = \Delta s = s(b)- s(a)\]

 In this section, we learned that \\
 (3) the displacement of the object over the time interval $[a,b]$ can be represented \\[0.5em]as the area under the curve $y=v(t)$, (if $v(t)\ge 0$ on $[a,b]$), which means that
\[ \text{Displacement} = \int_{a}^{b} v(t)\d t\]

(4) Combining the last two equations, we get 
\[s(b)-s(a)= \int_{a}^{b} v(t)\d t\]
 \textbf{Note: } The last formula states that a definite integral of a \emph{velocity} function can be computed by using an antiderivative of the function and evaluating it at the and points of the interval.
 This the first time that we make a connection between definite integrals and antiderivatives.
\newpage

\section*{Recitation Questions}
\fi


%problem 1
\begin{problem}
The graph of the velocity (t in min, v in ft/min) of a particle moving along a straight line is given in the figure. 
Assume that the particle was at the origin initially.
\begin{image}
\includegraphics[scale=.4]{figureAAA4.png}
\end{image}
\begin{enumerate}
\item Evaluate the displacement of the particle over the following intervals.
\begin{enumerate}
\item $[0,10]$
\WkstHop\begin{freeResponse}
		\[
			s(10)-s(0)=\int_0^{10} v(t)\d t= 3	ft		
		\]
		\end{freeResponse}
		\item $[0,7]$
\WkstHop\begin{freeResponse}
		\[
			s(7)-s(0)= -3	ft		
		\]
		\end{freeResponse}
		\item $[0,5]$
\WkstHop\begin{freeResponse}
		\[
			s(5)-s(0)= -5	ft		
		\]
		\end{freeResponse}
\end{enumerate}
\item  When was the particle farthest from the origin?
\WkstHop[3]\begin{freeResponse}
			
		We check the end points: $s(0)=0$, $s(10)=s(0)+3=3$, and critical points. Since  the only critical point of $s$ is at  $t=5$, since $s'(5)=v(5)=0$ there, we evaluate $s(5)=s(0)-5=-5$. It follows that the particle was farthest from the origin at $t=5$ min.
		\end{freeResponse}
		\item  What was  the total distance the particle has travelled over the time interval $[0,10]$?
\WkstHop
		\begin{freeResponse}The total distance travelled is given by the definite integral
		\[
			\int_0^{10}| v(t)|\d t=\int_0^{5}(- v(t))\d t+  \int_5^{10} v(t)\d t=5+8=13  ft		
		\]
				\end{freeResponse}
\end{enumerate}
\end{problem}
%problem 2

\WkstNew

%problem 3
\begin{problem}
The velocity of an object moving along a straight line is given by $v(t)$ (in ft/s) and  we only know the following:
$ \int_0^3 v(t)\d t = -3$, and 		$\int_3^4 v(t)\d t = 5$.

Compute the following displacements, if possible.  If it is not possible, give examples explaining why not.
	\begin{enumerate}
	
	%part a 
	\item  $s(3)-s(0)$
\WkstHop		\begin{freeResponse}
		\[
			s(3)-s(0)=\int_0^3 v(t)\d t  =-3	ft		
		\]
		\end{freeResponse}
		
			%part b
	\item  $s(4)-s(0) $
\WkstHop		\begin{freeResponse}
		First notice that
			\begin{equation*}
			s(4)-s(0)= \int_0^4 v(t) \d t 
			\end{equation*}
			Therefore,
			\begin{equation*}
			s(4)-s(0)=\int_0^3v(t) \d t + \int_3^4 v(t) \d t  =-3+5=2 ft
			\end{equation*}
			
		
			%\begin{align*}
			%\int_1^2 (-f(x)) \d x &= - \int_1^2 f(x) \d x  \\
			%&= - \left( \int_1^4 f(x)\d x + \int_4^2 f(x)\d x \right)  \\
			%&= - \left( \int_1^4 f(x)\d x - \int_2^4 f(x)\d x \right)  \\
			%&= - (7 - 5) = -2
			%\end{align*}
		\end{freeResponse}
		
		
		
	%part c
	\item Find the displacement during the interval $[0,4]$ if the velocity  at the time $t$,  $0\le t\le4$, was $v(t)+2$ ft/s instead?
\WkstHop		\begin{freeResponse}
		In that case, we would have that
		
				$s(4)-s(0)=\int_0^4(v(t)+2) \d t =\int_0^4v(t) \d t+\int_0^42\d t$
				
				We use the result in part (b) for the first integral, and geometry for the second interval.
				
				$s(4)-s(0) =2+2(4-0)=2+8=10$ ft
			\end{freeResponse}
		
	%part d
	\item Find the displacement during the interval $[0,4]$ if the velocity  at the time $t$, $0\le t\le 4$, was $5v(t)$ ft/s instead?
\WkstHop			\begin{freeResponse}
		$s(4)-s(0)=\int_0^4 5v(t) \d t =5\int_0^4v(t) \d t=5(2)=10$ ft

		\end{freeResponse}
	\end{enumerate}
\end{problem}

%problem 4
\WkstNew

%problem 5


%problem 6
\begin{problem}
The velocity of the object moving along a straight line is given by
$v(t)=t-5$, $0\le t\le10$, where $t$ is in seconds and $v$ in ft/s.
\begin{enumerate}
\item Use geometry to evaluate the displacement of the object on the time interval $[0,10]$. 
 Sketch the graph of the velocity and shade the relevant region.
\WkstHop[3]\begin{freeResponse}
$s(10)-s(0)=\int_{0}^{10} (t-5) \d t=$ area under the curve and the interval $[0,10]$ on the $t-$axis.
Therefore
$s(10)-s(0)=0$ ft.
\begin{image}
\includegraphics[scale=.4]{figureAAA2.png}
\end{image}
\end{freeResponse}
\item Compute the displacement on the interval $[0,10]$ using antiderivatives of the function $v$.  
\WkstHop    \begin{freeResponse}
    Since
   $ s(t)=\frac{t^2}{2}-5t+s(0)$ ,
   it follows that
    $ s(10)-s(0)=\frac{10^2}{2}-5(10)+s(0)-s(0)=50-50=0$ ft.
    \end{freeResponse}
    \item Use geometry to evaluate the displacement of the object on the time interval $[0,8]$. 
 Sketch the graph of the velocity and shade the relevant region.
\WkstHop[3]\begin{freeResponse}
\begin{image}
\includegraphics[scale=.4]{figureAAA3.png}
\end{image}
$s(8)-s(0)=\int_{0}^{8} (t-5) \d t=$ area under the curve and the interval $[0,8]$ on the $t-$axis.
Therefore
$s(8)-s(0)=-\frac{25}{2}+\frac{9}{2}=-8$ ft.

\end{freeResponse}
\item Compute the displacement on the interval $[0,8]$ using antiderivatives of the function $v$.  
\WkstHop    \begin{freeResponse}
    Since
   $ s(t)=\frac{t^2}{2}-5t+s(0)$,
   it follows that
   
    $ s(8)-s(0)=\frac{8^2}{2}-5(8)+s(0)-s(0)=32-40=-8$ ft.
    \end{freeResponse}
    \end{enumerate}
\end{problem}

\WkstNew

%problem 2
\begin{problem}
  The \dfn{velocity} function for a man walking along a straight road which runs east and west is given by $v(t) = -t^2 + 4t - 3$ ft/min.
  \begin{enumerate}
    
  \item  Set up a definite integral for the man's \dfn{displacement} during the time interval from $2$ minutes to $6$ minutes after he began running.
\WkstHop    \begin{freeResponse}
      \begin{align*}
        \int_2^6 v(t) \d t &= \lim_{n \to \infty} \sum_{k=1}^n v(t_k) \Delta t
      \end{align*}
      Where:  \\	
      $\Delta t = \frac{b-a}{n} = \frac{6-2}{n} = \frac{4}{n}$.
      
      $t_k = a + k \Delta t = 2 + k \frac{4}{n} = 2 + \frac{4k}{n}$.
    \end{freeResponse}
    
  \item  \dfn{At home:}  Evaluate the definite integral using the limit of a right Riemann sum.
\WkstHop
    \begin{freeResponse}
      \begin{align*}
        v(t_k) &= -\left(2 + \frac{4k}{n} \right)^2 + 4 \left( 2 + \frac{4k}{n} \right) - 3  \\
               &= - \left( 4 + \frac{16k}{n} + \frac{16k^2}{n^2} \right) + 8 + \frac{16k}{n} - 3  \\
               &= 1 - \frac{16k^2}{n^2}
      \end{align*}
      
      So we compute:
      \begin{align*}
        \int_2^6 v(t) \d t &= \lim_{n \to \infty} \sum_{k=1}^n \left[ \left( 1 - \frac{16k^2}{n^2} \right) \left( \frac{4}{n} \right) \right]  \\
                           &= \lim_{n \to \infty} \sum_{k=1}^n \left( \frac{4}{n} - \frac{64 k^2}{n^3} \right)  \\
                           &= \lim_{n \to \infty} \left[ \frac{4}{n} \sum_{k=1}^n 1 - \frac{64}{n^3} \sum_{k=1}^n k^2 \right]  \\
                           &= \lim_{n \to \infty} \left[ \frac{4}{n} (n) - \frac{64}{n^3} \left( \frac{n(n+1)(2n+1)}{6} \right) \right]  \\
                           &= 4 - \frac{64}{3} = \frac{12-64}{3} = - \frac{52}{3}.
      \end{align*}
    \end{freeResponse}
    
  \item  Is this the same as the total \dfn{distance} the man walked from $2$ minutes to $6$ minutes?
    Why or why not?
\WkstHop
    \begin{freeResponse}
      This number is not the same as the total distance.
      The man starts his walk by going east (the positive direction) but eventually ends his walk west of where he started.
      
      The total distance that the man walks would be measured by computing 
      $$\int_2^6 \left| v(t) \right| \d t$$  
      \begin{image}
      \includegraphics[scale=.7]{figureAAA3a.png}
            \end{image}
            $\int_2^6 \left| v(t) \right| \d t=\int_2^3 \left| v(t) \right| \d t+\int_3^6 \left| v(t) \right| \d t=\int_2^3  v(t) \d t+\int_3^6(- v(t))  \d t=\int_2^3  v(t) \d t-\int_3^6v(t)  \d t$
    \end{freeResponse}
  \end{enumerate}
\end{problem}

\WkstNew

%problem 4



\end{document} 
\begin{problem}
Snow is starting to fall with a rate at any time $t$ after the start being 
$$ f'(t) = \frac{3}{2} t - \frac{1}{4} t^2 + \frac{3}{10} $$
inches per hour for $t$ in $[0,4]$ (i.e., the snow falls for 4 hours - from noon until 4pm).  
There were already $5$ inches of snow on the ground when the storm started.  
	\begin{enumerate}
	
	%part a 
	\item  Use the formula for a right Riemann sum to estimate how much snow fell during the storm using $n$ rectangles.
\WkstHop
		\begin{freeResponse}
		$\Delta x = \frac{b-a}{n} = \frac{4-0}{n} = \frac{4}{n}$.
		
		$x_i = a + i \Delta x = 0 + i \frac{4}{n} = \frac{4i}{n}$.
			\begin{align*}
			f'(x_i) = f' \left( \frac{4i}{n} \right) &= \frac{3}{10} + \frac{3}{2} \left( \frac{4i}{n} \right) - \frac{1}{4} \left( \frac{4i}{n} \right)^2  \\
			&= \frac{3}{10} + \frac{6i}{n} - \frac{4i^2}{n^2}
			\end{align*}
			
		So our approximate area is:
			\begin{align*}
			\sum_{i=1}^n f'(x_i) \Delta x &= \sum_{i=1}^n \left[ \left( \frac{3}{10} + \frac{6i}{n} - \frac{4i^2}{n^2} \right) \left( \frac{4}{n} \right) \right]  \\
			&= \frac{4}{n} \sum_{i=1}^n \left( \frac{3}{10} + \frac{6i}{n} - \frac{4i^2}{n^2} \right)  \\
			&= \frac{4}{n} \sum_{i=1}^n \left( \frac{3}{10} \right) + \frac{4}{n} \sum_{i=1}^n \left( \frac{6i}{n} \right) - \frac{4}{n} \sum_{i=1}^n \left( \frac{4i^2}{n^2} \right)  \\
			&= \frac{6}{5n} \sum_{i=1}^n 1 + \frac{24}{n^2} \sum_{i=1}^n i - \frac{16}{n^3} \sum_{i=1}^n i^2  \\
			&= \frac{6}{5n} (n) + \frac{24}{n^2} \left( \frac{n(n+1)}{2} \right) - \frac{16}{n^3} \left( \frac{n(n+1)(2n+1)}{6} \right)  \\
			&= \frac{6}{5} + \frac{12(n+1)}{n} - \frac{8(n+1)(2n+1)}{3n^2}.
			\end{align*}
		\end{freeResponse}
		
		
		
	%part b
	\item  Take the limit as $n$ goes to infinity to find the exact amount of snow that fell.
\WkstHop
		\begin{freeResponse}
			\begin{align*}
			&  \lim_{n \to \infty} \left( \frac{6}{5} + \frac{12(n+1)}{n} - \frac{8(n+1)(2n+1)}{3n^2} \right)  \\
			&= \lim_{n \to \infty} \left( \frac{6}{5} + 12 \left( 1 + \frac{1}{n} \right) - \frac{8(1 + \frac{1}{n})(2 + \frac{1}{n})}{3} \right)  \\
			&= \frac{6}{5} + 12(1 + 0) - \frac{8(1+0)(2+0)}{3}  \\
			&= \frac{6}{5} + 12 - \frac{16}{3} = \frac{18 + 180 - 80}{15} = \frac{118}{15}.
			\end{align*}
		\end{freeResponse}
		
		
		
	\end{enumerate}
	
\end{problem}

\WkstNew

\begin{problem}
  The \dfn{velocity} function for a man walking along a straight road which runs east and west is given by $v(t) = -t^2 + 4t - 3$ ft/min.
  \begin{enumerate}
    
  \item  Set up a definite integral for the man's \dfn{displacement} during the time interval from $2$ minutes to $6$ minutes after he began running.
\WkstHop
    \begin{freeResponse}
      \begin{align*}
        \int_2^6 v(t) \d t &= \lim_{n \to \infty} \sum_{i=1}^n v(t_i) \Delta t
      \end{align*}
      Where:  \\	
      $\Delta t = \frac{b-a}{n} = \frac{6-2}{n} = \frac{4}{n}$.
      
      $t_i = a + i \Delta t = 2 + i \frac{4}{n} = 2 + \frac{4i}{n}$.
    \end{freeResponse}
    
  \item  \dfn{At home:}  Evaluate the definite integral using the limit of a right Riemann sum.
\WkstHop
    \begin{freeResponse}
      \begin{align*}
        v(t_i) &= -\left(2 + \frac{4i}{n} \right)^2 + 4 \left( 2 + \frac{4i}{n} \right) - 3  \\
               &= - \left( 4 + \frac{16i}{n} + \frac{16i^2}{n^2} \right) + 8 + \frac{16i}{n} - 3  \\
               &= 1 - \frac{16i^2}{n^2}
      \end{align*}
      
      So we compute:
      \begin{align*}
        \int_2^6 v(t) \d t &= \lim_{n \to \infty} \sum_{i=1}^n \left[ \left( 1 - \frac{16i^2}{n^2} \right) \left( \frac{4}{n} \right) \right]  \\
                           &= \lim_{n \to \infty} \sum_{i=1}^n \left( \frac{4}{n} - \frac{64 i^2}{n^3} \right)  \\
                           &= \lim_{n \to \infty} \left[ \frac{4}{n} \sum_{i=1}^n 1 - \frac{64}{n^3} \sum_{i=1}^n i^2 \right]  \\
                           &= \lim_{n \to \infty} \left[ \frac{4}{n} (n) - \frac{64}{n^3} \left( \frac{n(n+1)(2n+1)}{6} \right) \right]  \\
                           &= 4 - \frac{64}{3} = \frac{12-64}{3} = - \frac{52}{3}.
      \end{align*}
    \end{freeResponse}
    
  \item  Is this the same as the total \dfn{distance} the man walked from $2$ minutes to $6$ minutes?
    Why or why not?
\WkstHop
    \begin{freeResponse}
      This number is not the same as the total distance.
      The man starts his walk by going east (the positive direction) but eventually ends his walk west of where he started.
      
      The total distance that the man walks would be measured by computing 
      $$\int_2^6 \left| v(t) \right| \d t$$  
      \begin{image}
      \includegraphics[scale=.7]{figure1.png}
            \end{image}
    \end{freeResponse}
  \end{enumerate}
\end{problem}


