\documentclass{ximera}

\newcommand{\RR}{\mathbb R}
\renewcommand{\d}{\,d}
\newcommand{\dd}[2][]{\frac{d #1}{d #2}}
\renewcommand{\l}{\ell}
\newcommand{\ddx}{\frac{d}{dx}}
\newcommand{\dfn}{\textbf}
\newcommand{\eval}[1]{\bigg[ #1 \bigg]}
\renewcommand{\theenumii}{\textup{(\roman{enumii})}}
\renewcommand{\labelenumii}{\theenumii}

\usepackage{graphicx}
\usepackage{multicol}
\usepackage{tkz-euclide}
%\usepackage{unicode-math}

\usepackage{pgfplots}   % <- for graphics
\pgfplotsset{compat=newest}


\renewenvironment{freeResponse}{
\ifhandout\setbox0\vbox\bgroup\else
\begin{trivlist}\item[\hskip \labelsep\bfseries Solution:\hspace{2ex}]
\fi}
{\ifhandout\egroup\else
\end{trivlist}
\fi}

\newcommand*{\ZeroOverZero}{\ensuremath{\dfrac{0}{0}}}

\providecommand{\HCCondition}{0}
\newcommand{\WkstHop}[1][1]{\if\HCCondition 0
	\vspace*{\stretch{#1}} \fi} 
\newcommand{\WkstNew}{\if\HCCondition 0
	\newpage
	 \fi} 

\title[Problem 3]{Problem 3}

\begin{document}
\begin{abstract} \end{abstract}
\maketitle

% Extracted from approximatingTheAreaUnderACurve.tex, problem #3
\begin{problem}
The graph of the function $f(x)=x^2$ is given in the figure.   
      \begin{image}
      \includegraphics[scale=0.4]{figureR001.png}
      \end{image}
        Will a right Riemann sum approximation, for any value of $n$, be an underestimate or overestimate?
        
Approximate the shaded area using a right Riemann sum with $n=1,2,4$ , and $8$ rectangles, as illustrated in the figure below.
        \begin{image}
      \includegraphics[scale=0.5]{figureR002.png}
      \end{image}

\begin{enumerate}
\item Approximate the shaded area using a right Riemann sum with $n=1$ rectangles.

\item Approximate the shaded area using a right Riemann sum with $n=2$ rectangles.

\item Approximate the shaded area using a right Riemann sum with $n=4$ rectangles.

\item Approximate the shaded area using a right Riemann sum with $n=8$ rectangles.

\end{enumerate}	
\end{problem}

\end{document}
