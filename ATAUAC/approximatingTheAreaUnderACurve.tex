%Add code to compile both versions from makefile at same time
\providecommand{\HCCondition}{0}
%Define each of the conditions
\ifcase\HCCondition
	%\condition=0 -> handout
	\documentclass[nooutcomes,noauthor,space,handout]{ximera}
	\title{Approximating the area under a curve (ATAUAC)}
\or	%\condition=1 -> Soln
	\documentclass[nooutcomes,noauthor]{ximera}
	\title{Approximating the area under a curve (ATAUAC) - Solutions}  
\fi

\usepackage{fullpage}
\newcommand{\RR}{\mathbb R}
\renewcommand{\d}{\,d}
\newcommand{\dd}[2][]{\frac{d #1}{d #2}}
\renewcommand{\l}{\ell}
\newcommand{\ddx}{\frac{d}{dx}}
\newcommand{\dfn}{\textbf}
\newcommand{\eval}[1]{\bigg[ #1 \bigg]}
\renewcommand{\theenumii}{\textup{(\roman{enumii})}}
\renewcommand{\labelenumii}{\theenumii}

\usepackage{graphicx}
\usepackage{multicol}
\usepackage{tkz-euclide}
%\usepackage{unicode-math}

\usepackage{pgfplots}   % <- for graphics
\pgfplotsset{compat=newest}


\renewenvironment{freeResponse}{
\ifhandout\setbox0\vbox\bgroup\else
\begin{trivlist}\item[\hskip \labelsep\bfseries Solution:\hspace{2ex}]
\fi}
{\ifhandout\egroup\else
\end{trivlist}
\fi}

\newcommand*{\ZeroOverZero}{\ensuremath{\dfrac{0}{0}}}

\providecommand{\HCCondition}{0}
\newcommand{\WkstHop}[1][1]{\if\HCCondition 0
	\vspace*{\stretch{#1}} \fi} 
\newcommand{\WkstNew}{\if\HCCondition 0
	\newpage
	 \fi}  


\begin{document}
\begin{abstract}

\end{abstract}
\maketitle

\ifcase\HCCondition
%summary in here
\section*{SUMMARY of Sigma Notation:}

Useful formulas:\hspace{0.4in}   (1) $\sum_{k=1}^{n}C =n\cdot C$ ;\hspace{0.4in}    (2)  $\sum_{k=1}^{n} k =\frac{n(n+1)}{2}$;  \hspace{0.4in}      (3) $\sum_{k=1}^{n} k^2 =\frac{n(n+1)(2n+1)}{6}$ \hspace{0.4in}      (3) $\sum_{k=1}^{n} k^3 =\frac{n^2(n+1)^2}{4}$\\[1.5em]
Useful example of application of rules for sums: \\[1em] $\sum_{k=1}^{n}(ak+b) =\sum_{k=1}^{n}ak + \sum_{k=1}^{n}b=a\sum_{k=1}^{n}k + \sum_{k=1}^{n}b=a\frac{n(n+1)}{2}+nb$,\hspace{0.1in}where a and b constants.\hspace{0.1in}  \\[2.5em]
\section*{SUMMARY of Riemann Sums:}

Riemann sum:\hspace{0.1in}  $\sum_{k=1}^{n}f(x_k^*)\Delta x$, \\[1.5em]
 Right Riemann sum: $x_k^*=x_k$; \hspace{0.1in} Left Riemann sum: $x_k^*=x_{k-1}$; \hspace{0.1in}Midpoint Riemann sum: $x_k^*=\frac{x_{k-1}+x_k}{2}$.\\[1em]
Width of each  of $n$ rectangles on the interval $[a,b]$:\hspace{0.1in} $\Delta x=\frac{b-a}{n}$\\[1em]
Grid points for interval $[a,b]$:\\[1em]
$x_0=a$,\\[0.2em]
                                     $.$\\[0.2em]
                                            $ .$\\ [0.2em]
                                            $ .$\\[0.2em]
                                            $x_{k-1}=a+(k-1)\Delta x=a+(k-1)\frac{b-a}{n}$,\\[1em] \hspace{0.2in} $x_k=a+k\Delta x=a+k\frac{b-a}{n}$,\\[0.3em]
                                             $.$\\[0.2em]
                                           $ .$\\ [0.2em]
                                            $.$\\[0.2em]
                                               $x_n=a+n\Delta x=a+n\frac{b-a}{n}=b$

\newpage
\section*{Recitation Questions}
\fi


%problem 1
\begin{problem} Evaluate the sum.
\begin{enumerate}
	\item  $\sum_{k=1}^{n}5$
		\begin{freeResponse}
		  $\sum_{k=1}^{n}5=n\cdot5$
		\end{freeResponse}	
\WkstHop

	\item  $\sum_{k=1}^{10}5$
		\begin{freeResponse}
			Using the previous result for $n=10$, we get
 			$\sum_{k=1}^{10}5=10\cdot5=50$\\
  			or\\

   			$\sum_{k=1}^{10}5=5+5+5+5+5+5+5+5+5+5=50$
		\end{freeResponse}	
\WkstHop

	\item  $\sum_{k=1}^{n}8k$
		\begin{freeResponse}
  			$\sum_{k=1}^{n}8k=8\sum_{k=1}^{n}k=8\frac{n(n+1)}{2}=4n(n+1)$
		\end{freeResponse}	

\vspace*{\stretch{1}}
	\item  $\sum_{k=1}^{4}8k$
		\begin{freeResponse}
			Using the previous result for $n=4$, we get
			  $\sum_{k=1}^{4}8k=4(4)(4+1)=80$\\
			  or\\
			  $\sum_{k=1}^{4}8k=8(1)+8(2)+8(3)+8(4)=8+16+24+32=80$
		\end{freeResponse}	
\WkstHop

	\item  $\sum_{k=1}^{n}(6k^2-8k-5)$	
		\begin{freeResponse}
			$\sum_{k=1}^{n}(6k^2-8k-5)=\sum_{k=1}^{n}6k^2+\sum_{k=1}^{n}(-8k)+\sum_{k=1}^{n}(-5)=$\\
			$=6\sum_{k=1}^{n}k^2-8\sum_{k=1}^{n}k-\sum_{k=1}^{n}5=$ \\
			Now we use the formulas.\\[1em]
			$=6\frac{n(n+1)(2n+1)}{6}-8\frac{n(n+1)}{2}-\sum_{k=1}^{n}5=n(n+1)(2n+1)-4n(n+1)-5n$=\\[1em]
			$=2n^3-n^2-8n$
		\end{freeResponse}	
\WkstHop

	\item  $\sum_{k=0}^{2} \cos\left( \dfrac{\pi}{2}k\right)$
		\begin{freeResponse}
		\begin{align*}
			\sum_{k=0}^{2} \cos\left( \dfrac{\pi}{2}k\right) &= \cos\left( \dfrac{\pi}{2}(0)\right) + \cos\left( \dfrac{\pi}{2}(1)\right) + \cos\left( \dfrac{\pi}{2}(2)\right)\\
			&= \cos\left( 0\right) + \cos\left( \dfrac{\pi}{2}\right) + \cos\left( \pi\right) \\
			&= 1 + 0 + -1 = 0.
		\end{align*}
		\end{freeResponse}
\WkstHop

\end{enumerate}
\end{problem}

\WkstNew
\begin{problem}
\begin{enumerate}
	\item  If a function is positive and decreasing on an interval $[a,b]$,  will a right Riemann sum underestimate or overestimate the area of the region under the graph of the function?
  Justify your answer.

\begin{freeResponse}
  Because the function is decreasing, a right Riemann sum will be an underestimate.
  By definition, $f(x_k)< f(x)$ for all $x$ in the interval $[x_{k-1},x_k]$ .
  
\end{freeResponse}	

\WkstHop

		\item  If a function is positive and decreasing on an interval $[a,b]$,  will a left Riemann sum underestimate or overestimate the area of the region under the graph of the function?
  Justify your answer.

\begin{freeResponse}
  Because the function is decreasing, a right Rieimann sum will be an overestimate.
  By definition, $f(x_{k-1})> f(x)$ for all $x$ in the interval $[x_{k-1},x_k]$ .
  
\end{freeResponse}	
\WkstHop

\end{enumerate}

\end{problem}




\WkstNew

\begin{problem}
The graph of the function $f(x)=x^2$ is given in the figure.   
      \begin{image}
      \includegraphics[scale=0.4]{figureR001.png}
      \end{image}
        Will a right Riemann sum approximation, for any value of $n$, be an underestimate or overestimate?
        \begin{freeResponse}
        	Since the function $f$ is positive and increasing on this interval, the right Riemann sum will always give an overestimate of the actual area.
        \end{freeResponse}
\WkstHop
         
      Approximate the shaded area using a right Riemann sum with $n=1,2,4$ , and $8$ rectangles, as illustrated in the figure below.
        \begin{image}
      \includegraphics[scale=0.5]{figureR002.png}
      \end{image}

  
\begin{enumerate}
\item Approximate the shaded area using a right Riemann sum with $n=1$ rectangles.

\begin{freeResponse}
 $\Delta x=\frac{2}{1}=2$\\
 $ x_1^*=x_1=2$\\[1em]
 $A\approx f(2)\Delta x=4(2)=8$
\end{freeResponse}	
\WkstHop
\WkstNew

\item Approximate the shaded area using a right Riemann sum with $n=2$ rectangles.
\begin{freeResponse}
 $\Delta x=\frac{2}{2}=1$\\
 $ x_k^*=x_k=k\Delta x=k(1)$, for $k=1,2$\\
 $A\approx \sum_{k=1}^{2}  f ( x_k^* ) \Delta x =f(1)\Delta x+ f(2)\Delta x=1(1)+4(1)=5$
\end{freeResponse}
\WkstHop


	
\item Approximate the shaded area using a right Riemann sum with $n=4$ rectangles.
\begin{freeResponse}
 $\Delta x=\frac{2}{4}=\frac{1}{2}$\\
  $ x_k^*=x_k=k\Delta x=k\left(\frac{1}{2}\right)=\frac{k}{2}$, for $k=1,2,3,4$\\

 $A\approx \sum_{k=1}^{4}  f ( x_k^* ) \Delta x =f\left(\frac{1}{2}\right)\Delta x+ f(1)\Delta x+f\left(\frac{3}{2}\right)+ f(2)\Delta x=\Delta x\left(f\left(\frac{1}{2}\right)+f(1)+f\left(\frac{3}{2}\right)+f(2)\right)$\\
 $A\approx \frac{1}{2}\left(\frac{1}{4}+1+\frac{9}{4}+4\right)=\frac{5}{2}+5=\frac{15}{2}$\\

\end{freeResponse}
\WkstHop

\item Approximate the shaded area using a right Riemann sum with $n=8$ rectangles.
\begin{freeResponse}
 $\Delta x=\frac{2}{8}=\frac{1}{4}$\\
  $ x_k^*=x_k=k\Delta x=k\left(\frac{1}{4}\right)=\frac{k}{4}$, for $k=1,...,8$\\

 $A\approx \sum_{k=1}^{8}  f ( x_k^* ) \Delta x = \Delta x\cdot\sum_{k=1}^{8}  f ( x_k^* )=\frac{1}{4}\cdot\sum_{k=1}^{8} f \left( \frac{k}{4} \right)=\frac{1}{4}\cdot\sum_{k=1}^{8}  \left( \frac{k}{4} \right)^2= \frac{1}{4}\cdot\sum_{k=1}^{8}  \frac{k^2}{16} = \frac{1}{64}\cdot\sum_{k=1}^{8}k^2 $ \\[1em]
Recall: $\sum_{k=1}^{n} k^2 =\frac{n(n+1)(2n+1)}{6}$
Therefore\\
 $A\approx \frac{1}{64}\cdot\sum_{k=1}^{8}k^2 = \frac{1}{64}\cdot\frac{8(9)(17)}{6}= \frac{1}{8}\cdot\frac{3(17)}{2}=\frac{51}{16}$ 
\end{freeResponse}
\WkstHop

	  \end{enumerate}	
\end{problem}

\WkstNew
%prob 4
\begin{problem}
	A positive continuous function will have area approximated on the interval $[1, 6]$ using $n$ rectangles. 
	\begin{enumerate}
		\item Find a formula for the grid point, $x_k$. 
	
			\begin{freeResponse}
				Here $\Delta x = \dfrac{b-a}{n} = \dfrac{5}{n}$. 
				
				The basic grid point formula is $x_k = a + k \Delta x = 1 + \dfrac{5k}{n}$.
			\end{freeResponse}
\WkstHop

		\item Find a formula for the sample point $x_k^*$ if using a right Riemann sum.
			\begin{freeResponse}
				For right endpoints, $x^*_k = x_{k}  = 1 + \dfrac{5k}{n}$.
			\end{freeResponse}
\WkstHop

		\item Find a formula for the sample point $x_k^*$ if using a left Riemann sum.
			\begin{freeResponse}
				For left endpoints, $x^*_k = x_{k-1}  = 1 + \dfrac{5(k_1)}{n}$.
			\end{freeResponse}
\WkstHop

		\item Find a formula for the sample point $x_k^*$ if using a midpoint Riemann sum.
			\begin{freeResponse}
				For midpoints, $x^*_k = \dfrac{x_{k}+x_{k-1}}{2}  = 1 + \dfrac{5(k-\frac{1}{2})}{n}$.
			\end{freeResponse}
\WkstHop

\end{enumerate}
\end{problem}


\WkstNew

\begin{problem}
The graph of the function $f(x)=x^2$ is given in the figure.

   
      \begin{image}
      \includegraphics[scale=0.4]{figureR001.png}
      \end{image}
    Find the exact value of the area of the shaded region. HINT: Use a right Riemann sum with $n$ rectangles, and  then take the limit as $n\to\infty$.
      \begin{freeResponse}
 $\Delta x=\frac{2}{n}$\\
  $ x_k^*=x_k=k\Delta x=k\left(\frac{2}{n}\right)=\frac{2k}{n}$, for $k=1,...,n$\\

 $A\approx \sum_{k=1}^{n}  f ( x_k^* ) \Delta x = \Delta x\cdot\sum_{k=1}^{n}  f ( x_k^* )=\frac{2}{n}\cdot\sum_{k=1}^{n} f \left( \frac{2k}{n} \right)=\frac{2}{n}\cdot\sum_{k=1}^{n}  \left( \frac{2k}{n} \right)^2= \frac{2}{n}\cdot\sum_{k=1}^{n}  \frac{4k^2}{n^2} = \frac{8}{n^3}\cdot\sum_{k=1}^{n}k^2 $ \\[1em]
Recall: $\sum_{k=1}^{n} k^2 =\frac{n(n+1)(2n+1)}{6}$\\
Therefore\\
 $A\approx\frac{8}{n^3}\cdot\sum_{k=1}^{n}k^2 =\frac{8}{n^3}\cdot\frac{n(n+1)(2n+1)}{6} =\frac{4}{3}\cdot\frac{(n+1)(2n+1)}{n^2}=\frac{4}{3}\cdot\frac{2n^2+3n+1}{n^2}$.\\[1em]
 Let's take the limit of Riemann sums as $n\to\infty$.\\[1em]
  $A=\lim_{n\to\infty} \sum_{k=1}^{n}  f ( x_k^* ) \Delta x =\lim_{n\to\infty}\frac{4}{3}\cdot\frac{2n^2+3n+1}{n^2}=\lim_{n\to\infty}\frac{4}{3}\cdot\ \left(2+\frac{3}{n}+\frac{1}{n^2}\right)=\frac{8}{3}$.

\end{freeResponse}
\WkstHop
\end{problem}

\WkstNew

\begin{problem}
Consider  a  Riemann sum with $n$ rectangles for the  function $f$ on the interval $[a,b]$.\\
Use geometry to find the limit of Riemann sums as $n\to\infty$.\\
$\lim_{n\to\infty} \sum_{k=1}^{n}  f ( x_k^* ) \Delta x $ ?
\begin{enumerate}
\item $f(x)=x$, $[0,3]$;
\begin{freeResponse}
$\lim_{n\to\infty} \sum_{k=1}^{n}  f ( x_k^* ) \Delta x =\lim_{n\to\infty} \sum_{k=1}^{n}   x_k^*  \Delta x =\frac{9}{2}$,\\
since $\lim_{n\to\infty} \sum_{k=1}^{n}  f ( x_k^* ) \Delta x =A$, the  area of the shaded region in the figure.

      \begin{image}
      \includegraphics[scale=0.3]{figureR004.png}
      \end{image}
 
\end{freeResponse}	
\WkstHop

\item $f(x)=|x|$, $[-3,3]$;
\begin{freeResponse}
$\lim_{n\to\infty} \sum_{k=1}^{n}  f ( x_k^* ) \Delta x =\lim_{n\to\infty} \sum_{k=1}^{n}  | x_k^*|  \Delta x =9$,\\
since $\lim_{n\to\infty} \sum_{k=1}^{n}  f ( x_k^* ) \Delta x =A$, the  area of the shaded region in the figure.

      \begin{image}
      \includegraphics[scale=0.4]{figureR005.png}
      \end{image}
 
\end{freeResponse}	
\WkstHop
\end{enumerate}
\end{problem}

\WkstNew

\begin{problem}
A part of a circle is shown in the figure.
\begin{image}
      \includegraphics[scale=0.3]{figureR006.png}
      \end{image}
\begin{enumerate}
\item If we express the area of the shaded region in the figure as the limit of Riemann sums \\

$A=\lim_{n\to\infty} \sum_{k=1}^{n}  f ( x_k^* ) \Delta x$,\\
 find the function $f$ and the interval $[a,b]$.

      \begin{freeResponse}
      $f(x)=\sqrt{49-x^2}$, and the interval $=[0,7]$.
      \end{freeResponse}
\WkstHop

      \item Compute the limit. \\
      $\lim_{n\to\infty}  \sum_{k=1}^{n} \sqrt{49- (x_k^* )^2 }\cdot  \frac{7}{n}$
     \begin{freeResponse}
      $\lim_{n\to\infty}  \sum_{k=1}^{n} \sqrt{49- (x_k^* )^2 }\cdot  \frac{7}{n}=\frac{49\pi}{4}$
      \end{freeResponse}
\WkstHop

      \end{enumerate}
\end{problem}
\WkstNew

\begin{problem}
We want to approximate the area under the curve  using a right Riemann Sum with the given value of $n$. Write the sum in summation notation and evaluate it.
	\begin{enumerate}
	%part a
	\item  $\sin (x)$, \; $\left[ 0, \frac{\pi}{2} \right]$, \; $n=3$
	
		\begin{freeResponse}
		
		$\Delta x=\frac{b-a}{n}=\frac{\frac{\pi }{2}-0}{3}=\frac{\pi }{6}$.  \\
		
		$x_k^*=x_k=a+k\Delta x=0+\frac{\pi }{6}k=\frac{\pi }{6}k.$
		\begin{align*}
		\sum_{k=1}^{3}  f ( x_k^* ) \Delta x  &= \sum_{k=1}^{3} \left( f \left( \frac{\pi}{6}k \right) \cdot \frac{\pi}{6} \right) \\
		&= \frac{\pi}{6} \sum_{k=1}^{3} \sin \left( \frac{\pi}{6} k \right) \\
		&= \frac{\pi}{6} \left( \sin \left( \frac{\pi}{6} \right) + \sin \left( \frac{2 \pi}{6} \right) + \sin \left( \frac{3 \pi}{6} \right) \right) \\
		&= \frac{\pi}{6} \left( \sin \left( \frac{\pi}{6} \right) + \sin \left( \frac{\pi}{3} \right) + \sin \left( \frac{\pi}{2} \right) \right) \\
		&= \frac{\pi}{6} \left( \frac{1}{2} + \frac{\sqrt{3}}{2} + 1 \right)  \\
		&= \frac{\pi}{6} \left( \frac{3}{2} + \frac{\sqrt{3}}{2} \right)  \\
		&= \frac{\pi}{12} (3 + \sqrt{3} )
		\end{align*}
		\end{freeResponse}
\WkstHop
		
	\item  $f(x) = x^2 - 9x + 18$, \; $[7,10]$, \; $n=6$
		\begin{freeResponse}
		$\Delta x=\frac{b-a}{n}=\frac{10-7}{6}=\frac{1}{2}.$  \\
		$x_k^* =x_k=a+k\Delta x=7+\frac{1}{2}k.$  
		\begin{align*}
		\sum_{k=1}^{6} f(x_k^*) \Delta x &=f(7+\frac{1}{2})\cdot\frac{1}{2}+f(7+1)\cdot\frac{1}{2}+f(7+\frac{3}{2})\cdot\frac{1}{2}+f(7+2)\cdot\frac{1}{2}+f(7+\frac{5}{2})\cdot\frac{1}{2}+f(7+3)\cdot\frac{1}{2}\\
		&=f(7.5)\cdot\frac{1}{2}+f(8)\cdot\frac{1}{2}+f(8.5)\cdot\frac{1}{2}+f(9)\cdot\frac{1}{2}+f(9.5)\cdot\frac{1}{2}+f(10)\cdot\frac{1}{2}\\
&=\frac{1}{2}\left(f(7.5)+f(8)+f(8.5)+f(9)+f(9.5)+f(10)\right)\\
&=\frac{1}{2}\left((7.5^2-9(7.5)+18)+(8^2-9(8)+18)+(8.5^2-9(8.5)+18)+(9^2-9(9)+18)+(9.5^2-9(9.5)+18)+(10^2-9(10)+18)\right)\\
&=\frac{1}{2}\left((7.5^2+8^2+8.5^2+9^2+9.5^2+10^2)-9(7.5+8+8.5+9+9.5+10)+6(18)\right)\\
&= \frac{397}{8}
		\end{align*}
		Or in sigma notation:
		\begin{align*}
		\sum_{k=1}^{6} f(a + k \Delta x) \Delta x &= \sum_{k=1}^{6} \left( f \left( 7 + \frac{1}{2} k \right) \cdot \frac{1}{2} \right)  \\
		&= \frac{1}{2} \sum_{k=1}^{6} \left( \left( 7 + \frac{1}{2} k \right)^2 - 9 \left( 7 + \frac{1}{2} k \right) + 18 \right) \\
		&= \frac{1}{2} \sum_{k=1}^{6} \left( 49 + 7k + \frac{1}{4} k^2 - 63 - \frac{9}{2} k + 18 \right) \\
		&= \frac{1}{2} \sum_{k=1}^{6} \left( 4 + \frac{5}{2} k+ \frac{1}{4} k^2 \right) \\
		&= \frac{1}{2} \left( 4 \sum_{k=1}^{6} 1 + \frac{5}{2} \sum_{k=1}^{6} k+ \frac{1}{4} \sum_{k=1}^{6} k^2 \right) \\
		&= \frac{1}{2} \left( 4(6) + \frac{5}{2} \cdot \frac{(6)(7)}{2} + \frac{1}{4} \cdot \frac{(6)(7)(13)}{6} \right) \\
		&= \frac{1}{2} \left( 24 + \frac{105}{2} + \frac{91}{4} \right) \\
		&= \frac{1}{2} \cdot \frac{96 + 210 + 91}{4} = \frac{397}{8}
		\end{align*}
		\end{freeResponse}
\WkstHop
	%part b
	
		
		
		
	\end{enumerate}
		
		
		

\end{problem}
		
	\end{document} 
	
	
%problem 3
\begin{problem}
  Snow is accumulating on the ground at a rate of  
  $$f^\prime (t)=1.5t-.25 t^2+.3$$
  inches per hour for $t$ in $[0,4]$ (i.e., the snow falls for 4 hours- from noon until 4PM).  

  There were already $5$ inches of snow on the ground when the storm started.  What does this statement say notation-wise?

  A natural question would be to ask how much the amount of snow on the ground changed during the storm.  But because the rate is always changing, this is a difficult question to answer (yet, we will eventually answer it!).  Let’s take what we know about constant rates and amounts and use that to help us answer our question.

  \begin{enumerate}
    
%part a
  \item  Assume the rate stays the same as it was at the start of the storm: 0.3 inches per hour.  How much did the height of the snow on the ground change?  Is this a realistic estimate?
    \begin{freeResponse}
      We are assuming that snow is falling at a constant rate of $f^\prime (0)=.3$ inches per hour for the full four hours.  We estimate that the net amount of snow fallen is:
      
      (rate snow is falling)$\times$(change in time)=$f^\prime (0) \cdot (4-0) =.3 \cdot 4=1.2$ inches of snow.  
      
      No this is not a realistic estimate because $f^\prime$ is not constant.
    \end{freeResponse}

%part b
  \item  Now assume the rate is the same as it is at the start for the first two hours, then changes to what it is at 2PM for the final two hours.  How much did the height of the snow on the ground change?  Is this a realistic estimate?  Is it likely to be better or worse than that of part (a)?
    \begin{freeResponse}
      We are assuming that snow is falling at a constant rate of $f^\prime (0)=.3$ inches per hour for the first two hours, then at the constant rate of $f^\prime (2)=2.3$ inches per hour for the final two hours.  Thus, we estimate that the net amount of snow fallen is:
      
      $f^\prime (0)\cdot (2-0)+ f^\prime (2) \cdot (4-2)= (0.3)(2) + (2.3)(2) = 0.6+4.6=5.2$ inches.
      
      No this is not a realistic estimate, but it is better than the estimate from part (a).  
    \end{freeResponse}
 
    %part c   
  \item  Now assume the rate stays constant by the hour (i.e., it only changes on the hour to its rate at those times of noon, 1PM, 2PM, and 3PM).  How much did the height of the snow change?
    \begin{freeResponse}
      We are assuming that snow is falling at a constant rate of $f^\prime (0)=.3$ inches per hour for the first hour, then at the constant rate of $f^\prime (1)=1.55$ inches per hour for the second hour, then at a constant rate of $f^\prime (2)=2.3$ inches per hour for the third hour, and finally at a constant rate of $f^\prime (3)=2.55$ inches per hour for the final hour.  Thus, we estimate that the net amount of snow fallen is:
      $$ (0.3)(1) + (1.55)(1) + (2.3)(1) + (2.55)(1) = 0.3 + 1.55 + 2.3 + 2.55 = 6.7 $$
      
      So we estimate that 6.7 inches of snow fell throughout the four hours.
    \end{freeResponse}
    
    %part d
  \item  Now do the same, but it changes on the half-hour.
    \begin{freeResponse}
      Snow falls at the constant rate of $f^\prime (0)=.3$ inches per hour for the first half hour, then at the constant rate of $f^\prime (.5)=.9875$ inches per hour for the next half hour, then at the constant rate of $f^\prime (1)=1.55$ inches per hour for the next half hour, and so on, changing rates at each half hour to the constant rates of $f^\prime (1.5)=1.9875, f^\prime (2)=2.3, f^\prime (2.5)=2.4875, f^\prime (3)=2.55, $ and $f^\prime (3.5)=2.4875$ inches per hour, respectively.  Thus, we estimate that the net amount of snow fallen is:
      \begin{align*}
        & [0.3 + 0.9875 + 1.55 + 1.9875 + 2.3 + 2.4875 + 2.55 + 2.4875](0.5) \\
        &= 7.325 \text{ inches}.
      \end{align*}
    \end{freeResponse}
    
    
    
%part e
  \item  What would we need to do to find the exact amount that the height of the snow on the ground changed? Describe this in words, do not try to calculate them.
    \begin{freeResponse}
      To find the exact amount of snow that fell during $n$ time intervals of length $\frac{4}{n}$.  That amount is equal to the rate $f'\left(\frac{4k}{n}\right)$ multiplied by the length of the interval $\frac{4}{n}$.  We then need to add these up and take the limit as $n$ goes to infinity.
      
      In the figures below, you can see how as $n$ gets larger, the approximation becomes more accurate.  The light pink is an upper estimate and the dark pink in a lower estimate.
      
      
      \begin{image}
      \includegraphics[scale=.4]{figure1.png}
      \includegraphics[scale=.4]{figure2.png}
            \includegraphics[scale=.4]{figure3.png}
      \end{image}
      
      
    \end{freeResponse}
  \end{enumerate}
\end{problem}

\end{document} 
