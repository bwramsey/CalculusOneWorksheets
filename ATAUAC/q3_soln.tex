\documentclass{ximera}

\newcommand{\RR}{\mathbb R}
\renewcommand{\d}{\,d}
\newcommand{\dd}[2][]{\frac{d #1}{d #2}}
\renewcommand{\l}{\ell}
\newcommand{\ddx}{\frac{d}{dx}}
\newcommand{\dfn}{\textbf}
\newcommand{\eval}[1]{\bigg[ #1 \bigg]}
\renewcommand{\theenumii}{\textup{(\roman{enumii})}}
\renewcommand{\labelenumii}{\theenumii}

\usepackage{graphicx}
\usepackage{multicol}
\usepackage{tkz-euclide}
%\usepackage{unicode-math}

\usepackage{pgfplots}   % <- for graphics
\pgfplotsset{compat=newest}


\renewenvironment{freeResponse}{
\ifhandout\setbox0\vbox\bgroup\else
\begin{trivlist}\item[\hskip \labelsep\bfseries Solution:\hspace{2ex}]
\fi}
{\ifhandout\egroup\else
\end{trivlist}
\fi}

\newcommand*{\ZeroOverZero}{\ensuremath{\dfrac{0}{0}}}

\providecommand{\HCCondition}{0}
\newcommand{\WkstHop}[1][1]{\if\HCCondition 0
	\vspace*{\stretch{#1}} \fi} 
\newcommand{\WkstNew}{\if\HCCondition 0
	\newpage
	 \fi} 


\title[Problem 3]{Problem 3}

\begin{document}
\begin{abstract} \end{abstract}
\maketitle


% Extracted from approximatingTheAreaUnderACurve.tex, problem #3
\begin{problem}
The graph of the function $f(x)=x^2$ is given in the figure.   
      \begin{image}
      \includegraphics[scale=0.4]{figurer001.png}
      \end{image}
        Will a right Riemann sum approximation, for any value of $n$, be an underestimate or overestimate?
        \begin{explanation}
        	Since the function $f$ is positive and increasing on this interval, the right Riemann sum will always give an overestimate of the actual area.
        \end{explanation}
Approximate the shaded area using a right Riemann sum with $n=1,2,4$ , and $8$ rectangles, as illustrated in the figure below.
        \begin{image}
      \includegraphics[scale=0.5]{figurer002.png}
      \end{image}

  
\begin{enumerate}
\item Approximate the shaded area using a right Riemann sum with $n=1$ rectangles.

\begin{explanation}
 $\Delta x=\frac{2}{1}=2$\\
 $ x_1^*=x_1=2$\\[1em]
 $A\approx f(2)\Delta x=4(2)=8$
\end{explanation}	
\item Approximate the shaded area using a right Riemann sum with $n=2$ rectangles.
\begin{explanation}
 $\Delta x=\frac{2}{2}=1$\\
 $ x_k^*=x_k=k\Delta x=k(1)$, for $k=1,2$\\
 $A\approx \sum_{k=1}^{2}  f ( x_k^* ) \Delta x =f(1)\Delta x+ f(2)\Delta x=1(1)+4(1)=5$
\end{explanation}
\item Approximate the shaded area using a right Riemann sum with $n=4$ rectangles.
\begin{explanation}
 $\Delta x=\frac{2}{4}=\frac{1}{2}$\\
  $ x_k^*=x_k=k\Delta x=k\left(\frac{1}{2}\right)=\frac{k}{2}$, for $k=1,2,3,4$\\

 $A\approx \sum_{k=1}^{4}  f ( x_k^* ) \Delta x =f\left(\frac{1}{2}\right)\Delta x+ f(1)\Delta x+f\left(\frac{3}{2}\right)+ f(2)\Delta x=\Delta x\left(f\left(\frac{1}{2}\right)+f(1)+f\left(\frac{3}{2}\right)+f(2)\right)$\\
 $A\approx \frac{1}{2}\left(\frac{1}{4}+1+\frac{9}{4}+4\right)=\frac{5}{2}+5=\frac{15}{2}$\\

\end{explanation}
\item Approximate the shaded area using a right Riemann sum with $n=8$ rectangles.
\begin{explanation}
 $\Delta x=\frac{2}{8}=\frac{1}{4}$\\
  $ x_k^*=x_k=k\Delta x=k\left(\frac{1}{4}\right)=\frac{k}{4}$, for $k=1,...,8$\\

 $A\approx \sum_{k=1}^{8}  f ( x_k^* ) \Delta x = \Delta x\cdot\sum_{k=1}^{8}  f ( x_k^* )=\frac{1}{4}\cdot\sum_{k=1}^{8} f \left( \frac{k}{4} \right)=\frac{1}{4}\cdot\sum_{k=1}^{8}  \left( \frac{k}{4} \right)^2= \frac{1}{4}\cdot\sum_{k=1}^{8}  \frac{k^2}{16} = \frac{1}{64}\cdot\sum_{k=1}^{8}k^2 $ \\[1em]
Recall: $\sum_{k=1}^{n} k^2 =\frac{n(n+1)(2n+1)}{6}$
Therefore\\
 $A\approx \frac{1}{64}\cdot\sum_{k=1}^{8}k^2 = \frac{1}{64}\cdot\frac{8(9)(17)}{6}= \frac{1}{8}\cdot\frac{3(17)}{2}=\frac{51}{16}$ 
\end{explanation}
\end{enumerate}	
\end{problem}



\end{document}
