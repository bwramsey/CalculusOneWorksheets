\documentclass{ximera}

\newcommand{\RR}{\mathbb R}
\renewcommand{\d}{\,d}
\newcommand{\dd}[2][]{\frac{d #1}{d #2}}
\renewcommand{\l}{\ell}
\newcommand{\ddx}{\frac{d}{dx}}
\newcommand{\dfn}{\textbf}
\newcommand{\eval}[1]{\bigg[ #1 \bigg]}
\renewcommand{\theenumii}{\textup{(\roman{enumii})}}
\renewcommand{\labelenumii}{\theenumii}

\usepackage{graphicx}
\usepackage{multicol}
\usepackage{tkz-euclide}
%\usepackage{unicode-math}

\usepackage{pgfplots}   % <- for graphics
\pgfplotsset{compat=newest}


\renewenvironment{freeResponse}{
\ifhandout\setbox0\vbox\bgroup\else
\begin{trivlist}\item[\hskip \labelsep\bfseries Solution:\hspace{2ex}]
\fi}
{\ifhandout\egroup\else
\end{trivlist}
\fi}

\newcommand*{\ZeroOverZero}{\ensuremath{\dfrac{0}{0}}}

\providecommand{\HCCondition}{0}
\newcommand{\WkstHop}[1][1]{\if\HCCondition 0
	\vspace*{\stretch{#1}} \fi} 
\newcommand{\WkstNew}{\if\HCCondition 0
	\newpage
	 \fi} 


\title[Problem 2]{Problem 2}

\begin{document}
\begin{abstract} \end{abstract}
\maketitle


% Extracted from approximatingTheAreaUnderACurve.tex, problem #2
\begin{problem}
\begin{enumerate}
	\item  If a function is positive and decreasing on an interval $[a,b]$,  will a right Riemann sum underestimate or overestimate the area of the region under the graph of the function?
  Justify your answer.

\begin{explanation}
  Because the function is decreasing, a right Riemann sum will be an underestimate.
  By definition, $f(x_k)< f(x)$ for all $x$ in the interval $[x_{k-1},x_k]$ .
  
\end{explanation}	

\item  If a function is positive and decreasing on an interval $[a,b]$,  will a left Riemann sum underestimate or overestimate the area of the region under the graph of the function?
  Justify your answer.

\begin{explanation}
  Because the function is decreasing, a right Rieimann sum will be an overestimate.
  By definition, $f(x_{k-1})> f(x)$ for all $x$ in the interval $[x_{k-1},x_k]$ .
  
\end{explanation}	
\end{enumerate}

\end{problem}



\end{document}
