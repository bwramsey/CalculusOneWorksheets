\documentclass{ximera}

\newcommand{\RR}{\mathbb R}
\renewcommand{\d}{\,d}
\newcommand{\dd}[2][]{\frac{d #1}{d #2}}
\renewcommand{\l}{\ell}
\newcommand{\ddx}{\frac{d}{dx}}
\newcommand{\dfn}{\textbf}
\newcommand{\eval}[1]{\bigg[ #1 \bigg]}
\renewcommand{\theenumii}{\textup{(\roman{enumii})}}
\renewcommand{\labelenumii}{\theenumii}

\usepackage{graphicx}
\usepackage{multicol}
\usepackage{tkz-euclide}
%\usepackage{unicode-math}

\usepackage{pgfplots}   % <- for graphics
\pgfplotsset{compat=newest}


\renewenvironment{freeResponse}{
\ifhandout\setbox0\vbox\bgroup\else
\begin{trivlist}\item[\hskip \labelsep\bfseries Solution:\hspace{2ex}]
\fi}
{\ifhandout\egroup\else
\end{trivlist}
\fi}

\newcommand*{\ZeroOverZero}{\ensuremath{\dfrac{0}{0}}}

\providecommand{\HCCondition}{0}
\newcommand{\WkstHop}[1][1]{\if\HCCondition 0
	\vspace*{\stretch{#1}} \fi} 
\newcommand{\WkstNew}{\if\HCCondition 0
	\newpage
	 \fi} 


\title[Problem 4]{Problem 4}

\begin{document}
\begin{abstract} \end{abstract}
\maketitle


% Extracted from conceptsOfGraphingFunctions.tex, problem #4
\begin{problem}
  \outcome{Determine how the graph of a function looks based on an analytic description of the function.}
  \mbox{}
  \begin{enumerate}
    \item
      You are given that $f''(x) > 0$ for all $x$.
      Which of the following must be true about $f(x)$ on the region $0 \leq x \leq 2$?
      \begin{enumerate}
        \item
          There is a critical point between $0$ and $2$.

      
        \item
          There is a local maximum, but not enough information is given to determine where.

        \item
          $f$ need not have a local maximum.
      \end{enumerate}
\begin{explanation}
        Only  (iii) must be true.
        The following picture provides a counterexample for both (i) and (ii):
        \begin{image}
          \includegraphics[trim= 70 470 250 190]{figure1_pdf.png}
        \end{image}
      \end{explanation}
		
     \item
       You are told that $f''(x) > 0$ for all $x$.
       Which of the following must be true about the graph of $y=f(x)$?
       \begin{enumerate}
         \item
           The graph is a straight line.

         \item 
           The graph crosses the $x$-axis at most once.
         
         \item
           The graph is concave down.

         \item
           The graph crosses the $y$-axis more than once.

         \item
           The graph is concave up.
       \end{enumerate}
\begin{explanation}
         Only (v) must be true.
         For (i), the function $f(x) = e^x$ provides a counterexample:
         \begin{image}
           \includegraphics[trim= 70 470 250 190]{figure2_pdf.png}
         \end{image}
			
         For part (ii), the function $f(x) = x^2 -2$ provides a counterexample:
         \begin{image}
           \includegraphics[trim= 70 470 250 190]{figure3_pdf.png}
         \end{image}
			
         Part (iii) is clearly false since $f''(x) > 0$ means that $f$ is concave up.
         Part (iv) is false for any function.
       \end{explanation}
    \end{enumerate}
\end{problem}



\end{document}
