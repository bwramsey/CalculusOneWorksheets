\documentclass{ximera}

\newcommand{\RR}{\mathbb R}
\renewcommand{\d}{\,d}
\newcommand{\dd}[2][]{\frac{d #1}{d #2}}
\renewcommand{\l}{\ell}
\newcommand{\ddx}{\frac{d}{dx}}
\newcommand{\dfn}{\textbf}
\newcommand{\eval}[1]{\bigg[ #1 \bigg]}
\renewcommand{\theenumii}{\textup{(\roman{enumii})}}
\renewcommand{\labelenumii}{\theenumii}

\usepackage{graphicx}
\usepackage{multicol}
\usepackage{tkz-euclide}
%\usepackage{unicode-math}

\usepackage{pgfplots}   % <- for graphics
\pgfplotsset{compat=newest}


\renewenvironment{freeResponse}{
\ifhandout\setbox0\vbox\bgroup\else
\begin{trivlist}\item[\hskip \labelsep\bfseries Solution:\hspace{2ex}]
\fi}
{\ifhandout\egroup\else
\end{trivlist}
\fi}

\newcommand*{\ZeroOverZero}{\ensuremath{\dfrac{0}{0}}}

\providecommand{\HCCondition}{0}
\newcommand{\WkstHop}[1][1]{\if\HCCondition 0
	\vspace*{\stretch{#1}} \fi} 
\newcommand{\WkstNew}{\if\HCCondition 0
	\newpage
	 \fi} 


\title[Problem 5]{Problem 5}

\begin{document}
\begin{abstract} \end{abstract}
\maketitle


% Extracted from conceptsOfGraphingFunctions.tex, problem #5
\begin{problem}
Suppose a function $f$ satisfies the following conditions:
	\begin{enumerate}
		\item $f(0)=0$  and  $f'(-4)=f'(2)=f'(10)=0$\\
		\item $\displaystyle \lim_{x \to 6} f(x)=-\infty$,  and   $\displaystyle \lim_{x \to +\infty} f(x)=6$\\
		\item $f'(x)<0$ \hspace{0.2in} on \hspace{0.2in} $(-\infty, -4)$, \hspace{0.2in}$(2,6)$, \hspace{0.2in}  and \hspace{0.2in}  $(10, +\infty)$\\
		\item $f'(x)>0$ \hspace{0.2in} on \hspace{0.2in} $(-4, 2)$, \hspace{0.2in}  and \hspace{0.2in}  $(6, 10)$\\
		\item $f''(x)>0$ \hspace{0.2in} on \hspace{0.2in} $(-\infty, 0)$, \hspace{0.2in}  and \hspace{0.2in}  $(14, +\infty)$\\
		
		\item $f''(x)<0$ \hspace{0.2in} on \hspace{0.2in} $(0,6)$, \hspace{0.2in}  and \hspace{0.2in}  $(6, 14)$\\
	\end{enumerate}

\begin{enumerate}
\item List the \textbf{interval(s)} where the function $f$ is \textbf{both increasing} and \textbf{concave UP}.\\
\begin{explanation}
\hspace{1in}		{\color{blue}$(-4,0)$}\\[0.25em]
  \end{explanation}
\item List the \textbf{interval(s)} where the function $f$ is \textbf{both increasing} and \textbf{concave DOWN}.\\
\begin{explanation}
\hspace{1in}		{\color{blue}$(0, 2),\hspace{0.2in}(6, 10)$}\\[0.25em]
\end{explanation}
\item List the \textbf{interval(s)} where the function $f$ is \textbf{both decreasing} and \textbf{concave UP}.\\
\begin{explanation}
\hspace{1in}		{\color{blue}$(-\infty, -4),\hspace{0.2in}(14, +\infty)$}\\[0.25em]
 \end{explanation}
\item List the \textbf{interval(s)} where the function $f$ is \textbf{both decreasing} and \textbf{concave DOWN}.\\
\begin{explanation}
\hspace{1in}		{\color{blue}$(2, 6),\hspace{0.2in}(10, 14)$}\\[0.25em]
 \end{explanation}
\item List the \textbf{x-coordinates} at which $f$ has a \textbf{local minimum}. Write "none" if appropriate.\\
\begin{explanation}
\hspace{1in}		{\color{blue}$x=-4$}\\[0.25em]
 \end{explanation}
\item List the \textbf{x-coordinates} at which  $f$ has a \textbf{local maximum}. Write "none" if appropriate.\\
\begin{explanation}
\hspace{1in}		{\color{blue}$x=2,\hspace{0.2in}x=10$\\ ($f$ can also have a local maximum at $x=6$, if we include $x=6$ in the domain)}\\[0.25em]
 \end{explanation}
\item List the \textbf{x-coordinates} of all  \textbf{inflection points} of  $f$. Write "none" if appropriate.\\
\begin{explanation}
\hspace{1in}		{\color{blue}$x=0,\hspace{0.2in}x=14$}\\[0.25em]
 \end{explanation}

\item Sketch the graph  of $f$.\\
\begin{center}
\includegraphics[height=6.7in]{image10.pdf}
\end{center}
\begin{explanation}
  {\color{blue}This is just one of many possible graphs.\\The domain of $f$ could be $(-\infty, +\infty)$, and $f$ could satisfy the given conditions. }\\[0.25em]

\begin{center}
\includegraphics[height=6.7in]{image11.png}
\end{center}
Purple point :\hspace{0.1in}  local minimum\\[1em]
Red points : \hspace{0.1in}  local maxima\\[1em]
Blue points :\hspace{0.1in} inflection points\\[1em]
\end{explanation}
\end{enumerate}

\end{problem}



\end{document}
