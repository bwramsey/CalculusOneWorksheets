% Extracted from implicitDifferentiation.tex, problem #1
\begin{problem}
On the graph below, sketch the tangent lines at $x=0$.  Then, explain why both the $x$-coordinate and the $y$-coordinate are generally needed to find the slope of the tangent line at a point on the graph of an equation of the form $F(x,y)=0$
\begin{image}
\includegraphics[scale=.2]{figure1.png}
\end{image}
\WkstHop

	\begin{freeResponse}
		Given the equation $F(x,y)=0$, its graph may not pass the ``vertical line test".  As illustrated in the figure below, fixing a value for $x$ (in this case $x=0$) will not typically specify a unique point on the graph because there may be more than one corresponding $y$-value.  That is why you need to specify both the $x$ and the $y$ coordinate, to make sure that you are giving the coordinates for a unique point on the graph.  In this case, there are two tangent lines to the curve at $x=0$.  One passes through the point $(0,1)$ and the other through $(0,-1)$.
	\begin{image}
\includegraphics[scale=.5]{figure2.png}
\end{image}		


	\end{freeResponse}	
\end{problem}
