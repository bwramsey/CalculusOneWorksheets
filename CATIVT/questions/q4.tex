% Extracted from continuityAndIVT.tex, problem #4
\begin{problem}
  Use the Intermediate value theorem to find an interval in which you can guarantee that there is a solution to the equation $x^3 = x + \sin(x) + 1$. \textbf{EXPLAIN}.
  (Do not use a graphing device or calculator to solve this problem!) 
\WkstHop
  \begin{freeResponse}
    We have
    \begin{align*}
      x^3 = x + \sin(x) + 1 &\iff x^3 - x - \sin(x) - 1 = 0.
    \end{align*}
  Define $f(x) =  x^3 - x - \sin(x) - 1$.

    Since:\\
 $f(0) = (0)^3 - (0) - \sin(0)-1 = -1$ and \\
$f(\pi) = \pi^3 - \pi - \sin(\pi)-1 = \pi(\pi^2 - 1) - 1 > 3 \cdot (3^2 - 1) - 1 = 23$\\
We have: $-1<0<23$\\
and $f$ is continuous on $[0, \pi]$, the Intermediate Value Theorem implies there is some $c$ with $0 < c < \pi$ such that $f(c)=L= 0$, that is, $c^3 = c + \sin (c) + 1$.  That $c$ is a solution on the interval $[0,\pi]$.  (Note: There are many possible intervals, e.g. $[1,4]$, etc. )

	\begin{image}
	\includegraphics[scale=.5]{Figure3.png}
	\end{image}


  \end{freeResponse}
\end{problem}
