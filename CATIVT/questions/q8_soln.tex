\documentclass{ximera}

\newcommand{\RR}{\mathbb R}
\renewcommand{\d}{\,d}
\newcommand{\dd}[2][]{\frac{d #1}{d #2}}
\renewcommand{\l}{\ell}
\newcommand{\ddx}{\frac{d}{dx}}
\newcommand{\dfn}{\textbf}
\newcommand{\eval}[1]{\bigg[ #1 \bigg]}
\renewcommand{\theenumii}{\textup{(\roman{enumii})}}
\renewcommand{\labelenumii}{\theenumii}

\usepackage{graphicx}
\usepackage{multicol}
\usepackage{tkz-euclide}
%\usepackage{unicode-math}

\usepackage{pgfplots}   % <- for graphics
\pgfplotsset{compat=newest}


\renewenvironment{freeResponse}{
\ifhandout\setbox0\vbox\bgroup\else
\begin{trivlist}\item[\hskip \labelsep\bfseries Solution:\hspace{2ex}]
\fi}
{\ifhandout\egroup\else
\end{trivlist}
\fi}

\newcommand*{\ZeroOverZero}{\ensuremath{\dfrac{0}{0}}}

\providecommand{\HCCondition}{0}
\newcommand{\WkstHop}[1][1]{\if\HCCondition 0
	\vspace*{\stretch{#1}} \fi} 
\newcommand{\WkstNew}{\if\HCCondition 0
	\newpage
	 \fi} 


\title[Problem 8]{Problem 8}

\begin{document}
\begin{abstract} \end{abstract}
\maketitle


% Extracted from continuityAndIVT.tex, problem #8
\begin{problem}
Suppose a taxi ride costs $\$7.50$ for the first mile (or any part of the first mile), plus an additional $\$1.00$ for each additional mile (or any part of a mile).
\begin{enumerate}
	\item Graph the function $c=f(t)$ that gives the cost of a taxi ride for $t$ miles, for $0 \le t \le 5$.
\begin{explanation} \hfil
	\begin{image}
	\includegraphics[scale=.5]{Figure5.png}
	\end{image}
		\end{explanation}

	\item Evaluate $\lim_{t \to 2.9}f(t)$
\begin{explanation}
$\lim_{t \to 2.9}f(t)=9.5$
		\end{explanation}
	\item Evaluate $\lim_{t \to 3^-}f(t)$ and $\lim_{t \to 3^+}f(t)$		
\begin{explanation}
$\lim_{t \to 3^-}f(t)=9.5$ and $\lim_{t \to 3^+}f(t)=10.5$	
		\end{explanation}
	\item Interpret the meaning of the limits in part (c).
\begin{explanation}
As the number of miles the taxi drives approaches 3, the cost of the taxi ride is \$9.50.  If one drives for a bit more than 3 miles, the cost is \$10.50.
		\end{explanation}
	\item On what intervals is the function $c$ continuous?  Explain.
\begin{explanation}
			$(0,1],(1,2],(2,3],(3,4],(4,5]$
		\end{explanation}
\end{enumerate}
\end{problem}



\end{document}
