\documentclass{ximera}

\newcommand{\RR}{\mathbb R}
\renewcommand{\d}{\,d}
\newcommand{\dd}[2][]{\frac{d #1}{d #2}}
\renewcommand{\l}{\ell}
\newcommand{\ddx}{\frac{d}{dx}}
\newcommand{\dfn}{\textbf}
\newcommand{\eval}[1]{\bigg[ #1 \bigg]}
\renewcommand{\theenumii}{\textup{(\roman{enumii})}}
\renewcommand{\labelenumii}{\theenumii}

\usepackage{graphicx}
\usepackage{multicol}
\usepackage{tkz-euclide}
%\usepackage{unicode-math}

\usepackage{pgfplots}   % <- for graphics
\pgfplotsset{compat=newest}


\renewenvironment{freeResponse}{
\ifhandout\setbox0\vbox\bgroup\else
\begin{trivlist}\item[\hskip \labelsep\bfseries Solution:\hspace{2ex}]
\fi}
{\ifhandout\egroup\else
\end{trivlist}
\fi}

\newcommand*{\ZeroOverZero}{\ensuremath{\dfrac{0}{0}}}

\providecommand{\HCCondition}{0}
\newcommand{\WkstHop}[1][1]{\if\HCCondition 0
	\vspace*{\stretch{#1}} \fi} 
\newcommand{\WkstNew}{\if\HCCondition 0
	\newpage
	 \fi} 


\title[Problem 5]{Problem 5}

\begin{document}
\begin{abstract} \end{abstract}
\maketitle


% Extracted from continuityAndIVT.tex, problem #5
\begin{problem}
	\begin{enumerate}
	\item  True or False: If $f$ and $g$ are two functions defined on $(-1, 1)$, and if $\displaystyle \lim_{x \to 0} g(x) = 0$, then it must be true that $\displaystyle \lim_{x \to 0} (f(x) \cdot g(x)) = 0$.
\begin{explanation}
    False: Suppose 
    \[
      f(x) =
      \begin{cases}
        \frac{1}{x} & \mbox{if $x \ne 0$}\\
        0 & \mbox{if $x = 0$}
      \end{cases}
    \]
    and $g(x) = x$.
    Then $\lim_{x \to 0} g(x) = 0$ but
    \[
      \lim_{x \to 0} (f(x) \cdot g(x)) = \lim_{x \to 0} \left( \frac{1}{x} \cdot x \right) = \lim_{x \to 0} 1 = 1.
    \]
  \end{explanation}

  \item True or False: If $f$ is continuous on $(-1, 1)$, and if $f(0) = 10$ and $\displaystyle \lim_{x \to 0} g(x) = 2$, then
  \[
    \lim_{x \to 0} \frac{f(x)}{g(x)} = 5.
  \]
\begin{explanation}
    True: application of quotient rule.  Because $f$ is continuous, $lim_{x \to 0} f(x)=f(0)=10$
  \end{explanation}

	\item  True or False: If $f$ is continuous on $[1, 3]$, and if $f(1) = 0$ and $f(3) = 4$, then the equation $f(x) = \pi$ has a solution in $(1, 3)$.
\begin{explanation}
    True: $f$ is continuous on $[1, 3]$, $f(1) = 0 < \pi < 4 = f(3)$, and the Intermediate Value Theorem implies there is some $x$ in $(1, 3)$ with $f(x) = \pi$.
  \end{explanation}

  \item True or False: Let $f$ be a positive function with vertical asymptote $x = 5$. Then
  \[
    \lim_{x \to 5} f(x) = \infty.
  \]
\begin{explanation}
    False: Suppose $f$ is defined by
    \[
      f(x) =
      \begin{cases}
        \frac{1}{x - 5} & \mbox{if $x > 5$}\\
        2 & \mbox{if $x \le 5$}.
      \end{cases}
    \]
    
    Then $f$ has a vertical asymptote at $x = 5$:
    \begin{align*}
      \lim_{x \to 5^+} \underbrace{\frac{1}{x-5}} = \infty , 
    \end{align*}
because the limit is of the form $ \frac{\#}{0}$, the numerator positive, and the denominator positive and goes to $0$.\\
    But, $\displaystyle \lim_{x \to 5^-} f(x) = 2$.\\
	Therefore, $\lim_{x \to 5} f(x)$ does not exist

  \end{explanation}
\end{enumerate}
\end{problem}



\end{document}
