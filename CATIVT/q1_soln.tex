\documentclass{ximera}

\newcommand{\RR}{\mathbb R}
\renewcommand{\d}{\,d}
\newcommand{\dd}[2][]{\frac{d #1}{d #2}}
\renewcommand{\l}{\ell}
\newcommand{\ddx}{\frac{d}{dx}}
\newcommand{\dfn}{\textbf}
\newcommand{\eval}[1]{\bigg[ #1 \bigg]}
\renewcommand{\theenumii}{\textup{(\roman{enumii})}}
\renewcommand{\labelenumii}{\theenumii}

\usepackage{graphicx}
\usepackage{multicol}
\usepackage{tkz-euclide}
%\usepackage{unicode-math}

\usepackage{pgfplots}   % <- for graphics
\pgfplotsset{compat=newest}


\renewenvironment{freeResponse}{
\ifhandout\setbox0\vbox\bgroup\else
\begin{trivlist}\item[\hskip \labelsep\bfseries Solution:\hspace{2ex}]
\fi}
{\ifhandout\egroup\else
\end{trivlist}
\fi}

\newcommand*{\ZeroOverZero}{\ensuremath{\dfrac{0}{0}}}

\providecommand{\HCCondition}{0}
\newcommand{\WkstHop}[1][1]{\if\HCCondition 0
	\vspace*{\stretch{#1}} \fi} 
\newcommand{\WkstNew}{\if\HCCondition 0
	\newpage
	 \fi} 


\title[Problem 1]{Problem 1}

\begin{document}
\begin{abstract} \end{abstract}
\maketitle


% Extracted from continuityAndIVT.tex, problem #1
\begin{problem}
	\begin{enumerate}
   	\item  Let $f(x) = \frac{x-1}{x^2 - 5x}$.  Then $f(2)=-\frac{1}{6}$ and $f(6)=\frac{5}{6}$, but there is no value of $c$ between $2$ and $6$ for which $f(c)=0$.  Does this fact violate the Intermediate Value Theorem?
\begin{explanation}
        It does not violate the Intermediate Value Theorem.  $f$ is not continuous at 5 so the conditions of the IVT do not hold and therefore the IVT does not apply.
      \end{explanation}

	\item	True or False: At some time since you were born your weight in pounds exactly equaled your height in inches.
\begin{explanation}
        True: if $w(t)$ represents your weight in pounds at time at time $t$ and $h(t)$ represents your height in inches at time $t$, then $w$ and $h$ are both continuous functions.  This implies $w - h$ is also continuous.
        If $t = 0$ is the moment you where born and $t = T_0$ is the present time, then $w(0) - h(0) < 0$ and $w(T_0) - h(T_0) > 0$.
        Hence by the Intermediate Value Theorem there is a point in the past, $t$, when $w(t)-h(t)=0$ and therefore your weight in pounds equaled your height in inches.
      \end{explanation}
	
	\end{enumerate}

\end{problem}



\end{document}
