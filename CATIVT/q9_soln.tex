\documentclass{ximera}

\newcommand{\RR}{\mathbb R}
\renewcommand{\d}{\,d}
\newcommand{\dd}[2][]{\frac{d #1}{d #2}}
\renewcommand{\l}{\ell}
\newcommand{\ddx}{\frac{d}{dx}}
\newcommand{\dfn}{\textbf}
\newcommand{\eval}[1]{\bigg[ #1 \bigg]}
\renewcommand{\theenumii}{\textup{(\roman{enumii})}}
\renewcommand{\labelenumii}{\theenumii}

\usepackage{graphicx}
\usepackage{multicol}
\usepackage{tkz-euclide}
%\usepackage{unicode-math}

\usepackage{pgfplots}   % <- for graphics
\pgfplotsset{compat=newest}


\renewenvironment{freeResponse}{
\ifhandout\setbox0\vbox\bgroup\else
\begin{trivlist}\item[\hskip \labelsep\bfseries Solution:\hspace{2ex}]
\fi}
{\ifhandout\egroup\else
\end{trivlist}
\fi}

\newcommand*{\ZeroOverZero}{\ensuremath{\dfrac{0}{0}}}

\providecommand{\HCCondition}{0}
\newcommand{\WkstHop}[1][1]{\if\HCCondition 0
	\vspace*{\stretch{#1}} \fi} 
\newcommand{\WkstNew}{\if\HCCondition 0
	\newpage
	 \fi} 


\title[Problem 9]{Problem 9}

\begin{document}
\begin{abstract} \end{abstract}
\maketitle


% Extracted from continuityAndIVT.tex, problem #9
\begin{problem}
\begin{enumerate}

\item Given function $f$ on an interval $[a,b]$:

\begin{enumerate}
\item What are the conditions of the Intermediate Value Theorem?
\begin{explanation}
$f$ is continuous on the interval $[a,b]$ 
	\end{explanation}
\item What is the conclusion of the Intermediate Value Theorem?
\begin{explanation}
For any number $L$ strictly between $f(a)$ and $f(b)$, there is at least one number $c$ in $(a,b)$ satisfying $f(c)=L$.  This means that for any $L$ strictly between $f(a)$ and $f(b)$, the horizontal line $y=L$ intersects the graph of $f$. \\\\
	\end{explanation}

\end{enumerate}
	\item Given the four functions on the interval $[1,6]$, answer the questions below. 
	
	\begin{image}
	\includegraphics{figure2.png}
	\end{image}

\begin{enumerate}
	\item For each of the functions I through IV, indicate $f(1)$ and $f(6)$.  Then mark the interval of all numbers strictly between $f(1)$ and $f(6)$, on the y-axis.
\begin{explanation} \hfil
	\begin{image}
	\includegraphics[scale=.6]{figure4.png}
	\end{image}
	\end{explanation}

	\item For each of the functions I through IV, write an interval of all numbers strictly between $f(1)$ and $f(6)$
\begin{explanation}
		$f=I$: $(3,4)$\\
		$f=II$: $(2,3)$\\
		$f=III$: $(1,5)$\\
		$f=IV$: $(1,5)$
		\end{explanation}
	\item List the functions that satisfy the conditions of the Intermediate Value Theorem on $[1,6]$
\begin{explanation}
		Only function IV is continuous on $[1,6]$
		\end{explanation}
	\item For which of the functions is the following statement true: For any number $L$ strictly between $f(1)$ and $f(6)$, there exists a number $c$ in $(1,6)$ satisfying $f(c)=L$.
\begin{explanation}
		The statement is true for functions I and IV
		\end{explanation}

	\item Does the function III satisfy the conclusion of the Intermediate Value Theorem?  Why or why not?
\begin{explanation}
	No, function $III$ does not satisfy the conclusion of the IVT.  For example, take $L=1.5$.  $f(c)=1.5$ has no solution in $(1,6)$.  Draw the horizontal line $y=1.5$.  What do you notice?  This line does not intersect the graph of function III.  What does this mean?  The function does not attain that value, $1.5$.  So there is no $c$ in $(1,6)$ such that $f(c)=1.5$.
	\end{explanation}


	\end{enumerate}
\end{enumerate}

\end{problem}



\end{document}
