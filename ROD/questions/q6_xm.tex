\documentclass{ximera}

\newcommand{\RR}{\mathbb R}
\renewcommand{\d}{\,d}
\newcommand{\dd}[2][]{\frac{d #1}{d #2}}
\renewcommand{\l}{\ell}
\newcommand{\ddx}{\frac{d}{dx}}
\newcommand{\dfn}{\textbf}
\newcommand{\eval}[1]{\bigg[ #1 \bigg]}
\renewcommand{\theenumii}{\textup{(\roman{enumii})}}
\renewcommand{\labelenumii}{\theenumii}

\usepackage{graphicx}
\usepackage{multicol}
\usepackage{tkz-euclide}
%\usepackage{unicode-math}

\usepackage{pgfplots}   % <- for graphics
\pgfplotsset{compat=newest}


\renewenvironment{freeResponse}{
\ifhandout\setbox0\vbox\bgroup\else
\begin{trivlist}\item[\hskip \labelsep\bfseries Solution:\hspace{2ex}]
\fi}
{\ifhandout\egroup\else
\end{trivlist}
\fi}

\newcommand*{\ZeroOverZero}{\ensuremath{\dfrac{0}{0}}}

\providecommand{\HCCondition}{0}
\newcommand{\WkstHop}[1][1]{\if\HCCondition 0
	\vspace*{\stretch{#1}} \fi} 
\newcommand{\WkstNew}{\if\HCCondition 0
	\newpage
	 \fi} 

\title[Problem 6]{Problem 6}

\begin{document}
\begin{abstract} \end{abstract}
\maketitle

% Extracted from rulesOfDifferentiation.tex, problem #6
\begin{problem}
A company is producing cell phones. The cost of producing  $x$ cell phones is given by $C(x)$, defined by
\[
C(x)=-0.01x^2+40x+400, \text{  for } 0\le x<1000.
\]
AVERAGE cost of producing the first $x$ cell phones is given by

\[
C_{AVG}=\frac{C(x)}{x}
\]

If the company has produced $x$ cell phones, the cost of producing one more item is given by

\[
\text{COST of producing one more item}=C(x+1)-C(x)
\]

MARGINAL COST is approximation of the cost of producing one more cell phone

\[
\text{MARGINAL COST}=C'(x)
\]

\begin{enumerate}
\item Compute the average cost of the first $300$ cellphones that the company produces.
\item Compute the cost of producing one more cell phone, if the company has produced $300$ cell phones.
	
\item Compute the marginal cost, if $300$ cell phones have been produced.
\item Why is the marginal cost a good approximation of the cost of producing one more item?  Explain!
\end{enumerate} 
\end{problem}

\end{document}
