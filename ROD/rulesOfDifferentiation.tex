
%Add code to compile both versions from makefile at same time
\providecommand{\HCCondition}{0}
%Define each of the conditions
\ifcase\HCCondition
	%\condition=0 -> handout
	\documentclass[nooutcomes,noauthor,space,handout]{ximera}
	\title{ Rules of differentiation (ROD)} 
\or	%\condition=1 -> Soln
	\documentclass[nooutcomes,noauthor]{ximera}
	\title{ Rules of differentiation (ROD) - Solutions}
\fi

\newcommand{\RR}{\mathbb R}
\renewcommand{\d}{\,d}
\newcommand{\dd}[2][]{\frac{d #1}{d #2}}
\renewcommand{\l}{\ell}
\newcommand{\ddx}{\frac{d}{dx}}
\newcommand{\dfn}{\textbf}
\newcommand{\eval}[1]{\bigg[ #1 \bigg]}
\renewcommand{\theenumii}{\textup{(\roman{enumii})}}
\renewcommand{\labelenumii}{\theenumii}

\usepackage{graphicx}
\usepackage{multicol}
\usepackage{tkz-euclide}
%\usepackage{unicode-math}

\usepackage{pgfplots}   % <- for graphics
\pgfplotsset{compat=newest}


\renewenvironment{freeResponse}{
\ifhandout\setbox0\vbox\bgroup\else
\begin{trivlist}\item[\hskip \labelsep\bfseries Solution:\hspace{2ex}]
\fi}
{\ifhandout\egroup\else
\end{trivlist}
\fi}

\newcommand*{\ZeroOverZero}{\ensuremath{\dfrac{0}{0}}}

\providecommand{\HCCondition}{0}
\newcommand{\WkstHop}[1][1]{\if\HCCondition 0
	\vspace*{\stretch{#1}} \fi} 
\newcommand{\WkstNew}{\if\HCCondition 0
	\newpage
	 \fi}  %% we can turn off input when making a master document
\usepackage{fullpage}

\begin{document}
\begin{abstract}		\end{abstract}
\makeTagTitle

\ifcase\HCCondition
%summary in here

\begin{itemize}
	\item \textbf{The Power Rule}: For any real number $n$, $\displaystyle \ddx x^n = n x^{n-1}$.
	\item \textbf{The Natural Exponential Function}: $\displaystyle \ddx e^x = e^x$.
	\item \textbf{The Sine Function}: $\displaystyle \ddx \sin(x) = \cos(x)$.
		
	\item \textbf{The Sum Rule}: If $f(x)$ and $g(x)$ are differentiable functions and $c$ is a constant, then:
		\begin{itemize}
			\item $\displaystyle \ddx\left( f(x) + g(x) \right) = f'(x) + g'(x)$,
			\item $\displaystyle \ddx\left( f(x) - g(x) \right) = f'(x) - g'(x)$,
			\item $\displaystyle \ddx\left( c \cdot f(x) \right) = c \cdot f'(x)$.			
		\end{itemize}
\end{itemize}
\WkstHop

\section*{Recitation Questions}
\fi



%problem 1
\begin{problem}
For each of the following functions, use the "short cut derivative rules" to compute their derivative.
	
	\begin{enumerate}
	
	\item $f(x) = \sqrt{x}$
	\WkstHop
		\begin{freeResponse}
		$$f(x) = \sqrt{x} = x^{\frac{1}{2}}.$$ 
		\begin{align*}
		f'(x) &= \frac{1}{2} x^{\frac{1}{2} - 1}\\
		&= \frac{1}{2} x^{-\frac{1}{2}} \\
		&= \frac{1}{2\sqrt{x}}
		\end{align*}
		\end{freeResponse}
			

	\item $s(u) = \frac{5}{u^2}$
	\WkstHop
		\begin{freeResponse}
		$$s(u) = \frac{5}{u^2} = 5u^{-2}.$$  
		\begin{align*}
		s'(u) &= 5(-2) u^{-2-1}\\
		 &= -10u^{-3}\\
		&= \frac{-10}{u^3} 
		\end{align*}
		\end{freeResponse}
			
			

	\item $p(t) = t^5 + 4t^3 + \pi $
	\WkstHop
		\begin{freeResponse}
		\begin{align*}
		p'(t) &= 5t^{5-1} + 4(3)t^{3-1} + 0\\ 
		&= 5t^4 + 12t^2
		\end{align*}  
		Note that $\ddx (\pi) = 0$ because $\pi$ is a constant.	
		\end{freeResponse}
			
			
	\end{enumerate}
\end{problem}

\WkstNew
	
%problem2	
\begin{problem}
  Given the polynomial function $q$ defined by $q(v) = 2v^3 - 5v^2 + 7v - 9$ find:
  \begin{enumerate}

    \item
      The slope of the tangent line to the graph of $q$ at the point where $v = 3$ using the limit definition of a derivative.
      \WkstHop
      \begin{freeResponse}
        To compute $q'(3)$ from the limit definition we use either 
        \[
          \lim_{v \to 3} \frac{q(v)-q(3)}{v-3} \quad \text{or} \quad \lim_{h \to 0} \frac{q(3+h)-q(3)}{h}.
        \]

        
        Using the second formula:
        \begin{align*}
          q'(3) &= \lim_{h \to 0} \frac{q(3+h)-q(3)}{h}  \textrm{ which has form } \ZeroOverZero\\
          &= \lim_{h \to 0} \frac{(2(3+h)^3 - 5(3+h)^2 + 7(3+h) - 9) - (54 - 45 + 21 - 9)}{h}\\
          &= \lim_{h \to 0} \frac{(2(27+27h+9h^2+h^3) - 5(9+6h+h^2) + 21+7h - 9) - 21}{h} \\
          &= \lim_{h \to 0} \frac{54+54h+18h^2+2h^3-45-30h-5h^2+21+7h-9-21}{h} \\
          &= \lim_{h \to 0} \frac{31h+13h^2+2h^3}{h} \\
          &= \lim_{h \to 0} (31+13h+2h^2)\\
          &= 31.
        \end{align*}
      \end{freeResponse}


    \item
      The slope of the tangent line to the graph of $q$ at the point where $v = 3$ using the ``short-cut derivative rules'' to find a formula for $q'$ and evaluating $q'(3)$.
\WkstHop
      \begin{freeResponse} \hfil

        \begin{align*}
          q'(v) &= (2v^3 - 5v^2 + 7v - 9)'\\
          &= (2v^3)' + (-5v^2)' + (7v)' + (-9)' \\
          &= 3\cdot 2 \cdot v^{3-1} +  -5\cdot(2)\cdot v^{2-1} +  7\cdot1\cdot v^{1-1} + 0 \\
          &= 6v^2 - 10v + 7\\
          &\implies q'(3) = 6\cdot3^2 - 10\cdot3 + 7 = 31.
        \end{align*}
      \end{freeResponse}


    \item     The equation of the tangent line to the graph of $q$ at $v = 3$.
\WkstHop
      \begin{freeResponse}
        Slope of tangent line to the graph of $q$ at the point where $v=3$: $q'(3) = 6\cdot3^2 - 10\cdot3 + 7 = 31$.

        Point on tangent line at the point where $v=3$: $(3, q(3)) = (3, 21)$.

        Equation of tangent line to the graph of $q$ at the point where $v=3$:
        \begin{align*}
          y-21&=31(x-3)\\
              &\implies y = 31x -72.
        \end{align*}
      \end{freeResponse}

  \end{enumerate}
\end{problem}

			
%problem3			
\begin{problem}
 Find $s'$ of the function $s$ defined by $s(t) = 3t^2 + 5e^t - \frac{1}{t}$.
\WkstHop
  % Find $s'$, $s''$, and $s'''$ of the function $s$ defined by $s(t) = 3t^2 + 5e^t - \frac{1}{t}$.
   \begin{freeResponse}
     $s'(t) = 6t + 5e^t + \frac{1}{t^2}$.

   %  $s''(t) = 6 + 5e^t - \frac{2}{t^3}$.

    % $s'''(t) = 5e^t + \frac{6}{t^4}$.
   \end{freeResponse}

\end{problem}
\WkstNew

%problem4
\begin{problem}

  Use the given graphs of $f$ and $g$ and their accompanying derivatives to answer the following questions.
  \begin{image}
    \hspace*{-7em}
     \includegraphics[scale = 0.3]{graphSquaringFunction.png}
     \hspace{2em}
     \includegraphics[scale = 0.3]{graphSquaringFunctionDeriv.png}
   \end{image}
   \begin{image}
     \includegraphics[scale = 0.23]{graphIdentityFunction.png}
   \end{image}

   \begin{enumerate}
     \item
	Write an equation for the tangent line to $f$ at $x=2$.
\WkstHop
	\begin{freeResponse}
	$f(2)=4$,\quad $f'(2)=4$, \quad $y-4=4(x-2)$, \quad $y=4x-4$
	\end{freeResponse}

	\item Draw the graph of $g'$
	\WkstHop
		\begin{freeResponse}
	   \begin{image}
     \includegraphics[scale = 0.23]{graphIdentityFunctionDeriv.png}
   \end{image}	
		\end{freeResponse}
       \item Compute the value of the $(5f+3g)'(2)$.
\WkstHop
       \begin{freeResponse}
         \begin{align*}
           (5f+3g)'(2) &= 5\cdot f'(2) + 3 \cdot g'(2)\\
                       &= 5 \cdot 4 + 3 \cdot 1 \\
                       &= 20 + 3 = 23.
         \end{align*}
       \end{freeResponse}

	\item Find the equation of the tangent line to the graph of $(5f+3g)$ at the point where $x=2$.
\WkstHop
	\begin{freeResponse}
         \begin{align*}
           (5f+3g)(2) &= 5\cdot f(2) + 3 \cdot g(2)\\
                       &= 5 \cdot 4 + 3 \cdot 2 \\
                       &= 20 + 6 = 26\\\\
	y-26 &=23(x-2)\\
	y&=23x-20
         \end{align*}
	\end{freeResponse}

	\item Use the given graph of $f'$ and $g'$ to find the following:
	\begin{enumerate}
		\item A formula for $f'$
\WkstHop
			\begin{freeResponse}
				$f'(x)=2x$.  The graph shows a linear function through the point $(0,0)$ and $(1,2)$.
			\end{freeResponse}
	
		\item A formula for $g'$
\WkstHop
			\begin{freeResponse}
				$g'(x)=1$.  The graph shows a linear function with slope $0$ and through the point $(0,1)$.
			\end{freeResponse}
	\end{enumerate}

	\item Find the expression for $(5f+3g)'(x)$.
\WkstHop	
	\begin{freeResponse}
         \begin{align*}
           (5f+3g)'(x) &= 5\cdot f'(x) + 3 \cdot g'(x)\\
                       &= 5 \cdot 2x + 3 \cdot 1 \\
                       &= 10x+3
         \end{align*}
	\end{freeResponse}

  

   \end{enumerate}

\end{problem}
\WkstNew
%problem5
\begin{problem}
Sketch the graph of the derivative of the given function:

  \begin{image}
   \hspace*{-3em}
    \includegraphics[scale = 0.2]{graphOfDeriv.png}
  \end{image}
\WkstHop
\begin{freeResponse} \hfil
  \begin{image}
    \hspace*{-3em}
    \includegraphics[scale=.9]{graphOfDerivSolution.png}
  \end{image}

\end{freeResponse}
\end{problem}
\WkstNew

\begin{problem}
A company is producing cell phones. The cost of producing  $x$ cell phones is given by $C(x)$, defined by
\[
C(x)=-0.01x^2+40x+400, \text{  for } 0\le x<1000.
\]
AVERAGE cost of producing the first $x$ cell phones is given by

\[
C_{AVG}=\frac{C(x)}{x}
\]

If the company has produced $x$ cell phones, the cost of producing one more item is given by


\[
\text{COST of producing one more item}=C(x+1)-C(x)
\]

MARGINAL COST is approximation of the cost of producing one more cell phone

\[
\text{MARGINAL COST}=C'(x)
\]
 

\begin{enumerate}
\item Compute the average cost of the first $300$ cellphones that the company produces.
\WkstHop
	\begin{freeResponse}
		The first 300 cell phones costs, on average, 
		
		\[
		C_{AVG}=\frac{C(300)}{300}=38.33
		\]
	\end{freeResponse}
	\item Compute the cost of producing one more cell phone, if the company has produced $300$ cell phones.
	\begin{freeResponse}
\[
C(301)-C(300)=-0.01(301)^2+40(301)+400-(-0.01(300)^2+40(300)+400)=33.99
\]
\end{freeResponse}
\item Compute the marginal cost, if $300$ cell phones have been produced.
\WkstHop
\begin{freeResponse}
\[
\text{MARGINAL COST}=C'(300)=\bigl[0.02x+40\bigr]_{x=300} =34
\]
\end{freeResponse}
\item Why is the marginal cost a good approximation of the cost of producing one more item?  Explain!
\WkstHop
\begin{freeResponse}
		By the definition of the derivative we have that
		
		\[
		C'(300)=\lim_{h \to 0}\dfrac{C(300+h)-C(300)}{h}\approx \dfrac{C(300+h)-C(300)}{h}
		\]
		for any $h$ near $0$. In particular, for $h=1$, we obtain
		\[
		C'(300)\approx \dfrac{C(300+1)-C(300)}{1}=C(301)-C(300),
		\]
Therefore, marginal cost at $x=300 \approx$ the cost of producing one more item!
	\end{freeResponse}
\end{enumerate} 
\end{problem}
	

\end{document} 
 %% \item
       Compute the value of $(5f+3g)''(2)$.
       \begin{freeResponse}
         \begin{align*}
		(5f+3g)''(2)&=[(5f+3g)]'(2)\\
		&=[5f'+3g']'(2)\\
		&=[5f''+3g''](2)
	\end{align*}
	Finding $f''$:
         \begin{align*}
           \mbox{$f'$ is a linear function with slope 2} &\implies f''(x) = 2.
         \end{align*}

         Finding $g''$:
         \begin{align*}
           \mbox{$g'$ is a constant function} &\implies g''(x) = 0.
         \end{align*}
         
         Computing $(5f+3g)''(2)$:
         \begin{align*}
           (5f+3g)''(2) &= 5\cdot f''(2) + 3 \cdot g''(2) \\
                        &= 5 \cdot 2 + 3 \cdot 0 = 10.
         \end{align*}
       \end{freeResponse}

  %% \item
       Compute the value of $(5f+3g)''(2)$.
       \begin{freeResponse}
         \begin{align*}
		(5f+3g)''(2)&=[(5f+3g)]'(2)\\
		&=[5f'+3g']'(2)\\
		&=[5f''+3g''](2)
	\end{align*}
	Finding $f''$:
         \begin{align*}
           \mbox{$f'$ is a linear function with slope 2} &\implies f''(x) = 2.
         \end{align*}

         Finding $g''$:
         \begin{align*}
           \mbox{$g'$ is a constant function} &\implies g''(x) = 0.
         \end{align*}
         
         Computing $(5f+3g)''(2)$:
         \begin{align*}
           (5f+3g)''(2) &= 5\cdot f''(2) + 3 \cdot g''(2) \\
                        &= 5 \cdot 2 + 3 \cdot 0 = 10.
         \end{align*}
       \end{freeResponse}
\begin{problem}
A company is producing cell phones. The cost of producing  $x$ cellphones is given by $C(x)$, defined by
\[
C(x)=-0.01x^2+40x+400, \text{  for } 0\le x<1000.
\]
AVERAGE cost of producing the first $x$ cellphones is given by

\[
C_{AVG}=\frac{C(x)}{x}
\]

If the company has produced $x$ cellphones, the cost of producing one more item is given by


\[
\text{COST of producing one more item}=C(x+1)-C(x)
\]

MARGINAL COST is approximation of the cost of producing one more cellphone

\[
\text{MARGINAL COST}=C'(x)
\]
 

\begin{enumerate}
\item Compute the average cost of the first $300$ cellphones that the company produces.

	\begin{freeResponse}
		The first 300 cellphones costs, on average, 
		
		\[
		C_{AVG}=\frac{C(300)}{300}=38.33
		\]
	\end{freeResponse}
	\item Compute the cost of producing one more cellphone, if the company has produced $300$ cellphones.
	\begin{freeResponse}
\[
C(301)-C(300)=-0.01(301)^2+40(301)+400-(-0.01(300)^2+40(300)+400)=33.99
\]
\end{freeResponse}
\item Compute the marginal cost, if $300$ cellphones have been produced.
\begin{freeResponse}
\[
\text{MARGINAL COST}=C'(300)=\bigl[0.02x+40\bigr]_{x=300} =34
\]
\end{freeResponse}
\item Why is the marginal cost a good approximation of the cost of producing one more item?  Explain!
\begin{freeResponse}
		By the definition of the derivative we have that
		
		\[
		C'(300)=\lim_{h \to 0}\dfrac{C(300+h)-C(300)}{h}\approx \dfrac{C(300+h)-C(300)}{h}
		\]
		for any $h$ near $0$. In particular, for $h=1$, we obtain
		\[
		C'(300)\approx \dfrac{C(300+1)-C(300)}{1}=C(301)-C(300),
		\]
Therefore, marginal cost at $x=300 \approx$ the cost of producing one more item!
	\end{freeResponse}
\end{enumerate} 
\end{problem}
	
