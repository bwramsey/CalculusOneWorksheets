\documentclass{ximera}

\newcommand{\RR}{\mathbb R}
\renewcommand{\d}{\,d}
\newcommand{\dd}[2][]{\frac{d #1}{d #2}}
\renewcommand{\l}{\ell}
\newcommand{\ddx}{\frac{d}{dx}}
\newcommand{\dfn}{\textbf}
\newcommand{\eval}[1]{\bigg[ #1 \bigg]}
\renewcommand{\theenumii}{\textup{(\roman{enumii})}}
\renewcommand{\labelenumii}{\theenumii}

\usepackage{graphicx}
\usepackage{multicol}
\usepackage{tkz-euclide}
%\usepackage{unicode-math}

\usepackage{pgfplots}   % <- for graphics
\pgfplotsset{compat=newest}


\renewenvironment{freeResponse}{
\ifhandout\setbox0\vbox\bgroup\else
\begin{trivlist}\item[\hskip \labelsep\bfseries Solution:\hspace{2ex}]
\fi}
{\ifhandout\egroup\else
\end{trivlist}
\fi}

\newcommand*{\ZeroOverZero}{\ensuremath{\dfrac{0}{0}}}

\providecommand{\HCCondition}{0}
\newcommand{\WkstHop}[1][1]{\if\HCCondition 0
	\vspace*{\stretch{#1}} \fi} 
\newcommand{\WkstNew}{\if\HCCondition 0
	\newpage
	 \fi} 


\title[Problem 1]{Problem 1}

\begin{document}
\begin{abstract} \end{abstract}
\maketitle


% Extracted from lHopitalsRule.tex, problem #1
\begin{problem}
State the form of the limit. Determine whether the form is determinate or indeterminate. Evaluate each limit.  
  \begin{enumerate}
      \item  $\lim_{x \to 0}\frac{\sin(x)-\cos(x)+1}{x^2-x}$
    \begin{explanation}
    The form is $\frac{0}{0}$; indeterminate. we can apply L.R.\\
     $\lim_{x \to 0}\frac{\sin(x)-\cos(x)+1}{x^2-x}=\lim_{x \to 0}\frac{\cos(x)+\sin(x)}{2x-1}=\frac{\cos(0)+\sin(0)}{2(0)-1}=\frac{1}{-1}=-1$

    \end{explanation}
  \item  $\lim_{x \to 0}\frac{e^x-1-x}{x^2}$
    \begin{explanation}
    The form is $\frac{0}{0}$; indeterminate.\\
     $\lim_{x \to 0}\frac{e^x-1-x}{x^2}= \lim_{x \to 0}\frac{e^x-1}{2x}$, by L.R.\\
     The last limit also has  the form $\frac{0}{0}$, so, we can apply L.R. again.\\
     $\lim_{x \to 0}\frac{e^x-1-x}{x^2}= \lim_{x \to 0}\frac{e^x-1}{2x}= \lim_{x \to 0}\frac{e^x}{2}=\frac{1}{2}$
    \end{explanation}
     \item  $\lim_{x \to \infty}\frac{e^x}{x^4}$
    \begin{explanation}
    The form is $\frac{\infty}{\infty}$; indeterminate.\\
    $\lim_{x \to \infty}\frac{e^x}{x^4}=\lim_{x \to \infty}\frac{e^x}{4x^3}$ , by L.R.\\
     The last limit also has  the form $\frac{\infty}{\infty}$. It turns out that we can apply L.R. again, and again, and again:\\
     $\lim_{x \to \infty}\frac{e^x}{x^4}=\lim_{x \to \infty}\frac{e^x}{4x^3}= \lim_{x \to \infty}\frac{e^x}{12x^2}= \lim_{x \to \infty}\frac{e^x}{24x}= \lim_{x \to \infty}\frac{e^x}{24}=\infty$\\
     This result shows that the function $e^x$ grows much faster than $x^4$, as $x$ goes to $\infty$.
        \end{explanation}
  
      \end{enumerate}
      \end{problem}



\end{document}
