%Add code to compile both versions from makefile at same time
\providecommand{\HCCondition}{0}
%Define each of the conditions
\ifcase\HCCondition
	%\condition=0 -> handout
	\documentclass[nooutcomes,noauthor,space,handout]{ximera}
	\title{ L'H\^opital's Rule (LHR)}
\or	%\condition=1 -> Soln
	\documentclass[nooutcomes,noauthor]{ximera}
	\title{ L'H\^opital's Rule (LHR) - Solutions}  
\fi

\newcommand{\RR}{\mathbb R}
\renewcommand{\d}{\,d}
\newcommand{\dd}[2][]{\frac{d #1}{d #2}}
\renewcommand{\l}{\ell}
\newcommand{\ddx}{\frac{d}{dx}}
\newcommand{\dfn}{\textbf}
\newcommand{\eval}[1]{\bigg[ #1 \bigg]}
\renewcommand{\theenumii}{\textup{(\roman{enumii})}}
\renewcommand{\labelenumii}{\theenumii}

\usepackage{graphicx}
\usepackage{multicol}
\usepackage{tkz-euclide}
%\usepackage{unicode-math}

\usepackage{pgfplots}   % <- for graphics
\pgfplotsset{compat=newest}


\renewenvironment{freeResponse}{
\ifhandout\setbox0\vbox\bgroup\else
\begin{trivlist}\item[\hskip \labelsep\bfseries Solution:\hspace{2ex}]
\fi}
{\ifhandout\egroup\else
\end{trivlist}
\fi}

\newcommand*{\ZeroOverZero}{\ensuremath{\dfrac{0}{0}}}

\providecommand{\HCCondition}{0}
\newcommand{\WkstHop}[1][1]{\if\HCCondition 0
	\vspace*{\stretch{#1}} \fi} 
\newcommand{\WkstNew}{\if\HCCondition 0
	\newpage
	 \fi}  





\begin{document}
\begin{abstract}
\end{abstract}
\maketitle

\ifcase\HCCondition
%summary in here
\section*{SUMMARY of L'H\^{o}pital's Rule:}

\subsection*{Indeterminate Forms - Part I}

(1) $\frac{0}{0}$ refers to a limit of the form $\lim_{x \to a} \frac{f(x)}{g(x)}$, where $\lim_{x \to a}f(x)= 0 $ and $\lim_{x \to a}g(x)=0$\\
(2) $\frac{\infty}{\infty}$ refers to a limit of the form $\lim_{x \to a} \frac{f(x)}{g(x)}$, where $\lim_{x \to a}f(x) = \pm \infty$ and $\lim_{x \to a}g(x)=\pm \infty$. (The signs do not need to match.)\\[0.8em]
How do we determine whether the limit of this form  exists and, if it does, \\how do we evaluate the limit? We can apply L'H\^{o}pital's Rule to limits of the form $\frac{0}{0}$  or $\frac{\infty}{\infty}$.\\[0.4em]
\textbf{L'H\^{o}pital's Rule} (LHR)\\[0.3em]
Let $f(x)$ and $g(x)$ be functions that are differentiable near $a$.\\  If
$\lim_{x \to a} f(x) = \lim_{x \to a}g(x) = 0$ 
 and $g'(x) \neq 0$
for all $x$ near $a$, then \\

\hspace{1in}$\lim_{x \to a} \frac{f(x)}{g(x)} = \lim_{x \to a} \frac{f'(x)}{g'(x)}$,

provided that $\lim_{x \to a} \frac{f'(x)}{g'(x)}$ exists.

  L'H\^{o}pital's rule applies also when $\lim_{x \to a}f(x) = \pm \infty$ and $\lim_{x \to a}g(x)=\pm \infty$.\\[1em]

\subsection*{Indeterminate Forms - Part II}

(3) $0\cdot \infty$ refers to a limit of the form $\lim_{x \to a} f(x)\cdot g(x)$, where $\lim_{x \to a}f(x)=0$, and $\lim_{x \to a}g(x)=\pm \infty$\\
(4)$\infty-\infty$ refers to a limit of the form $\lim_{x \to a}(f(x)-g(x))$, where $\lim_{x \to a}f(x)=\lim_{x \to a}g(x)=\infty$\\[1em]
How do we determine whether the limit of this form  exists and, if it does, \\how do we evaluate the limit? By performing convenient algebraic operations, \\we show that the limit is equivalent to a limit of the form $\frac{0}{0}$ or $\frac{\infty}{\infty}$, so that we can apply the LHR.\\[1em]
 
 \subsection*{Indeterminate Forms - Part III} 

(5) $1^{\infty}$ refers to a limit of the form $\lim_{x \to a} f(x)^{ g(x)}$, where $\lim_{x \to a}f(x)=1$, and $\lim_{x \to a}g(x)=\infty$\\
(6) $0^0$ refers to a limit of the form $\lim_{x \to a} f(x)^{ g(x)}$, where $\lim_{x \to a}f(x)=\lim_{x \to a}g(x)=0$\\
(7)  $\infty^{0}$ refers to a limit of the form $\lim_{x \to a} f(x)^{ g(x)}$, where $\lim_{x \to a}f(x)=\infty$, and $\lim_{x \to a}g(x)=0$\\[1em]
How do we determine whether the limit of this form  exists and, if it does, \\how do we evaluate the limit? By performing the following steps

\[\lim_{x \to a} f(x)^{g(x)}=\lim_{x \to a} [e^{\ln{(f(x))}}]^{g(x)}=\lim_{x \to a} e^{ g(x)\ln{(f(x))}}={\large{e}}^{\lim_{x \to a} g(x)\ln{(f(x))}}\]
we end up with the limit of simpler form (in the exponent).

\newpage
\section*{Recitation Questions}
\fi



\begin{problem}
State the form of the limit. Determine whether the form is determinate or indeterminate. Evaluate each limit.  
  \begin{enumerate}
      \item  $\lim_{x \to 0}\frac{\sin(x)-\cos(x)+1}{x^2-x}$
    \WkstHop \begin{freeResponse}
    The form is $\frac{0}{0}$; indeterminate. we can apply L.R.\\
     $\lim_{x \to 0}\frac{\sin(x)-\cos(x)+1}{x^2-x}=\lim_{x \to 0}\frac{\cos(x)+\sin(x)}{2x-1}=\frac{\cos(0)+\sin(0)}{2(0)-1}=\frac{1}{-1}=-1$

    \end{freeResponse}
  \item  $\lim_{x \to 0}\frac{e^x-1-x}{x^2}$
    \WkstHop \begin{freeResponse}
    The form is $\frac{0}{0}$; indeterminate.\\
     $\lim_{x \to 0}\frac{e^x-1-x}{x^2}= \lim_{x \to 0}\frac{e^x-1}{2x}$, by L.R.\\
     The last limit also has  the form $\frac{0}{0}$, so, we can apply L.R. again.\\
     $\lim_{x \to 0}\frac{e^x-1-x}{x^2}= \lim_{x \to 0}\frac{e^x-1}{2x}= \lim_{x \to 0}\frac{e^x}{2}=\frac{1}{2}$
    \end{freeResponse}
     \item  $\lim_{x \to \infty}\frac{e^x}{x^4}$
    \WkstHop \begin{freeResponse}
    The form is $\frac{\infty}{\infty}$; indeterminate.\\
    $\lim_{x \to \infty}\frac{e^x}{x^4}=\lim_{x \to \infty}\frac{e^x}{4x^3}$ , by L.R.\\
     The last limit also has  the form $\frac{\infty}{\infty}$. It turns out that we can apply L.R. again, and again, and again:\\
     $\lim_{x \to \infty}\frac{e^x}{x^4}=\lim_{x \to \infty}\frac{e^x}{4x^3}= \lim_{x \to \infty}\frac{e^x}{12x^2}= \lim_{x \to \infty}\frac{e^x}{24x}= \lim_{x \to \infty}\frac{e^x}{24}=\infty$\\
     This result shows that the function $e^x$ grows much faster than $x^4$, as $x$ goes to $\infty$.
        \end{freeResponse}
  
      \end{enumerate}
      \end{problem} \WkstNew
%problem 1
\begin{problem}

State the form of the limit. Determine whether the form is determinate or indeterminate. Evaluate each limit.  
  \begin{enumerate}
  \item  $\lim_{x \to \infty}\left( \ln (1 + e^{-x}) \right)^x $
    \WkstHop \begin{freeResponse}
      Since $\lim_{x \to \infty} (1 + e^{-x}) = 1 + 0 = 1$ and $\ln (1) = 0$, this limit is of the form $0^{\infty}$.
      This is a determinate form.
     $\lim_{x \to \infty} \left( \ln (1 + e^{-x}) \right)^x = 0$.
    \end{freeResponse}
    
  \item  $\lim_{x \to \infty} \left( \frac{1}{x} + 1 \right)^{\frac{1}{x}} $
    \WkstHop \begin{freeResponse}
      This limit is of the form $1^0$, which is a determinate form.\\
      Thus, $\lim_{x \to \infty} \left( \frac{1}{x} + 1 \right)^{\frac{1}{x}} = 1 $
    \end{freeResponse}
    
  \item  $\lim_{x \to \infty} \left( \frac{\arctan(x)}{x} \right) $
    \WkstHop \begin{freeResponse}
      Since $\lim_{x \to \infty} \arctan(x) = \frac{\pi}{2}$, this limit is of the form $\frac{\text{\#}}{\infty}$, which is a determinate form.\\[1em]
      Thus, $\lim_{x \to \infty} \left( \frac{\arctan(x)}{x} \right) = 0$  
    \end{freeResponse}
    \WkstNew
    
  \item  $\lim_{x \to \infty} (x-\ln(x)) $
    \WkstHop \begin{freeResponse}
      This limit is of the form $\infty - \infty$, which is an indeterminate form.  We can rewrite this as:
   
      $$\lim_{x \to \infty} (x-\ln(x))= \lim_{x \to \infty} \left(x\left(1-\frac{\ln(x)}{x}\right)\right)$$
     
     We can see that $\lim_{x \to \infty}\left(\frac{\ln(x)}{x}\right)$ is of the form $\frac{\infty}{\infty}$,  so we can use L'Hopital's Rule.
     
      $$\stackrel{L.R.}{\implies} \lim_{x \to \infty}\left(\frac{1/x}{1}\right)=\lim_{x \to \infty}\left(\frac{1}{x}\right)=0$$
      
      Now we have: 
      $$\lim_{x \to \infty} \left(x\left(1-\frac{\ln(x)}{x}\right)\right)$$
      This limit has the form $\infty \cdot 1$.  This is a determinate form, and, therefore,

	 $\lim_{x \to \infty} (x-\ln(x)) =\infty$
   
    \end{freeResponse}

  \item  $\lim_{x \to \infty} \left( x \ln \left( \frac{1}{x} \right) \right) $
    \WkstHop \begin{freeResponse}
      As $x$ approaches $\infty$, $\frac{1}{x}$ approaches $0$ from the right.
      So  $$\lim_{x \to \infty} \ln \left( \frac{1}{x} \right) = - \infty $$
      Therefore, the limit in question is of the form $\infty \cdot - \infty$, which is a determinate form.\\
      Thus, 
      $$ \lim_{x \to \infty} \left( x \ln \left( \frac{1}{x} \right) \right) = - \infty $$
    \end{freeResponse}

  \item  $\lim_{x \to 0^+} (\sin(x) \cot(x) ) $
    \WkstHop \begin{freeResponse}
      Since $\lim_{x \to 0^+} \cot(x) = \infty$, this limit is of the form $0 \cdot \infty$.
      This is an indeterminate form.
      
     Note: $\cot(x) = \frac{\cos(x) }{\sin(x)}$.  So
      $$\lim_{x \to 0^+} (\sin(x) \cot(x) ) = \lim_{x \to 0^+} \cos(x) = 1 .$$
    \end{freeResponse}
  \end{enumerate}
\end{problem} \WkstNew


%problem 2

%problem 3
\begin{problem}

  Circle the correct answer in each part:
  \begin{enumerate}
    \item
      Consider the limit $\lim_{x \to 0} (\cos(x))^{\sin(x)}$.
      \begin{enumerate}
        \item
          Evaluate the limit.
         
          \begin{enumerate}
            \item
              the limit DNE
            \item
              $e$
             \item
              $1$
            \item
              $\infty$
            \item
              $-\infty$
            \item
              $0$
            \item
              none of the previous answers is correct
          \end{enumerate}
           \WkstHop \begin{freeResponse}
            \textbf{The correct choice is (iii).}

            Evaluation of limit:
            \begin{align*}
              \lim_{x \to 0} \underbrace{(\cos(x))^{\sin(x)}}_{\text{form $1^0$}} &= 1
            \end{align*}
          \end{freeResponse}
        \item
          What Limit Law, rule or technique did you use to find this limit?
                    \begin{enumerate}
            \item
              The Squeeze Theorem;
            \item
              L'H\^{o}pital's Rule;
            \item
              The Product Law;
            \item
              evaluated the function at $x = 0$, since the function is continuous at $x = 0$;
            \item
              none of the previous answers is correct
          \end{enumerate}
          \WkstHop \begin{freeResponse}
            The correct choice is (iv).
          \end{freeResponse}

      \end{enumerate}
\WkstNew

    \item
      Evaluate the limit $\lim_{x \to 4^-} \frac{\ln(x)}{x - 4}$.
  
      \begin{enumerate}
        \item
          the limit DNE
        \item
          $e$
        \item
          $1$
        \item
          $\infty$
        \item
          $-\infty$
        \item
          $0$
        \item
          none of the previous answers is correct
      \end{enumerate}
  \WkstHop \begin{freeResponse}
        The correct choice is (v).

        Evaluation of limit:
  \(
          \lim_{x \to 4^-}  \frac{\ln(x)}{x - 4}= - \infty, 
 \)
 \\[1em]
 \text{since the limit is of the form $\dfrac{\#}{0}$, and since  the numerator  is positive and the denominator negative.}
      \end{freeResponse}

    \item
      Evaluate the limit $\lim_{x \to \infty} \frac{\ln(x)}{x - 4}$.
           \begin{enumerate}
        \item
          the limit DNE
        \item
          $e$
        \item
          $1$
        \item
          $\infty$
        \item
          $-\infty$
        \item
          $0$
        \item
          none of the previous answers is correct
      \end{enumerate}
 \WkstHop \begin{freeResponse}
        The correct choice is (vi).

        Evaluation of limit:
        \begin{align*}
          \lim_{x \to \infty} \underbrace{\frac{\ln(x)}{x - 4}}_{\text{form $\infty/\infty$}} &\stackrel{L.H.}{=} \lim_{x \to \infty} \frac{1/x}{1} \\
          &= 0
        \end{align*}

      \end{freeResponse}

\WkstNew
    \item
      Consider the limit\\[1em]
       $\lim_{h \to 0} \frac{(2+h)^3 - 8}{h} = f'(2)$.\\[1em]
      Determine the function $f$.
    
      \begin{enumerate}
       \item
         such a function DNE;
       \item
         $f(x) = x^3$;
       \item
         $f(x) = (2 + x)^3$;
       \item
         $f(x) = \frac{(2+x)^3}{x}$;
       \item
         none of the previous answers is correct
      \end{enumerate}
        \WkstHop \begin{freeResponse}
        The correct choice is (ii).
      \end{freeResponse}
    \item
      Consider the limit: $\lim_{x \to 0^+} \left( \frac{\sin(x)}{x} \right)^{|\ln(x)|}$.\\[1em]
      Determine the form of this limit.
           \begin{enumerate}
       \item
         $\frac{0}{0}$;
       \item
         $\frac{\infty}{\infty}$;
       \item
         $1^0$;
       \item
         $0^0$;
       \item
         $1^\infty$;
       \item
         $\infty^\infty$;
       \item
         none of the previous answers is correct
      \end{enumerate}
       \WkstHop \begin{freeResponse}
        The correct choice is (v).
      \end{freeResponse}

  \end{enumerate}
\end{problem} \WkstNew



%problem 4
\begin{problem}
  Determine the following limits.
  Use L'H\^{o}pital's Rule if applicable.
  \begin{enumerate}
  
    %part a  
  \item  $\lim_{x \to \infty} \frac{x}{\sqrt{x^2 + 1}}  $

    \WkstHop \begin{freeResponse}
    
    This limit is of the form:  $\frac{\infty}{\infty}$. L.R. is applicable, but we don't need to apply it.
      \begin{align*}
        \lim_{x \to \infty} \frac{x}{\sqrt{x^2 + 1}} &= \lim_{x \to \infty} \frac{x}{\sqrt{x^2 \left(1 + \frac{1}{x^2} \right)}} \\
                                                     &=  \lim_{x \to \infty} \frac{x}{|x| \sqrt{1 + \frac{1}{x^2} }} \\
                                                     &=  \lim_{x \to \infty} \frac{x}{x \sqrt{1 + \frac{1}{x^2} }} \\
                                                     &=  \lim_{x \to \infty} \frac{1}{\sqrt{1 + \frac{1}{x^2} }} \\
                                                     &= \frac{1}{\sqrt{1 + 0}} = 1
      \end{align*}
    \end{freeResponse}
    
    
    %part b
  \item  $\lim_{x \to - \infty} x^2 e^x $
    \WkstHop \begin{freeResponse}
    This limit is of the form: $\infty \cdot 0$
      \begin{align*}
        \lim_{x \to - \infty} x^2 e^x &= \lim_{x \to - \infty} \frac{x^2}{ e^{-x}} \; \left( \text{of the form } \frac{\infty}{\infty} \right) \\
                                      &\stackrel{L.R.}{=}  \lim_{x \to - \infty} \frac{2x}{- e^{-x}} \; \left( \text{of the form } \frac{\infty}{\infty} \right) \\
                                      &\stackrel{L.R.}{=}  \lim_{x \to - \infty} \frac{2}{ e^{-x}}  \\
                                      &= 0
      \end{align*}
      where ``L.R.'' above an equals sign means that that equality is due to ``L'H\^{o}pital's Rule''.  
    \end{freeResponse}
    
    

    % part c
  \item  $\lim_{x \to \infty} x^{\frac{1}{x}} $
    \WkstHop \begin{freeResponse}
    This limit is of the form: $\infty^0$
      \begin{align*}
        \lim_{x \to \infty} x^{\frac{1}{x}} &= \lim_{x \to \infty} e^{\ln \left( x^{\frac{1}{x}} \right) } \\
                                            &= \lim_{x \to \infty} e^{\frac{1}{x} \ln x } \\
                                            &= e^{ \lim_{x \to \infty} \frac{\ln x}{x} } \; \left( \text{limit is of the form } \frac{\infty}{\infty} \right) \\
                                            &\stackrel{L.R.}{=} e^{\lim_{x \to \infty}\frac{\frac{1}{x}}{1}} \\
                                            &= e^{\lim_{x \to \infty} \frac{1}{x}} \\
                                            &= e^0 = 1
      \end{align*}
    \end{freeResponse}
    \WkstNew
    %part d
    \item $\lim_{x \to \infty} \left(1+\frac{2}{x}\right)^x$
        \WkstHop \begin{freeResponse}
    This limit is of the form: $1^\infty$
     \begin{align*}
        \lim_{x \to \infty} \left(1+\frac{2}{x}\right)^x &= \lim_{x \to \infty} e^{\ln \left(\left(1+\frac{2}{x}\right)^x \right) } \\
                                            &= \lim_{x \to \infty} e^{x \ln \left(1+\frac{2}{x}\right) } \\
                                            &= e^{ \lim_{x \to \infty} \frac{\ln \left(1+\frac{2}{x}\right)}{1/x} } \; \left( \text{limit is of the form } \frac{0}{0} \right) \\
                                            &\stackrel{L.R.}{=} e^{\lim_{x \to \infty}\frac{-2x^{-2}\cdot \frac{1}{(1+2/x)}}{-x^{-2}}} \\
                                            &= e^{\lim_{x \to \infty} \left(2\left(\frac{1}{1+\frac{2}{x}}\right)\right)} \\
                                            &= e^2 
        \end{align*}
    \end{freeResponse}
    
    
        %part e
    \item $\lim_{\theta \to 0^+} \left(\sin(\theta)\right)^{\tan(\theta)}$
        \WkstHop \begin{freeResponse}
    This limit is of the form: $0^0$
    
     \begin{align*}
        \lim_{\theta \to 0^+} \left(\sin(\theta)\right)^{\tan\theta} &= \lim_{\theta \to 0^+} e^{\ln \left((\sin(\theta))^{\tan(\theta)} \right) } \\
                                            &= \lim_{\theta \to 0^+} e^{\tan(\theta) \ln (\sin(\theta)) } \\
                                            &= e^{ \lim_{\theta \to 0^+} \frac{\ln (\sin(\theta))}{\cot(\theta)} } \; \left( \text{limit is of the form } \frac{\infty}{\infty} \right) \\
                                            &\stackrel{L.R.}{=} e^{\lim_{\theta \to 0^+}\frac{\cos(\theta) \cdot \frac{1}{\sin(\theta)}}{-\csc^2(\theta)}} \\
                                            &= e^{\lim_{\theta \to 0^+} \left(\frac{\cos(\theta)}{\sin(\theta)}\cdot\frac{-\sin^2(\theta)}{1}\right)} \\
                                              &= e^{\lim_{\theta \to 0^+} (-\cos(\theta)\cdot\sin(\theta))} \\
                                            &= e^0=1
        \end{align*}
    \end{freeResponse}
    

    
  \end{enumerate}
\end{problem} \WkstNew

%problem5
\begin{problem}
Let $f$ and $g$ be two functions such that $\lim_{x\to\infty}f(x)=\lim_{x\to\infty}g(x)=\infty$.\\[1em]

We say that $f$   \textbf{grows faster} than $g$ as $x$ goes to $\infty$  if  $\lim_{x\to\infty}\frac{f(x)}{g(x)}=\infty$, or, equivalently, if\\

 $\lim_{x\to\infty}\frac{g(x)}{f(x)}=0$.\\

If the limit exists, namely, if  $\lim_{x\to\infty}\frac{f(x)}{g(x)}=M$, for some positive number $M$,\\

 then we say that the functions $f$ and $g$ have \textbf{comparable growth rates}.\\

In other words, we compare the growth rates of  functions $f$ and $g$ by computing the limit of the form $\frac{\infty}{\infty}$: $\lim_{x\to\infty}\frac{f(x)}{g(x)}$ .\\

For each of the following pairs of functions, determine which of the pair grows faster or state that they have comparable growth rates.  Justify your answer by computing the limit $\lim_{x\to\infty}\frac{f(x)}{g(x)}$.



\begin{enumerate}
	\item$b^x; x^x; b>1$
		\WkstHop \begin{freeResponse}
		$\lim_{x \to \infty}\frac{b^x}{x^x}=\lim_{x \to \infty}\left(\frac{b}{x}\right)^x=0$.  Since this limit is of the form $0^{\infty}$, which is a determinate form.
		Therefore, $x^x$ grows faster.
		
		\end{freeResponse}
	
	\item $x^x; \left(\frac{x}{e}\right)^x$
		\WkstHop \begin{freeResponse}
		$\lim_{x \to \infty}\frac{x^x}{(x/e)^x}=\lim_{x \to \infty}\frac{x^x}{\frac{x^x}{e^x}}=\lim_{x \to \infty}{x^x}\cdot{\frac{e^x}{x^x}}=\lim_{x \to \infty}{e^x}=\infty$  Therefore, $x^x$ grows faster.
		
		\end{freeResponse}
	

\WkstNew
	
	\item $x^3;  x^3 \cdot \ln(x)$
		\WkstHop \begin{freeResponse}
		$\lim_{x \to \infty}\frac{x^3}{x^3 \cdot \ln(x)}=\lim_{x \to \infty}\frac{1}{\ln(x)}=0$.  
		Therefore, $x^3 \cdot \ln(x)$ grows faster.
		
		\end{freeResponse}
	\item $a^x; b^x; 0<a<b$
	\WkstHop \begin{freeResponse}
		$\lim_{x \to \infty}\frac{a^x}{b^x}=\lim_{x \to \infty}\left(\frac{a}{b}\right)^x=0$
		Therefore, $b^x$ grows faster
	
	
	\end{freeResponse}
	\item $\log_a(x); \log_b(x); 1<a<b$
	\WkstHop \begin{freeResponse}
	$\lim_{x \to \infty}\frac{\log_a(x)}{\log_b(x)}$\\
	By L'Hopital's Rule: $\lim_{x \to \infty}\frac{\frac{1}{x\ln(a)}}{\frac{1}{x\ln(b)}}=\lim_{x \to \infty}\left(\frac{1}{x\ln(a)} \cdot \frac{x\ln(b)}{1}\right)=\frac{\ln b}{\ln a}$\\
	Therefore, $\log_a(x)$ and $\log_b(x)$ grow at comparable rates.
	
		\end{freeResponse}
	\item $\ln^{3} (x); x^{1/2}$
		\WkstHop \begin{freeResponse}
	$\lim_{x \to \infty}\frac{\ln^{3}(x)}{x^{1/2}}$\\
	By L'Hopital's Rule:$\lim_{x \to \infty}\frac{\ln^{3}(x)}{x^{1/2}}=\lim_{x \to \infty}\frac{3\ln^2 (x) \cdot (1/x)}{(1/2)x^{-1/2}}=\lim_{x \to \infty}\frac{6 \cdot \ln^2 (x)}{x^{1/2}}$\\
	By L'Hopital's Rule:$\lim_{x \to \infty}\frac{12\ln (x) \cdot (1/x)}{(1/2)x^{-1/2}}=\lim_{x \to \infty}\frac{24 \cdot \ln (x)}{x^{1/2}}$\\
	By L'Hopital's Rule: $\lim_{x \to \infty}\frac{48}{x^{1/2}}=0$
	Therefore, $x^{1/2}$ grows faster
		\end{freeResponse}

\WkstNew	
		%part g	
	\item $x; \ln(x)\sqrt{x}$
		\WkstHop \begin{freeResponse}
	
	$\lim_{x \to \infty}\frac{x}{\ln (x)\sqrt{x}}=\lim_{x \to \infty}\frac{x^{1/2}}{\ln(x)}$\\
	By L'Hopital's Rule:$\lim_{x \to \infty}\frac{(1/2)x^{-1/2}}{x^{-1}}=\lim_{x \to \infty}{(1/2)x^{1/2}}=\infty$\\
		Therefore, $x$ grows faster
		\end{freeResponse}
		
		%part h
		\item Challenge: $x^{40}; 1.004^x$ (Hint: Use the substitution $x=\ln(t)$.)
			\WkstHop \begin{freeResponse}
			\begin{align*}
			\lim_{x \to \infty}\frac{x^{40}}{1.004^x} &=\lim_{t \to \infty}\frac{[\ln(t)]^{40}}{1.004^{\ln(t)}}\\
			&=\lim_{t \to \infty}\frac{\ln(t)^{40}}{{e^{\ln(1.004)}}^{\ln(t)}}\\
			&=\lim_{t \to \infty}\frac{\ln(t)^{40}}{t^{\ln(1.004)}}\\
			&=\lim_{t \to \infty}\left(\frac{\ln(t)}{t^{1.004/40}}\right)^{40}\\
			&=\left[\lim_{t \to \infty}\left(\frac{\ln(t)}{t^{1.004/40}}\right)\right]^{40}\\
			&\stackrel{L.R.}{=} \left[\lim_{t \to \infty}\frac{\frac{1}{t}}{(1.004/40)t^{(1.004/40)-1}}\right]^{40}\\
			&=\left[\lim_{t \to \infty}\frac{1}{(1.004/40)t^{((1.004/40)-1+1)}}\right]^{40}\\
			&=0^{40}\\
			&=0	
				\end{align*}
		Therefore, $1.004^x$ grows faster.
		\end{freeResponse}
	%part i
	\item Put the following functions in order of growth rate.\\
	$$x^3 \cdot \ln(x), \ln^{3} (x), x^x, 1.004^x, x^{40}, x^3$$
	
	\WkstHop \begin{freeResponse}
	
	$$ \ln^{3} (x)<<x^{3}<<x^3 \cdot \ln(x)<< x^{40}<< 1.004^x <<x^x$$
	
	\end{freeResponse}
	
\end{enumerate}
\end{problem} \WkstNew
\end{document} 
\begin{problem}

  True or False: You can use L'H\^{o}pital's Rule to compute
  $\lim_{x \to 0} \frac{|x|}{x}$.
  \WkstHop \begin{freeResponse}
    False.
    The function $|x|$ is not differentiable at $x=0$, and so L'H\^{o}pital's Rule is not applicable.
  \end{freeResponse}
\end{problem} \WkstNew
