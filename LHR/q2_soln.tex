\documentclass{ximera}

\newcommand{\RR}{\mathbb R}
\renewcommand{\d}{\,d}
\newcommand{\dd}[2][]{\frac{d #1}{d #2}}
\renewcommand{\l}{\ell}
\newcommand{\ddx}{\frac{d}{dx}}
\newcommand{\dfn}{\textbf}
\newcommand{\eval}[1]{\bigg[ #1 \bigg]}
\renewcommand{\theenumii}{\textup{(\roman{enumii})}}
\renewcommand{\labelenumii}{\theenumii}

\usepackage{graphicx}
\usepackage{multicol}
\usepackage{tkz-euclide}
%\usepackage{unicode-math}

\usepackage{pgfplots}   % <- for graphics
\pgfplotsset{compat=newest}


\renewenvironment{freeResponse}{
\ifhandout\setbox0\vbox\bgroup\else
\begin{trivlist}\item[\hskip \labelsep\bfseries Solution:\hspace{2ex}]
\fi}
{\ifhandout\egroup\else
\end{trivlist}
\fi}

\newcommand*{\ZeroOverZero}{\ensuremath{\dfrac{0}{0}}}

\providecommand{\HCCondition}{0}
\newcommand{\WkstHop}[1][1]{\if\HCCondition 0
	\vspace*{\stretch{#1}} \fi} 
\newcommand{\WkstNew}{\if\HCCondition 0
	\newpage
	 \fi} 


\title[Problem 2]{Problem 2}

\begin{document}
\begin{abstract} \end{abstract}
\maketitle


% Extracted from lHopitalsRule.tex, problem #2
\begin{problem}

State the form of the limit. Determine whether the form is determinate or indeterminate. Evaluate each limit.  
  \begin{enumerate}
  \item  $\lim_{x \to \infty}\left( \ln (1 + e^{-x}) \right)^x $
    \begin{explanation}
      Since $\lim_{x \to \infty} (1 + e^{-x}) = 1 + 0 = 1$ and $\ln (1) = 0$, this limit is of the form $0^{\infty}$.
      This is a determinate form.
     $\lim_{x \to \infty} \left( \ln (1 + e^{-x}) \right)^x = 0$.
    \end{explanation}
    
  \item  $\lim_{x \to \infty} \left( \frac{1}{x} + 1 \right)^{\frac{1}{x}} $
    \begin{explanation}
      This limit is of the form $1^0$, which is a determinate form.\\
      Thus, $\lim_{x \to \infty} \left( \frac{1}{x} + 1 \right)^{\frac{1}{x}} = 1 $
    \end{explanation}
    
  \item  $\lim_{x \to \infty} \left( \frac{\arctan(x)}{x} \right) $
    \begin{explanation}
      Since $\lim_{x \to \infty} \arctan(x) = \frac{\pi}{2}$, this limit is of the form $\frac{\text{\#}}{\infty}$, which is a determinate form.\\[1em]
      Thus, $\lim_{x \to \infty} \left( \frac{\arctan(x)}{x} \right) = 0$  
    \end{explanation}
    \item  $\lim_{x \to \infty} (x-\ln(x)) $
    \begin{explanation}
      This limit is of the form $\infty - \infty$, which is an indeterminate form.  We can rewrite this as:
   
      $$\lim_{x \to \infty} (x-\ln(x))= \lim_{x \to \infty} \left(x\left(1-\frac{\ln(x)}{x}\right)\right)$$
     
     We can see that $\lim_{x \to \infty}\left(\frac{\ln(x)}{x}\right)$ is of the form $\frac{\infty}{\infty}$,  so we can use L'Hopital's Rule.
     
      $$\stackrel{L.R.}{\implies} \lim_{x \to \infty}\left(\frac{1/x}{1}\right)=\lim_{x \to \infty}\left(\frac{1}{x}\right)=0$$
      
      Now we have: 
      $$\lim_{x \to \infty} \left(x\left(1-\frac{\ln(x)}{x}\right)\right)$$
      This limit has the form $\infty \cdot 1$.  This is a determinate form, and, therefore,

	 $\lim_{x \to \infty} (x-\ln(x)) =\infty$
   
    \end{explanation}

  \item  $\lim_{x \to \infty} \left( x \ln \left( \frac{1}{x} \right) \right) $
    \begin{explanation}
      As $x$ approaches $\infty$, $\frac{1}{x}$ approaches $0$ from the right.
      So  $$\lim_{x \to \infty} \ln \left( \frac{1}{x} \right) = - \infty $$
      Therefore, the limit in question is of the form $\infty \cdot - \infty$, which is a determinate form.\\
      Thus, 
      $$ \lim_{x \to \infty} \left( x \ln \left( \frac{1}{x} \right) \right) = - \infty $$
    \end{explanation}

  \item  $\lim_{x \to 0^+} (\sin(x) \cot(x) ) $
    \begin{explanation}
      Since $\lim_{x \to 0^+} \cot(x) = \infty$, this limit is of the form $0 \cdot \infty$.
      This is an indeterminate form.
      
     Note: $\cot(x) = \frac{\cos(x) }{\sin(x)}$.  So
      $$\lim_{x \to 0^+} (\sin(x) \cot(x) ) = \lim_{x \to 0^+} \cos(x) = 1 .$$
    \end{explanation}
  \end{enumerate}
\end{problem}



\end{document}
