\documentclass{ximera}

\newcommand{\RR}{\mathbb R}
\renewcommand{\d}{\,d}
\newcommand{\dd}[2][]{\frac{d #1}{d #2}}
\renewcommand{\l}{\ell}
\newcommand{\ddx}{\frac{d}{dx}}
\newcommand{\dfn}{\textbf}
\newcommand{\eval}[1]{\bigg[ #1 \bigg]}
\renewcommand{\theenumii}{\textup{(\roman{enumii})}}
\renewcommand{\labelenumii}{\theenumii}

\usepackage{graphicx}
\usepackage{multicol}
\usepackage{tkz-euclide}
%\usepackage{unicode-math}

\usepackage{pgfplots}   % <- for graphics
\pgfplotsset{compat=newest}


\renewenvironment{freeResponse}{
\ifhandout\setbox0\vbox\bgroup\else
\begin{trivlist}\item[\hskip \labelsep\bfseries Solution:\hspace{2ex}]
\fi}
{\ifhandout\egroup\else
\end{trivlist}
\fi}

\newcommand*{\ZeroOverZero}{\ensuremath{\dfrac{0}{0}}}

\providecommand{\HCCondition}{0}
\newcommand{\WkstHop}[1][1]{\if\HCCondition 0
	\vspace*{\stretch{#1}} \fi} 
\newcommand{\WkstNew}{\if\HCCondition 0
	\newpage
	 \fi} 


\title[Problem 3]{Problem 3}

\begin{document}
\begin{abstract} \end{abstract}
\maketitle


% Extracted from applicationsOfIntegrals.tex, problem #3
\begin{problem}
	Assume that the {\it rate of change} (in dollars per day) of the price of shares of stock
	in the WeSaySo Company (with $t$ in days) is modeled by the equation $r(t) = -3t^2+30t-63$
	(note that this is technically a {\it discrete} function, but prices change so often with stocks that modeling this with a continuous function makes sense).
	Assume also that the price of a share of stock on day $1$ (i.e., $t=1$) is $\$51$.
	Answer the following questions:
	\begin{enumerate}

	\item  Find the rate of change of price at $t=5$.
		\begin{explanation}
			$r(5) = -3(25) + 30(5) - 63 = -75+150-63=12 \text{ dollars/day}$.
		\end{explanation}




	\item  Find the price of a share of stock at $t=5$.
		\begin{explanation}
			Let $p(t)$ denote the price of a share of stock at any time $t$.
			Then notice that $r(t) = p^\prime (t)$.
			We will first find $p(t)$ for general $t$, and then substitute $t=5$ to solve this problem.
			\begin{align*}
				p(t) &= \int_1^t r(s) \d s + p(1)  \\
					&= \int_1^t \left( -3s^2 + 30s - 63 \right) \d s + 51  \\
					&= \eval{ - s^3 + 15s^2 - 63s}_1^t + 51  \\
					&= \left( - t^3 + 15t^2 - 63t \right) - (-1+15-63) + 51  \\
				&= -t^3 + 15t^2 - 63t + 100.
			\end{align*}
			So, $p(5) = -125+375-315+100=35\$.$
		\end{explanation}




	\item  How fast is the rate of change of price changing at $t=5$?
		\begin{explanation}
			$r'(t) = -6t + 30$.  So, $r'(5) = -30+30=0$.
		\end{explanation}

		\item  How much did the price of a share of stock change in the first $6$ days (i.e., on $[1,6]$)?
		\begin{explanation}
			$p(6) - p(1) = (-216+540-378+100) - 51 = -5\$. $
		\end{explanation}


		

	\item  What was the greatest rate of change of price during the first $6$ days (i.e., on $[1,6]$)?
		[3]
		\begin{explanation}
			This question wants us to maximize $r(t)$ on the closed interval $[1,6]$.
			So we need to find all critical points of $r(t)$ on $[1,6]$.
			$$ r'(t) = -6t+30:=0 \quad \Longrightarrow \quad t=5  $$
			Then, using the closed interval method, we simply plug $t=1,5,6$ into $r(t)$ and check which has the greatest output.
			\begin{align*}
				&r(1) = -3+30-63=-36  \\
				&r(5) = -75+150-63=12  \\
				&r(6) = -108+180-63=9.
			\end{align*}
			Thus, the greatest rate of change of price is $12$ dollars/day when $t=5$.
		\end{explanation}




	\item  What was the greatest price of a share of stock during the first $6$ days (i.e., on $[1,6]$)?
		[3]
		\begin{explanation}
			This question wants us to maximize $p(t)$ on the closed interval $[1,6]$.
			So we need to find all critical points of $p(t)$ on $[1,6]$.
			But note that $p'(t) = r(t)$.  So critical points of $p(t)$ are just roots of $r(t)$.
			$$ r(t) = 0 $$
			$$ -3t^2+30t-63 = 0 $$
			$$ -3(t^2-10t+21)=0 $$
			$$ -3(t-3)(t-7) = 0 $$
			$$ t=3,7 \quad \Longrightarrow \quad t=3 $$
			since $7$ is not in the interval $[1,6]$.
			So we use the Closed Interval Method:
			\begin{align*}
				&p(1) = -1+15-63+100 = 51  \\
				&p(3) = -27 + 135 - 189 + 100 = 19  \\
				&p(6) = -216+540-378+100=46  \\
			\end{align*}
			Thus, the greatest price of the stock is $\$51$ when $t=1$.
		\end{explanation}
	\end{enumerate}
\end{problem}



\end{document}
