\documentclass{ximera}

\newcommand{\RR}{\mathbb R}
\renewcommand{\d}{\,d}
\newcommand{\dd}[2][]{\frac{d #1}{d #2}}
\renewcommand{\l}{\ell}
\newcommand{\ddx}{\frac{d}{dx}}
\newcommand{\dfn}{\textbf}
\newcommand{\eval}[1]{\bigg[ #1 \bigg]}
\renewcommand{\theenumii}{\textup{(\roman{enumii})}}
\renewcommand{\labelenumii}{\theenumii}

\usepackage{graphicx}
\usepackage{multicol}
\usepackage{tkz-euclide}
%\usepackage{unicode-math}

\usepackage{pgfplots}   % <- for graphics
\pgfplotsset{compat=newest}


\renewenvironment{freeResponse}{
\ifhandout\setbox0\vbox\bgroup\else
\begin{trivlist}\item[\hskip \labelsep\bfseries Solution:\hspace{2ex}]
\fi}
{\ifhandout\egroup\else
\end{trivlist}
\fi}

\newcommand*{\ZeroOverZero}{\ensuremath{\dfrac{0}{0}}}

\providecommand{\HCCondition}{0}
\newcommand{\WkstHop}[1][1]{\if\HCCondition 0
	\vspace*{\stretch{#1}} \fi} 
\newcommand{\WkstNew}{\if\HCCondition 0
	\newpage
	 \fi} 


\title[Problem 2]{Problem 2}

\begin{document}
\begin{abstract} \end{abstract}
\maketitle


% Extracted from applicationsOfIntegrals.tex, problem #2
\begin{problem}

Suppose that $r(t) = r_0 e^{-kt}$ (with $k>0$) is the rate at which a nation extracts oil.
The current rate of extraction is $r(0) = 10^7$ barrels/yr.
Also assume that the estimate of the total oil reserve (ie, the amount of oil remaining beneath the ground in this country) is $2 \times 10^9$ barrels.

	\begin{enumerate}

	\item  Find $A(t)$, the total amount of oil extracted by the nation after $t$ years.
		\begin{explanation}
			$A(t) =  \int_0^t r(s) \d s$
			\begin{align*}
				A(t) &= \int_0^t r(s) \d s  \\
					&= \int_0^t r_0 e^{-ks} \d s  \\
					&= - \frac{r_0}{k} \eval{e^{-ks}}_0^t  \\
					&= - \frac{r_0}{k} \left( e^{-kt} - 1 \right)  \\
					&= - \frac{1}{k} 10^7 \left(e^{-kt}-1 \right)  \\
			\end{align*}
		\end{explanation}

	\item  Evaluate $\lim_{t \to \infty} A(t)$ and explain the meaning of this limit.
		\begin{explanation}
			\begin{align*}
			\lim_{t \to \infty} A(t) &= \lim_{t \to \infty} - \frac{1}{k} 10^7 \left(e^{-kt}-1 \right)  \\
			&= - \frac{1}{k} 10^7 (0-1)  \\
			&= \frac{1}{k} 10^7.
			\end{align*}
		\end{explanation}
		%		
	\item  Find the minimum constant $k$ for which the total oil reserves will last forever.
		\begin{explanation}
		For the oil reserves to last forever, we need that
			\begin{align*}
			&\lim_{t \to \infty} A(t) \leq 2 \times 10^9  \\
			 &\Longleftrightarrow \qquad \frac{1}{k} 10^7 \leq 2 \times 10^9  \\
			 &\Longleftrightarrow \qquad \frac{1}{k} \leq 2 \times 10^2 = 200  \\
			 &\Longleftrightarrow \qquad \frac{1}{200} \leq k.
			\end{align*}
		So the minimum value for $k$ is $\frac{1}{200}$.
		\end{explanation}

	\item  Suppose that the  constant k is half the minimum value found in part (c). When will the nation deplete its oil reserve?
		\begin{explanation}
			First note that $k = \frac{1}{2} \cdot \frac{1}{200} = \frac{1}{400}$.
			We want to find the value of $t$ such that:
			\begin{equation*}
				A(t) = 2 \times 10^9
			\end{equation*}
			\begin{equation*}
				- 400 \times 10^7 \left(e^{-\frac{1}{400}t}-1 \right) = 2 \times 10^9
			\end{equation*}
			\begin{equation*}
				\left(e^{-\frac{1}{400}t}-1 \right) = \frac{2 \times 10^9}{-400 \times 10^7} = - \frac{1}{2}
			\end{equation*}
			\begin{equation*}
				e^{-\frac{1}{400}t} = \frac{1}{2}
			\end{equation*}
			\begin{equation*}
				-\frac{1}{400}t=\ln \left(\frac{1}{2} \right) = -\ln(2)
			\end{equation*}
			\begin{equation*}
				t = 400 \ln(2) \approx 277.259 \text{ years}.
			\end{equation*}
		\end{explanation}
	\end{enumerate}
\end{problem}



\end{document}
