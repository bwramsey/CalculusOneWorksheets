%Add code to compile both versions from makefile at same time
\providecommand{\HCCondition}{0}
%Define each of the conditions
\ifcase\HCCondition
	%\condition=0 -> handout
	\documentclass[nooutcomes,noauthor,space,handout]{ximera}
	\title{Applications of Integrals (AOI)}
\or	%\condition=1 -> Soln
	\documentclass[nooutcomes,noauthor]{ximera}
	\title{Applications of Integrals (AOI) - Solutions}  
\fi

\usepackage{fullpage}

\newcommand{\RR}{\mathbb R}
\renewcommand{\d}{\,d}
\newcommand{\dd}[2][]{\frac{d #1}{d #2}}
\renewcommand{\l}{\ell}
\newcommand{\ddx}{\frac{d}{dx}}
\newcommand{\dfn}{\textbf}
\newcommand{\eval}[1]{\bigg[ #1 \bigg]}
\renewcommand{\theenumii}{\textup{(\roman{enumii})}}
\renewcommand{\labelenumii}{\theenumii}

\usepackage{graphicx}
\usepackage{multicol}
\usepackage{tkz-euclide}
%\usepackage{unicode-math}

\usepackage{pgfplots}   % <- for graphics
\pgfplotsset{compat=newest}


\renewenvironment{freeResponse}{
\ifhandout\setbox0\vbox\bgroup\else
\begin{trivlist}\item[\hskip \labelsep\bfseries Solution:\hspace{2ex}]
\fi}
{\ifhandout\egroup\else
\end{trivlist}
\fi}

\newcommand*{\ZeroOverZero}{\ensuremath{\dfrac{0}{0}}}

\providecommand{\HCCondition}{0}
\newcommand{\WkstHop}[1][1]{\if\HCCondition 0
	\vspace*{\stretch{#1}} \fi} 
\newcommand{\WkstNew}{\if\HCCondition 0
	\newpage
	 \fi} 



\begin{document}
\begin{abstract}
\end{abstract}
\makeTagTitle

\ifcase\HCCondition
%summary in here
\section*{SUMMARY: Velocity, Speed, Position, Displacement, Distance}

\textbf{Position}, $s(t)$, of an object at time $t$:
\hspace{0.5in}$s(t)= s(a) + \int_{a}^{ t} v(z) dz $\\[1em]
\textbf{Displacement}, $\Delta s = s(b)-s(a)$,\\ of an object over the time interval $[a,b]$:
\hspace{0.4in}$s(b)- s(a) = \int_{a}^{ b} v(t) dt $\\[1em]
\textbf{Distance} traveled by an object\\ over the time interval $[a,b]$:
\hspace{1.5in}$ \int_{a}^{ b} |v(t)| dt $\\[1.5em]


\section*{SUMMARY: Rate of Accumulation, Amount, Change in the Amount}

The amount, $A(t)$, of some\\ substance/population 
 at the time $t$:\hspace{0.72in}$A(t)= A(a) + \int_{a}^{ t} A'(z) dz $\\[1em]
 The amount, $A(b)$,\\
over the time interval  $[a,b]$:\hspace{1.2in}$A(b)= A(a) + \int_{a}^{ b} A'(t) dt $\\[1em]
The change in the amount, $A(b)-A(a)$,\\
over the time interval  $[a,b]$:\hspace{1.2in}$A(b)- A(a) = \int_{a}^{ b} A'(t) dt $\\[1.3em]

\section*{SUMMARY: Average Value, the Mean Value Theorem for Integrals}

\textbf{Average value} $\bar{f}$,\\
 of the function $f$ on the interval $[a,b]$:\hspace{0.6in}$\bar{f} = \frac{1}{b-a}\int_{a}^{ b} f(x) dx $\\[1.3em]
 \textbf{Mean Value Theorem for Integrals}\\[1em]
 Let $f$ be continuous on $[a,b]$. There exists a value $c$ in $(a,b)$ such that\\[1em]
                                       \[
                                  \frac{1}{b-a}\int_{a}^{ b} f(x) dx=f(c)
                                       \]\\[1em]
      \textbf{Note 1:} $ f(c)=\bar{f}$, the average value of $f$ on $[a,b]$.\\[1em]  
       \textbf{Note 2:} The net area of the region between the curve $y=f(x)$ and the $x-$axis is given by \\

      \[
       \int_{a}^{ b} f(x) dx=f(c)(b-a)
       \] 

\newpage
       
\section*{Recitation Questions}
\fi


\begin{problem}

Solve the following word problems:

	\begin{enumerate}

	\item  The velocity function for an object moving along a line east/west is given by
	$v(t) = -t^2 + 4t - 3$ feet per minute.

		\begin{enumerate}

		\item  Find the total displacement the object traveled from $2$ minutes to $6$ minutes (assume east is positive).
		\WkstHop
			\begin{freeResponse}
				The object's total displacement is given by $\int_2^6 v(t) \d t$.  So we compute:
					\begin{align*}
						\int_2^6 v(t) \d t &= \int_2^6 (-t^2 + 4t - 3) \d t  \\
							&= \eval{- \frac{1}{3} t^3 + 2t^2 - 3t}_2^6  \\
							&= (-72 + 72 - 18) - \left( - \frac{8}{3} + 8 - 6 \right)  \\
							&= \frac{8}{3} - 20 = - \frac{52}{3}.
						\end{align*}
				So the object's displacement is $\frac{52}{3}$ feet west of its original location.
			\end{freeResponse}

		\item  Find the total distance the object traveled from $2$ minutes to $6$ minutes.
		\WkstHop
			\begin{freeResponse}
				First notice that $v(t) = -(t^2 - 4t + 3) = -(t-1)(t-3)$.  So we can see that
					\begin{align*}
						v(t) &> 0 \text{ when }  2 \leq t < 3.  \\
						v(t) &< 0 \text{ when } 3 < t \leq 6.
					\end{align*}
					Thus, the total distance that the object traveled from $2$ minutes to $6$ minutes is:
					\begin{align*}
						\int_2^6 \left| v(t) \right| \d t &= \int_2^3 \left| v(t) \right| \d t + \int_3^6 \left| v(t) \right| \d t  \\
							&= \int_2^3 v(t) \d t + \int_3^6 - v(t) \d t  \\
							&= \int_2^3 (-t^2 + 4t - 3) \d t - \int_3^6 (-t^2 + 4t - 3) \d t  \\
							&= \eval{- \frac{1}{3} t^3 + 2t^2 - 3t}_2^3 - \eval{- \frac{1}{3} t^3 + 2t^2 - 3t}_3^6  \\
							&= \left( (-9+18-9) - (- \frac{8}{3} + 8 - 6) \right) -  \\
							& \left( (-72+72-18)-(-9+18-9) \right)  \\
							&= \left(0 - 2 + \frac{8}{3} \right) - \left( -18 - 0 \right)  \\
							&=  16 + \frac{8}{3} = \frac{56}{3}.
					\end{align*}
				So, the total distance that the object traveled is $\frac{56}{3}$ feet.
			\end{freeResponse}

		\item  Suppose that the object's position $2$ minutes into the trip is $5$ feet of a placement marker.  What is its position (relative to the placement marker) at $6$ minutes.
		\WkstHop
			\begin{freeResponse}
				$s(6) = s(2) + \int_2^6 v(t) \d t = 5 + \left(- \frac{52}{3} \right) = - \frac{37}{3}. $

				So the object's position at $6$ minutes is $\frac{37}{3}$ feet west of the placement marker.
			\end{freeResponse}
		\end{enumerate}
		
		\WkstNew
		

	\item  Sammy the Snail sets up camp in the median of I-70 and, starting at noon and ending at 6pm, hikes back and forth along the highway.  He starts his hike at his campsite.  His velocity at time $t$ hours (after noon)  is given by $v(t)=(t-2)(t-5)$ inches per hour.  Find the total distance Sammy travelled on his hike.
		\WkstHop
		\begin{freeResponse}
			The total distance that Sammy travels is $\int_0^6 \left| v(t) \right| \d t$.
			The following picture indicates where $v(t)$ is positive and negative:
				\begin{image}
				\includegraphics{figure16.png}
				\end{image}
				So we compute:
				\begin{align*}
					\int_0^6 \left| v(t) \right| \d t &= \int_0^2 \left| v(t) \right| \d t + \int_2^5 \left| v(t) \right| \d t + \int_5^6 \left| v(t) \right| \d t  \\
						&= \int_0^2 v(t) \d t - \int_2^5 v(t) \d t + \int_5^6 v(t) \d t  \\
						&= \int_0^2 (t^2 - 7t + 10) \d t - \int_2^5 (t^2 - 7t + 10) \d t + \int_5^6 (t^2 - 7t + 10) \d t  \\
						&= \eval{\frac{1}{3}t^3-\frac{7}{2}t^2+10t}_0^2-\eval{\frac{1}{3}t^3-\frac{7}{2}t^2+10t}_2^5+\eval{\frac{1}{3}t^3-\frac{7}{2}t^2+10t}_5^6  \\
						&= \left( \left(\frac{8}{3}-14+20 \right)-0\right)-\left( \left( \frac{125}{3}-\frac{175}{2}+50 \right)-\left( \frac{8}{3}+6 \right) \right)+  \\
						&\left( \left( 72-126+60 \right) - \left( \frac{125}{3} - \frac{175}{2} + 50 \right) \right)  \\
						&= -82+175-78=15.
				\end{align*}
			So Sammy has traveled a distance of $15$ inches.
		\end{freeResponse}
	\end{enumerate}
\end{problem}

\WkstNew

\begin{problem}

Suppose that $r(t) = r_0 e^{-kt}$ (with $k>0$) is the rate at which a nation extracts oil.
The current rate of extraction is $r(0) = 10^7$ barrels/yr.
Also assume that the estimate of the total oil reserve (ie, the amount of oil remaining beneath the ground in this country) is $2 \times 10^9$ barrels.

	\begin{enumerate}

	\item  Find $A(t)$, the total amount of oil extracted by the nation after $t$ years.
		\WkstHop
		\begin{freeResponse}
			$A(t) =  \int_0^t r(s) \d s$
			\begin{align*}
				A(t) &= \int_0^t r(s) \d s  \\
					&= \int_0^t r_0 e^{-ks} \d s  \\
					&= - \frac{r_0}{k} \eval{e^{-ks}}_0^t  \\
					&= - \frac{r_0}{k} \left( e^{-kt} - 1 \right)  \\
					&= - \frac{1}{k} 10^7 \left(e^{-kt}-1 \right)  \\
			\end{align*}
		\end{freeResponse}

	\item  Evaluate $\lim_{t \to \infty} A(t)$ and explain the meaning of this limit.
		\WkstHop
		\begin{freeResponse}
			\begin{align*}
			\lim_{t \to \infty} A(t) &= \lim_{t \to \infty} - \frac{1}{k} 10^7 \left(e^{-kt}-1 \right)  \\
			&= - \frac{1}{k} 10^7 (0-1)  \\
			&= \frac{1}{k} 10^7.
			\end{align*}
		\end{freeResponse}
		\WkstNew
		
%		
	\item  Find the minimum constant $k$ for which the total oil reserves will last forever.
		\WkstHop
		\begin{freeResponse}
		For the oil reserves to last forever, we need that
			\begin{align*}
			&\lim_{t \to \infty} A(t) \leq 2 \times 10^9  \\
			 &\Longleftrightarrow \qquad \frac{1}{k} 10^7 \leq 2 \times 10^9  \\
			 &\Longleftrightarrow \qquad \frac{1}{k} \leq 2 \times 10^2 = 200  \\
			 &\Longleftrightarrow \qquad \frac{1}{200} \leq k.
			\end{align*}
		So the minimum value for $k$ is $\frac{1}{200}$.
		\end{freeResponse}

	\item  Suppose that the  constant k is half the minimum value found in part (c). When will the nation deplete its oil reserve?
		\WkstHop

		\begin{freeResponse}
			First note that $k = \frac{1}{2} \cdot \frac{1}{200} = \frac{1}{400}$.
			We want to find the value of $t$ such that:
			\begin{equation*}
				A(t) = 2 \times 10^9
			\end{equation*}
			\begin{equation*}
				- 400 \times 10^7 \left(e^{-\frac{1}{400}t}-1 \right) = 2 \times 10^9
			\end{equation*}
			\begin{equation*}
				\left(e^{-\frac{1}{400}t}-1 \right) = \frac{2 \times 10^9}{-400 \times 10^7} = - \frac{1}{2}
			\end{equation*}
			\begin{equation*}
				e^{-\frac{1}{400}t} = \frac{1}{2}
			\end{equation*}
			\begin{equation*}
				-\frac{1}{400}t=\ln \left(\frac{1}{2} \right) = -\ln(2)
			\end{equation*}
			\begin{equation*}
				t = 400 \ln(2) \approx 277.259 \text{ years}.
			\end{equation*}
		\end{freeResponse}
	\end{enumerate}
\end{problem}


		\WkstNew


\begin{problem}
	Assume that the {\it rate of change} (in dollars per day) of the price of shares of stock
	in the WeSaySo Company (with $t$ in days) is modeled by the equation $r(t) = -3t^2+30t-63$
	(note that this is technically a {\it discrete} function, but prices change so often with stocks that modeling this with a continuous function makes sense).
	Assume also that the price of a share of stock on day $1$ (i.e., $t=1$) is $\$51$.
	Answer the following questions:
	\begin{enumerate}

	\item  Find the rate of change of price at $t=5$.
		\WkstHop
		\begin{freeResponse}
			$r(5) = -3(25) + 30(5) - 63 = -75+150-63=12 \text{ dollars/day}$.
		\end{freeResponse}




	\item  Find the price of a share of stock at $t=5$.
		\WkstHop
		\begin{freeResponse}
			Let $p(t)$ denote the price of a share of stock at any time $t$.
			Then notice that $r(t) = p^\prime (t)$.
			We will first find $p(t)$ for general $t$, and then substitute $t=5$ to solve this problem.
			\begin{align*}
				p(t) &= \int_1^t r(s) \d s + p(1)  \\
					&= \int_1^t \left( -3s^2 + 30s - 63 \right) \d s + 51  \\
					&= \eval{ - s^3 + 15s^2 - 63s}_1^t + 51  \\
					&= \left( - t^3 + 15t^2 - 63t \right) - (-1+15-63) + 51  \\
				&= -t^3 + 15t^2 - 63t + 100.
			\end{align*}
			So, $p(5) = -125+375-315+100=35\$.$
		\end{freeResponse}




	\item  How fast is the rate of change of price changing at $t=5$?
		\WkstHop
		\begin{freeResponse}
			$r'(t) = -6t + 30$.  So, $r'(5) = -30+30=0$.
		\end{freeResponse}

		\WkstNew



	\item  How much did the price of a share of stock change in the first $6$ days (i.e., on $[1,6]$)?
		\WkstHop
		\begin{freeResponse}
			$p(6) - p(1) = (-216+540-378+100) - 51 = -5\$. $
		\end{freeResponse}


		

	\item  What was the greatest rate of change of price during the first $6$ days (i.e., on $[1,6]$)?
		\WkstHop[3]
		\begin{freeResponse}
			This question wants us to maximize $r(t)$ on the closed interval $[1,6]$.
			So we need to find all critical points of $r(t)$ on $[1,6]$.
			$$ r'(t) = -6t+30:=0 \quad \Longrightarrow \quad t=5  $$
			Then, using the closed interval method, we simply plug $t=1,5,6$ into $r(t)$ and check which has the greatest output.
			\begin{align*}
				&r(1) = -3+30-63=-36  \\
				&r(5) = -75+150-63=12  \\
				&r(6) = -108+180-63=9.
			\end{align*}
			Thus, the greatest rate of change of price is $12$ dollars/day when $t=5$.
		\end{freeResponse}




	\item  What was the greatest price of a share of stock during the first $6$ days (i.e., on $[1,6]$)?
		\WkstHop[3]
		\begin{freeResponse}
			This question wants us to maximize $p(t)$ on the closed interval $[1,6]$.
			So we need to find all critical points of $p(t)$ on $[1,6]$.
			But note that $p'(t) = r(t)$.  So critical points of $p(t)$ are just roots of $r(t)$.
			$$ r(t) = 0 $$
			$$ -3t^2+30t-63 = 0 $$
			$$ -3(t^2-10t+21)=0 $$
			$$ -3(t-3)(t-7) = 0 $$
			$$ t=3,7 \quad \Longrightarrow \quad t=3 $$
			since $7$ is not in the interval $[1,6]$.
			So we use the Closed Interval Method:
			\begin{align*}
				&p(1) = -1+15-63+100 = 51  \\
				&p(3) = -27 + 135 - 189 + 100 = 19  \\
				&p(6) = -216+540-378+100=46  \\
			\end{align*}
			Thus, the greatest price of the stock is $\$51$ when $t=1$.
		\end{freeResponse}
	\end{enumerate}
\end{problem}


\WkstNew


%problem 1
\begin{problem}
	A cup of coffee has temperature $20 + 75e^{-.02t}$ degrees (Celsius) $t$ minutes after being poured into a cup.  
      What is the average temperature of the coffee during the first half hour?
		\WkstHop
      \begin{freeResponse}
        Let $T(t) := 20 + 75e^{-.02t}$.  We want the average value of $T$ on $[0,30]$.
        \begin{align*}
          T_{\text{avg}} &= \frac{1}{30-0} \int_0^{30} \left( 20 + 75e^{-.02t} \right) \d t  \\
                         &= \frac{1}{30} \eval{20t - \frac{75}{.02} e^{-.02t}}_0^{30}  \\
                         &= \frac{1}{30} \eval{20t - 3750 e^{-.02t}}_0^{30}  \\
                         &= \frac{1}{30} \left[ \left( 20(30) - 3750e^{-0.6} \right) - (0 - 3750) \right]  \\
                         &= \frac{1}{30} \left( 4350 - 3750e^{-0.6} \right)  \\
                         &= 145 - 125e^{-0.6} 
        \end{align*}
        So the average temperature of the coffee during the first half hour is $145 - 125e^{-0.6} \approx 76.4$ degrees Celsius.
      \end{freeResponse}
    

\end{problem}

\WkstNew


%problem 2
\begin{problem}
	The graph of a function $f$ defined on the interval $[0,6]$  is given in the figure.
  
        \begin{image}
          \includegraphics[scale = 0.6]{figure5.png}
        \end{image}
     \begin{enumerate}
	     \item Compute the net area of the region between the graph of $f$ and the $x$-axis, on the interval $[0,6]$.
		\WkstHop[3]
		     \begin{freeResponse}
		     The net area $= -1/2 + 1/2 +1 +3+1=5$
		     \end{freeResponse}
        
     \item Draw a rectangle with base on the $x$-axis, $0\leq x \leq 6$, whose area is equal to the net area in part (a).
  		\WkstHop
     	\begin{freeResponse}
	     Let $A$ be the area of the rectangle.  Then $A=$(base)(height)$=6h=5$.  \\
	     So, $6h=5 \implies h=\frac{5}{6}$
	            \begin{image}
  			        \includegraphics[scale = 0.5]{figure6.png}
        		\end{image}
     	\end{freeResponse}
     
       \item In the figure, mark a point $c$ in $(0,6)$ such that $f(c)$ is the height of the rectangle from part (b).
		\WkstHop
	     \begin{freeResponse}
    		$f(c)=h=\frac{5}{6}$
            \begin{image}
          		\includegraphics[scale = 0.5]{figure7.png}
        	\end{image}
        	Note: In this example, there is only one such point $c$.  In some cases there may be more than one.
     	\end{freeResponse}
     
     
     \item Using the figure and parts (a-c), what is the relationship between the rectangle from part (b), the 
     	net area from part (a), and the average value of $f$ on $[0, 6]$?
		\WkstHop[3]
	      \begin{freeResponse}
    		    The Mean Value Theorem for integrals states that there exists a point $d$ in $(a, b)$ such that
		        \[
		          f(d) = \frac{1}{b-a}\int_a^b f(x) \d x.
		        \]
		        The right-hand side of this equation is the average value of $f$ on $[a, b]$.
		
		        Rewriting we have
		        \[
		        f(d) \cdot (b-a) = \int_a^b f(x) \d x.
		        \]
		      The right-hand side of this equation is the net area found in part (a): $\int_0^6 f(x) \d x=5$.\\
		       The left-hand side of this equation is the area of the rectangle in part (b).  The point $d$ is the point $c$ we found in part (c).  From (b) and (c), $f(c)=\frac{5}{6}$ is the average value of $f$ on $[0,6]$.
		 \end{freeResponse}
    \end{enumerate}
\end{problem}


\WkstNew


%problem 3
\begin{problem}

  Find all points at which the given function equals its average value on the given interval.
  \begin{enumerate}
    \item $f(x) = e^x	\qquad	[0,4]$
		\WkstHop
      \begin{freeResponse}
        First, we need to find $f_{\text{avg}}$:
        \begin{align*}
          f_{\text{avg}} &= \frac{1}{4-0} \int_0^4 e^x \d x  \\
                         &= \frac{1}{4} \eval{e^x}_0^4  \\
                         &= \frac{1}{4} \left( e^4 - 1 \right)  
        \end{align*}
        So we are looking for all values $c \in [0,4]$ such that:
        \begin{align*}
          f(c) &= \frac{1}{4} (e^4 - 1)  \\
          \Longrightarrow \quad e^c &= \frac{1}{4} (e^4 - 1)  \\
          \Longrightarrow \quad c &= \ln \left( \frac{1}{4} (e^4 - 1) \right)
        \end{align*}
        Therefore, our answer is $\ln \left( \frac{1}{4} (e^4 - 1) \right)$.
      \end{freeResponse}

    \item $g(x) = \frac{\pi}{4} \sin(x)	\qquad	[0,\pi]$
		\WkstHop
      \begin{freeResponse}
        First, we need to find $g_{\text{avg}}$:
        \begin{align*}
          g_{\text{avg}} &= \frac{1}{\pi-0} \int_0^{\pi} \frac{\pi}{4} \sin(x) \d x  \\
                         &= \frac{1}{4} \eval{-\cos(x)}_0^{\pi}  \\
                         &= \frac{1}{4} \left( - (-1-1) \right)  \\
                         &= \frac{1}{2}
        \end{align*}
        So we are looking for all values $c \in [0,\pi]$ such that:
        \begin{align*}
          &g(c) = \frac{1}{2}  \\
          &\Longrightarrow \quad \frac{\pi}{4} \sin(c) = \frac{1}{2}  \\
          &\Longrightarrow \quad \sin(c) = \frac{2}{\pi}  \\
          &\Longrightarrow \quad c = \arcsin \left( \frac{2}{\pi} \right), \pi - \arcsin \left( \frac{2}{\pi} \right)
          &\end{align*}
        Therefore, our two answers are $\arcsin \left( \frac{2}{\pi} \right), \pi - \arcsin \left( \frac{2}{\pi} \right)$.
    \end{freeResponse}
  \end{enumerate}
\end{problem}


\WkstNew


\begin{problem}
	 Find the average value of the function $g(t) = 4t e^{-t^2}$ on the interval $[0, 3]$.
		\WkstHop
	\begin{freeResponse}
		We know $\displaystyle g_{\text{avg}} = \frac{1}{b-a} \int_a^{b} g(t) \d t$ with $[a, b]$ as $[0,3]$. We'll make a substitution, $w = t^2$ with $dw = 2t dt$. Notice that when $t = 0$, $w = (0)^2 = 0$ and when $t=3$, $w=(3)^2 = 9$.
	        \begin{align*}
		          g_{\text{avg}} &= \frac{1}{3-0} \int_0^{3} 4 t e^{-t^2} \d t  \\
                         &= \frac{1}{3}  \int_0^9 2 e^{-w} \d w\\
                         &= \frac{1}{3} \eval{-2e^{-w}}_{0}^{9}\\
                         &= \frac{1}{3} \left[ (-2 e^{-9}) - (-2 e^{0}) \right] \\
                         &= \frac{1}{3} \left( 2 - \frac{2}{e^9} \right) 
        \end{align*}
	\end{freeResponse}
\end{problem}



\end{document}

