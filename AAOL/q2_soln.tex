\documentclass{ximera}

\newcommand{\RR}{\mathbb R}
\renewcommand{\d}{\,d}
\newcommand{\dd}[2][]{\frac{d #1}{d #2}}
\renewcommand{\l}{\ell}
\newcommand{\ddx}{\frac{d}{dx}}
\newcommand{\dfn}{\textbf}
\newcommand{\eval}[1]{\bigg[ #1 \bigg]}
\renewcommand{\theenumii}{\textup{(\roman{enumii})}}
\renewcommand{\labelenumii}{\theenumii}

\usepackage{graphicx}
\usepackage{multicol}
\usepackage{tkz-euclide}
%\usepackage{unicode-math}

\usepackage{pgfplots}   % <- for graphics
\pgfplotsset{compat=newest}


\renewenvironment{freeResponse}{
\ifhandout\setbox0\vbox\bgroup\else
\begin{trivlist}\item[\hskip \labelsep\bfseries Solution:\hspace{2ex}]
\fi}
{\ifhandout\egroup\else
\end{trivlist}
\fi}

\newcommand*{\ZeroOverZero}{\ensuremath{\dfrac{0}{0}}}

\providecommand{\HCCondition}{0}
\newcommand{\WkstHop}[1][1]{\if\HCCondition 0
	\vspace*{\stretch{#1}} \fi} 
\newcommand{\WkstNew}{\if\HCCondition 0
	\newpage
	 \fi} 


\title[Problem 2]{Problem 2}

\begin{document}
\begin{abstract} \end{abstract}
\maketitle


% Extracted from anApplicationOfLimits.tex, problem #2
\begin{problem}
 Part of the given parabola can be used to model the ``position-time'' graph of a ball thrown straight up into the air.  The graph gives the height of the ball in feet $t$ seconds after being thrown into the air.\\
  Let the function $f$ be defined by  $f(t)=-16t^2+128t+144$.\\
  Use this graph, and the given function, $f$, to answer the following questions.
  \begin{image}
    \includegraphics[scale = 0.5]{figure2.png}
  \end{image}
	
	
		\begin{enumerate}
		\item Mark the part of the parabola that can be used to model the position of the ball.		
		\begin{explanation} \hfil
       		 \begin{center}
       	  	 \includegraphics[scale = 0.4]{ptGraphOfBall.png}
        		\end{center}
		\end{explanation}		

		 \item  What are the units on the $t$ axis?  What are the units on the $y$ axis?
		 \begin{explanation}		 
	The units on the $t$ axis are ``seconds'' (for time), while the units on the $y$ axis are ``feet'' (for height).        
		\end{explanation}
			
		\item  If you were watching a movie of the ball being thrown, is the graph a picture of the path that the ball follows?  Why or why not?
		\begin{explanation}		 
	 No, the position-time graph is \emph{not} the path the ball follows.
        The graph shows the height of the ball at a given \emph{time}.
        The ball is thrown straight up and has no horizontal movement so its path is on a vertical line.
		\end{explanation}
	
		\item Let $f(t)$ denote the height of the ball at time $t, t\geq 0$.  What is the height of the ball at time $t=0$?
		\begin{explanation}
		The height can be found by finding $f(0)$
		\begin{align*}
			f(0)&=-16(0)^2+128(0)+144 \\
			f(0)&=144\  \text{feet}
 			\end{align*}

		\end{explanation}
		\item  When will the ball hit the ground?
		\begin{explanation}		 
		The ball will hit the ground when the height $f(t)$ equals zero.
			\begin{align*}
			0&=-16t^2+128t+144 \\
			0&=-16(t^2-8t-9) \\
 			0&=-16(t+1)(t-9) \\
 			t&=-1, 9 
 			\end{align*}
		The ball will hit the ground at $t=9$ or 9 seconds after the ball is thrown into the air.  
		\end{explanation}

		\item  What is the domain of the position function, $f$, of the ball?

		 \begin{explanation}
     	   The domain of $f$ is the interval $[0, 9]$.   The ball is thrown straight up at $t=0$ and hits the ground at $t=9$.
     	   With this domain, the position-time graph of the ball is given by
       		 \begin{center}
       	  	 \includegraphics[scale = 0.4]{ptGraphOfBall.png}
        		\end{center}
     	 \end{explanation}

		
		
		\item Use the table of values to find the average velocity of the ball between $t=8.9$ and $t=9$ seconds.
	\[
\begin{array}{|l|l|}
			\hline
			t & \approx f(t)  \\
			\hline
			8.9 & 15.84  \\
			\hline
			8.99 & 1.6  \\
			\hline
			8.999 & 0.159984  \\
			\hline
			8.9999 &  0.015998  \\
			\hline
			9 &  0  \\
			\hline
			\end{array}
		\]
  \begin{explanation}
    The average velocity of the ball between $t = 8.9$ seconds and $t = 9$ seconds is
    \[
       \frac{f(9) - f(8.9)}{9- 8.9} = \frac{0- 15.84}{0.1} = -158.4\  \text{feet per second}
    \]
  \end{explanation}


		\item  Use the table of average velocities to approximate the instantaneous velocity of the ball when it hits the ground.
			 \[
			\renewcommand*{\arraystretch}{2.5}	
			\begin{array}{|l|l|}
			\hline
			\text{Time Interval} & \text{Average Velocity}  \\
			\hline
			[8.9, 9] & \frac{f(9)-f(8.9)}{.1}=\frac{0-15.84}{.1}=-158.4  \\
			\hline
			[8.99,9] & \frac{f(9)-f(8.99)}{.01}=\frac{0-1.5984}{.01}=-159.84 \\
			\hline
			[8.999, 9] & \frac{f(9)-f(8.999)}{.001}=\frac{0-.159984}{.001}=-159.984 \\
			\hline
			[8.9999, 9] &  \frac{f(9)-f(8.999)}{.0001}=\frac{0-.0159998}{.0001}=-159.998  \\
			\hline
			\end{array} 
			\]
		

		\begin{explanation}
		 The instantaneous velocity of the ball hitting the ground appears to be $-160$ ft/sec.
		\end{explanation}
		
		\item   Compute $v_{AV(t)}$ the average velocity of the ball on the time interval  $[t, 9]$, where $t<9$.

		\begin{explanation}
		
		\begin{align*}		 
		v_{AV(t)}&=\dfrac{f(9)-f(t)}{9-t}\\
		&=\dfrac{0-(-16t^2+128t+144)}{9-t}\\
		&=\dfrac{16t^2-128t-144)}{9-t}\\
         &=\dfrac{16(t^2-8t-9 )}{9-t}\\
         &=\dfrac{16(t-9)(t+1)}{9-t}\\
          &=\dfrac{-16(9-t)(t+1)}{9-t}\\
            &=-16(t+1)\  \text{feet per second}\\
            	\end{align*}
	\end{explanation}
\item   Compute $v(9)$, the \textbf{instantaneous} velocity of the ball at $t=9$.

		\begin{explanation}

		\begin{align*}		 
		v(9)&=\lim_{t \to 9^{-}}v_{AV(t)}\\
		&=\lim_{t \to 9^{-}}\dfrac{f(9)-f(t)}{9-t}\\
		\end{align*}
		Notice that this is an indeterminate form, with form $\ZeroOverZero$.
				
		\begin{align*}		 
		v(9)&=\lim_{t \to 9^{-}}v_{AV(t)}\\
		&=\lim_{t \to 9^{-}}\dfrac{f(9)-f(t)}{9-t}\\
            &=\lim_{t \to 9^{-}}-16(t+1)\\
            &=-16(9+1)\\
              &=-160 \  \text{feet per second}
	\end{align*}
     		\end{explanation}			
		\end{enumerate}	
\end{problem}



\end{document}
