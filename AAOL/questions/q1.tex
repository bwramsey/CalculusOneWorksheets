% Extracted from anApplicationOfLimits.tex, problem #1
\begin{problem} 
The position, $s(t)$, of an object moving along a horizontal line is given by $s(t) = t^2 - 4$, where $s$ is in meters and $t$ is in seconds, $0\leq t <5$.
\begin{enumerate}
\item
Mark the position of the object on the line at time $t = 1$:
\begin{image}
\includegraphics[scale = 1]{Figure9.png}
\end{image} 

\begin{freeResponse}
$s(1)=1^2-4=-3$
\begin{image}
\includegraphics[scale = .8]{Figure8.png}
\end{image}
\end{freeResponse}


\item
Find the average velocity, $v_{\mathrm{AV}}$, of the object during the time interval $[1, 3]$.
\WkstHop \begin{freeResponse}
The average velocity over $[1, 3]$ is
\[
\frac{s(3) - s(1)}{3-1} = \frac{5 - (-3)}{2} = \frac{8}{2} = 4 \text{m/s}
\]
\end{freeResponse}


\item
Compute the average velocity, $v_{\mathrm{AV}}(t)$, of the object during the time interval
\begin{enumerate}
\item
$[1, t]$, for $1<t<5$;
\WkstHop \begin{freeResponse}
The average velocity,$v_{\mathrm{AV}}(t)$ over $[1, t]$ is
\begin{align*}
v_{\mathrm{AV}}(t)=\frac{s(t) - s(1)}{t-1} &= \frac{(t^2-4) - (-3)}{t-1}\\
&= \frac{t^2-1}{t-1} = t+1,\quad 1<t<5
\end{align*}
\end{freeResponse}

\item
$[t, 1]$, for $0\leq t < 1$.
\WkstHop \begin{freeResponse}
The average velocity over $[t, 1]$ is
\begin{align*}
v_{\mathrm{AV}}(t)=\frac{s(1) - s(t)}{1-t} &=\frac{(-3) - (t^2-4)}{1-t}\\
&= \frac{1-t^2}{1-t} = 1+t, \quad 0<t<1
\end{align*}

Note: $v_{\mathrm{AV}}(t)=\frac{s(1) - s(t)}{1-t}= \frac{s(t) - s(1)}{t-1}=1+t, \quad 0\leq t<5$
\end{freeResponse}
\end{enumerate}

\item 
Find the instantaneous velocity, $v_{\mathrm{inst}}$, of the object at $t = 1$.
Justify your answer.
\WkstHop \begin{freeResponse}
The instantaneous velocity of the object at $t = 1$ is given by $\displaystyle \lim_{t\to 1} \dfrac{s(t)-s(1)}{t-1}$. This limit has form $\ZeroOverZero$.
\begin{align*}
v_{\mathrm{inst}} &=\lim_{t \to 1}v_{\mathrm{AV}}(t)= \lim_{t \to 1} \frac{s(t) - s(1)}{t-1} \\
&= \lim_{t \to 1} (t+1) = 2.
\end{align*}
\end{freeResponse}
\WkstNew

\item
The position-time graph of the function $s$ is given in the figure below.
\begin{image}
\includegraphics[scale = .7]{Figure10.png}
\end{image}
\begin{enumerate}
\item
Assume $P$ is a point on the graph of $s$.
Fill in the blank.
\[
P = (1, \mbox{\underline{\hspace{2em}}}).
\]
\WkstHop \begin{freeResponse}
$P = (1, \mbox{\underline{$-3$}})$
\end{freeResponse}


\item
Plot the point $P$ and draw the tangent line at this point in the figure above.
\WkstHop \begin{freeResponse} \hfil
\begin{image}
\includegraphics[scale = .7]{Figure11.png}
\end{image}
\end{freeResponse}


\item
Find the slope, $m_{\mathrm{tan}}$, of the tangent line in part (ii).
Explain.
\WkstHop \begin{freeResponse}
The slope of the tangent line at $t = 1$ is the same as the instantaneous velocity at $t = 1$.
Therefore $m_{\mathrm{tan}} = v_{\mathrm{inst}} = 2$.
\end{freeResponse}
\end{enumerate}
\end{enumerate}
\end{problem}
