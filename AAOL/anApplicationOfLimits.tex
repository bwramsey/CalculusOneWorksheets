%Add code to compile both versions from makefile at same time
\providecommand{\HCCondition}{0}
%Define each of the conditions
\ifcase\HCCondition
	%\condition=0 -> handout
	\documentclass[nooutcomes,noauthor,space,handout]{ximera}
	\title{An application of limits (AAOL)}  
\or	%\condition=1 -> Soln
	\documentclass[nooutcomes,noauthor]{ximera}
	\title{An application of limits (AAOL) - Solutions} 
\fi


\newcommand{\RR}{\mathbb R}
\renewcommand{\d}{\,d}
\newcommand{\dd}[2][]{\frac{d #1}{d #2}}
\renewcommand{\l}{\ell}
\newcommand{\ddx}{\frac{d}{dx}}
\newcommand{\dfn}{\textbf}
\newcommand{\eval}[1]{\bigg[ #1 \bigg]}
\renewcommand{\theenumii}{\textup{(\roman{enumii})}}
\renewcommand{\labelenumii}{\theenumii}

\usepackage{graphicx}
\usepackage{multicol}
\usepackage{tkz-euclide}
%\usepackage{unicode-math}

\usepackage{pgfplots}   % <- for graphics
\pgfplotsset{compat=newest}


\renewenvironment{freeResponse}{
\ifhandout\setbox0\vbox\bgroup\else
\begin{trivlist}\item[\hskip \labelsep\bfseries Solution:\hspace{2ex}]
\fi}
{\ifhandout\egroup\else
\end{trivlist}
\fi}

\newcommand*{\ZeroOverZero}{\ensuremath{\dfrac{0}{0}}}

\providecommand{\HCCondition}{0}
\newcommand{\WkstHop}[1][1]{\if\HCCondition 0
	\vspace*{\stretch{#1}} \fi} 
\newcommand{\WkstNew}{\if\HCCondition 0
	\newpage
	 \fi}  %% we can turn off input when making a master document

\usepackage{fullpage}




\begin{document}
\begin{abstract}		\end{abstract}
\maketitle
\ifcase\HCCondition
%summary in here
Suppose an object is traveling along a straight line, with displacement given by a function $s(t)$.
\begin{itemize}
	\item The \textit{average velocity}, $v_{av}$, of the object between time $t=a$ and time $t=b$ is given by 
		\begin{align*} 
			v_{av} &= \frac{\Delta s}{\Delta t} \\
				&= \frac{s(b)-s(a)}{b-a}
		\end{align*}
	\item The \textit{instantaneous velocity} of the object at time $t=a$ is given by
		\begin{align*}
			v(a) &= \lim_{\Delta t \to 0} \frac{\Delta s}{\Delta t}\\
				&= \lim_{t\to a} \frac{s(t)-s(a)}{t-a}\\
				&= \lim_{t \to a} v_{av}(t)
		\end{align*}
		where $v_{av}(t)$ denotes the average velocity on the interval $[a,t]$ if $t > a$, and $[t, a]$ if $t < a$.
\end{itemize}
\newpage
\section*{Recitation Questions}
\fi

%problem 1
\begin{problem} 
The position, $s(t)$, of an object moving along a horizontal line is given by $s(t) = t^2 - 4$, where $s$ is in meters and $t$ is in seconds, $0\leq t <5$.
\begin{enumerate}
\item
Mark the position of the object on the line at time $t = 1$:
\begin{image}
\includegraphics[scale = 1]{Figure9.png}
\end{image} 

\begin{freeResponse}
$s(1)=1^2-4=-3$
\begin{image}
\includegraphics[scale = .8]{Figure8.png}
\end{image}
\end{freeResponse}


\item
Find the average velocity, $v_{\mathrm{AV}}$, of the object during the time interval $[1, 3]$.
\WkstHop \begin{freeResponse}
The average velocity over $[1, 3]$ is
\[
\frac{s(3) - s(1)}{3-1} = \frac{5 - (-3)}{2} = \frac{8}{2} = 4 \text{m/s}
\]
\end{freeResponse}


\item
Compute the average velocity, $v_{\mathrm{AV}}(t)$, of the object during the time interval
\begin{enumerate}
\item
$[1, t]$, for $1<t<5$;
\WkstHop \begin{freeResponse}
The average velocity,$v_{\mathrm{AV}}(t)$ over $[1, t]$ is
\begin{align*}
v_{\mathrm{AV}}(t)=\frac{s(t) - s(1)}{t-1} &= \frac{(t^2-4) - (-3)}{t-1}\\
&= \frac{t^2-1}{t-1} = t+1,\quad 1<t<5
\end{align*}
\end{freeResponse}

\item
$[t, 1]$, for $0\leq t < 1$.
\WkstHop \begin{freeResponse}
The average velocity over $[t, 1]$ is
\begin{align*}
v_{\mathrm{AV}}(t)=\frac{s(1) - s(t)}{1-t} &=\frac{(-3) - (t^2-4)}{1-t}\\
&= \frac{1-t^2}{1-t} = 1+t, \quad 0<t<1
\end{align*}

Note: $v_{\mathrm{AV}}(t)=\frac{s(1) - s(t)}{1-t}= \frac{s(t) - s(1)}{t-1}=1+t, \quad 0\leq t<5$
\end{freeResponse}
\end{enumerate}

\item 
Find the instantaneous velocity, $v_{\mathrm{inst}}$, of the object at $t = 1$.
Justify your answer.
\WkstHop \begin{freeResponse}
The instantaneous velocity of the object at $t = 1$ is given by $\displaystyle \lim_{t\to 1} \dfrac{s(t)-s(1)}{t-1}$. This limit has form $\ZeroOverZero$.
\begin{align*}
v_{\mathrm{inst}} &=\lim_{t \to 1}v_{\mathrm{AV}}(t)= \lim_{t \to 1} \frac{s(t) - s(1)}{t-1} \\
&= \lim_{t \to 1} (t+1) = 2.
\end{align*}
\end{freeResponse}
\WkstNew

\item
The position-time graph of the function $s$ is given in the figure below.
\begin{image}
\includegraphics[scale = .7]{Figure10.png}
\end{image}
\begin{enumerate}
\item
Assume $P$ is a point on the graph of $s$.
Fill in the blank.
\[
P = (1, \mbox{\underline{\hspace{2em}}}).
\]
\WkstHop \begin{freeResponse}
$P = (1, \mbox{\underline{$-3$}})$
\end{freeResponse}


\item
Plot the point $P$ and draw the tangent line at this point in the figure above.
\WkstHop \begin{freeResponse} \hfil
\begin{image}
\includegraphics[scale = .7]{Figure11.png}
\end{image}
\end{freeResponse}


\item
Find the slope, $m_{\mathrm{tan}}$, of the tangent line in part (ii).
Explain.
\WkstHop \begin{freeResponse}
The slope of the tangent line at $t = 1$ is the same as the instantaneous velocity at $t = 1$.
Therefore $m_{\mathrm{tan}} = v_{\mathrm{inst}} = 2$.
\end{freeResponse}
\end{enumerate}
\end{enumerate}
\end{problem}\WkstNew 

          
%problem2
\begin{problem}
 Part of the given parabola can be used to model the ``position-time'' graph of a ball thrown straight up into the air.  The graph gives the height of the ball in feet $t$ seconds after being thrown into the air.\\
  Let the function $f$ be defined by  $f(t)=-16t^2+128t+144$.\\
  Use this graph, and the given function, $f$, to answer the following questions.
  \begin{image}
    \includegraphics[scale = 0.5]{figure2.png}
  \end{image}
	
	
		\begin{enumerate}
		\item Mark the part of the parabola that can be used to model the position of the ball.		
		\WkstHop \begin{freeResponse} \hfil
       		 \begin{center}
       	  	 \includegraphics[scale = 0.4]{ptGraphOfBall.png}
        		\end{center}
		\end{freeResponse}		

		 \item  What are the units on the $t$ axis?  What are the units on the $y$ axis?
		 \WkstHop \begin{freeResponse}		 
	The units on the $t$ axis are ``seconds'' (for time), while the units on the $y$ axis are ``feet'' (for height).        
		\end{freeResponse}
			
		\item  If you were watching a movie of the ball being thrown, is the graph a picture of the path that the ball follows?  Why or why not?
		\WkstHop \begin{freeResponse}		 
	 No, the position-time graph is \emph{not} the path the ball follows.
        The graph shows the height of the ball at a given \emph{time}.
        The ball is thrown straight up and has no horizontal movement so its path is on a vertical line.
		\end{freeResponse}
	
		\item Let $f(t)$ denote the height of the ball at time $t, t\geq 0$.  What is the height of the ball at time $t=0$?
		\WkstHop \begin{freeResponse}
		The height can be found by finding $f(0)$
		\begin{align*}
			f(0)&=-16(0)^2+128(0)+144 \\
			f(0)&=144\  \text{feet}
 			\end{align*}

		\end{freeResponse}
		\WkstNew
		
		\item  When will the ball hit the ground?
		\WkstHop \begin{freeResponse}		 
		The ball will hit the ground when the height $f(t)$ equals zero.
			\begin{align*}
			0&=-16t^2+128t+144 \\
			0&=-16(t^2-8t-9) \\
 			0&=-16(t+1)(t-9) \\
 			t&=-1, 9 
 			\end{align*}
		The ball will hit the ground at $t=9$ or 9 seconds after the ball is thrown into the air.  
		\end{freeResponse}

		\item  What is the domain of the position function, $f$, of the ball?

		 \WkstHop \begin{freeResponse}
     	   The domain of $f$ is the interval $[0, 9]$.   The ball is thrown straight up at $t=0$ and hits the ground at $t=9$.
     	   With this domain, the position-time graph of the ball is given by
       		 \begin{center}
       	  	 \includegraphics[scale = 0.4]{ptGraphOfBall.png}
        		\end{center}
     	 \end{freeResponse}

		
		
		\item Use the table of values to find the average velocity of the ball between $t=8.9$ and $t=9$ seconds.
	\[
\begin{array}{|l|l|}
			\hline
			t & \approx f(t)  \\
			\hline
			8.9 & 15.84  \\
			\hline
			8.99 & 1.6  \\
			\hline
			8.999 & 0.159984  \\
			\hline
			8.9999 &  0.015998  \\
			\hline
			9 &  0  \\
			\hline
			\end{array}
		\]
  \WkstHop \begin{freeResponse}
    The average velocity of the ball between $t = 8.9$ seconds and $t = 9$ seconds is
    \[
       \frac{f(9) - f(8.9)}{9- 8.9} = \frac{0- 15.84}{0.1} = -158.4\  \text{feet per second}
    \]
  \end{freeResponse}


		\item  Use the table of average velocities to approximate the instantaneous velocity of the ball when it hits the ground.
			 \[
			\renewcommand*{\arraystretch}{2.5}	
			\begin{array}{|l|l|}
			\hline
			\text{Time Interval} & \text{Average Velocity}  \\
			\hline
			[8.9, 9] & \frac{f(9)-f(8.9)}{.1}=\frac{0-15.84}{.1}=-158.4  \\
			\hline
			[8.99,9] & \frac{f(9)-f(8.99)}{.01}=\frac{0-1.5984}{.01}=-159.84 \\
			\hline
			[8.999, 9] & \frac{f(9)-f(8.999)}{.001}=\frac{0-.159984}{.001}=-159.984 \\
			\hline
			[8.9999, 9] &  \frac{f(9)-f(8.999)}{.0001}=\frac{0-.0159998}{.0001}=-159.998  \\
			\hline
			\end{array} 
			\]
		

		\WkstHop \begin{freeResponse}
		 The instantaneous velocity of the ball hitting the ground appears to be $-160$ ft/sec.
		\end{freeResponse}
		
		\WkstNew
			
		\item   Compute $v_{AV(t)}$ the average velocity of the ball on the time interval  $[t, 9]$, where $t<9$.

		\WkstHop \begin{freeResponse}
		
		\begin{align*}		 
		v_{AV(t)}&=\dfrac{f(9)-f(t)}{9-t}\\
		&=\dfrac{0-(-16t^2+128t+144)}{9-t}\\
		&=\dfrac{16t^2-128t-144)}{9-t}\\
         &=\dfrac{16(t^2-8t-9 )}{9-t}\\
         &=\dfrac{16(t-9)(t+1)}{9-t}\\
          &=\dfrac{-16(9-t)(t+1)}{9-t}\\
            &=-16(t+1)\  \text{feet per second}\\
            	\end{align*}
	\end{freeResponse}
\item   Compute $v(9)$, the \textbf{instantaneous} velocity of the ball at $t=9$.

		\WkstHop \begin{freeResponse}

		\begin{align*}		 
		v(9)&=\lim_{t \to 9^{-}}v_{AV(t)}\\
		&=\lim_{t \to 9^{-}}\dfrac{f(9)-f(t)}{9-t}\\
		\end{align*}
		Notice that this is an indeterminate form, with form $\ZeroOverZero$.
				
		\begin{align*}		 
		v(9)&=\lim_{t \to 9^{-}}v_{AV(t)}\\
		&=\lim_{t \to 9^{-}}\dfrac{f(9)-f(t)}{9-t}\\
            &=\lim_{t \to 9^{-}}-16(t+1)\\
            &=-16(9+1)\\
              &=-160 \  \text{feet per second}
	\end{align*}
     		\end{freeResponse}			
		\end{enumerate}	
\end{problem}\WkstNew						
				
\end{document} 


















