\documentclass{ximera}

\newcommand{\RR}{\mathbb R}
\renewcommand{\d}{\,d}
\newcommand{\dd}[2][]{\frac{d #1}{d #2}}
\renewcommand{\l}{\ell}
\newcommand{\ddx}{\frac{d}{dx}}
\newcommand{\dfn}{\textbf}
\newcommand{\eval}[1]{\bigg[ #1 \bigg]}
\renewcommand{\theenumii}{\textup{(\roman{enumii})}}
\renewcommand{\labelenumii}{\theenumii}

\usepackage{graphicx}
\usepackage{multicol}
\usepackage{tkz-euclide}
%\usepackage{unicode-math}

\usepackage{pgfplots}   % <- for graphics
\pgfplotsset{compat=newest}


\renewenvironment{freeResponse}{
\ifhandout\setbox0\vbox\bgroup\else
\begin{trivlist}\item[\hskip \labelsep\bfseries Solution:\hspace{2ex}]
\fi}
{\ifhandout\egroup\else
\end{trivlist}
\fi}

\newcommand*{\ZeroOverZero}{\ensuremath{\dfrac{0}{0}}}

\providecommand{\HCCondition}{0}
\newcommand{\WkstHop}[1][1]{\if\HCCondition 0
	\vspace*{\stretch{#1}} \fi} 
\newcommand{\WkstNew}{\if\HCCondition 0
	\newpage
	 \fi} 


\title[Problem 8]{Problem 8}

\begin{document}
\begin{abstract} \end{abstract}
\maketitle


% Extracted from optimization.tex, problem #8
\begin{problem}
  A rectangle is constructed with one side on the positive $x$-axis, one side on the positive $y$-axis, and the vertex opposite the origin on the line $y=10-2x$.
  What dimensions maximize the area of the rectangle?
  What is the maximum area?
\begin{explanation}
    The four vertices of the rectangle are $(0, 0)$, $(x, 0)$, $(x, y)$, and $(0, y)$:
    \begin{center}
      \includegraphics[scale = .5]{figure8.png}
    \end{center}
    We're trying to maximize the area $A = x \cdot y$.

    First, we want to turn this area formula into a function of $x$.
    To do that we use the constraint $y = 10 - 2x$.
    Substituting, we find the function we're trying to maximize is $A(x) = x (10 - 2x) = 10x - 2x^2$.

    Next, we need to identify the domain of $A$.
  
    To find the largest value we compute the $x$-intercept of $y = 10 - 2x$:
    \begin{align*}
      10 - 2x = 0 &\implies x = 5.
    \end{align*}
    Therefore $\mathrm{Domain}(A) = (0, 5)$.

    Next we locate the critical points of $A$.
    Since $A'(x) = 10 - 4x$ is defined on $(0, 5)$, to locate critical points solve 
    \begin{align*}
      A'(x) = 0 &\iff 10 - 4x = 0\\
                &\iff x = \frac{5}{2}.
    \end{align*}
To finish, we check the sign of the derivative an intervals $(0,\frac{5}{2})$, and $(\frac{5}{2},5)$.
Since $A'(x) = 10 - 4x=-4(x-\frac{5}{2})$, it follows that $A'(x)>0$ on the interval $(0,\frac{5}{2})$ and $A'(x)<0$ on the interval $(\frac{5}{2},5)$.
Therfore the function $A$ has a local maximum at $x=\frac{5}{2}$. This also a global maximum, since the function is increasing  on $(0,\frac{5}{2})$, and decreasing on $(\frac{5}{2},5)$.
    Therefore the maximum area occurs when the rectangle has a base with length $5/2$ units and height with length of $5$ units.
    The area with these dimensions is $25/2$.
  \end{explanation}
\end{problem}



\end{document}
