\documentclass{ximera}

\newcommand{\RR}{\mathbb R}
\renewcommand{\d}{\,d}
\newcommand{\dd}[2][]{\frac{d #1}{d #2}}
\renewcommand{\l}{\ell}
\newcommand{\ddx}{\frac{d}{dx}}
\newcommand{\dfn}{\textbf}
\newcommand{\eval}[1]{\bigg[ #1 \bigg]}
\renewcommand{\theenumii}{\textup{(\roman{enumii})}}
\renewcommand{\labelenumii}{\theenumii}

\usepackage{graphicx}
\usepackage{multicol}
\usepackage{tkz-euclide}
%\usepackage{unicode-math}

\usepackage{pgfplots}   % <- for graphics
\pgfplotsset{compat=newest}


\renewenvironment{freeResponse}{
\ifhandout\setbox0\vbox\bgroup\else
\begin{trivlist}\item[\hskip \labelsep\bfseries Solution:\hspace{2ex}]
\fi}
{\ifhandout\egroup\else
\end{trivlist}
\fi}

\newcommand*{\ZeroOverZero}{\ensuremath{\dfrac{0}{0}}}

\providecommand{\HCCondition}{0}
\newcommand{\WkstHop}[1][1]{\if\HCCondition 0
	\vspace*{\stretch{#1}} \fi} 
\newcommand{\WkstNew}{\if\HCCondition 0
	\newpage
	 \fi} 


\title[Problem 3]{Problem 3}

\begin{document}
\begin{abstract} \end{abstract}
\maketitle


% Extracted from optimization.tex, problem #3
\begin{problem}

  A part of a circle centered at the origin with radius $r = 7 \text{ cm}$ is given in the figure (A) below.
  A right triangle is formed in the first quadrant (see figure (A)).
  One of its sides lies on the $x$-axis.
  Its hypotenuse runs from the origin to a point on the circle.
  The hypotenuse makes an angle $\theta$ with the $x$-axis.

  \begin{image}
    \includegraphics[scale = 0.1]{triangleInCircle.png}
    \includegraphics[scale = 0.1]{figure7.png}
  \end{image}
  Make sure to label the picture.

  \begin{enumerate}
    \item
      Draw 2 more examples of such a triangle in the figure (B).
      \begin{explanation} \hfil
      
  \begin{image}
    \includegraphics[scale = 0.15]{twoMoreTriangles.png}
      \end{image}
 
      \end{explanation}

    \item
      Express the area of such a triangle as a function of $\theta$ and state its domain.
\begin{explanation}
        The area of a triangle is $(1/2)\cdot\mbox{ base } \cdot \mbox{ height }$.
        The base of the triangle is $7\cos(\theta)$ and the height is $7\sin(\theta)$.

        Therefore:
        \begin{align*}
          A(\theta) &= \frac{1}{2}\cdot 7\cos(\theta) \cdot 7\sin(\theta)\\
                    &= \frac{49}{2}\cos(\theta)\sin(\theta)
        \end{align*}

        and
        \begin{align*}
          \mbox{Domain of $A$ = $[0,\pi/2]$}
        \end{align*}
      \end{explanation}

    \item
      Find the value of $\theta$ which maximizes the area in part (b).
      Show your work and justify your answer.
[2]
      \begin{explanation}
        Finding critical points:
        \begin{align*}
          A'(\theta) &= \frac{-49}{2} \sin(\theta)\sin(\theta) + \frac{49}{2}\cos(\theta)\cos(\theta) \\
          &\implies A'(\theta) = 0\\
          &\implies \cos^2(\theta) = \sin^2(\theta)\\
          &\implies \cos(\theta) = \sin(\theta)\\
          &\implies \theta = \frac{\pi}{4}
        \end{align*}

        Locating global maximum:
        \begin{align*}
          A(0) = 0, A(\pi/4) = \frac{49}{4}, A(\pi/2) = 0 &\implies \mbox{global maximum at $\theta = \pi/4$}
        \end{align*}
      \end{explanation}
  \end{enumerate}
\end{problem}



\end{document}
