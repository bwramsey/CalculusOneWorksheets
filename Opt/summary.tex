\documentclass{ximera}

\newcommand{\RR}{\mathbb R}
\renewcommand{\d}{\,d}
\newcommand{\dd}[2][]{\frac{d #1}{d #2}}
\renewcommand{\l}{\ell}
\newcommand{\ddx}{\frac{d}{dx}}
\newcommand{\dfn}{\textbf}
\newcommand{\eval}[1]{\bigg[ #1 \bigg]}
\renewcommand{\theenumii}{\textup{(\roman{enumii})}}
\renewcommand{\labelenumii}{\theenumii}

\usepackage{graphicx}
\usepackage{multicol}
\usepackage{tkz-euclide}
%\usepackage{unicode-math}

\usepackage{pgfplots}   % <- for graphics
\pgfplotsset{compat=newest}


\renewenvironment{freeResponse}{
\ifhandout\setbox0\vbox\bgroup\else
\begin{trivlist}\item[\hskip \labelsep\bfseries Solution:\hspace{2ex}]
\fi}
{\ifhandout\egroup\else
\end{trivlist}
\fi}

\newcommand*{\ZeroOverZero}{\ensuremath{\dfrac{0}{0}}}

\providecommand{\HCCondition}{0}
\newcommand{\WkstHop}[1][1]{\if\HCCondition 0
	\vspace*{\stretch{#1}} \fi} 
\newcommand{\WkstNew}{\if\HCCondition 0
	\newpage
	 \fi} 

\title[Summary]{Summary}

\begin{document}
\begin{abstract} \end{abstract}
\maketitle


\subsection{How to Solve an Optimization Problem }

\begin{enumerate}
	\item Identify \textbf{variables}; draw and label the picture of the problem\\
	\item Identify the \textbf{objective function} (a quantity to be optimized);\\
		 write a \textbf{formula} for the objective function in terms of variables of the problem\\
	\item Identify the \textbf{constraint(s)};\\ use the constraint(s)  to express all the variables in terms of a \textbf{single variable}\\
	\item Write the \textbf{objective function in terms of a single variable};\\ find the \textbf{interval of interest}\\
	\item Using the methods of calculus, find the \textbf{global maximum/minimum}; \\\textbf{justify} your answer\\
 \end{enumerate}

\textbf{REMINDER: The Interval of Interest tells you the method to use to solve an optimization problem.}\\
 \textbf{CASE 1}: The interval of interest is a closed interval  $[a,b]$ \\[0.8em]
In this case, the Extreme Value Theorem (EVT)  guarantees that both global extrema of $f$ exist!\\
If $f$ has an global minimum and an global maximum which either occur at a critical point or at a boundary point (which means $a$ or $b$). 
To find an extreme values of $f$ on $[a,b]$, we:
\begin{itemize}
	\item Find all critical points and plug them into $f$. 
	\item Evaluate $f$ at both boundary points. 
	\item Compare the values. The biggest of those values is the maximum value of $f$ on $[a,b]$, and the least one is the minimum value of $f$ on $[a,b]$.
 \end{itemize}
 
\textbf{CASE 2}: The interval of interest is any interval.\\[0.5em]
In this case, the global minimum/maximum \textbf{may not exist}. 
Our approach is then:
\begin{itemize}
	\item Find all critical points of $f$ on the interval.
	\item Use 1st or 2nd Derivative Test to classify those critical points as \textbf{local maxima/minima}.
	\item If there is \emph{exactly one local extremum}, then it is an global extremum of the same type. (That is, if it is a local minimum then it is automatically a global minimum...)
\end{itemize}

OTHER CASES (e.g., An interval with multiple local extrema, etc) can be approached similar to how we graphed functions: check the boundary points, check the critical points, use the sign chart etc.




\end{document}
