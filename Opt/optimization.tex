

%Add code to compile both versions from makefile at same time
\providecommand{\HCCondition}{0}
%Define each of the conditions
\ifcase\HCCondition
	%\condition=0 -> handout
	\documentclass[nooutcomes,noauthor,space,handout]{ximera}
	\title{Optimization (O)}
\or	%\condition=1 -> Soln
	\documentclass[nooutcomes,noauthor]{ximera}
	\title{Optimization (O)- Solutions}  
\fi
\newcommand{\RR}{\mathbb R}
\renewcommand{\d}{\,d}
\newcommand{\dd}[2][]{\frac{d #1}{d #2}}
\renewcommand{\l}{\ell}
\newcommand{\ddx}{\frac{d}{dx}}
\newcommand{\dfn}{\textbf}
\newcommand{\eval}[1]{\bigg[ #1 \bigg]}
\renewcommand{\theenumii}{\textup{(\roman{enumii})}}
\renewcommand{\labelenumii}{\theenumii}

\usepackage{graphicx}
\usepackage{multicol}
\usepackage{tkz-euclide}
%\usepackage{unicode-math}

\usepackage{pgfplots}   % <- for graphics
\pgfplotsset{compat=newest}


\renewenvironment{freeResponse}{
\ifhandout\setbox0\vbox\bgroup\else
\begin{trivlist}\item[\hskip \labelsep\bfseries Solution:\hspace{2ex}]
\fi}
{\ifhandout\egroup\else
\end{trivlist}
\fi}

\newcommand*{\ZeroOverZero}{\ensuremath{\dfrac{0}{0}}}

\providecommand{\HCCondition}{0}
\newcommand{\WkstHop}[1][1]{\if\HCCondition 0
	\vspace*{\stretch{#1}} \fi} 
\newcommand{\WkstNew}{\if\HCCondition 0
	\newpage
	 \fi} 



\begin{document}
\begin{abstract}
\end{abstract}
\makeTagTitle
\ifcase\HCCondition
%summary in here
\section*{SUMMARY: How to Solve an Optimization Problem }

\begin{enumerate}
	\item Identify \textbf{variables}; draw and label the picture of the problem\\
	\item Identify the \textbf{objective function} (a quantity to be optimized);\\
		 write a \textbf{formula} for the objective function in terms of variables of the problem\\
	\item Identify the \textbf{constraint(s)};\\ use the constraint(s)  to express all the variables in terms of a \textbf{single variable}\\
	\item Write the \textbf{objective function in terms of a single variable};\\ find the \textbf{interval of interest}\\
	\item Using the methods of calculus, find the \textbf{global maximum/minimum}; \\\textbf{justify} your answer\\
 \end{enumerate}

\textbf{REMINDER: The Interval of Interest tells you the method to use to solve an optimization problem.}\\
 \textbf{CASE 1}: The interval of interest is a closed interval  $[a,b]$ \\[0.8em]
In this case, the Extreme Value Theorem (EVT)  guarantees that both global extrema of $f$ exist!\\
If $f$ has an global minimum and an global maximum which either occur at a critical point or at a boundary point (which means $a$ or $b$). 
To find an extreme values of $f$ on $[a,b]$, we:
\begin{itemize}
	\item Find all critical points and plug them into $f$. 
	\item Evaluate $f$ at both boundary points. 
	\item Compare the values. The biggest of those values is the maximum value of $f$ on $[a,b]$, and the least one is the minimum value of $f$ on $[a,b]$.
 \end{itemize}
 
\textbf{CASE 2}: The interval of interest is any interval.\\[0.5em]
In this case, the global minimum/maximum \textbf{may not exist}. 
Our approach is then:
\begin{itemize}
	\item Find all critical points of $f$ on the interval.
	\item Use 1st or 2nd Derivative Test to classify those critical points as \textbf{local maxima/minima}.
	\item If there is \emph{exactly one local extremum}, then it is an global extremum of the same type. (That is, if it is a local minimum then it is automatically a global minimum...)
\end{itemize}

OTHER CASES (e.g., An interval with multiple local extrema, etc) can be approached similar to how we graphed functions: check the boundary points, check the critical points, use the sign chart etc.

\newpage

\section*{Recitation Questions}
\fi
%problem 1
\begin{problem}

  Suppose you want to maximize a continuous function on a closed interval, but you find that it only has one local extremum on the  interval which happens to be a local minimum.
  Where else should you check for the solution? \textbf{EXPLAIN}.
\WkstHop
\begin{freeResponse}
  By the Extreme Value Theorem, the global maximum must occur either at a critical point or at an endpiont of the domain. If the only local extremum is a local minimum, the maximum will occur at one of the endpoints.
\end{freeResponse}	
\end{problem}


\WkstNew

%problem 2
\begin{problem}

  A rectangular flower garden with an area of $30 \, m^2$ is surrounded by a grass border $1 \, m$ wide on two sides and $2 \, m$ wide on the other two sides (see figure).
  What dimensions of the garden minimize the combined area of the garden and borders?
  
  \begin{image}
    \includegraphics[trim= 100 480 250 250]{Figure1.pdf}
  \end{image}

  \begin{enumerate}
    \item  Label the picture with variables.
      \begin{freeResponse}
        \begin{image}
          \includegraphics[trim= 640 500 250 210]{Figure6.pdf}
	\end{image}
      \end{freeResponse}

    \item  What are you trying to maximize or minimize?
      Write an equation for it in terms of the variables from (a).
 \WkstHop
     \begin{freeResponse}
        We want to minimize the combined area of the garden and border.  If $A$ denotes this area, then an equation for $A$ is
        $$ A = (x+4)(y+2) $$
      \end{freeResponse}

    \item  What is your constraint?
      Write a constraint equation in terms of the variables from (a).
\WkstHop
      \begin{freeResponse}
        We know that the area of the flower garden is $30 \, m^2$.  So our constraint equation is
        $$ xy = 30 $$
      \end{freeResponse}
      
    \item  Reduce your optimization equation to one variable using the constraint equation.

\WkstHop[2]
      \begin{freeResponse}
        $$ x = \frac{30}{y} \quad (\text{Note that } y \neq 0) $$
        \begin{align}
          A &= \left( \frac{30}{y} + 4 \right)(y+2) \\
            &= 30 + 4y + \frac{60}{y} + 8 \\
            &=4y + \frac{60}{y} + 38 \label{eqn3}
        \end{align}
      \end{freeResponse}
		
    \item  What is the interval on which your variable makes sense?
      Is it open or closed?
      What does this mean for the method of finding the global max or min?
\WkstHop[2]
      \begin{freeResponse}
        $0 < y < \infty$, which is an open interval.
        So we need there to only be one critical point to equation \eqref{eqn3} in the domain of $y$, and then we need to show that this critical point is a local minimum.
        This will imply that the critical point is a global minimum (since there is only one local extremum).  
      \end{freeResponse}
		
    \item  Use the appropriate method to find and justify your global extremum.
\WkstHop[4]
      \begin{freeResponse}
        We need to differentiate equation \eqref{eqn3} with respect to $y$, set this derivative equal to $0$, and then solve:
        $$ \dd[A]{y} = 4 - \frac{60}{y^2} :=0 $$
        $$ \frac{60}{y^2} = 4 $$
        $$ 4y^2 = 60 $$
        $$ y^2 = 15 $$
        $$ y = \pm \sqrt{15} $$
        Since $-\sqrt{15}$ is not in our domain, the only critical point of $A(y)$ in the interval $(0,\infty)$ is $\sqrt{15}$.  
        Thus, if $y=\sqrt{15}$ is a local minimum for $A$, then it will be a global minimum.  
        Using the second derivative test:
        $$ \dd[^2A]{y^2} = \frac{120}{y^3} $$
        $$ \eval{\dd[^2A]{y^2}}_{y=\sqrt{15}} = \frac{120}{15 \sqrt{15}} > 0. $$
        
        Since $A(y)$ is concave up at $y=\sqrt{15}$, this point is a local (and thus, global) minimum of $A(y)$.  
        Note that we also could have used the first derivative test to show that $y=\sqrt{15}$ was a local minimum of $A$.
      \end{freeResponse}
      
    \item  Be sure to answer the question asked in the original problem.
      \begin{freeResponse}
        Since $y=\sqrt{15} \, m$, we have that
        $$ x = \frac{30}{y} = \frac{30}{\sqrt{15}} = \frac{30 \sqrt{15}}{15} = 2 \sqrt{15} \, m $$
      \end{freeResponse}
    \end{enumerate}
\end{problem}


\WkstNew

%problem 3
\begin{problem}

  A part of a circle centered at the origin with radius $r = 7 \text{ cm}$ is given in the figure (A) below.
  A right triangle is formed in the first quadrant (see figure (A)).
  One of its sides lies on the $x$-axis.
  Its hypotenuse runs from the origin to a point on the circle.
  The hypotenuse makes an angle $\theta$ with the $x$-axis.

  \begin{image}
    \includegraphics[scale = 0.1]{triangleInCircle.png}
    \includegraphics[scale = 0.1]{figure7.png}
  \end{image}
  Make sure to label the picture.

  \begin{enumerate}
    \item
      Draw 2 more examples of such a triangle in the figure (B).
      \begin{freeResponse} \hfil
      
  \begin{image}
    \includegraphics[scale = 0.15]{twoMoreTriangles.png}
      \end{image}
 
      \end{freeResponse}

    \item
      Express the area of such a triangle as a function of $\theta$ and state its domain.
\WkstHop

      \begin{freeResponse}
        The area of a triangle is $(1/2)\cdot\mbox{ base } \cdot \mbox{ height }$.
        The base of the triangle is $7\cos(\theta)$ and the height is $7\sin(\theta)$.

        Therefore:
        \begin{align*}
          A(\theta) &= \frac{1}{2}\cdot 7\cos(\theta) \cdot 7\sin(\theta)\\
                    &= \frac{49}{2}\cos(\theta)\sin(\theta)
        \end{align*}

        and
        \begin{align*}
          \mbox{Domain of $A$ = $[0,\pi/2]$}
        \end{align*}
      \end{freeResponse}

    \item
      Find the value of $\theta$ which maximizes the area in part (b).
      Show your work and justify your answer.
\WkstHop[2]
      \begin{freeResponse}
        Finding critical points:
        \begin{align*}
          A'(\theta) &= \frac{-49}{2} \sin(\theta)\sin(\theta) + \frac{49}{2}\cos(\theta)\cos(\theta) \\
          &\implies A'(\theta) = 0\\
          &\implies \cos^2(\theta) = \sin^2(\theta)\\
          &\implies \cos(\theta) = \sin(\theta)\\
          &\implies \theta = \frac{\pi}{4}
        \end{align*}

        Locating global maximum:
        \begin{align*}
          A(0) = 0, A(\pi/4) = \frac{49}{4}, A(\pi/2) = 0 &\implies \mbox{global maximum at $\theta = \pi/4$}
        \end{align*}
      \end{freeResponse}
  \end{enumerate}
\end{problem}

\WkstNew

 % for drawing cube in Optimization problem
\usetikzlibrary{quotes,arrows.meta}
\tikzset{
  annotated cuboid/.pic={
    \tikzset{%
      every edge quotes/.append style={midway, auto},
      /cuboid/.cd,
      #1
    }
    \draw [every edge/.append style={pic actions, densely dashed, opacity=.5}, pic actions]
    (0,0,0) coordinate (o) -- ++(-\cubescale*\cubex,0,0) coordinate (a) -- ++(0,-\cubescale*\cubey,0) coordinate (b) edge coordinate [pos=1] (g) ++(0,0,-\cubescale*\cubez)  -- ++(\cubescale*\cubex,0,0) coordinate (c) -- cycle
    (o) -- ++(0,0,-\cubescale*\cubez) coordinate (d) -- ++(0,-\cubescale*\cubey,0) coordinate (e) edge (g) -- (c) -- cycle
    (o) -- (a) -- ++(0,0,-\cubescale*\cubez) coordinate (f) edge (g) -- (d) -- cycle;
    \path [every edge/.append style={pic actions, |-|}]
    (b) +(0,-5pt) coordinate (b1) edge ["y"'] (b1 -| c)
    (b) +(-5pt,0) coordinate (b2) edge ["x"] (b2 |- a)
    (c) +(3.5pt,-3.5pt) coordinate (c2) edge ["x"'] ([xshift=3.5pt,yshift=-3.5pt]e)
    ;
  },
  /cuboid/.search also={/tikz},
  /cuboid/.cd,
  width/.store in=\cubex,
  height/.store in=\cubey,
  depth/.store in=\cubez,
  units/.store in=\cubeunits,
  scale/.store in=\cubescale,
  width=10,
  height=10,
  depth=10,
  units=cm,
  scale=.1,
}
\begin{problem}
A box with two square sides is constructed to have a volume of 27. Denote the dimensions of the length of the base as $y$, and the width and height as $x$, as labeled in the diagram below. Find the dimensions needed to minimize the surface area of the box.
		\begin{center}
		\begin{tikzpicture}
			\pic {annotated cuboid={width=120, height=95, depth=95, scale=.02, units=cm}};
		\end{tikzpicture}
		\end{center}
		Solve the problem by performing the following steps.

		
		\begin{enumerate}
			\item Find a formula for $S(x)$ the surface area of the box, as a function of only the variable $x$.
 
 	\WkstHop

			\begin{freeResponse}
        				The volume of the box is given by $V=x^2 y$ and the surface area is given by $S = 2x^2 + 4xy$. We are told that the volume of the box is 27, which means our constraint is that $x^2y = 27$, so that $y = \frac{27}{x^2}$. Plugging this into the surface area formula gives 
        				\[ S(x) = 2x^2 + 4x\left( \frac{27}{x^2} \right) = 2x^2 + \frac{108}{x} \]
			      \end{freeResponse}
			\item Find the{interval of interest} for $S(x)$. 
\WkstHop
 				\begin{freeResponse}
        				Because $x$ represents a side-length of this box, it cannot be negative. Moreover, the function $S(x)$ is not defined at $x=0$,
        				so $x$ has to be positive. The Interval of Interest is $(0,\infty)$.
			      \end{freeResponse}
			      
			\item Find the $x$-value that gives the minimum surface area over all 
				such boxes.
\WkstHop[2]
				\begin{freeResponse}
        				Our objective function $S(x) = 2x^2 + \frac{108}{x}$ is continuous on the open interval $(0,\infty)$. This domain is an 
        				open interval, so to find the global minimum we start by finding the local extrema.
        				
        				$S'(x) = 4x - \frac{108}{x^2}$. This exists for all nonzero values of $x$, and since $x=0$ is not in our interval of interest,
        				our critical points must have $S'(x)=0$.
        				\begin{align*}
        					S'(x) &= 0\\
        					4x - \frac{108}{x^2} &= 0\\
						4x &= \frac{108}{x^2}\\
						x^3 &= 27\\
						x &= 3.
        				\end{align*} 
        				
        				There is one critical point, at $x=3$. Since $S''(x) = 4 + \frac{216}{x^3}$, we have $S''(3) = 4 + \frac{216}{27} > 0$. By the
        				Second Derivative Test, $x=3$ is a local minimum.
        				
        				Since $x=3$ was the only local extremum on the open interval $(0,\infty)$, it is the global minimum as well.
			      \end{freeResponse}
			      
	\end{enumerate}	

\end{problem}


\WkstNew



%problem 4
\begin{problem}
Find the radius of a cylindrical container with a volume of $2 \pi\ m^3$ that minimizes the surface area.


  \begin{image}
    \includegraphics[scale = 0.4]{figure1.png}
  \end{image}
\WkstHop

\begin{freeResponse}

We have two formulas: $V=\pi r^2h$ and $S=2 \pi rh+2r^2\pi$.  We know $V=2\pi$ and want to find the $r$ that minimizes $S$.  Since we need a forumula for $S$ only in terms of $r$ we are going to rewrite $V=\pi r^2h$ to give us a formula for for $h$ in terms of $r$.

\begin{align*}
V=2\pi \text{and}\ V=\pi r^2h \implies 2\pi &= \pi r^2h\\
\frac{2\pi}{\pi r^2}&=h\\
\frac{2}{r^2}&=h\\
\implies S(r)&=2 \pi r\left(\frac{2}{r^2}\right)+2r^2\pi\\
S(r)&=\frac{4\pi}{r}+2r^2\pi
\end{align*}

The domain is $(0,\infty)$.

Next, we need to minimize $S$.
\begin{align*}
S'(r)&=\frac{-4\pi}{r^2}+4\pi r\\
&=4\pi \left(r-\frac{1}{r^2}\right)
\end{align*}

To find the critical points:
\begin{align*}
r-\frac{1}{r^2}&=0\\
r&=\frac{1}{r^2}\\
r^3&=1\\
r&=1
\end{align*}

The only critical point of $S$ is at $r=1$.  We need to verify $r=1$ is a local extremum.

We can use the second derivative test:
\begin{align*}
S''(r)&=8\pi r^{-3}+4\pi\\
\implies S''(1)&=8\pi+4\pi>0
\end{align*}

The function $S$ has a local minimum at $r=1$.  So $(1,S(1)$ is the global minimum since this is the only local extremum.  (See Theorem 4.5).


\end{freeResponse}

\end{problem}



\WkstNew




\begin{problem}
  A cone is constructed by cutting a sector from a circular sheet of metal with radius 20.
  The cut sheet is then folded and welded.
  Find the radius and height of the cone with maximum volume that can be formed this way.

  \begin{image}
    \includegraphics[trim= 100 530 250 190]{Figure2.pdf}
  \end{image}
\WkstHop
	
  \begin{freeResponse}
    First, recall that the volume of a cone is
    \begin{equation}
      \label{cone volume}
      V = \frac{1}{3} \pi r^2 h
    \end{equation}
		
    Equation \eqref{cone volume} has two variables, $r$ and $h$.  
    So we need to find a constraint equation.  
    Notice that the length of the cone is a fixed length of 20.  
    So using Pythagorean's Theorem we have that:
    \begin{equation}
      \label{constraint}
      r^2 + h^2 = 20^2 = 400
    \end{equation}
		
    Solving equation \eqref{constraint} for $r^2$, we get $r^2 = 400 - h^2$.  
    Plugging this into equation \eqref{cone volume} gives
    \begin{equation}
      \label{V(h)}
      V = \frac{1}{3} \pi (400-h^2) h = \frac{1}{3} \pi (400h - h^3) 
    \end{equation}
    Note that, due to the settings of the problem, the domain for $h$ is $[0,20]$.  
		
    It is worth pointing out that it is a lot easier algebraically to solve for $r^2$ in equation \eqref{constraint} and then plug into equation \eqref{cone volume} instead of doing the same thing for $h$.

    Now we need to differentiate equation \eqref{V(h)} with respect to $h$, set this equal to $0$, and solve for $h$.
    $$ \dd[V]{h} = \frac{1}{3} \pi (400 - 3h^2) := 0 $$
    $$ 400 - 3h^2 = 0 $$
    $$ 3h^2 = 400 $$
    $$ h^2 = \frac{400}{3} $$
    $$ h = \pm \sqrt{\frac{400}{3}} = \pm \frac{20}{\sqrt{3}} $$
    but $- \frac{20}{\sqrt{3}}$ is not in our domain for $h$, and so the only critical point for $V(h)$ is $h = \frac{20}{\sqrt{3}}$.  
    We need to show that this is a global maximum for $V(h)$ on $[0,20]$.  
    Since $[0,20]$ is a closed interval, the Extreme Value Theorem says we need to evaluate $V(h)$ at $h=0, \frac{20}{\sqrt{3}}, 20$.  
    \begin{align}
      V(0) &= \frac{1}{3} \pi (0-0) = 0 \\
      V \left( \frac{20}{\sqrt{3}} \right) &= \frac{1}{3} \pi \left( \frac{20^3}{\sqrt{3}} - \frac{20^3}{\left( \sqrt{3} \right)^3} \right) > 0 \label{inequality} \\
      V(20) &= \frac{1}{3} \pi (20^3 - 20^3) = 0 
    \end{align}
		
    Thus $h=\frac{20}{\sqrt{3}}$ maximizes the volume of the cone.
    Then $$r^2 = 400 - h^2 = 400 - \left( \frac{20}{\sqrt{3}} \right)^2 = 400 - \frac{400}{3} = \frac{800}{3}$$ and so $$ r = 20 \sqrt{\frac{2}{3}}. $$
		
    If you are not comfortable with inequality \eqref{inequality} above without a calculator, then a nice alternative is to use the second derivative test instead to check that $h=\frac{20}{\sqrt{3}}$ maximizes the volume of the cone.
    This works since this is the only critical point in the domain of $h$.
    To do this, compute
    $$ \dd[^2V]{h^2} = \frac{1}{3} \pi (-6h) $$
    $$ \eval{\dd[^2V]{h^2}}_{h=\frac{20}{\sqrt{3}}} = \frac{1}{3} \pi \left( -6 \left(\frac{20}{\sqrt{3}} \right) \right) < 0 $$
    and thus this value for $h$ gives a local (and therefore, global) maximum value for $V(h)$.  
  \end{freeResponse}
\end{problem}

%problem 5

\WkstNew


%problem 6
\begin{problem}
  What point on the parabola $y=5-x^2$ is closest to the point $(4,7)$?
\WkstHop

  \begin{freeResponse}
    \begin{image}
      \includegraphics[scale = 0.3]{distToParabola.png}
    \end{image}

    We have ``to minimize $d \iff$ to minimize $d^2$''.

    Objective function:
    \begin{align*}
      d^2 = (x-4)^2 + (y-7)^2 \mbox{ and } y = 5 - x^2 &\implies d^2 = (x-4)^2 + \left((5 - x^2) -7 \right)^2\\
      &\implies f(x) = (x-4)^2 + (x^2 + 2)^2
    \end{align*}
    We may assume $\mbox{Domain}(f) = [0, \sqrt{5}]$.

    Critical points:
    \begin{align*}
      f'(x) &= 2(x-4) + 2(x^2 + 2) \cdot 2x\\
      &= 4x^3 + 10x - 8.
    \end{align*}
    $f'$ is defined everywhere on $(0, \sqrt{5}) \implies$ critical points only occur where $f'(x) = 0$:
    \begin{align*}
      f'(x) = 0 &\iff 4x^3 + 10x - 8 = 0\\
     \text{using a calculator to find an approximate value of}\ x &\implies x \approx 0.67628 
    \end{align*}

    Find global minimum:
    \begin{align*}
      f(0) &= 20\\
      f(0.67628) &\approx 17.1\\
      f(\sqrt{5}) &\approx 52.1\\
      &\implies \mbox{global minimum at $x \approx 0.67628$}
    \end{align*}
  \end{freeResponse}
\end{problem}

\WkstNew

%problem 7
\begin{problem}
  A rectangle is constructed with one side on the positive $x$-axis, one side on the positive $y$-axis, and the vertex opposite the origin on the line $y=10-2x$.
  What dimensions maximize the area of the rectangle?
  What is the maximum area?
\WkstHop

  \begin{freeResponse}
    The four vertices of the rectangle are $(0, 0)$, $(x, 0)$, $(x, y)$, and $(0, y)$:
    \begin{center}
      \includegraphics[scale = .5]{figure8.png}
    \end{center}
    We're trying to maximize the area $A = x \cdot y$.

    First, we want to turn this area formula into a function of $x$.
    To do that we use the constraint $y = 10 - 2x$.
    Substituting, we find the function we're trying to maximize is $A(x) = x (10 - 2x) = 10x - 2x^2$.

    Next, we need to identify the domain of $A$.
  
    To find the largest value we compute the $x$-intercept of $y = 10 - 2x$:
    \begin{align*}
      10 - 2x = 0 &\implies x = 5.
    \end{align*}
    Therefore $\mathrm{Domain}(A) = (0, 5)$.

    Next we locate the critical points of $A$.
    Since $A'(x) = 10 - 4x$ is defined on $(0, 5)$, to locate critical points solve 
    \begin{align*}
      A'(x) = 0 &\iff 10 - 4x = 0\\
                &\iff x = \frac{5}{2}.
    \end{align*}
To finish, we check the sign of the derivative an intervals $(0,\frac{5}{2})$, and $(\frac{5}{2},5)$.
Since $A'(x) = 10 - 4x=-4(x-\frac{5}{2})$, it follows that $A'(x)>0$ on the interval $(0,\frac{5}{2})$ and $A'(x)<0$ on the interval $(\frac{5}{2},5)$.
Therfore the function $A$ has a local maximum at $x=\frac{5}{2}$. This also a global maximum, since the function is increasing  on $(0,\frac{5}{2})$, and decreasing on $(\frac{5}{2},5)$.
    Therefore the maximum area occurs when the rectangle has a base with length $5/2$ units and height with length of $5$ units.
    The area with these dimensions is $25/2$.
  \end{freeResponse}
\end{problem}
\begin{problem}
Suppose you own a tour bus and you book groups of $20$ to $80$ people for a day tour. The cost per person is $ \$30$ minus $ \$ 0.25$ for every ticket sold. If gas and other miscellaneous costs are $ \$300$, how many tickets should you sell to maximize your profit? Treat the number of tickets as a nonnegative real number.
  \begin{freeResponse}
  Let $x$ denote the number of tickets sold.
  Then the revenue, $R$ is given by
  \[
  R(x)=x(30-x\cdot 0.25).
  \]
  Therefore, the profit, P,  is given by
  \[
  P(x)=R(x)-C(x)=x(30-x\cdot 0.25)-300.
  \]
  So, we have to find the maximum of the function $P$ on its domain, $[20,80]$.\\
  Since $P$ is continuous on the closed interval, the Extreme Value Theorem guarantees that the maximum of $P$ is attained on $[20,80]$.
  We have to find the critical points of $P$ and evaluate the function at critical points and at end points, and compare the values.\\
  Since
  \[
  P'(x)=30-0.5x
  \]
  we have to solve the equation 
  \[
30-0.5x=0.
  \]
 It follows that the function $P$ has its only critical points at $x=60$.
  Since 
  \[
  P(60)=60(30-60\cdot 0.25)-300=60\cdot 15-300=600,
  \]
  \[
  P(20)=20(30-20\cdot 0.25)-300=20\cdot 25-300=200,
  \]
  \[
   P(80)=80(30-80\cdot 0.25)-300=800-300=500,
  \]
  it follows that the maximum profit is $\$ 600$ and it is attained  when $60$ tickets are sold, i.e. at $x=60$.
    \end{freeResponse}
\end{problem}




\end{document} 
