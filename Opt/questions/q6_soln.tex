\documentclass{ximera}

\newcommand{\RR}{\mathbb R}
\renewcommand{\d}{\,d}
\newcommand{\dd}[2][]{\frac{d #1}{d #2}}
\renewcommand{\l}{\ell}
\newcommand{\ddx}{\frac{d}{dx}}
\newcommand{\dfn}{\textbf}
\newcommand{\eval}[1]{\bigg[ #1 \bigg]}
\renewcommand{\theenumii}{\textup{(\roman{enumii})}}
\renewcommand{\labelenumii}{\theenumii}

\usepackage{graphicx}
\usepackage{multicol}
\usepackage{tkz-euclide}
%\usepackage{unicode-math}

\usepackage{pgfplots}   % <- for graphics
\pgfplotsset{compat=newest}


\renewenvironment{freeResponse}{
\ifhandout\setbox0\vbox\bgroup\else
\begin{trivlist}\item[\hskip \labelsep\bfseries Solution:\hspace{2ex}]
\fi}
{\ifhandout\egroup\else
\end{trivlist}
\fi}

\newcommand*{\ZeroOverZero}{\ensuremath{\dfrac{0}{0}}}

\providecommand{\HCCondition}{0}
\newcommand{\WkstHop}[1][1]{\if\HCCondition 0
	\vspace*{\stretch{#1}} \fi} 
\newcommand{\WkstNew}{\if\HCCondition 0
	\newpage
	 \fi} 


\title[Problem 6]{Problem 6}

\begin{document}
\begin{abstract} \end{abstract}
\maketitle


% Extracted from optimization.tex, problem #6
\begin{problem}
  A cone is constructed by cutting a sector from a circular sheet of metal with radius 20.
  The cut sheet is then folded and welded.
  Find the radius and height of the cone with maximum volume that can be formed this way.

  \begin{image}
    \includegraphics[trim= 100 530 250 190]{Figure2.pdf}
  \end{image}
\begin{explanation}
    First, recall that the volume of a cone is
    \begin{equation}
      \label{cone volume}
      V = \frac{1}{3} \pi r^2 h
    \end{equation}
		
    Equation \eqref{cone volume} has two variables, $r$ and $h$.  
    So we need to find a constraint equation.  
    Notice that the length of the cone is a fixed length of 20.  
    So using Pythagorean's Theorem we have that:
    \begin{equation}
      \label{constraint}
      r^2 + h^2 = 20^2 = 400
    \end{equation}
		
    Solving equation \eqref{constraint} for $r^2$, we get $r^2 = 400 - h^2$.  
    Plugging this into equation \eqref{cone volume} gives
    \begin{equation}
      \label{V(h)}
      V = \frac{1}{3} \pi (400-h^2) h = \frac{1}{3} \pi (400h - h^3) 
    \end{equation}
    Note that, due to the settings of the problem, the domain for $h$ is $[0,20]$.  
		
    It is worth pointing out that it is a lot easier algebraically to solve for $r^2$ in equation \eqref{constraint} and then plug into equation \eqref{cone volume} instead of doing the same thing for $h$.

    Now we need to differentiate equation \eqref{V(h)} with respect to $h$, set this equal to $0$, and solve for $h$.
    $$ \dd[V]{h} = \frac{1}{3} \pi (400 - 3h^2) := 0 $$
    $$ 400 - 3h^2 = 0 $$
    $$ 3h^2 = 400 $$
    $$ h^2 = \frac{400}{3} $$
    $$ h = \pm \sqrt{\frac{400}{3}} = \pm \frac{20}{\sqrt{3}} $$
    but $- \frac{20}{\sqrt{3}}$ is not in our domain for $h$, and so the only critical point for $V(h)$ is $h = \frac{20}{\sqrt{3}}$.  
    We need to show that this is a global maximum for $V(h)$ on $[0,20]$.  
    Since $[0,20]$ is a closed interval, the Extreme Value Theorem says we need to evaluate $V(h)$ at $h=0, \frac{20}{\sqrt{3}}, 20$.  
    \begin{align}
      V(0) &= \frac{1}{3} \pi (0-0) = 0 \\
      V \left( \frac{20}{\sqrt{3}} \right) &= \frac{1}{3} \pi \left( \frac{20^3}{\sqrt{3}} - \frac{20^3}{\left( \sqrt{3} \right)^3} \right) > 0 \label{inequality} \\
      V(20) &= \frac{1}{3} \pi (20^3 - 20^3) = 0 
    \end{align}
		
    Thus $h=\frac{20}{\sqrt{3}}$ maximizes the volume of the cone.
    Then $$r^2 = 400 - h^2 = 400 - \left( \frac{20}{\sqrt{3}} \right)^2 = 400 - \frac{400}{3} = \frac{800}{3}$$ and so $$ r = 20 \sqrt{\frac{2}{3}}. $$
		
    If you are not comfortable with inequality \eqref{inequality} above without a calculator, then a nice alternative is to use the second derivative test instead to check that $h=\frac{20}{\sqrt{3}}$ maximizes the volume of the cone.
    This works since this is the only critical point in the domain of $h$.
    To do this, compute
    $$ \dd[^2V]{h^2} = \frac{1}{3} \pi (-6h) $$
    $$ \eval{\dd[^2V]{h^2}}_{h=\frac{20}{\sqrt{3}}} = \frac{1}{3} \pi \left( -6 \left(\frac{20}{\sqrt{3}} \right) \right) < 0 $$
    and thus this value for $h$ gives a local (and therefore, global) maximum value for $V(h)$.  
  \end{explanation}
\end{problem}



\end{document}
