% Extracted from optimization.tex, problem #2
\begin{problem}

  A rectangular flower garden with an area of $30 \, m^2$ is surrounded by a grass border $1 \, m$ wide on two sides and $2 \, m$ wide on the other two sides (see figure).
  What dimensions of the garden minimize the combined area of the garden and borders?
  
  \begin{image}
    \includegraphics[trim= 100 480 250 250]{Figure1.pdf}
  \end{image}

  \begin{enumerate}
    \item  Label the picture with variables.
      \begin{freeResponse}
        \begin{image}
          \includegraphics[trim= 640 500 250 210]{Figure6.pdf}
	\end{image}
      \end{freeResponse}

    \item  What are you trying to maximize or minimize?
      Write an equation for it in terms of the variables from (a).
 \WkstHop
     \begin{freeResponse}
        We want to minimize the combined area of the garden and border.  If $A$ denotes this area, then an equation for $A$ is
        $$ A = (x+4)(y+2) $$
      \end{freeResponse}

    \item  What is your constraint?
      Write a constraint equation in terms of the variables from (a).
\WkstHop
      \begin{freeResponse}
        We know that the area of the flower garden is $30 \, m^2$.  So our constraint equation is
        $$ xy = 30 $$
      \end{freeResponse}
      
    \item  Reduce your optimization equation to one variable using the constraint equation.

\WkstHop[2]
      \begin{freeResponse}
        $$ x = \frac{30}{y} \quad (\text{Note that } y \neq 0) $$
        \begin{align}
          A &= \left( \frac{30}{y} + 4 \right)(y+2) \\
            &= 30 + 4y + \frac{60}{y} + 8 \\
            &=4y + \frac{60}{y} + 38 \label{eqn3}
        \end{align}
      \end{freeResponse}
		
    \item  What is the interval on which your variable makes sense?
      Is it open or closed?
      What does this mean for the method of finding the global max or min?
\WkstHop[2]
      \begin{freeResponse}
        $0 < y < \infty$, which is an open interval.
        So we need there to only be one critical point to equation \eqref{eqn3} in the domain of $y$, and then we need to show that this critical point is a local minimum.
        This will imply that the critical point is a global minimum (since there is only one local extremum).  
      \end{freeResponse}
		
    \item  Use the appropriate method to find and justify your global extremum.
\WkstHop[4]
      \begin{freeResponse}
        We need to differentiate equation \eqref{eqn3} with respect to $y$, set this derivative equal to $0$, and then solve:
        $$ \dd[A]{y} = 4 - \frac{60}{y^2} :=0 $$
        $$ \frac{60}{y^2} = 4 $$
        $$ 4y^2 = 60 $$
        $$ y^2 = 15 $$
        $$ y = \pm \sqrt{15} $$
        Since $-\sqrt{15}$ is not in our domain, the only critical point of $A(y)$ in the interval $(0,\infty)$ is $\sqrt{15}$.  
        Thus, if $y=\sqrt{15}$ is a local minimum for $A$, then it will be a global minimum.  
        Using the second derivative test:
        $$ \dd[^2A]{y^2} = \frac{120}{y^3} $$
        $$ \eval{\dd[^2A]{y^2}}_{y=\sqrt{15}} = \frac{120}{15 \sqrt{15}} > 0. $$
        
        Since $A(y)$ is concave up at $y=\sqrt{15}$, this point is a local (and thus, global) minimum of $A(y)$.  
        Note that we also could have used the first derivative test to show that $y=\sqrt{15}$ was a local minimum of $A$.
      \end{freeResponse}
      
    \item  Be sure to answer the question asked in the original problem.
      \begin{freeResponse}
        Since $y=\sqrt{15} \, m$, we have that
        $$ x = \frac{30}{y} = \frac{30}{\sqrt{15}} = \frac{30 \sqrt{15}}{15} = 2 \sqrt{15} \, m $$
      \end{freeResponse}
    \end{enumerate}
\end{problem}
